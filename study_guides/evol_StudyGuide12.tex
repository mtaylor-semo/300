%!TEX TS-program = lualatex
%!TEX encoding = UTF-8 Unicode

\documentclass[letterpaper]{tufte-handout}

%\geometry{showframe} % display margins for debugging page layout

\usepackage{fontspec}
\def\mainfont{Linux Libertine O}
\setmainfont[Ligatures={Common,TeX}, Contextuals={NoAlternate}, BoldFont={* Bold}, ItalicFont={* Italic}, Numbers={OldStyle}]{\mainfont}
\setsansfont[Scale=MatchLowercase]{Linux Biolinum O} 
\usepackage{microtype}
\usepackage{graphicx} % allow embedded images
  \setkeys{Gin}{width=\linewidth,totalheight=\textheight,keepaspectratio}
  \graphicspath{{img/}} % set of paths to search for images
\usepackage{amsmath}  % extended mathematics
\usepackage{booktabs} % book-quality tables
\usepackage{units}    % non-stacked fractions and better unit spacing
\usepackage{siunitx}
\usepackage{multicol} % multiple column layout facilities
\usepackage{microtype}   % filler text
\usepackage{hyperref}
%\usepackage{fancyvrb} % extended verbatim environments
%  \fvset{fontsize=\normalsize}% default font size for fancy-verbatim environments
\makeatletter
% Paragraph indentation and separation for normal text
\renewcommand{\@tufte@reset@par}{%
  \setlength{\RaggedRightParindent}{1.0pc}%
  \setlength{\JustifyingParindent}{1.0pc}%
  \setlength{\parindent}{1pc}%
  \setlength{\parskip}{0pt}%
}
\@tufte@reset@par

% Paragraph indentation and separation for marginal text
\renewcommand{\@tufte@margin@par}{%
  \setlength{\RaggedRightParindent}{0pt}%
  \setlength{\JustifyingParindent}{0.5pc}%
  \setlength{\parindent}{0.5pc}%
  \setlength{\parskip}{0pt}%
}
\makeatother

% Set up the spacing using fontspec features
\renewcommand\allcapsspacing[1]{{\addfontfeatures{LetterSpace=15}#1}}
\renewcommand\smallcapsspacing[1]{{\addfontfeatures{LetterSpace=10}#1}}

\title{Study Guide 12\hfill}
\author{Evolution of Humans and Flowering Plants}

\date{} % without \date command, current date is supplied

\begin{document}

\maketitle	% this prints the handout title, author, and date

%\printclassoptions

\section{Vocabulary}\marginnote{\textbf{Read:} 305 for plant genetic toolkit; 108--113, 267--268, 279--281, 553-573. Read the remainder of chapter 17 if curious about human evolution.\\
\noindent\textbf{Questions:} pg. 116--117, MC 2, 9; SA 5; pgs. 598--599, MC\ 1--3, 6; SA\ 1--4.}
\vspace{-1\baselineskip}
\begin{multicols}{2}
toolkit genes\\
FOXP2\\
MADS genes\\
\end{multicols}

\section{Concepts}

You should \emph{write} clear and concise answers to each question in the Concepts section.  The questions are not necessarily independent.  Think broadly across lectures to see ``the big picture.'' 

\begin{enumerate}
	\item Briefly state and explain 3--4 hypotheses that have been proposed to explain the selective advantage of bipedalism over quadrupedalism.  Do you think any one hypothesis is more likely than the others?  Why or why not?

	\item Two hypotheses have been proposed for the dispersal of the genus \textit{Homo} from central Africa.  Briefly explain the two leading hypotheses for this dispersal.  Which one is supported by the genetic evidence.\sidenote{You do not have to explain the genetic evidence.}  During what geological period or periods did this occur? What era?

	\item Why might FOXP2 have been important in the early evolution of human civilizations?  What evidence from other organisms supports this role for this toolkit gene? Explain.

	\item How many classes of MADS genes are there?  List them.

	\item Describe the model of floral morphology based on the classes of MADS genes.  Be sure to describe which class of MADS genes are associated with each floral whorl (sepals, petals, stamens, carpels)

	\item Discuss the evidence from the Ranunculaceae flowers (buttercup family) that suggests that the evolutionary history of MADS genes may be closely associated with diversity of angiosperms.\sidenote{Look it up if you do not know!}  Explain the evolution of floral diversity, as represented by the Ranunculaceae, due to natural selection for different types of pollination / pollinators.

\end{enumerate}

\end{document}