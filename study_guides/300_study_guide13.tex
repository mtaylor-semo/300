%!TEX TS-program = lualatex
%!TEX encoding = UTF-8 Unicode

\documentclass[letterpaper]{tufte-handout}

%\geometry{showframe} % display margins for debugging page layout

\usepackage{fontspec}
\def\mainfont{Linux Libertine O}
\setmainfont[Ligatures={Common,TeX}, Contextuals={NoAlternate}, BoldFont={* Bold}, ItalicFont={* Italic}, Numbers={OldStyle}]{\mainfont}
\setsansfont[Scale=MatchLowercase]{Linux Biolinum O} 
\usepackage{microtype}

\usepackage{graphicx} % allow embedded images
  \setkeys{Gin}{width=\linewidth,totalheight=\textheight,keepaspectratio}
  \graphicspath{{img/}} % set of paths to search for images
\usepackage{amsmath}  % extended mathematics
\usepackage{booktabs} % book-quality tables
\usepackage{units}    % non-stacked fractions and better unit spacing
\usepackage{siunitx}
\usepackage{multicol} % multiple column layout facilities
\usepackage{microtype}   % filler text
\usepackage{hyperref}

\usepackage{enumitem}

%\usepackage{fancyvrb} % extended verbatim environments
%  \fvset{fontsize=\normalsize}% default font size for fancy-verbatim environments
\makeatletter
% Paragraph indentation and separation for normal text
\renewcommand{\@tufte@reset@par}{%
  \setlength{\RaggedRightParindent}{1.0pc}%
  \setlength{\JustifyingParindent}{1.0pc}%
  \setlength{\parindent}{0pc}%
  \setlength{\parskip}{0.5\baselineskip}%
}
\@tufte@reset@par

% Paragraph indentation and separation for marginal text
\renewcommand{\@tufte@margin@par}{%
  \setlength{\RaggedRightParindent}{0pt}%
  \setlength{\JustifyingParindent}{0.5pc}%
  \setlength{\parindent}{0.5pc}%
  \setlength{\parskip}{0pt}%
}
\makeatother

% Set up the spacing using fontspec features
\renewcommand\allcapsspacing[1]{{\addfontfeatures{LetterSpace=15}#1}}
\renewcommand\smallcapsspacing[1]{{\addfontfeatures{LetterSpace=10}#1}}


\title{Study Guide 13\hfill}
\author{Species and speciation}

\date{} % without \date command, current date is supplied

\begin{document}

\maketitle	% this prints the handout title, author, and date

\section{Vocabulary}\marginnote{\textbf{Read:} to be updated.}

\begin{multicols}{2}
morphological species\\
biological species\\
hybrid zone \\
introgression \\
speciation\\
reproductive isolating barrier\\
prezygotic barrier \\
%prezygotic, postmating\\\hspace{1em}barrier\\
postzygotic barrier \\
ecological inviability\\
hybrid sterility\\
reinforcement \\
character displacement \\
geographic barrier \\
allopatric \\
sympatric \\
parapatric \\
vicariance \\
founder event
\end{multicols}

%\printclassoptions

\section{Concepts}

You should \emph{write} clear and concise answers to each question in the Concepts section.  The questions are not necessarily independent.  Think broadly across lectures to see ``the big picture.'' 

\begin{enumerate}
	
	\item Explain the advantages and disadvantages of the biological species concept.  Why is this one most widely used?  What mode of reproduction is required for the biological species concept to apply?  What mode of reproduction would not fit the biological species concept?

\item Explain how geographical barriers and biological barriers contribute to the evolution of reproductive isolation between two populations.  Why are “geographical” and “biological” not synonyms in the context of reproductive isolation?

\item Given an example, hypothetical or real, be able to determine whether the reproductive barriers are prezygotic or postzygotic.  Be able to provide a real-world example of each, different from any examples provided in class.

\item Compare and contrast different types of prezygotic isolation.

\item Compare and contrast allopatric, parapatric, and sympatric speciation in terms of geographical overlap or proximity and gene flow. What are the key factors that distinguish among them?

\item How does allopatry enhance speciation by sexual or ecological selection?

\item Why is sympatric speciation thought to be the least common form of speciation?

\end{enumerate}

\end{document}