%!TEX TS-program = lualatex
%!TEX encoding = UTF-8 Unicode

\documentclass[letterpaper]{tufte-handout}

%\geometry{showframe} % display margins for debugging page layout

\usepackage{fontspec}
\def\mainfont{Linux Libertine O}
\setmainfont[Ligatures={Common,TeX}, Contextuals={NoAlternate}, BoldFont={* Bold}, ItalicFont={* Italic}, Numbers={OldStyle}]{\mainfont}
\setsansfont[Scale=MatchLowercase]{Linux Biolinum O} 
\usepackage{microtype}

\usepackage{graphicx} % allow embedded images
  \setkeys{Gin}{width=\linewidth,totalheight=\textheight,keepaspectratio}
  \graphicspath{{img/}} % set of paths to search for images
\usepackage{amsmath}  % extended mathematics
\usepackage{booktabs} % book-quality tables
\usepackage{units}    % non-stacked fractions and better unit spacing
\usepackage{siunitx}
\usepackage{multicol} % multiple column layout facilities
\usepackage{microtype}   % filler text
%\usepackage{hyperref}
%\usepackage{fancyvrb} % extended verbatim environments
%  \fvset{fontsize=\normalsize}% default font size for fancy-verbatim environments
\makeatletter
% Paragraph indentation and separation for normal text
\renewcommand{\@tufte@reset@par}{%
  \setlength{\RaggedRightParindent}{1.0pc}%
  \setlength{\JustifyingParindent}{1.0pc}%
  \setlength{\parindent}{1pc}%
  \setlength{\parskip}{0pt}%
}
\@tufte@reset@par

% Paragraph indentation and separation for marginal text
\renewcommand{\@tufte@margin@par}{%
  \setlength{\RaggedRightParindent}{0pt}%
  \setlength{\JustifyingParindent}{0.5pc}%
  \setlength{\parindent}{0.5pc}%
  \setlength{\parskip}{0pt}%
}
\makeatother

% Set up the spacing using fontspec features
\renewcommand\allcapsspacing[1]{{\addfontfeatures{LetterSpace=15}#1}}
\renewcommand\smallcapsspacing[1]{{\addfontfeatures{LetterSpace=10}#1}}

\title{Study Guide 09\hfill}
\author{Evolution of Jaws and Limbs}

\date{} % without \date command, current date is supplied

\begin{document}

\maketitle	% this prints the handout title, author, and date

%\printclassoptions

\section{Vocabulary}\marginnote{\textbf{Read:} Chap. 3; 108--114.\\\textbf{Questions:} pg. 124, SA 4, 5.}
\vspace{-1\baselineskip}
\begin{multicols}{2}
toolkit genes\\
calmodulin (CaM1)\\
bone morphogenetic protein\ 4\\\hspace{1em}(BMP4)\\
sonic hedgehog (shh)\\
hox genes\\
co-option
\end{multicols}

\section{Concepts}

You should \emph{write} clear and concise answers to each question in the Concepts section.  The questions are not necessarily independent.  Think broadly across lectures to see ``the big picture.'' 

\begin{enumerate}
	\item Describe the basic evolution of jaws.  Use the original model, not the latest evidence (as that is still being figured out).

	\item Name two toolkit genes that have been implicated in the evolution of jaws.  What evidence exists from studies of modern vertebrates to support the hypothesis that these toolkit genes could be partly responsible for the evolution of jaws.

	\item About when did the transition to land occur?  What fossil evidence from this time period is consistent with the evolutionary origin of tetrapods?  What recent evidence casts the original hypothesis in to doubt?

	\item What homologies to fish and tetrapods does Tiktaalik have that supports this fossil as transitional form between lobe-finned fishes and amphibians?

	\item Discuss the evolutionary importance of co-option at the genetic level.  Use the evolution of tetrapod limbs as an example. 

	\item Describe how the expression of co-opted hox genes in the tetrapod limb consistent with their original expression along the anterior-posterior vertebrate axis.

	\item Scientists have long held that the evolution of limbs has occurred independently several times within the metazoan kingdom.  For example, there is no structural homology between the appendages of crabs and the appendages of birds. Yet, analyses of the toolkit genes responsible for appendage development suggests homology at the genetic level.  Discuss the evidence that suggests the development of metazoan appendages is genetically homologous.  Be sure to name the genes (acronyms are fine, e.g., shh) and describe and illustrate the general patterns of expression for the genes in the developing limb.
\end{enumerate}

\end{document}