%!TEX TS-program = lualatex
%!TEX encoding = UTF-8 Unicode

\documentclass[letterpaper]{tufte-handout}

%\geometry{showframe} % display margins for debugging page layout

\usepackage{fontspec}
\def\mainfont{Linux Libertine O}
\setmainfont[Ligatures={Common,TeX}, Contextuals={NoAlternate}, BoldFont={* Bold}, ItalicFont={* Italic}, Numbers={OldStyle}]{\mainfont}
\setsansfont[Scale=MatchLowercase]{Linux Biolinum O} 
\usepackage{microtype}

\usepackage{graphicx} % allow embedded images
  \setkeys{Gin}{width=\linewidth,totalheight=\textheight,keepaspectratio}
  \graphicspath{{img/}} % set of paths to search for images
\usepackage{amsmath}  % extended mathematics
\usepackage{booktabs} % book-quality tables
\usepackage{units}    % non-stacked fractions and better unit spacing
\usepackage{siunitx}
\usepackage{multicol} % multiple column layout facilities
\usepackage{microtype}   % filler text
\usepackage{hyperref}
%\usepackage{fancyvrb} % extended verbatim environments
%  \fvset{fontsize=\normalsize}% default font size for fancy-verbatim environments
\makeatletter
% Paragraph indentation and separation for normal text
\renewcommand{\@tufte@reset@par}{%
  \setlength{\RaggedRightParindent}{1.0pc}%
  \setlength{\JustifyingParindent}{1.0pc}%
  \setlength{\parindent}{1pc}%
  \setlength{\parskip}{0pt}%
}
\@tufte@reset@par

% Paragraph indentation and separation for marginal text
\renewcommand{\@tufte@margin@par}{%
  \setlength{\RaggedRightParindent}{0pt}%
  \setlength{\JustifyingParindent}{0.5pc}%
  \setlength{\parindent}{0.5pc}%
  \setlength{\parskip}{0pt}%
}
\makeatother

% Set up the spacing using fontspec features
\renewcommand\allcapsspacing[1]{{\addfontfeatures{LetterSpace=15}#1}}
\renewcommand\smallcapsspacing[1]{{\addfontfeatures{LetterSpace=10}#1}}

\title{Study Guide 13\hfill}
\author{Evolutionary Trends in the Fossil Record}

\date{} % without \date command, current date is supplied

\begin{document}

\maketitle	% this prints the handout title, author, and date

%\printclassoptions

\section{Vocabulary}\marginnote{\textbf{Read:} 451--453, 457--465 (skip Box\ 14.1).\\
	\noindent\textbf{Questions:} pgs. 484--485, MC 1, 4, 6; SC 1.}
\vspace{-1\baselineskip}
\begin{multicols}{2}
Diversification Rate ($R$)\\
Origination Rate\\\hspace{1em}(Speciation; $S$)\\
Extinction Rate ($E$)\\
Turnover Rate\\
Ecological Release\\
Ecological Divergence\\
Coevolution
\end{multicols}

\section{Concepts}

You should \emph{write} clear and concise answers to each question in the Concepts section.  The questions are not necessarily independent.  Think broadly across lectures to see ``the big picture.'' 

\begin{enumerate}
	\item The fossil record is unfortunately very incomplete.  Discuss the three principle reasons the fossil record is incomplete, and the consequences of inferring evolutionary history from the incomplete fossil record.

	\item The fossil record is also biased.  Discuss the three types of bias found in the fossil record and the consequences of inferring evolutionary history from the biased fossil record.  The term ``consequences'' has a negative undertone.  Can you think of a positive benefit that could result from this bias?

	\item If the fossil record of a particular geological formation is found to have a diversification rate $R > 0$, what does this mean in terms of origination rate and extinction rate?  What if $R < 0$?

	\item What trends have been observed in the fossil record in terms of:
	\begin{itemize}
		\item overall diversity
		\item rate of origination
		\item rate of extinction
	\end{itemize}
	
	\item How would you define turnover rate?  What trends have been observed when turnover rate is compared across multiple taxa in the fossil record?

	\item What mechanisms have been proposed to explain the link between rates of origination and rates of extinction?  That is, what mechanisms have been proposed to explain observed turnover rates (high or low)?  What justification do evolutionary biologists have for these explanations?  

	\item We discussed three hypotheses have been proposed to explain why some taxa show higher rates of origination compared to other taxa.  What are these three hypotheses?  Can you explain each of them, and how they support increased origination rates?

\end{enumerate}

\end{document}