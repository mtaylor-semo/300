\documentclass[letterpaper]{tufte-handout}

%\geometry{showframe} % display margins for debugging page layout

\usepackage{graphicx} % allow embedded images
  \setkeys{Gin}{width=\linewidth,totalheight=\textheight,keepaspectratio}
  \graphicspath{{img/}} % set of paths to search for images
\usepackage{amsmath}  % extended mathematics
\usepackage{booktabs} % book-quality tables
\usepackage{units}    % non-stacked fractions and better unit spacing
\usepackage{siunitx}
\usepackage{multicol} % multiple column layout facilities
\usepackage{microtype}   % filler text
\usepackage{hyperref}
\usepackage{enumitem}
%\usepackage{fancyvrb} % extended verbatim environments
%  \fvset{fontsize=\normalsize}% default font size for fancy-verbatim environments
\makeatletter
% Paragraph indentation and separation for normal text
\renewcommand{\@tufte@reset@par}{%
  \setlength{\RaggedRightParindent}{1.0pc}%
  \setlength{\JustifyingParindent}{1.0pc}%
  \setlength{\parindent}{1pc}%
  \setlength{\parskip}{0pt}%
}
\@tufte@reset@par

% Paragraph indentation and separation for marginal text
\renewcommand{\@tufte@margin@par}{%
  \setlength{\RaggedRightParindent}{0pt}%
  \setlength{\JustifyingParindent}{0.5pc}%
  \setlength{\parindent}{0.5pc}%
  \setlength{\parskip}{0pt}%
}
\makeatother


\title{Study Guides 15 and 16\hfill}
\author{Evolution of Sex; Sexual Selection}

\date{} % without \date command, current date is supplied

\begin{document}

\maketitle	% this prints the handout title, author, and date

%\printclassoptions

\section{Vocabulary}\marginnote{\textbf{Read:} 345--376.\\
	\noindent\textbf{Questions:} pgs. 377--378, MC 1--, 4--6, 8--12; SA 1--6.}
\vspace{-1\baselineskip}
\begin{multicols}{2}
asexual reproduction\\
sexual reproduction\\
cost of sex\\
sneaking males\\
sequential hermaphroditism\\
protogyny\\
protandry\\
sexual dimorphism\\
intrasexual selection\\
intersexual selection\\
mate choice\\
same-sex competition\\
sensory bias\\
sensory drive\\
sensory exploitation\\
antagonistic coevolution\\
sperm competition\\
sperm precedence\\\hspace{1em}(last male precedence)\\
social monogamy
\end{multicols}

\section{Concepts}

You should \emph{write} clear and concise answers to each question in the Concepts section.  The questions are not necessarily independent.  Think broadly across lectures to see ``the big picture.'' 

\begin{enumerate}
	\item Discuss in terms of fitness and adaptive flexibility the advantages and disadvantages of both asexual and sexual reproduction. That is, what are the advantages and disadvantages of maintaining the same genome (asexual) versus a variable genome (sexual) in relation to the current environment (in a broad sense) and possible future changes to the environment.

	\item Which has a greater inherent fitness, asexual or sexual reproduction?  Why?  Explain in terms of ``cost of sex,'' using words and illustration. 

	\item What are the advantages of sex?  What two major ideas or themes that explain the evolution and maintenance of sexual reproduction, despite its inherent cost.

	\item What are sneaking males?  What selective forces might lead to the evolution of sneaking?  Would you expect sneaking to evolve in a sexually selected species?  Why or why not?

	\item In terms of male and female fitness, what conditions favor the evolution of sequential hermaphroditism?

	\item Discuss the relative fitness of males and of females at different ages or size classes that lead to protogyny and to protandry.  How does this compare to fitness of both sexes for species that do not have sequential hermaphroditism? 

	\item Explain how female investment of energy compared to male investment of energy helps to explain the evolution of sexual selection.

	\item Sexual selection is a form of natural selection.  However, there is an important distinction to be made between how sexual and natural selection act on reproductive success.  One operates directly on reproductive success and the other operates indirectly. Tell which form of selection operates directly and which operates indirectly, and then explain why this distinction is evolutionarily important.

	\item List and explain various forms of sexual selection (e.g., antagonistic coevolution, indirect male-male competition, etc). Search the scientific literature for examples of antagonistic coevolution.\sidenote{I won't ask you questions about what you find but the literature is full of many bizarrely wonderful examples.}

	\item Explain the difference between inter- and intrasexual selection.  Explain the direct or indirect role of the female, as appropriate, for each type.

	\item Compare and contrast the good genes hypothesis with the runaway sexual selection hypothesis.  

%	\item Explain how sensory bias, sensory exploitation and sensory drive differ from each other.  How might these interact to lead to the evolution of sexually dimorphic traits?  How might ecological factors (e.g., predation) limit the evolution of the trait?

	\item Explain how sensory bias and sensory drive differ from each other.  How might these interact to lead to the evolution of sexually dimorphic traits?  How might ecological factors (e.g., predation) limit the evolution of the trait?

	\item Are exaggerated traits in males always beneficial from the standpoint of sexual selection?  What about natural selection?  Why or why not?  If not (for one or both sexual and natural selection), then what hypotheses explain the evolution of exaggerated traits?

	\item Explain why natural selection could favor the evolution of sexual cannibalism (sacrificial males).  Use the Australian redback spider and Chinese praying mantis in your example.  Think in terms of fitness, the interests of the male, and the interests of the female.

	\item Penis-fencing in marine flatworms involves a cost to the ``loser'' of the fencing match. What is this ``cost?''  Explain how the loser is determined and why losing is costly.  

	\item Explain social monogamy. Explain why some species of birds are sexually dimorphic when they display social monogamy.

\end{enumerate}

\end{document}