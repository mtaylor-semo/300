%!TEX TS-program = lualatex
%!TEX encoding = UTF-8 Unicode

\documentclass[letterpaper]{tufte-handout}

%\geometry{showframe} % display margins for debugging page layout

\usepackage{fontspec}
\def\mainfont{Linux Libertine O}
\setmainfont[Ligatures={Common,TeX}, Contextuals={NoAlternate}, BoldFont={* Bold}, ItalicFont={* Italic}, Numbers={OldStyle}]{\mainfont}
\setsansfont[Scale=MatchLowercase]{Linux Biolinum O} 
\usepackage{microtype}
\usepackage{graphicx} % allow embedded images
  \setkeys{Gin}{width=\linewidth,totalheight=\textheight,keepaspectratio}
  \graphicspath{{img/}} % set of paths to search for images
\usepackage{amsmath}  % extended mathematics
\usepackage{booktabs} % book-quality tables
\usepackage{units}    % non-stacked fractions and better unit spacing
\usepackage{siunitx}
\usepackage{multicol} % multiple column layout facilities
\usepackage{microtype}   % filler text
\usepackage{hyperref}
%\usepackage{fancyvrb} % extended verbatim environments
%  \fvset{fontsize=\normalsize}% default font size for fancy-verbatim environments
\makeatletter
% Paragraph indentation and separation for normal text
\renewcommand{\@tufte@reset@par}{%
  \setlength{\RaggedRightParindent}{1.0pc}%
  \setlength{\JustifyingParindent}{1.0pc}%
  \setlength{\parindent}{1pc}%
  \setlength{\parskip}{0pt}%
}
\@tufte@reset@par

% Paragraph indentation and separation for marginal text
\renewcommand{\@tufte@margin@par}{%
  \setlength{\RaggedRightParindent}{0pt}%
  \setlength{\JustifyingParindent}{0.5pc}%
  \setlength{\parindent}{0.5pc}%
  \setlength{\parskip}{0pt}%
}
\makeatother

% Set up the spacing using fontspec features
\renewcommand\allcapsspacing[1]{{\addfontfeatures{LetterSpace=15}#1}}
\renewcommand\smallcapsspacing[1]{{\addfontfeatures{LetterSpace=10}#1}}

\title{Study Guide 08\hfill}
\author{Cambrian Radiation}

\date{} % without \date command, current date is supplied

\begin{document}

\maketitle	% this prints the handout title, author, and date

%\printclassoptions

\section{Vocabulary}\marginnote{\textbf{Read:} Chapter 3; 469--473. We will highlight many of the examples covered in this chapter.\\
	\noindent\textbf{Questions:} pg. 88, SA\ 2, 4, 5, 7.\\}
\vspace{-1\baselineskip}
\begin{multicols}{2}
Ediacaran\\
Ediacaran fauna\\
Cambrian\\
Proterozoic\\
Paleozoic\\
latent evolutionary potential\\
Cambrian radiation
\end{multicols}

\section{Concepts}

You should \emph{write} clear and concise answers to each question in the Concepts section.  The questions are not necessarily independent.  Think broadly across lectures to see ``the big picture.'' 

\begin{enumerate}
	\item Approximately when did life first appear on earth?  What about the first eukaryotes? The first multicellular eukaryotes?

	\item Metazoans showed two dramatic increases in body size during the Proterozoic and early Paleozoic.  What biological and abiotic changes occurred during this time that have been proposed to explain the two increases of body size?

	\item Why was cyanobacteria photosynthesis important to the evolution of aerobic eukaryotes?

	\item What is the Ediacaran fauna?  Why is it an important component of the fossil record?

	\item What is the so-called ``Cambrian explosion?''  Why is Cambrian radiation a more appropriate term?

	\item Discuss, in some depth, the biotic and abiotic factors that may have contributed to the Cambrian radiation.

	\item Summarize the evolutionary changes that occurred in the late Ediacaran and through the Cambrian.  Consider overall diversity, trace fossils, and skeletonized animals.

	\item Describe how the expansion of the Hox gene complex and co-option of those genes could have contributed to the Cambrian radiation.

	\item Briefly explain how the molecular clock supports the fossil evidence of the Cambrian radiation. 

\end{enumerate}

\end{document}