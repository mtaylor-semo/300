%!TEX TS-program = lualatex
%!TEX encoding = UTF-8 Unicode

\documentclass[letterpaper]{tufte-handout}

%\geometry{showframe} % display margins for debugging page layout

\usepackage{fontspec}
\def\mainfont{Linux Libertine O}
\setmainfont[Ligatures={Common,TeX}, Contextuals={NoAlternate}, BoldFont={* Bold}, ItalicFont={* Italic}, Numbers={OldStyle}]{\mainfont}
\setsansfont[Scale=MatchLowercase]{Linux Biolinum O} 
\usepackage{microtype}

\usepackage{graphicx} % allow embedded images
  \setkeys{Gin}{width=\linewidth,totalheight=\textheight,keepaspectratio}
  \graphicspath{{img/}} % set of paths to search for images
\usepackage{amsmath}  % extended mathematics
\usepackage{booktabs} % book-quality tables
\usepackage{units}    % non-stacked fractions and better unit spacing
\usepackage{siunitx}
\usepackage{multicol} % multiple column layout facilities
\usepackage{microtype}   % filler text
\usepackage{hyperref}

\usepackage{enumitem}

%\usepackage{fancyvrb} % extended verbatim environments
%  \fvset{fontsize=\normalsize}% default font size for fancy-verbatim environments
\makeatletter
% Paragraph indentation and separation for normal text
\renewcommand{\@tufte@reset@par}{%
  \setlength{\RaggedRightParindent}{1.0pc}%
  \setlength{\JustifyingParindent}{1.0pc}%
  \setlength{\parindent}{0pc}%
  \setlength{\parskip}{0.5\baselineskip}%
}
\@tufte@reset@par

% Paragraph indentation and separation for marginal text
\renewcommand{\@tufte@margin@par}{%
  \setlength{\RaggedRightParindent}{0pt}%
  \setlength{\JustifyingParindent}{0.5pc}%
  \setlength{\parindent}{0.5pc}%
  \setlength{\parskip}{0pt}%
}
\makeatother

% Set up the spacing using fontspec features
\renewcommand\allcapsspacing[1]{{\addfontfeatures{LetterSpace=15}#1}}
\renewcommand\smallcapsspacing[1]{{\addfontfeatures{LetterSpace=10}#1}}


\title{Study Guide 07\hfill}
\author{Phenotypic evolution and quantitative traits}

\date{} % without \date command, current date is supplied

\begin{document}

\maketitle	% this prints the handout title, author, and date

\section{Vocabulary}\marginnote{\textbf{Read:} Chapter 6, pages 135--148, 151--160.}

\begin{multicols}{2}
discrete traits \\
quantitative traits \\
polygenic traits \\
fitness functions \\
directional selection \\
stabilizing selection \\
disruptive selection \\
selection gradient \\
breeder's equation $\left(\Delta z = h^2S\right)$ \\
heritability $\left(h^2\right)$ \\
additive genetic variance \\
\end{multicols}

%\printclassoptions

\section{Concepts}

You should \emph{write} clear and concise answers to each question in the Concepts section.  The questions are not necessarily independent.  Think broadly across lectures to see ``the big picture.'' 

\begin{enumerate}
	
	\item Darwin's basic idea was that natural selection affects the ability of an organism to survive and reproduce.  Thus, natural selection acts on the phenotype.  However, evolutionary change occurs at the genetic level.  Explain the relationship between phenotype, genotype and fitness to explain how natural selection determines whether evolutionary change can occur in a population.
	
	\item Describe the three fitness functions for selection given in class.  Illustrate and explain each mode.  Provide examples (real or hypothetical) of each.  Relate each to different values of relative fitness $\left(w\right)$. Describe and illustrate how survivorship and frequency changes for each function changes as a trait value changes.
	
	\item Evolution \emph{by} selection requires correlation between a phenotypic trait and fitness, and between parental and offspring phenotypes. Know and explain why both of these correlations are necessary for selection to occur.
	
	\item Use the breeder's equation to show that a population will not evolve if a trait is heritable $(h^2 > 0)$ but there is no selection $(S = 0)$, or if there is selection $(s \ne 0)$ but the trait is not heritable $(h^2 = 0)$.
	
	\item Use the breeder's equation to calculate the predicted amount of phenotypic change given the appropriate values. I would give you $\beta$, $P$, and $h^2$. Be sure you understand what each of these variables represents.


\end{enumerate}

\end{document}