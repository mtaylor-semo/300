\documentclass[letterpaper]{tufte-handout}

%\geometry{showframe} % display margins for debugging page layout

\usepackage{graphicx} % allow embedded images
  \setkeys{Gin}{width=\linewidth,totalheight=\textheight,keepaspectratio}
  \graphicspath{{img/}} % set of paths to search for images
\usepackage{amsmath}  % extended mathematics
\usepackage{booktabs} % book-quality tables
\usepackage{units}    % non-stacked fractions and better unit spacing
\usepackage{siunitx}
\usepackage{multicol} % multiple column layout facilities
\usepackage{microtype}   % filler text
\usepackage{hyperref}
%\usepackage{fancyvrb} % extended verbatim environments
%  \fvset{fontsize=\normalsize}% default font size for fancy-verbatim environments
\makeatletter
% Paragraph indentation and separation for normal text
\renewcommand{\@tufte@reset@par}{%
  \setlength{\RaggedRightParindent}{1.0pc}%
  \setlength{\JustifyingParindent}{1.0pc}%
  \setlength{\parindent}{1pc}%
  \setlength{\parskip}{0pt}%
}
\@tufte@reset@par

% Paragraph indentation and separation for marginal text
\renewcommand{\@tufte@margin@par}{%
  \setlength{\RaggedRightParindent}{0pt}%
  \setlength{\JustifyingParindent}{0.5pc}%
  \setlength{\parindent}{0.5pc}%
  \setlength{\parskip}{0pt}%
}
\makeatother


\title{Study Guides 10 and 11\hfill}
\author{Evolution of Mammals and Birds}

\date{} % without \date command, current date is supplied

\begin{document}

\maketitle	% this prints the handout title, author, and date

%\printclassoptions

\section{Vocabulary}\marginnote{\textbf{Read:} 101--108.\\\textbf{Questions:} None.}
\vspace{-1\baselineskip}
\begin{multicols}{2}
toolkit genes\\
bone morphogenetic protein\ 2\\\hspace{1em}(BMP2)\\
BMP4\\
sonic hedgehog (shh)\\
noggin
\end{multicols}

\section{Concepts}

You should \emph{write} clear and concise answers to each question in the Concepts section.  The questions are not necessarily independent.  Think broadly across lectures to see ``the big picture.'' 

\begin{enumerate}
	\item During what geological period or periods did the reptile-mammal transition occur?

	\item What fossil evidence is consistent with the evolution of mammals from reptilian ancestors?

	\item Are dinosaurs extinct?  Explain why or why not.  Use the concepts of monophyly (or not monophyletic, if necessary) to justify your explanation.

	\item During what geological period or periods did the evolution of birds occur?  What group of dinosaurs is the likely ancestor of birds?  Are these the only group of dinosaurs for which there is evidence of feathers?  

	\item If yes to the above question, does it mean that feathers will not be found in other dinosaur groups?  Explain why, using your understanding of the inherent problems with the fossil record (e.g., biases, etc).

	\item If no to the above question, what other group of dinosaurs shows evidence of feathers?  Does this have implications for the evolutionary history of feathers?  Does this have implications for the evolutionary history of flight?

	\item Describe some of the skeletal homologies and genetic evidence that suggest birds evolved from dinosaurs.

	\item Explain what might have been the earliest adaptive functions of feathers.  Explain how flight might have evolved.

	\item What toolkit genes have been implicated in the evolution of archosaurian scales (you may have to look up archosaurs) and bird feathers?  Which toolkit genes potentially determine the types of feathers found in birds?  What are the implications of this evidence for the evolution of mammalian hair? 

	\item Thought question.  If birds evolved from dinosaurs, should feathered dinosaurs molt?

	\item Based on what you know about dinosaur-bird ancestry, construct a scientific answer for this question: Which came first, the chicken or the egg?

\end{enumerate}

\end{document}