%!TEX TS-program = lualatex
%!TEX encoding = UTF-8 Unicode

\documentclass[letterpaper]{tufte-handout}

%\geometry{showframe} % display margins for debugging page layout

\usepackage{fontspec}
\def\mainfont{Linux Libertine O}
\setmainfont[Ligatures={Common,TeX}, Contextuals={NoAlternate}, BoldFont={* Bold}, ItalicFont={* Italic}, Numbers={OldStyle}]{\mainfont}
\setsansfont[Scale=MatchLowercase]{Linux Biolinum O} 
\usepackage{microtype}

\usepackage{graphicx} % allow embedded images
  \setkeys{Gin}{width=\linewidth,totalheight=\textheight,keepaspectratio}
  \graphicspath{{img/}} % set of paths to search for images
\usepackage{amsmath}  % extended mathematics
\usepackage{booktabs} % book-quality tables
\usepackage{units}    % non-stacked fractions and better unit spacing
\usepackage{siunitx}
\usepackage{multicol} % multiple column layout facilities
\usepackage{microtype}   % filler text
\usepackage{hyperref}

\usepackage{enumitem}

%\usepackage{fancyvrb} % extended verbatim environments
%  \fvset{fontsize=\normalsize}% default font size for fancy-verbatim environments
\makeatletter
% Paragraph indentation and separation for normal text
\renewcommand{\@tufte@reset@par}{%
  \setlength{\RaggedRightParindent}{1.0pc}%
  \setlength{\JustifyingParindent}{1.0pc}%
  \setlength{\parindent}{0pc}%
  \setlength{\parskip}{0.5\baselineskip}%
}
\@tufte@reset@par

% Paragraph indentation and separation for marginal text
\renewcommand{\@tufte@margin@par}{%
  \setlength{\RaggedRightParindent}{0pt}%
  \setlength{\JustifyingParindent}{0.5pc}%
  \setlength{\parindent}{0.5pc}%
  \setlength{\parskip}{0pt}%
}
\makeatother

% Set up the spacing using fontspec features
\renewcommand\allcapsspacing[1]{{\addfontfeatures{LetterSpace=15}#1}}
\renewcommand\smallcapsspacing[1]{{\addfontfeatures{LetterSpace=10}#1}}


\title{Study Guide 08\hfill}
\author{Genetical theory of natural selection}

\date{} % without \date command, current date is supplied

\begin{document}

\maketitle	% this prints the handout title, author, and date

\section{Vocabulary}\marginnote{\textbf{Read:} Chapter 5, pages 103--115, 117--118 (hitchhiking); 119--126, 130--131.}

\begin{multicols}{2}
absolute fitness $\left(W\right)$ \\
relative fitness $\left(w\right)$ \\
fitness component \\
selection coefficient $\left(s\right)$\\
positive selection \\
negative selection \\
purifying selection \\
hitchhiking (genetic hitchhiking) \\
selective sweep \\
balancing selection \\
overdominance \\
heterozygote advantage \\
negative frequency-dependent \\
  \quad selection\\
underdominance \\
positive frequency-dependent \\
  \quad selection \\
mutation-selection balance \\
mutation load
\end{multicols}

%\printclassoptions

\section{Concepts}

You should \emph{write} clear and concise answers to each question in the Concepts section.  The questions are not necessarily independent.  Think broadly across lectures to see ``the big picture.'' 

\begin{enumerate}
	
	\item Natural selection requires a correlation between parental and offspring genotypes, and a correlation between a phenotypic trait and fitness. Explain why these correlations are \emph{necessary} for natural selection to occur.
	
	\item Tell mathematically and in words how relative fitness is obtained from absolute fitness.
	
	\item Be sure you can read and interpret $w_{11}, W_{11}$, etc.~in context. For example if you read $w_{12} > w_{22}$, you should understand and be able to explain what that relationship means.
	
	\item Explain the selection coefficient.\marginnote{Strength of selection is an synonym.} Be able to interpret and diagram an illustration of weak and strong $s$ relative to a reference genotype.
	
	\item Given values of $p$ and $s$, be able to calculate the predicted frequency change $\left(\Delta p\right)$ for an allele.
	
	\item Be able to use calculations of $\Delta p$ to show how $s$ affects the rate of adaptation in a population.
	
	\item Explain how relative fitness of genotypes is affected by alleles that are dominant or recessive, relative to an allele that is neither.
	
	\item Explain how a selective sweep for a beneficial mutation could lead to an increased frequency of a weakly detrimental allele due to hitchhiking.
	
	\item Briefly explain balancing selection. Explain why balancing selection maintains genetic polymorphism in a population.
	
	\item Explain the difference between positive and negative frequency-dependent selection. Which maintains genetic polymorphism in a population and why? Which eliminates genetic polymorphism from a population and why? Be able to recognize examples of each given the proper context. (In other words, don't just memorize the examples given in class. )
	
	\item Repeat the previous question for overdominance and underdominance. 
	
	\item What does it mean to say heterozygote advantage? What do you think would be a corresponding term in the context of underdominance?
	
	\item Purifying selection should remove detrimental mutations from a population but detrimental mutations remain in populations of most species. Explain why.
	
	\item Explain mutation load and how it affects mean fitness of a population.


\end{enumerate}

\end{document}