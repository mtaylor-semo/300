%!TEX TS-program = lualatex
%!TEX encoding = UTF-8 Unicode

\documentclass[letterpaper]{tufte-handout}

%\geometry{showframe} % display margins for debugging page layout

\usepackage{fontspec}
\def\mainfont{Linux Libertine O}
\setmainfont[Ligatures={Common,TeX}, Contextuals={NoAlternate}, BoldFont={* Bold}, ItalicFont={* Italic}, Numbers={OldStyle}]{\mainfont}
\setsansfont[Scale=MatchLowercase]{Linux Biolinum O} 
\usepackage{microtype}

\usepackage{graphicx} % allow embedded images
  \setkeys{Gin}{width=\linewidth,totalheight=\textheight,keepaspectratio}
  \graphicspath{{img/}} % set of paths to search for images
\usepackage{amsmath}  % extended mathematics
\usepackage{booktabs} % book-quality tables
\usepackage{units}    % non-stacked fractions and better unit spacing
\usepackage{siunitx}
\usepackage{multicol} % multiple column layout facilities
\usepackage{enumitem}   % filler text
\usepackage{hyperref}
%\usepackage{fancyvrb} % extended verbatim environments
%  \fvset{fontsize=\normalsize}% default font size for fancy-verbatim environments
\makeatletter
% Paragraph indentation and separation for normal text
\renewcommand{\@tufte@reset@par}{%
  \setlength{\RaggedRightParindent}{1.0pc}%
  \setlength{\JustifyingParindent}{1.0pc}%
  \setlength{\parindent}{0pc}%
  \setlength{\parskip}{0.5\baselineskip}%
}
\@tufte@reset@par

% Paragraph indentation and separation for marginal text
\renewcommand{\@tufte@margin@par}{%
  \setlength{\RaggedRightParindent}{0pt}%
  \setlength{\JustifyingParindent}{0.5pc}%
  \setlength{\parindent}{0.5pc}%
  \setlength{\parskip}{0pt}%
}
\makeatother

% Set up the spacing using fontspec features
\renewcommand\allcapsspacing[1]{{\addfontfeatures{LetterSpace=15}#1}}
\renewcommand\smallcapsspacing[1]{{\addfontfeatures{LetterSpace=10}#1}}


\title{Study Guide 07\hfill}
\author{Mutation and variation}

\date{} % without \date command, current date is supplied

\begin{document}

\maketitle	% this prints the handout title, author, and date

\section{Vocabulary}\marginnote{\textbf{Read:} Chapter 4, pages 79--85, 88--94; Read and study the Genetics Review document from the course web page.}

\begin{multicols}{2}
segregation \\
recombination \\
linkage disequilibrium \\
epistasis \\
horizontal gene transfer \\
mutation \\
point mutation \\
synonymous substitution \\
non-synonymous substitution \\
insertion \\
deletion \\
indel \\
frameshift mutation \\
gene duplication \\
neofunctionalization \\
subfunctionalization \\
paralog \\
ortholog \\
pleiotropy \\
beneficial mutation \\
deleterious mutation
\end{multicols}

%\printclassoptions

\section{Concepts}

You should \emph{write} clear and concise answers to each question in the Concepts section.  The questions are not necessarily independent.  Think broadly across lectures to see ``the big picture.'' 

\begin{enumerate}
	
	\item Read and study the Genetics Review document from the course web site. Just sayin’$\dots$. 
	
	\item Describe how the Mendelian “laws” of segregation and assortment (recombination) increase genetic polymorphism in a population.
	
	\item Explain linkage equilibrium. How does relative proximity of two genes on the same chromosome contribute (or not) to disequilibrium?
	
	\item Explain epistasis.\marginnote{Epistasis means “to stand on,” refering to the phenotypic effect of one gene influencing or “standing on” the phenotypic effect of a different gene.} Describe how epistasis can lead to linkage disequilibrium through natural selection.
	
	\item Describe ways\marginnote{Horizontal gene transfer is also known as lateral gene transfer.} in which genetic variation has been (in past evolutionary events) or can be influenced by horizontal gene transfer.
	
	\item Why are mutations the ultimate source of all genetic variation in a population? Why not segregation or recombination?
	
	\item Explain why non-synyonymous substitutions are subject to natural selection but synonymous substitutions are not.
	
	\item Does the concept of non-synonymous substitutions apply outside of a gene locus (e.g., in a non-coding region of a chromosome)? Why or why not?
	
	\item Describe why indels in a coding region are usually going to be subject to negative selection.
	
	\item Be able to identify a frameshift mutation in a \textsc{dna} sequence.
	
	\item Explain gene duplication. 
	
	\item Describe\marginnote{Review lecture~5 to be sure you understand orthologs and paralogs.} the difference between neofunctionalization and subfunctionalization after gene duplication.
	
	\item If gene duplicates evolve subfunctionalization, are the duplicates orthologs or paralogs? If one copy retains the original function and a duplicate copy evolves a new function, are the two copies orthologs or paralogs? Explain why for each question.
	
	\item If mutation rates are very low\marginnote{Human mutation rate is about 10\textsuperscript{-8} per bp per year.} then how can so many mutations occur in a genome every year? Be able to describe this in words and mathematically. (See below for a couple example problems.) 
	
	\item Describe pleiotropy\marginnote{One gene, many traits.} as it related to genes and phenotypes. 
	
	\item Describe the difference between beneficial and deleterious\marginnote{I also use “detrimental” to refer to deleterious mutations.} mutations. 	What does it mean to say that beneficial and deleterious mutations affect fitness?

\end{enumerate}

\subsection*{Mutation rate practice problems.}

\begin{enumerate}
\item The Asian Honey Bee \textit{(Apis cerana)} has a genome size of about $2.3 \times 10^{8}$ base pairs. The mutation rate is about $3.4 \times 10^{-9}.$ About how many new mutations would you expect in every gamete?\marginnote{\hfill\reflectbox{0.782}}

\item \textit{Caenorhabditis elegans} (a nematode), has a genome size of approximately 100,000,000 base pairs. It's mutation rate is about $2.1 \times 10^{-8}.$ About how many new mutations would you expect in every gamete?\marginnote{\hfill\reflectbox{2.1}}

\item \textit{Arabidopsis thaliana} (Thale Cress, a plant) has a genome size of about 135 megabases.\marginnote{A megabase (Mb) is 1,000,000 nucleotides.} The mutation rate is about $7 \times 10^{-9}.$ About how many new mutations would you expect in every gamete?\marginnote{\hfill\reflectbox{0.91}}

\end{enumerate}




\end{document}