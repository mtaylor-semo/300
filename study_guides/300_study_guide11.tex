%!TEX TS-program = lualatex
%!TEX encoding = UTF-8 Unicode

\documentclass[letterpaper]{tufte-handout}

%\geometry{showframe} % display margins for debugging page layout

\usepackage{fontspec}
\def\mainfont{Linux Libertine O}
\setmainfont[Ligatures={Common,TeX}, Contextuals={NoAlternate}, BoldFont={* Bold}, ItalicFont={* Italic}, Numbers={OldStyle}]{\mainfont}
\setsansfont[Scale=MatchLowercase]{Linux Biolinum O} 
\usepackage{microtype}

\usepackage{graphicx} % allow embedded images
  \setkeys{Gin}{width=\linewidth,totalheight=\textheight,keepaspectratio}
  \graphicspath{{img/}} % set of paths to search for images
\usepackage{amsmath}  % extended mathematics
\usepackage{booktabs} % book-quality tables
\usepackage{units}    % non-stacked fractions and better unit spacing
\usepackage{siunitx}
\usepackage{multicol} % multiple column layout facilities
\usepackage{microtype}   % filler text
\usepackage{hyperref}

\usepackage{enumitem}

%\usepackage{fancyvrb} % extended verbatim environments
%  \fvset{fontsize=\normalsize}% default font size for fancy-verbatim environments
\makeatletter
% Paragraph indentation and separation for normal text
\renewcommand{\@tufte@reset@par}{%
  \setlength{\RaggedRightParindent}{1.0pc}%
  \setlength{\JustifyingParindent}{1.0pc}%
  \setlength{\parindent}{0pc}%
  \setlength{\parskip}{0.5\baselineskip}%
}
\@tufte@reset@par

% Paragraph indentation and separation for marginal text
\renewcommand{\@tufte@margin@par}{%
  \setlength{\RaggedRightParindent}{0pt}%
  \setlength{\JustifyingParindent}{0.5pc}%
  \setlength{\parindent}{0.5pc}%
  \setlength{\parskip}{0pt}%
}
\makeatother

% Set up the spacing using fontspec features
\renewcommand\allcapsspacing[1]{{\addfontfeatures{LetterSpace=15}#1}}
\renewcommand\smallcapsspacing[1]{{\addfontfeatures{LetterSpace=10}#1}}


\title{Study Guide 09\hfill}
\author{Genetic drift, non-adaptive evolution, and adaptive evolution}

\date{} % without \date command, current date is supplied

\begin{document}

\maketitle	% this prints the handout title, author, and date

\section{Vocabulary}\marginnote{\textbf{Read:} to be updated.}

\begin{multicols}{2}
genetic drift \\
non-adaptive evolution \\
fixation \\
probability of fixation \\
haplotype substitution \\
neutral mutation rate \\
Neutral Theory \\
selective sweep \\
positive selection \\
codon bias
\end{multicols}

%\printclassoptions

\section{Concepts}

You should \emph{write} clear and concise answers to each question in the Concepts section.  The questions are not necessarily independent.  Think broadly across lectures to see ``the big picture.'' 

\begin{enumerate}
	
	\item Discuss the consequences of genetic drift.\marginnote{This item is also in the previous study guide but now I ask you specifically to relate drift to non-adaptive evolution.} Consider population size, the probability of any given haplotype going to fixation, what happens to heterozygosity in a population, and what happens among replicate populations (isolated populations with the same initial starting conditions and all subject to genetic drift.  Relate this to nonadaptive evolution.  

	\item Relate the neutral mutation rate to the rate of nonadaptive evolution\sidenote{Non-adaptive evolution is evolution by genetic drift. Adaptive evolution is evolution by natural selection.} by genetic drift in a population.  Does small vs.~large population size matter? If so, why?  If not, why not?\sidenote{Hint: $\mu = 2N_e\mu \times \frac{1}{2N_e}$\label{foot:hint}}

	\item What is meant by substitution of one haplotype for another in a population? 

	\item How does the neutral mutation rate and genetic drift relate to the neutral theory of molecular evolution?\sidenote{Commonly called Neutral Theory.}\sidenote{See Sidenote~\ref{foot:hint}}

	\item Explain the genetic evidence that supports the Neutral Theory.

	\item If multiple (e.g., 10), independent populations are subject only to non-adaptive evolution, would you predict that the end result would be the same for each population? Explain why or why not.
	
	\item Compare and contrast the Neutral Theory vs natural selection at the molecular genetic level ({\scshape dna} and proteins).  Does the Neutral Theory completely discount the importance of natural selection?  Does natural selection discount the importance of neutral variation?

	\item Selective sweeps, positive selection, and codon bias all provide evidence for natural selection at the genetic level. Would this evidence falsify the Neutral Theory?  Explain. 

\end{enumerate}

\end{document}