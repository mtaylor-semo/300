%!TEX TS-program = lualatex
%!TEX encoding = UTF-8 Unicode

\documentclass[letterpaper]{tufte-handout}

%\geometry{showframe} % display margins for debugging page layout

\usepackage{fontspec}
\def\mainfont{Linux Libertine O}
\setmainfont[Ligatures={Common,TeX}, Contextuals={NoAlternate}, BoldFont={* Bold}, ItalicFont={* Italic}, Numbers={OldStyle}]{\mainfont}
\setsansfont[Scale=MatchLowercase]{Linux Biolinum O} 
\usepackage{microtype}

\usepackage{graphicx} % allow embedded images
  \setkeys{Gin}{width=\linewidth,totalheight=\textheight,keepaspectratio}
  \graphicspath{{img/}} % set of paths to search for images
\usepackage{amsmath}  % extended mathematics
\usepackage{booktabs} % book-quality tables
\usepackage{units}    % non-stacked fractions and better unit spacing
\usepackage{siunitx}
\usepackage{multicol} % multiple column layout facilities
\usepackage{microtype}   % filler text
\usepackage{hyperref}

\usepackage{enumitem}

%\usepackage{fancyvrb} % extended verbatim environments
%  \fvset{fontsize=\normalsize}% default font size for fancy-verbatim environments
\makeatletter
% Paragraph indentation and separation for normal text
\renewcommand{\@tufte@reset@par}{%
  \setlength{\RaggedRightParindent}{1.0pc}%
  \setlength{\JustifyingParindent}{1.0pc}%
  \setlength{\parindent}{0pc}%
  \setlength{\parskip}{0.5\baselineskip}%
}
\@tufte@reset@par

% Paragraph indentation and separation for marginal text
\renewcommand{\@tufte@margin@par}{%
  \setlength{\RaggedRightParindent}{0pt}%
  \setlength{\JustifyingParindent}{0.5pc}%
  \setlength{\parindent}{0.5pc}%
  \setlength{\parskip}{0pt}%
}
\makeatother

% Set up the spacing using fontspec features
\renewcommand\allcapsspacing[1]{{\addfontfeatures{LetterSpace=15}#1}}
\renewcommand\smallcapsspacing[1]{{\addfontfeatures{LetterSpace=10}#1}}


\title{Study Guide 18\hfill}
\author{Geography of evolution}

\date{} % without \date command, current date is supplied

\begin{document}

\maketitle	% this prints the handout title, author, and date

\section{Vocabulary}\marginnote{\textbf{Read:} Chapter 18, pages 469–482; 484–486.}

\begin{multicols}{2}
biogeographic realm \\
biogeographic province \\
endemism \\
allopatry \\
sympatry \\
dispersal \\
disjunct distribution \\
vicariance \\
phylogeography \\
parsimony network \\
latitudinal diversity gradient \\
Out-of-the-tropics model \\
island biogeography model
\end{multicols}

%\printclassoptions

\section{Concepts}

You should \emph{write} clear and concise answers to each question in the Concepts section.  The questions are not necessarily independent.  Think broadly across lectures to see ``the big picture.'' 

\begin{enumerate}
	
	\item Name the eight biogeographic realms.\marginnote{The figure in your textbook only shows seven. What is the eighth?}
	
	\item How are biogeographic realms and provinces determined?.

	\item Compare and contrast dispersal and vicariance. Can each one potentially explain a disjunct distribution? Why or why not?  
	
	\item Explain why is vicariance a better null hypothesis\marginnote{Better  does not mean correct. It means it is a better starting point to test your ideas.} to explain disjunct distributions than dispersal.
	
	\item How does plate tectonics explain the distribution of some higher taxa? Is this an example of vicariance, dispersal, or both? Explain.
	
	\item Why is a parsimony network sometimes better than a phylogenetic tree for phylogeographic studies? 
	
	\item Be sure you can interpret a parsimony network,\marginnote{Why should the refugial populations have greater haplotypic diversity than the population in the expanded range?} especially for testing hypotheses of range expansion?
	
	\item How might net primary productivity explain the latitudinal diversity gradient?
	
	\item Explain how the tropics function as a cradle, museum, and immigration pump\marginnote{From the out-of-the-tropics model.} explain the latitudinal diversity gradient.
	
	\item Use the island model of biogeography to explain how island (or habitat patch) size and isolation influence the number of species present.

	
\end{enumerate}

\end{document}