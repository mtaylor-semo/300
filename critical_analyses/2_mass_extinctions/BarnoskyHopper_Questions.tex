%!TEX TS-program = lualatex
%!TEX encoding = UTF-8 Unicode

\documentclass[11pt, addpoints]{exam}
\usepackage{graphicx}
	\graphicspath{{/Users/goby/Pictures/teach/300/}
	{img/}} % set of paths to search for images

\usepackage{geometry}
\geometry{letterpaper, bottom=1in}                   
%\geometry{landscape}                % Activate for for rotated page geometry
%\usepackage[parfill]{parskip}    % Activate to begin paragraphs with an empty line rather than an indent
\usepackage{amssymb, amsmath}
\usepackage{mathtools}
	\everymath{\displaystyle}

\usepackage{fontspec}
\setmainfont[Ligatures={TeX}, BoldFont={* Bold}, ItalicFont={* Italic}, BoldItalicFont={* BoldItalic}, Numbers={Proportional}]{Linux Libertine O}
\setsansfont[Scale=MatchLowercase,Ligatures=TeX]{Linux Biolinum O}
\setmonofont[Scale=MatchLowercase]{Inconsolata}
\usepackage{microtype}

\usepackage{unicode-math}
\setmathfont[Scale=MatchLowercase]{Asana Math}

\newfontfamily{\tablenumbers}[Numbers={Monospaced}]{Linux Libertine O}
\newfontfamily{\libertinedisplay}{Linux Libertine Display O}

\usepackage{booktabs}
%\usepackage{tabularx}
\usepackage{longtable}
%\usepackage{siunitx}
\usepackage{array}
\newcolumntype{L}[1]{>{\raggedright\let\newline\\\arraybackslash\hspace{0pt}}p{#1}}
\newcolumntype{C}[1]{>{\centering\let\newline\\\arraybackslash\hspace{0pt}}p{#1}}
\newcolumntype{R}[1]{>{\raggedleft\let\newline\\\arraybackslash\hspace{0pt}}p{#1}}

\usepackage{enumitem}

\usepackage{titling}
\setlength{\droptitle}{-60pt}
\posttitle{\par\end{center}}
\predate{}\postdate{}

\renewcommand{\solutiontitle}{\noindent}
\unframedsolutions
\SolutionEmphasis{\bfseries}

\pagestyle{headandfoot}
\firstpageheader{BI 300: Evolution}{}{\ifprintanswers\textbf{KEY}\else Name: \enspace \makebox[2.5in]{\hrulefill}\fi}
\runningheader{}{}{\footnotesize{pg. \thepage}}
\footer{}{}{}
\runningheadrule

\printanswers

\begin{document}

These questions are based on Barnosky et al. 2011. Has the Earth's sixth
mass extinction already arrived? Nature 471: 51--57; and, Hooper et al.
2012. A global synthesis reveals biodiversity loss as a major driver of
ecosystem change. Nature 486: 105--109.

Read these papers carefully and then type your answers. Hand-written
assignments will not be accepted. You must have your completed answers
with you in class or your assignment will be considered late. Any
assignment e-mailed to me after the \emph{start} of class will be
considered late. Do not copy answers directly from the text (that would
be plagiarism). \emph{Interpret} what you read into your own words. Be
prepared to discuss your answers in small groups and as a class. Failure
to discuss these questions may result in a pop quiz with a point value
equal to this assignment.

\textbf{Failure to attend} \textbf{class} \textbf{on the due date} will
result in an automatic 50\% deduction.

\begin{questions}

\question[5]
 Interpet the important information provided by Figure 1 of Barnosky et 
  al.'s study. Interpret; do not quote or paraphrase. Show me you
  understand the figure.
\vspace*{\stretch{1}}

\question[5]
Interpet the important information provided by Figure 1 of Hooper et
  al.'s study. Interpret; do not quote or paraphrase. Show me you
  understand the figure.
  
\vspace*{\stretch{1}}

\question[5]
According to Barnosky et al., how does the magnitude of extinction compare to the last five mass extinctions? Do
  Barnosky et al. use these numbers to argue that a sixth mass
  extinction is currently underway? Explain.

\vspace*{\stretch{1}}

\question[10]
Barnosky et al. has estimated the percentage of species extinctions
  that have occurred within the last 500 years, which correspond to
  Hooper et al.'s estimated ``intermediate levels of species loss.''
  Summarize Hooper et al.'s predicted ecosystem changes that assume
  intermediate levels of species loss.

\ifprintanswers\begin{solution}
Intermediate species loss of 21--40\% is similar to or may:

\begin{itemize}
	\item Lead to reduced plant primary production of 5--10\%.

	\item Reduce decomposition by a smaller amount.

	\item affect capacity for carbon uptake and storage.
\end{itemize}
\end{solution}
\else
	\vspace*{\stretch{1}}
\fi

\question[5]
Type at least one key point from each paper (not related to the
  questions above) and two questions about either or both papers that
  you do not understand, or that you think follow up on some ideas
  presented in the paper. You can use the key point and the questions to
  help generate discussion in class. 
\vspace*{\stretch{1}}


\end{questions}

\end{document}  