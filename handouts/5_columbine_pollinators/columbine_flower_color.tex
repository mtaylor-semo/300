%!TEX TS-program = lualatex
%!TEX encoding = UTF-8 Unicode

\documentclass[11pt, addpoints]{exam}
\usepackage{graphicx}
	\graphicspath{{/Users/goby/Pictures/teach/300/}
	{img/}} % set of paths to search for images

\usepackage{geometry}
\geometry{letterpaper, bottom=1in}                   
%\geometry{landscape}                % Activate for for rotated page geometry
%\usepackage[parfill]{parskip}    % Activate to begin paragraphs with an empty line rather than an indent
\usepackage{amssymb, amsmath}
\usepackage{mathtools}
	\everymath{\displaystyle}

\usepackage{fontspec}
\setmainfont[Ligatures={TeX}, BoldFont={* Bold}, ItalicFont={* Italic}, BoldItalicFont={* BoldItalic}, Numbers={Proportional}]{Linux Libertine O}
\setsansfont[Scale=MatchLowercase,Ligatures=TeX]{Linux Biolinum O}
\setmonofont[Scale=MatchLowercase]{Inconsolata}
\usepackage{microtype}

\usepackage{unicode-math}
\setmathfont[Scale=MatchLowercase]{Asana Math}

\newfontfamily{\tablenumbers}[Numbers={Monospaced}]{Linux Libertine O}
\newfontfamily{\libertinedisplay}{Linux Libertine Display O}

\usepackage{booktabs}
%\usepackage{tabularx}
%\usepackage{longtable}
%\usepackage{siunitx}

\usepackage{hanging}

\usepackage{array}
\newcolumntype{L}[1]{>{\raggedright\let\newline\\\arraybackslash\hspace{0pt}}p{#1}}
\newcolumntype{C}[1]{>{\centering\let\newline\\\arraybackslash\hspace{0pt}}p{#1}}
\newcolumntype{R}[1]{>{\raggedleft\let\newline\\\arraybackslash\hspace{0pt}}p{#1}}

%\usepackage{enumitem}

%\usepackage{titling}
%\setlength{\droptitle}{-60pt}
%\posttitle{\par\end{center}}
%\predate{}\postdate{}

\renewcommand{\solutiontitle}{\noindent}
\unframedsolutions
\SolutionEmphasis{\bfseries}

\pagestyle{headandfoot}
\firstpageheader{BI 300: Evolution}{}{}
\runningheader{}{}{\footnotesize{pg. \thepage}}
\footer{}{}{}
\runningheadrule

%\printanswers

\begin{document}

\subsection*{Columbine Flower Colors}

Flower color is typically blue, red, white or yellow. In some species,
the color is very pale or faded. The color refers to the color of the
spurs and the sepals. In some species, the petals have a different color
than the sepals and spurs. For example, most of the species with red
flowers have yellow petals. I've simplified the information in the table
and the results of this exercise are not affected.

\vspace{\baselineskip}

\begin{tabular}[c]{@{}ll@{}}
\toprule
Species & Flower Color\tabularnewline
\midrule
BA & Red (often pale)\tabularnewline
BR & Blue\tabularnewline
CA & Red\tabularnewline
CH.CHI & Yellow\tabularnewline
CH.NM & Yellow\tabularnewline
CHAP & Yellow\tabularnewline
COAL & Pale blue to white\tabularnewline
COOC.CO & Pale blue to white\tabularnewline
COOC.UT & White\tabularnewline
COOC.WY & Pale blue to white\tabularnewline
DE & Red\tabularnewline
EL & Red\tabularnewline
EX & Red\tabularnewline
FL & Yellow\tabularnewline
FO.E & Red\tabularnewline
FO.W & Red\tabularnewline
HI & Yellow\tabularnewline
JO & Blue\tabularnewline
LA & Pale purple to white\tabularnewline
LO.AZ & Yellow\tabularnewline
LO.TX & Yellow\tabularnewline
MI & Pink (occasionally pale)\tabularnewline
PI & White\tabularnewline
PU & White\tabularnewline
SA & Blue\tabularnewline
SC & Blue\tabularnewline
SH & Red\tabularnewline
SK & Red\tabularnewline
Sp. nov.\footnotemark & White\tabularnewline
TR & Red\tabularnewline
\bottomrule
\end{tabular}

\footnotetext{Sp. nov. means \emph{species novum}, which designates a new
species without a formal scientific name.}

\end{document}  