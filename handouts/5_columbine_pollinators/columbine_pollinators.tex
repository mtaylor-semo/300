%!TEX TS-program = lualatex
%!TEX encoding = UTF-8 Unicode

\documentclass[11pt, addpoints]{exam}
\usepackage{graphicx}
	\graphicspath{{/Users/goby/Pictures/teach/300/}
	{img/}} % set of paths to search for images

\usepackage{geometry}
\geometry{letterpaper, bottom=1in}                   
%\geometry{landscape}                % Activate for for rotated page geometry
%\usepackage[parfill]{parskip}    % Activate to begin paragraphs with an empty line rather than an indent
\usepackage{amssymb, amsmath}
\usepackage{mathtools}
	\everymath{\displaystyle}

\usepackage{fontspec}
\setmainfont[Ligatures={TeX}, BoldFont={* Bold}, ItalicFont={* Italic}, BoldItalicFont={* BoldItalic}, Numbers={Proportional}]{Linux Libertine O}
\setsansfont[Scale=MatchLowercase,Ligatures=TeX]{Linux Biolinum O}
\setmonofont[Scale=MatchLowercase]{Inconsolata}
\usepackage{microtype}

\usepackage{unicode-math}
\setmathfont[Scale=MatchLowercase]{Asana Math}

\newfontfamily{\tablenumbers}[Numbers={Monospaced}]{Linux Libertine O}
\newfontfamily{\libertinedisplay}{Linux Libertine Display O}

\usepackage{booktabs}
%\usepackage{tabularx}
%\usepackage{longtable}
%\usepackage{siunitx}

\usepackage{hanging}

\usepackage{array}
\newcolumntype{L}[1]{>{\raggedright\let\newline\\\arraybackslash\hspace{0pt}}p{#1}}
\newcolumntype{C}[1]{>{\centering\let\newline\\\arraybackslash\hspace{0pt}}p{#1}}
\newcolumntype{R}[1]{>{\raggedleft\let\newline\\\arraybackslash\hspace{0pt}}p{#1}}

%\usepackage{enumitem}

%\usepackage{titling}
%\setlength{\droptitle}{-60pt}
%\posttitle{\par\end{center}}
%\predate{}\postdate{}

\renewcommand{\solutiontitle}{\noindent}
\unframedsolutions
\SolutionEmphasis{\bfseries}

\pagestyle{headandfoot}
\firstpageheader{BI 300: Evolution}{}{}
\runningheader{}{}{\footnotesize{pg. \thepage}}
\footer{}{}{}
\runningheadrule

%\printanswers

\begin{document}

\subsection*{Columbine Species and Pollinators}

Most columbines are pollinated by only bumblebees, only hummingbirds or
only hawkmoths. A few species have two pollinators, which is indicated
in the table. Both pollinators contribute about equally to pollination
success in columbines.

\vspace{\baselineskip}

\begin{tabular}[c]{@{}ll@{}}
\toprule
Species & Pollinator\tabularnewline
\midrule
BA & Hummingbirds / Hawkmoths\tabularnewline
BR & Bumblebees\tabularnewline
CA & Hummingbirds\tabularnewline
CH.CHI & Hawkmoths\tabularnewline
CH.NM & Hawkmoths\tabularnewline
CHAP & Hawkmoths\tabularnewline
COAL & Hawkmoths\tabularnewline
COOC.CO & Bumblebees\tabularnewline
COOC.UT & Hawkmoths\tabularnewline
COOC.WY & Bumblebees\tabularnewline
DE & Hummingbirds\tabularnewline
EL & Hummingbirds\tabularnewline
EX & Hummingbirds\tabularnewline
FL & Hummingbirds\tabularnewline
FO.E & Hummingbirds\tabularnewline
FO.W & Hummingbirds\tabularnewline
HI & Hawkmoths\tabularnewline
JO & Bumblebees\tabularnewline
LA & Bumblebees\tabularnewline
LO.AZ & Hawkmoths\tabularnewline
LO.TX & Hawkmoths\tabularnewline
MI & Hummingbirds / Hawkmoths\tabularnewline
PI & Hawkmoths\tabularnewline
PU & Hawkmoths\tabularnewline
SA & Bumblebees\tabularnewline
SC & Hawkmoths\tabularnewline
SH & Hummingbirds\tabularnewline
SK & Hummingbirds\tabularnewline
Sp. nov.\footnotemark & Hawkmoths\tabularnewline
TR & Hummingbirds\tabularnewline
\bottomrule
\end{tabular}

\footnotetext{Sp. nov. means \emph{species novum}, which designates a new
species without a formal scientific name.}

\end{document}  