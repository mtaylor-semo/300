%!TEX TS-program = lualatex
%!TEX encoding = UTF-8 Unicode

\documentclass[11pt, addpoints]{exam}
\usepackage{graphicx}
	\graphicspath{{/Users/goby/Pictures/teach/300/}
	{img/}} % set of paths to search for images

\usepackage{geometry}
\geometry{letterpaper, bottom=1in}                   
%\geometry{landscape}                % Activate for for rotated page geometry
%\usepackage[parfill]{parskip}    % Activate to begin paragraphs with an empty line rather than an indent
\usepackage{amssymb, amsmath}
\usepackage{mathtools}
	\everymath{\displaystyle}

\usepackage{fontspec}
\setmainfont[Ligatures={TeX}, BoldFont={* Bold}, ItalicFont={* Italic}, BoldItalicFont={* BoldItalic}, Numbers={Proportional}]{Linux Libertine O}
\setsansfont[Scale=MatchLowercase,Ligatures=TeX]{Linux Biolinum O}
\setmonofont[Scale=MatchLowercase]{Inconsolata}
\usepackage{microtype}

\usepackage{unicode-math}
\setmathfont[Scale=MatchLowercase]{Asana Math}

\newfontfamily{\tablenumbers}[Numbers={Monospaced}]{Linux Libertine O}
\newfontfamily{\libertinedisplay}{Linux Libertine Display O}

\usepackage{booktabs}
%\usepackage{tabularx}
%\usepackage{longtable}
%\usepackage{siunitx}

\usepackage{hanging}

\usepackage{array}
\newcolumntype{L}[1]{>{\raggedright\let\newline\\\arraybackslash\hspace{0pt}}p{#1}}
\newcolumntype{C}[1]{>{\centering\let\newline\\\arraybackslash\hspace{0pt}}p{#1}}
\newcolumntype{R}[1]{>{\raggedleft\let\newline\\\arraybackslash\hspace{0pt}}p{#1}}

%\usepackage{enumitem}

%\usepackage{titling}
%\setlength{\droptitle}{-60pt}
%\posttitle{\par\end{center}}
%\predate{}\postdate{}

\renewcommand{\solutiontitle}{\noindent}
\unframedsolutions
\SolutionEmphasis{\bfseries}

\pagestyle{headandfoot}
\firstpageheader{BI 300: Evolution}{}{\ifprintanswers\textbf{KEY} \else Name \rule{2.5in}{0.4pt}\fi}
\runningheader{}{}{\footnotesize{pg. \thepage}}
\footer{}{}{}
\runningheadrule

%\printanswers

\begin{document}

\subsection*{Pollinators and Speciation in Columbine Plants (\numpoints\ points)}

Justen Whittall and Scott Hodges study the evolution of flower
morphology. Of special interest to them is a group of plants called
columbines (genus \textit{Aquilegia}). \textit{Aquilegia} is a diverse group
with about 70 known species. Columbine flowers have an unusual
morphology. The inner petals have long spurs that extend backwards
between the outer sepals. The spurs are filled with nectar that attracts
pollinators, such as bumblebees, hummingbirds, and hawkmoths. In
columbines, spur length and flower color vary among species. Spurs vary from
1-15 cm and may be straight or curved. Sepal, petal and spur colors are
typically blue, red, white, or yellow.

\begin{center}
	\includegraphics{columbine_parts}

	{\footnotesize
	www.fs.fed.us/wildflowers/beauty/columbines/flower.shtml}
\end{center}

For this exercise, you will work in a group of three students. Together,
the three of you will use a phylogeny of 30 \textit{Aquilegia} species and
varieties to explore how changes of flower color and spur length
correspond to changes in columbine pollinators.

The phylogeny is on the next page. Each student is responsible for
matching the species on the phylogeny to the flower color (blue, red,
yellow or white), the spur length (short, medium or long), or the
pollinator (bumblebees, hummingbirds, or hawkmoths). You will then
search for evolutionary patterns. For example, do the columbine species
with red flowers share the same common ancestor or did red flower color
evolve multiple times? Do all columbine species with short spurs share a
common ancestor? What about columbine species that are pollinated by
hawkmoths?

Once you have established the patterns on the tree, compare your results
with the results of the other members in your group. Do you find a
relationship between flower color, spur length and pollinator?

\begin{center}
	\includegraphics[width=0.75\textwidth]{columbine_phylogeny}

	{\footnotesize Bayesian probabilities are not presented but nearly all branches enjoy
	Bayesian posterior probabilities \textgreater{} 0.95.}
\end{center}

\begin{questions}

\question[5]
How many pollinator shifts (from one pollinator group to
another) do you identify on the tree? Do you detect an 
association between pollinator, flower color, and spur length? Explain. 


\end{questions}

\end{document}  