%!TEX TS-program = lualatex
%!TEX encoding = UTF-8 Unicode

\documentclass[11pt, addpoints]{exam}
\usepackage{graphicx}
	\graphicspath{{/Users/goby/Pictures/teach/300/}
	{img/}} % set of paths to search for images

\usepackage{geometry}
\geometry{letterpaper, bottom=1in}                   
%\geometry{landscape}                % Activate for for rotated page geometry
%\usepackage[parfill]{parskip}    % Activate to begin paragraphs with an empty line rather than an indent
\usepackage{amssymb, amsmath}
\usepackage{mathtools}
	\everymath{\displaystyle}

\usepackage{fontspec}
\setmainfont[Ligatures={TeX}, BoldFont={* Bold}, ItalicFont={* Italic}, BoldItalicFont={* BoldItalic}, Numbers={Proportional}]{Linux Libertine O}
\setsansfont[Scale=MatchLowercase,Ligatures=TeX]{Linux Biolinum O}
\setmonofont[Scale=MatchLowercase]{Inconsolata}
\usepackage{microtype}

\usepackage{unicode-math}
\setmathfont[Scale=MatchLowercase]{Asana Math}

\newfontfamily{\tablenumbers}[Numbers={Monospaced}]{Linux Libertine O}
\newfontfamily{\libertinedisplay}{Linux Libertine Display O}

\usepackage{booktabs}
%\usepackage{tabularx}
%\usepackage{longtable}
%\usepackage{siunitx}

\usepackage{hanging}

\usepackage{array}
\newcolumntype{L}[1]{>{\raggedright\let\newline\\\arraybackslash\hspace{0pt}}p{#1}}
\newcolumntype{C}[1]{>{\centering\let\newline\\\arraybackslash\hspace{0pt}}p{#1}}
\newcolumntype{R}[1]{>{\raggedleft\let\newline\\\arraybackslash\hspace{0pt}}p{#1}}

%\usepackage{enumitem}

%\usepackage{titling}
%\setlength{\droptitle}{-60pt}
%\posttitle{\par\end{center}}
%\predate{}\postdate{}

\renewcommand{\solutiontitle}{\noindent}
\unframedsolutions
\SolutionEmphasis{\bfseries}

\pagestyle{headandfoot}
\firstpageheader{BI 300: Evolution}{}{}
\runningheader{}{}{\footnotesize{pg. \thepage}}
\footer{}{}{}
\runningheadrule

%\printanswers

\begin{document}

\subsection*{Columbine Spur Length}

Spurs are short (7--12 mm), medium (14--24 mm) or long (26--130 mm). Match
the spur length to the species codes on the phylogeny of the main
handout.

\vspace{\baselineskip}

\begin{tabular}[c]{@{}ll@{}}
\toprule
Species & Spur Length\tabularnewline
\midrule
BA & Medium\tabularnewline
BR & Short\tabularnewline
CA & Medium\tabularnewline
CH.CHI & Long\tabularnewline
CH.NM & Long\tabularnewline
CHAP & Long\tabularnewline
COAL & Long\tabularnewline
COOC.CO & Long\tabularnewline
COOC.UT & Long\tabularnewline
COOC.WY & Long\tabularnewline
DE & Medium\tabularnewline
EL & Medium\tabularnewline
EX & Medium\tabularnewline
FL & Medium\tabularnewline
FO.E & Medium\tabularnewline
FO.W & Medium\tabularnewline
HI & Long\tabularnewline
JO & Short\tabularnewline
LA & Short\tabularnewline
LO.AZ & Long\tabularnewline
LO.TX & Long\tabularnewline
MI & Long\tabularnewline
PI & Long\tabularnewline
PU & Long\tabularnewline
SA & Short\tabularnewline
SC & Long\tabularnewline
SH & Medium\tabularnewline
SK & Medium\tabularnewline
Sp. nov.\footnotemark & Long\tabularnewline
TR & Medium\tabularnewline
\bottomrule
\end{tabular}

\footnotetext{Sp. nov. means \emph{species novum}, which designates a new
species without a formal scientific name.}

\end{document}  