%!TEX TS-program = lualatex
%!TEX encoding = UTF-8 Unicode

\documentclass[12pt, addpoints, hidelinks]{exam}

\printanswers

\usepackage{fontspec}
\setmainfont[Ligatures={TeX, Common}, BoldFont={* Bold}, ItalicFont={* Italic}, BoldItalicFont={* BoldItalic}, Numbers={OldStyle,Proportional}]{Linux Libertine O}
\setsansfont[Scale=MatchLowercase,Ligatures={TeX,Common}, Numbers={OldStyle,Proportional}]{Linux Biolinum O}
%\setmonofont[Scale=MatchLowercase]{Inconsolata}
\usepackage{microtype}

\usepackage{geometry}
\geometry{letterpaper, bottom=1in}                   
%\geometry{landscape}                % Activate for for rotated page geometry
\usepackage[parfill]{parskip}    % Activate to begin paragraphs with an empty line rather than an indent
\usepackage[fleqn]{amsmath}
%\usepackage{amssymb}
\usepackage{cancel}
%\everymath{\displaystyle}


\usepackage{unicode-math}
\setmathfont[Scale=MatchLowercase, Numbers=Lining]{Asana Math}
%\setmathfont[Scale=MatchLowercase]{XITS Math}

% To define fonts for particular uses within a document. For example, 
% This sets the Libertine font to use tabular number format for tables.
\newfontfamily{\tablenumbers}[Numbers={Monospaced}]{Linux Libertine O}
\newfontfamily{\libertinedisplay}{Linux Libertine Display O}

%
%\usepackage{graphicx}
%\graphicspath{{/Users/goby/Pictures/teach/300/exercises/}
%	{img/}} % set of paths to search for images
%


%\usepackage{booktabs}
%%\usepackage{longtable}
%%\usepackage{siunitx}
%\usepackage{array}
%\newcolumntype{L}[1]{>{\raggedright\let\newline\\\arraybackslash\hspace{0pt}}p{#1}}
%\newcolumntype{C}[1]{>{\centering\let\newline\\\arraybackslash\hspace{0pt}}p{#1}}
%\newcolumntype{R}[1]{>{\raggedleft\let\newline\\\arraybackslash\hspace{0pt}}p{#1}}
%
%\usepackage{multicol}
%
%\usepackage{enumitem}
%\setlist{leftmargin=*}
%\setlist[1]{labelindent=\parindent}
%\setlist[enumerate]{label=\textsc{\alph*}., ref=\textsc{\alph*}}
%
%\usepackage{tikz}


\usepackage[sc]{titlesec}


\makeatletter
\def\SetTotalwidth{\advance\linewidth by \@totalleftmargin
	\@totalleftmargin=0pt}
\makeatother


\pagestyle{headandfoot}
\firstpageheader{BI 300: Evolution}{}{\ifprintanswers\textbf{Cheat Sheet}\else Name: \enspace \makebox[2.5in]{\hrulefill}\fi}
\runningheader{}{}{\footnotesize{pg. \thepage}}
\footer{}{}{}
\runningheadrule


\newcommand{\fst}{$F_{\mathrm{ST}}$}


\begin{document}
%\thispagestyle{firstpage}

\subsection*{Algebraic rearrangement of the fixation index \fst{}}


Solve for $\mu$, using the mutation form of the $F_{ST}$ equation. For clarity, I use $N$ and $F$ for $N_e$~and $F_{ST},$ respectively.

\vspace{-\baselineskip}

\begin{align*}
F &= \dfrac{1}{1 + 4N\mu}
%%
%\intertext{First, we want $m$ on the left side of the equation so rearrange the starting equation (not necessary, but easier to follow). Note too that $1 + 4N\mu$ is equivalent to $4N\mu + 1.$}
\intertext{Note that $1 + 4N\mu$ is equivalent to $4N\mu + 1.$}
%%
F &= \dfrac{1}{4N\mu + 1}
%%
\intertext{Multiply both sides by $4N\mu + 1$ and cancel.}
%%
F \times (4N\mu + 1) &= \frac{1}{4N\mu + 1} \times 4N\mu + 1\\[2\jot]
%%
F \times (4N\mu + 1) &= \frac{1}{\cancel{4N\mu + 1}} \times \cancel{4N\mu + 1}\\
%%
F (4N\mu + 1) &= 1
%%
\intertext{Expand $F(4N\mu + 1).$}
%%
4FN\mu + F &= 1
%%
\intertext{Add $-F$ to both sides (subtract $F$).}
4FN\mu + F - F &= 1 - F\\
4FN\mu &= 1 - F
%
\intertext{Divide both sides by $4FN$ and cancel.}
%%
\frac{\cancel{4FN}\mu}{\cancel{4FN}} &= \frac{1 - F}{4FN}\\[2\jot]
%
\mu &= \frac{1 - F}{4FN}
%%
\intertext{The final form with subscripts returned to $F_{ST}$ and $N_e$ is,}
%
\mu &= \dfrac{1-F_{ST}}{4F_{ST}N_e}
\end{align*}
%

Surely this can't work for the migration version of the equation. Right?

\end{document}  