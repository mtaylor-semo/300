%!TEX TS-program = lualatex
%!TEX encoding = UTF-8 Unicode

\documentclass[12pt, addpoints, hidelinks]{exam}

%\printanswers

\usepackage{fontspec}
\setmainfont[Ligatures={TeX, Common}, BoldFont={* Bold}, ItalicFont={* Italic}, BoldItalicFont={* BoldItalic}, Numbers={OldStyle,Proportional}]{Linux Libertine O}
\setsansfont[Scale=MatchLowercase,Ligatures={TeX,Common}, Numbers={OldStyle,Proportional}]{Linux Biolinum O}
%\setmonofont[Scale=MatchLowercase]{Inconsolata}
\usepackage{microtype}

\usepackage{geometry}
\geometry{letterpaper, bottom=1in}                   
%\geometry{landscape}                % Activate for for rotated page geometry
\usepackage[parfill]{parskip}    % Activate to begin paragraphs with an empty line rather than an indent
\usepackage[fleqn]{amsmath}
%\usepackage{amssymb}
%\usepackage{mathtools}
%\everymath{\displaystyle}


\usepackage{unicode-math}
\setmathfont[Scale=MatchLowercase, Numbers=Lining]{Asana Math}
%\setmathfont[Scale=MatchLowercase]{XITS Math}

% To define fonts for particular uses within a document. For example, 
% This sets the Libertine font to use tabular number format for tables.
\newfontfamily{\tablenumbers}[Numbers={Monospaced}]{Linux Libertine O}
\newfontfamily{\libertinedisplay}{Linux Libertine Display O}


\usepackage{graphicx}
\graphicspath{{/Users/goby/Pictures/teach/300/exercises/}
	{img/}} % set of paths to search for images



\usepackage{booktabs}
%\usepackage{longtable}
%\usepackage{siunitx}
\usepackage{array}
\newcolumntype{L}[1]{>{\raggedright\let\newline\\\arraybackslash\hspace{0pt}}p{#1}}
\newcolumntype{C}[1]{>{\centering\let\newline\\\arraybackslash\hspace{0pt}}p{#1}}
\newcolumntype{R}[1]{>{\raggedleft\let\newline\\\arraybackslash\hspace{0pt}}p{#1}}

\usepackage{multicol}

\usepackage{enumitem}
\setlist{leftmargin=*}
\setlist[1]{labelindent=\parindent}
\setlist[enumerate]{label=\textsc{\alph*}., ref=\textsc{\alph*}}

\usepackage{tikz}


\usepackage[sc]{titlesec}

\renewcommand{\solutiontitle}{\noindent}
\unframedsolutions
\SolutionEmphasis{\bfseries}

\renewcommand{\questionshook}{%
	\setlength{\leftmargin}{-\leftskip}%
}

\makeatletter
\def\SetTotalwidth{\advance\linewidth by \@totalleftmargin
	\@totalleftmargin=0pt}
\makeatother


\pagestyle{headandfoot}
\firstpageheader{BI 300: Evolution}{}{\ifprintanswers\textbf{KEY}\else Name: \enspace \makebox[2.5in]{\hrulefill}\fi}
\runningheader{}{}{\footnotesize{pg. \thepage}}
\footer{}{}{}
\runningheadrule

\newcommand*\AnswerBox[2]{%
	\parbox[t][#1]{0.92\textwidth}{%
		\begin{solution}#2\end{solution}}
	\vspace{\stretch{1}}
}

\newenvironment{AnswerPage}[1]
{\begin{minipage}[t][#1]{0.92\textwidth}%
		\begin{solution}}
		{\end{solution}\end{minipage}
	\vspace{\stretch{1}}}

\newlength{\basespace}
\setlength{\basespace}{5\baselineskip}

\newcommand{\fst}{$F_{\mathrm{ST}}$}


\begin{document}
%\thispagestyle{firstpage}

\subsection*{Fixation index \fst{} (\numpoints\ points)}

The fixation index $(F_\mathrm{ST})$ is useful for estimating the amount of genetic structuring and gene flow among populations. \fst{} can be calculated by several related equations. The equations you need for this exercise are,
\begin{multicols}{2}
\noindent	
\begin{equation}
F_\mathrm{ST} = \dfrac{2pq-\sum c_i2p_iq_i}{2pq},
\end{equation}

%\[F_\mathrm{ST} = \dfrac{2pq-\sum c_i2p_iq_i}{2pq}\]
	
	\columnbreak
	
\noindent	
\begin{equation}
F_\mathrm{ST} = \dfrac{1}{1+4N_em},
\end{equation}%\[F_\mathrm{ST} = \dfrac{1}{1+4N_em}, \]
\end{multicols}

where $p$ is the frequency of the first allele, $q$ is the frequency of the second allele, $c_i$ is the contribution of each population to the total (the “weight” of each sample), $N_e$ is effective population size, and $m$ is the migration rate.

Be sure to review the slides and your lecture notes to understand how to calculate the sample weights and mean heterozygosity when sample sizes are unequal.

\textbf{Round your final answers to four digits after the decimal point except for the final question. Show your work for each question!} You may use a spreadsheet to set up the calculations; if you do then upload it with your answers to show your work.

\subsubsection*{Problem set}

\begin{questions}
	\question[4]
	
	Calculate \fst{} for populations \textsc{a} and \textsc{b}.

	\begin{multicols}{2}
	
		\begin{tabular}{rccc}
			\toprule
			& $N$ & $p$ & $q$ \tabularnewline
			\midrule
			A & 50 & 0.44 & 0.56 \tabularnewline
			B & 50 & 0.72 & 0.28 \tabularnewline
			\bottomrule
		\end{tabular}
		
		\columnbreak
		
		\ifprintanswers \fst{} $= 0.080$ \else \phantom{\fst{} $= 0.080$} \fi
	\end{multicols}
	
	

\question[4]\label{q:migration}

Calculate \fst{} for populations \textsc{a}, \textsc{b}, \textsc{c}, and \textsc{d}.


\begin{multicols}{2}
	\begin{tabular}{rccc}
	\toprule
  & $N$ & $p$  & $q$ \tabularnewline
  \midrule
A & 50  & 0.44 & 0.56 \tabularnewline
B & 50  & 0.72 & 0.28 \tabularnewline
C & 50  & 0.20 & 0.80 \tabularnewline
D & 50  & 0.16 & 0.84 \tabularnewline
\bottomrule
\end{tabular}

\columnbreak

\ifprintanswers \fst{} $= 0.212$ \else \phantom{\fst{} $= 0.212$} \fi

\end{multicols}

\question[4]\label{q:unequal}

Calculate \fst{} for populations \textsc{a}, \textsc{b}, and \textsc{c}.


\begin{multicols}{2}
	\begin{tabular}{rccc}
	\toprule
	& $N$ & $p$  & $q$ \tabularnewline
	\midrule
	A & 40  & 0.81 & 0.19 \tabularnewline
	B & 100 & 0.27 & 0.73 \tabularnewline
	C & 60  & 0.35 & 0.65 \tabularnewline
	\bottomrule
\end{tabular}

\columnbreak

\ifprintanswers \fst{} $= 0.178$ \else \phantom{\fst{} $= 0.178$} \fi

\end{multicols}
	
\newpage

The last two questions require you to algebraically rearrange equation~(2) above to solve for $m$ or $N_e$.

\textsc{Hint:} use $F_\mathrm{ST} = 0.2$ and $N_e = 100$ or $m = 0.01$ as needed to test your rearranged equation. Using $F_\mathrm{ST} = 0.2$ and $N_e = 100$ should give you $m = 0.01$. Using $F_\mathrm{ST} = 0.2$ and $m = 0.01$ should give you $N_e = 100$. If so, then you are ready to proceed to the last two questions. If not, find and fix your mistake(s).

\question[4]
Use the \fst{} value you calculated for question~\ref{q:migration} and $N_e = 2,200$ individuals to estimate the migration rate, $m$. 

\ifprintanswers 
	$m = 0.000422384$ \quad $4.22 \times 10^{-4}$ \hfill $m = \dfrac{1-F_{ST}}{4F_{ST}N_e}$
	
	\vspace{2\baselineskip} 
\else 
	\vspace{3\baselineskip}
\fi

\question[4]
Use the \fst{} value you calculated for question~\ref{q:unequal} and $m = 0.000005\ (5\times 10^{-6})$ to estimate the effective population size, $N_e$. Round your answer to the nearest whole individual.

\ifprintanswers 
	$N_e = 230,898$ \hfill $N_e = \dfrac{1-F_{ST}}{4F_{ST}m}$\newline
	\vspace{\baselineskip}
\else 
	\vspace{2\baselineskip} 
\fi

\end{questions}


\end{document}  