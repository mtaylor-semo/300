%!TEX TS-program = lualatex
%!TEX encoding = UTF-8 Unicode

\documentclass[letterpaper,nofonts]{tufte-handout}

%\geometry{showframe} % display margins for debugging page layout

\usepackage{fontspec}
\def\mainfont{Linux Libertine O}
\setmainfont[Ligatures={Common,TeX}, Contextuals={NoAlternate}, BoldFont={* Bold}, ItalicFont={* Italic}, Numbers={OldStyle}]{\mainfont}
\setsansfont[Scale=MatchLowercase, Numbers={OldStyle}]{Linux Biolinum O} 
\usepackage{microtype}

\usepackage{graphicx} % allow embedded images
  \setkeys{Gin}{width=\linewidth}
  \graphicspath{	{/Users/goby/pictures/teach/300/review/}}%}%

%\usepackage[parfill]{parskip}

\usepackage{amsmath}  % extended mathematics
%\usepackage{amsfonts}
%\usepackage{amssymb}
\usepackage{unicode-math}
\setmathfont[Scale=MatchLowercase, Numbers=Lining]{Asana Math}
\usepackage{cancel}
%\usepackage{bm}

\usepackage{booktabs} % book-quality tables
%\usepackage{units}    % non-stacked fractions and better unit spacing
%\usepackage{multicol} % multiple column layout facilities
%\usepackage{fancyvrb} % extended verbatim environments
%  \fvset{fontsize=\normalsize}% default font size for fancy-verbatim environments

\usepackage{enumitem}

\makeatletter
% Paragraph indentation and separation for normal text
\renewcommand{\@tufte@reset@par}{%
  \setlength{\RaggedRightParindent}{1.0pc}%
  \setlength{\JustifyingParindent}{1.0pc}%
  \setlength{\parindent}{1pc}%
  \setlength{\parskip}{0pt}%
}
\@tufte@reset@par

% Paragraph indentation and separation for marginal text
\renewcommand{\@tufte@margin@par}{%
  \setlength{\RaggedRightParindent}{0pt}%
  \setlength{\JustifyingParindent}{0.5pc}%
  \setlength{\parindent}{0.5pc}%
  \setlength{\parskip}{0pt}%
}
\makeatother

\makeatletter
\long\def\@caption#1[#2]#3{%
  \par
  \addcontentsline{\csname ext@#1\endcsname}{#1}%
    {\protect\numberline{\csname the#1\endcsname}{\ignorespaces #2}}%
  \begingroup
    \@parboxrestore
    \if@minipage
      \@setminipage
    \fi
    \@tufte@caption@font\@tufte@caption@justification
    \noindent\csname fnum@#1\endcsname. \ignorespaces#3\par% changed : to .
  \endgroup}
\makeatother

% Set up the spacing using fontspec features
\renewcommand\allcapsspacing[1]{{\addfontfeatures{LetterSpace=15}#1}}
\renewcommand\smallcapsspacing[1]{{\addfontfeatures{LetterSpace=10}#1}}

% Put footnote tags in the margin from within
% the align environment. 
% See https://tex.stackexchange.com/a/63042/39194
\def\mathnote#1{%
  \tag*{\rlap{\hspace\marginparsep\smash{\parbox[t]{\marginparwidth}{%
  \footnotesize#1}}}}
}

\title{{\scshape bi} 300 Evolution by Selection on a Single Locus}

\date{} % without \date command, current date is supplied

\begin{document}

\maketitle	% this prints the handout title, author, and date


This example is modified from Futuyma and Kirkpatrick (2017).\footnote{\textit{Evolution}, 4th ed. Sinauer, Box 5\textsc{a}, page 109.} This document explains the mathematics of selection on a single locus with two alleles. Examples are worked in detail so  you can understand how the equations and numbers were derived.

\emph{Work the examples and derivations yourself.} Do not just read through them. You will improve your understanding of the math by working through each step. Read the margin notes for additional detail.

\section*{Genotype frequency}

For a population in Hardy-Weinberg equilibrium, the expected genotype frequencies for individuals born into a population are calculated as,

\begin{table}
\begin{tabular}{@{}lccc@{}}
\caption{Starting allele frequencies.}\label{tab:hwe}
Genotype	& 
$A_1A_1$	& 
$A_1A_2$	& 
$A_2A_2$ 	\tabularnewline[0.5em]
%
Frequency at birth		&
$\left(1-p\right)^2$	&
$2p\left(1-p\right)$	&
$p^2$ 					\tabularnewline
%
\end{tabular}
\end{table}\marginnote[-0.5\baselineskip]{Hardy-Weinberg equations often use $q = 1-p$. Thus, $q^2 = (1-p)^2$ and $2pq = 2p(1-p)$. I use $1-p$ here to simplify the math.}


\subsection*{Worked problem for genotype frequencies}

Given parameters $p = 0.3$ and $s = 0.02$, then initial genotype frequencies are
{\setlength{\jot}{0.8em}
\begin{align*}
A_1A_1 &= \left(1-p\right)^2 = \left(1 - 0.3\right)^2 = \left(0.7\right)^2 = 0.49\\[1ex]
A_1A_2 &= 2p(1-p) = 2(0.3)(0.7) = 0.42\\[1ex]
A_2A_2 &= (0.3)^2 = 0.09.\mathnote{The three frequencies sum to 1.}
\end{align*}
}

\section*{Allele frequencies}

The frequency of allele $A_2$ is the frequency of the $A_2A_2$ genotype plus half the frequency of the $A_1A_2$ genotype. 

\subsection*{Worked problem for allele frequencies}

From Table~\ref{tab:hwe}, the frequency of $A_2A_2$ is $p^2$ and the frequency of $A_1A_2$ is $2p(1-p)$. Given  $p^2 = 0.09$ and $2p(1-p) = 0.42$, then the frequency ($p$) of the $A_2$ allele is
{\setlength{\jot}{0.8em}
\begin{align*}
p &= 0.09 + \frac{1}{2}(0.42),\mathnote{Practice by solving for $A_1$ frequency if $A_1A_1$ frequency  is 0.49. $A_1A_2$ frequency is the same as the example.} \\
  &= 0.09 + 0.21  \\
  &= 0.3. 
\end{align*}
}

\section*{Genotype frequencies after selection}

After selection, the new genotype frequencies are calculated.

\begin{table}
\begin{tabular}{@{}lccc@{}}
\caption{Genotype frequencies in adults after selection.}\label{tab:selection_freqs}
Genotype	& 
$A_1A_1$	& 
$A_1A_2$	& 
$A_2A_2$ 	\tabularnewline[0.5em]
%
Relative fitness $(w)$	&
$1$			& 
$1 + s$		&
$1 + 2s$ 	\tabularnewline[0.5em]
%
Frequency in adults		&
$\dfrac{\left(1-p\right)^2}{\overline{w}}$	&
$\dfrac{2p\left(1-p\right)(1+s)}{\overline{w}}$	&
$\dfrac{p^2(1+2s)}{\overline{w}}$ \tabularnewline
\end{tabular}
\end{table}

\vspace{\baselineskip}

 Relative fitness $(w)$ is the proportion of offspring produced by a genotype compared to a reference genotype. Here, $A_1A_1$ is the reference genotype so it has relative fitness $w_{11} = 1$.
 %
 \marginnote[-2\baselineskip]{Subscripts $1$ and $2$ for $w$ indicate alleles $A_1$ and $A_2$, respectively, in the genotypes. Thus, $w_{12}$ indicates relative fitness of heterozygotes and $w_{22}$ indicates relative fitness of the $A_2A_2$ genotype.}
 
 The selection coefficient $(s)$ is a measure of the increase or decrease in relative fitness imparted by an allele in a genotype.  Positive $s$ increases relative fitness.
 %
 \marginnote{The selection coefficient is also called the “strength of selection.”} 
 %
 Negative $s$ reduces relative fitness. Here, allele $A_2$ imparts a selection coefficient of $s$ relative to allele $A_1.$
 
Selection modifies a genotype frequency if the genotype has the $A_2$ allele (Table~\ref{tab:selection_freqs}). $A_1A_1$ is not affected because the $A_2$ allele is not present.
%
\marginnote{This applies for additive selection where each copy of the allele in the genotype adds the same strength of selection $(s)$. This is not always true for all alleles.}
%
$A_1A_2$ frequency changes by a factor of $1 + s$ because that genotype has one $A_2$ allele. $A_2A_2$ frequency changes by a factor of $1+2s$ because that genotype has two $A_2$ alleles. 


%
The new genotype frequencies must be normalized
%
\marginnote{If $s = 0$, then $\overline{w} = 1$ because selection is absent. All genotypes have the same relative fitness. See Table~\ref{tab:selection_freqs}.}
%
to the average fitness of the population $(\overline{w})$ so the frequencies still sum to 1. The average fitness of the three genotypes in a population is calculated as


\marginnote[1.8\baselineskip]{See derivation on page~\pageref{deriv:w_mean_fitness}.}
%
\begin{equation}\label{eq:w_mean_fitness}
\overline{w} = 1 + 2sp.
\end{equation}



\subsection*{Worked problem for genotype frequencies after selection}

Calculate the genotype frequencies after selection, given the parameters $p = 0.3$ and $s = 0.02.$ 

First, use equation~\ref{eq:w_mean_fitness} to calculate the mean fitness of the population.
\begin{align*}
\overline{w} = 1 + 2(0.3)(0.02) = 1.012.
\end{align*}

Then, use the equations in Table~\ref{tab:selection_freqs} to calculate the genotype frequencies after selection.

{
\setlength{\jot}{0.8em}
\begin{align*}
%\begin{aligned}
A_1A_1 &= \dfrac{(1-0.3)^2}{1.012} \mathnote{Results rounded to four decimals to offset rounding error.}\\
       &= \dfrac{0.49}{1.012}\\
       &= 0.4842.\\
A_1A_2 &= \dfrac{2(0.3)(1-0.3)(1 + 0.02)}{1.012} \\
	   &= \dfrac{(0.6)(0.7)(1.02)}{1.012}\\
	   &= 0.4233.\\
A_2A_2 &= \dfrac{(0.3)^2(1+2(0.02))}{1.012} \\
       &= \dfrac{(0.09)(1.04)}{1.012} \\
       &= 0.0925. \mathnote{The frequencies after selection sum to 1.}
%\end{aligned}
\end{align*}
}

\section*{Allele frequencies after selection}

From Table~\ref{tab:hwe},\marginnote{From first page: $0.09 + \frac{1}{2}(0.42) = 0.3,$} the initial frequency $(p)$ of $A_2$  is calculated as the frequency of $A_2A_2$ plus half the frequency of $A_1A_2.$ Using the same logic, the frequency of allele $A_2$ after selection $(p^\prime)$ can be calculated directly from $s$, $p$, and $\overline{w}$.

\marginnote[2.3\baselineskip]{See derivation on page~\pageref{deriv:p_prime}.}
%
\begin{equation}\label{eq:p_prime}
p^\prime = \dfrac{p\left[1 + s(1+p)\right]}{\overline{w}}.
\end{equation}

\subsection*{Worked problem for allele frequencies after selection}

First, expand the numerator for $p^\prime$.
\begin{align*}
p^\prime &= \dfrac{p\left[1 +s(1 + p)\right]}{\overline{w}},\\
         &= \dfrac{p + sp + sp^2}{\overline{w}}.
\end{align*}

\newpage

Given $p = 0.3$, $s = 0.02$, and $\overline{w} = 1.012$ from above, and using the expanded $p^\prime$, then
{\setlength{\jot}{0.8em}
\begin{align*}
p^\prime &= \dfrac{0.3 + 0.02\left(0.3\right) + 0.02\left(0.3^2\right)}{1.012}, \\
         &= \dfrac{0.3 + 0.006 + 0.0018}{1.012}, \\
         &= \dfrac{0.3078}{1.012},\\
         &= 0.30415.
\end{align*}
}

After selection, the frequency of $A_2$ is $0.30415$, an increase of $0.00415$ over the starting frequency of $0.3$. The new frequency of the $A_1$ allele is $1-0.03415 = 0.69585$.

\section*{Allele frequency change}

Selection causes the allele frequencies to change. You can calculate directly the allele frequency change  $\left(\Delta p\right)$ after one generation of selection. 

\marginnote[2\baselineskip]{See derivation on page~\pageref{deriv:delta_p}.}
%
\begin{equation}\label{eq:delta_p}
\Delta p = p^\prime - p = \dfrac{sp(1-p)}{\overline{w}}.
\end{equation}

When $s < 0.1$, Equation~\ref{eq:delta_p} can be estimated accurately by
%
\marginnote{As $s$ approaches zero, $\overline{w}$ approaches 1. See Equation~\ref{eq:w_mean_fitness}.}
%
\begin{equation}\label{eq:delta_p_approx}
\Delta p = sp(1-p).
\end{equation}
%


\subsection{Worked problem for allele frequency change}

Using $p = 0.3$, $s = 0.02$, and $\overline{w} = 1.012$,\marginnote{Same values used previously.} then

{\setlength{\jot}{0.8em}\begin{align*}
\Delta p &= \dfrac{0.02(0.3)(1-0.3)}{1.012},\\
         &= \dfrac{(0.006)(0.7)}{\overline{w}},\\
         &= \dfrac{0.0042}{1.012}, \\
         &= 0.00415.
\end{align*}
}

This is the exact value obtained from the worked problem for Equation~\ref{eq:p_prime}. The approximation from Equation~\ref{eq:delta_p_approx} is $0.0042,$ which is very similar to the actual change.

\newpage

\section*{Derivations}

Below are the derivations for Equations~\ref{eq:w_mean_fitness}–\ref{eq:delta_p}. The derivation for Equation~\ref{eq:w_mean_fitness} shows every step, explained when necessary in the margin notes. Subsequent derivations show slightly less detail but with some explanation in the margin notes. 

\subsection*{Derivation of Equation~\ref{eq:w_mean_fitness}}
%
\label{deriv:w_mean_fitness}

\begin{equation*}
\overline{w} = 1 + 2sp.
\end{equation*}

First, expand each numerator.
\begin{align*}
%\setlength{\jot}{0.8em}
%\begin{aligned}
A_1A_1 &= (1-p)^2, \\
       &= (1-p)(1-p),\\
       &= 1-p-p-p^2, \\
       &= 1-2p+p^2.\\[0.5em]
A_1A_2 &= 2p(1-p)(1+s), \\
       &= 2p(1 - p + s - sp), \\
       &= 2p - 2p^2 + 2sp - 2sp^2.\\[0.5em]
A_2A_2 &= p^2(1 + 2s), \\
       &= p^2 + 2sp^2.
%\end{aligned}
\end{align*}

Second, add together the expanded numerators to calculate $\overline{w}$. 
\marginnote[1.3\baselineskip]{I've placed brackets around each numerator to help you identify them from the first step.}
\begin{equation*}%\mathnote{}
\overline{w} = [1-2p+p^2] + [2p - 2p^2 + 2sp - 2sp^2] + [p^2 + 2sp^2].
\end{equation*}

Finally, rearrange and combine like terms to reduce the equation to the final solution.
\begin{align*}
\overline{w} &= 1 + 2p - 2p + 2p^2 - p^2 - p^2 + 2p^2 - 2p^2 + 2sp + 2sp^2 - 2sp^2,\\
%
			 &= 1 + \cancel{2p} - \cancel{2p} + \cancel{2p^2} - \cancel{p^2} - \cancel{p^2} \ + \cancel{2p^2} - \cancel{2p^2} + 2sp + \cancel{2sp^2} - \cancel{2sp^2},\\
%
			 &= 1+2sp.
\end{align*}

\newpage

\subsection*{Derivation of Equation~\ref{eq:p_prime}}
%
\label{deriv:p_prime}

\begin{align*}
p^\prime = \dfrac{p[1 + s(1 + p)]}{\overline{w}}.
\end{align*}

The frequency of $p$ is the frequency of $A_2A_2$ plus half the frequency of $A_1A_2$. From Table~\ref{tab:selection_freqs}, the frequency of the genotypes with the $A_2$ allele after selection is
{
\setlength{\jot}{0.8em}
\begin{align*}
A_2A_2 &= \dfrac{p^2 + 2sp^2}{\overline{w}},\\
A_1A_2 &= \dfrac{2p(1-p)(1+s)}{\overline{w}},
\end{align*}
}
therefore,
{\setlength{\jot}{0.8em}
\begin{align*}
p^\prime &= \dfrac{p^2 + 2sp^2}{\overline{w}} + \dfrac{1}{2}\left[\dfrac{[2p(1-p)(1+s)]}{\overline{w}}\right],\\
%
         &= \dfrac{p^2 + 2sp^2}{\overline{w}} + \dfrac{1}{\cancel{2}}\left[\dfrac{\cancel{2}p(1-p)(1+s)}{\overline{w}}\right], \mathnote{Cancel the 2s.}\\
%
         &= \dfrac{p^2 + 2sp^2}{\overline{w}} + \dfrac{p(1-p)(1+s)}{\overline{w}},\\
%
         &= \dfrac{p^2 + 2sp^2 + p(1-p)(1+s)}{\overline{w}},\mathnote{Combine the terms over the common denominator.}\\
%
%         &= \dfrac{p^2 + 2sp^2 + p(1-p)(1+s)}{\overline{w}},\\
%
         &= \dfrac{p^2 + 2sp^2 + p(1-p +s-sp)}{\overline{w}},\mathnote{Expand $(1-p)(1+s)$.}\\
%
         &= \dfrac{p^2 + 2sp^2 + p - p^2 +sp - sp^2}{\overline{w}}, \mathnote{Multiply the expansion by $p$.}\\
%
         &= \dfrac{p + p^2 - p^2 + sp + 2sp^2 - [1]sp^2}{\overline{w}},\mathnote{Rearrange and combine like terms. I have shown the implicit [1] before the final term.}\\
%
         &= \dfrac{p + \cancel{p^2} - \cancel{p^2} + sp + \cancel{2}sp^2 - \cancel{[1]}sp^2}{\overline{w}},\\
%
         &= \dfrac{p + sp + sp^2}{\overline{w}},\\
%
         &= \dfrac{p[1 + s + sp]}{\overline{w}},\mathnote{Factor the $p.$}\\
%
         &= \dfrac{p[1 + s(1 + p)]}{\overline{w}}.\mathnote{Factor the $s.$}
\end{align*}
}

\newpage

\subsection{Derivation of Equation~\ref{eq:delta_p}}
%
\label{deriv:delta_p}

\begin{align*}
\Delta p = \dfrac{sp(1 -p)}{\overline{w}}.
\end{align*}

{\setlength{\jot}{0.8em}
\begin{align*}
\Delta p &= \dfrac{p[1+s(1+p)]}{\overline{w}} - p, \mathnote{Substitute equation~\ref{eq:p_prime} for $p^\prime$.}\\
%
         &= \dfrac{p + sp + sp^2}{\overline{w}} - p, \mathnote{Expand the numerator for $p^\prime$.}\\
%
         &= \dfrac{p + sp + sp^2}{\overline{w}} - \dfrac{p\overline{w}}{\overline{w}}, \mathnote{Multiply $p$ by $\overline{w}/\overline{w}$ so that $p$ can be subtracted from $p^\prime$.} \\
%
         &= \dfrac{p + sp + sp^2}{\overline{w}} - \dfrac{p(1 + 2sp)}{\overline{w}}, \mathnote{Substitute $1+2sp$ for $\overline{w}$ in the numerator (Equation~\ref{eq:w_mean_fitness}). No need to substitute for the denominator.} \\
%
         &= \dfrac{p + sp + sp^2}{\overline{w}} - \dfrac{p + 2sp^2}{\overline{w}}, \\
%
         &= \dfrac{p + sp + sp^2 -\left[p + 2sp^2\right]}{\overline{w}}, \mathnote{Combine terms of the common denominator.}\\
%
         &= \dfrac{p + sp + sp^2 -p - 2sp^2}{\overline{w}}, \\
%
         &= \dfrac{sp - sp^2}{\overline{w}}, \mathnote{Like terms canceled.}\\
%
         &= \dfrac{sp(1 -p)}{\overline{w}}. \mathnote{Factor $sp$ for final result.}
\end{align*}
}


\end{document}