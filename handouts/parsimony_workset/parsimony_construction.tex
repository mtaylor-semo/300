%!TEX TS-program = lualatex
%!TEX encoding = UTF-8 Unicode

\documentclass[11pt, addpoints, hidelinks]{exam}

%\printanswers

\usepackage{fontspec}
\setmainfont[Ligatures={TeX, Common}, BoldFont={* Bold}, ItalicFont={* Italic}, BoldItalicFont={* BoldItalic}, Numbers={Proportional}]{Linux Libertine O}
\setsansfont[Scale=MatchLowercase,Ligatures={TeX,Common}, Numbers={OldStyle,Proportional}]{Linux Biolinum O}
\usepackage{microtype}

\usepackage{geometry}
\geometry{letterpaper, bottom=1in}                   
\usepackage[parfill]{parskip}    % Activate to begin paragraphs with an empty line rather than an indent


\newfontfamily{\tablenumbers}[Numbers={Monospaced}]{Linux Libertine O}
\newfontfamily{\libertinedisplay}{Linux Libertine Display O}

\usepackage{graphicx}
\graphicspath{{/Users/goby/Pictures/teach/300/exercises/}
	{img/}} % set of paths to search for images

\usepackage{booktabs}
\usepackage{longtable}
\usepackage{array}
\newcolumntype{L}[1]{>{\raggedright\let\newline\\\arraybackslash\hspace{0pt}}p{#1}}
\newcolumntype{C}[1]{>{\centering\let\newline\\\arraybackslash\hspace{0pt}}p{#1}}
\newcolumntype{R}[1]{>{\raggedleft\let\newline\\\arraybackslash\hspace{0pt}}p{#1}}

\usepackage{enumitem}
\setlist{leftmargin=*}
\setlist[1]{labelindent=\parindent}
\setlist[enumerate]{label=\textsc{\alph*}.}

\usepackage[sc]{titlesec}

\usepackage{tikz}
\usetikzlibrary{trees}
\usepackage{forest}

\tikzstyle{block} = [rectangle, draw, fill=white, rounded corners,
minimum size=2em]
\tikzstyle{branch} = [thick, draw]

\usetikzlibrary{positioning, backgrounds}


\forestset{
	every leaf node/.style={
		if n children=0{#1}{}
	},
	every tree node/.style={
		if n children=0{}{#1}
	},
	mytree/.style={
		for tree={
			edge path={
				\noexpand\path [draw, thick, \forestoption{edge}] (!u.parent anchor) |- (.child anchor)\forestoption{edge label};
			},
			every tree node={draw=none,inner sep=0, outer sep=0, minimum size=0},
			every leaf node/.style={align=left},
			grow'=0,
			parent anchor=east, 
			child anchor=west,
			anchor=west,
			l sep=0.5cm,
			s sep=3mm,
			draw=none,
			if n children=0{tier=word}{}
		}
	}
}



\renewcommand{\solutiontitle}{\noindent}
\unframedsolutions
\SolutionEmphasis{\bfseries}

\renewcommand{\questionshook}{%
	\setlength{\leftmargin}{-\leftskip}%
}

\makeatletter
\def\SetTotalwidth{\advance\linewidth by \@totalleftmargin
	\@totalleftmargin=0pt}
\makeatother


\pagestyle{headandfoot}
\firstpageheader{BI 300: Evolution}{}{\ifprintanswers\textbf{KEY}\else Name: \enspace \makebox[2.5in]{\hrulefill}\fi}
\runningheader{}{}{\footnotesize{pg. \thepage}}
\footer{}{}{}
\runningheadrule

\newcommand*\AnswerBox[2]{%
	\parbox[t][#1]{0.92\textwidth}{%
		\begin{solution}#2\end{solution}}
	\vspace{\stretch{1}}
}

\newenvironment{AnswerPage}[1]
{\begin{minipage}[t][#1]{0.92\textwidth}%
		\begin{solution}}
		{\end{solution}\end{minipage}
	\vspace{\stretch{1}}}

\newlength{\basespace}
\setlength{\basespace}{5\baselineskip}

\usepackage{hyperref}

\begin{document}

\subsection*{Parsimony construction worksheet (\numpoints~points)}

For all datasets, find the most parsimonious tree. 0 = character absent,
1 = character present. Characters in columns (C1, C2, C3, etc.). Taxa in
rows (A, B, C, etc.). OG = outgroup. One set uses nucleotides as
characters. Draw the final tree in the space to the right of the data
matrix. You can use scrap paper as necessary to solve the problem.

\begin{questions}
	

\question[3]
An easy one to warm you up.

\begin{longtable}[l]{@{}lllllll@{}}
\toprule
\endhead
& \textbf{C1} & \textbf{C2} & \textbf{C3} & \textbf{C4} & \textbf{C5} &
\textbf{C6}\tabularnewline
\textbf{OG} & 0 & 0 & 0 & 0 & 0 & 0\tabularnewline
\textbf{A} & 1 & 0 & 0 & 0 & 0 & 0\tabularnewline
\textbf{B} & 1 & 1 & 0 & 0 & 0 & 0\tabularnewline
\textbf{C} & 1 & 1 & 1 & 0 & 0 & 0\tabularnewline
\textbf{D} & 1 & 1 & 1 & 1 & 0 & 0\tabularnewline
\textbf{E} & 1 & 1 & 1 & 1 & 1 & 0\tabularnewline
\textbf{F} & 1 & 1 & 1 & 1 & 1 & 1\tabularnewline
\textbf{G} & 1 & 1 & 1 & 1 & 1 & 1\tabularnewline
\bottomrule
\end{longtable}

\AnswerBox{0.12\textheight}{}

\question[3]
Still relatively easy, using nucleotides.

This time, the
nucleotide sequence (left to right) is compared to the outgroup. The
outgroup has the ancestral sequence, so changes relative to the outgroup
are evolutionary that occur in a descendant lineage. \textsc{Hint:} convert
the letters to zeros and ones.

\begin{longtable}[l]{@{}lllll@{}}
\toprule
& \textbf{Site 1} & \textbf{Site 2} & \textbf{Site 3} & \textbf{Site
4}\tabularnewline
\midrule
\endhead
\textbf{OG} & T & A & G & C\tabularnewline
\textbf{A} & C & G & T & C\tabularnewline
\textbf{B} & T & A & T & C\tabularnewline
\textbf{C} & C & G & T & C\tabularnewline
\textbf{D} & T & G & T & T\tabularnewline
\textbf{E} & T & G & T & T\tabularnewline
\bottomrule
\end{longtable}

\AnswerBox{0.1\textheight}{}

\newpage

\question[3]
Somewhat more complex.

\begin{longtable}[l]{@{}llllllll@{}}
\toprule
& \textbf{C1} & \textbf{C2} & \textbf{C3} & \textbf{C4} & \textbf{C5} &
\textbf{C6} & \textbf{C7}\tabularnewline
\midrule
\endhead
\textbf{OG} & 0 & 0 & 0 & 0 & 0 & 0 & 0\tabularnewline
\textbf{A} & 0 & 1 & 0 & 0 & 1 & 1 & 0\tabularnewline
\textbf{B} & 1 & 0 & 0 & 1 & 0 & 1 & 0\tabularnewline
\textbf{C} & 0 & 0 & 0 & 0 & 1 & 1 & 0\tabularnewline
\textbf{D} & 1 & 0 & 0 & 1 & 0 & 1 & 0\tabularnewline
\textbf{E} & 1 & 0 & 1 & 0 & 0 & 1 & 1\tabularnewline
\textbf{F} & 0 & 1 & 0 & 0 & 1 & 1 & 0\tabularnewline
\textbf{G} & 1 & 0 & 1 & 0 & 0 & 1 & 1\tabularnewline
\textbf{H} & 1 & 0 & 1 & 0 & 0 & 1 & 0\tabularnewline
\bottomrule
\end{longtable}

\AnswerBox{0.2\textheight}{}

\question[3]
This phylogeny will give you a polytomy, or an unresolved
node, which means that more than two branches will come from one node.

\begin{longtable}[l]{@{}lllllll@{}}
\toprule
\endhead
& \textbf{C1} & \textbf{C2} & \textbf{C3} & \textbf{C4} & \textbf{C5} &
\textbf{C6}\tabularnewline
\textbf{OG} & 0 & 0 & 0 & 0 & 0 & 0\tabularnewline
\textbf{A} & 1 & 0 & 0 & 1 & 1 & 1\tabularnewline
\textbf{B} & 1 & 0 & 0 & 1 & 1 & 1\tabularnewline
\textbf{C} & 1 & 0 & 0 & 1 & 1 & 0\tabularnewline
\textbf{D} & 1 & 0 & 0 & 1 & 0 & 0\tabularnewline
\textbf{E} & 1 & 0 & 0 & 1 & 0 & 0\tabularnewline
\textbf{F} & 1 & 1 & 0 & 0 & 0 & 0\tabularnewline
\textbf{G} & 1 & 1 & 1 & 0 & 0 & 0\tabularnewline
\textbf{H} & 1 & 1 & 1 & 0 & 0 & 0\tabularnewline
\bottomrule
\end{longtable}

\AnswerBox{0.1\textheight}{}

\newpage

\question[3]
You can build two different trees with these data. Find the most parsimonious tree.

Characters C5 and C7 will conflict with C9. They will give different
trees. Create the first tree using all characters except C9. Once
you have the tree, then add C9. It will have evolved (0$rightarrow$1) 
independently at different places in the tree. Next, build a 
second tree but this time build the tree with C9 so it evolved only once.
Once you have the second tree, add C5 and C7. Draw both trees below and indicate which 
is most parsimonious.

\begin{longtable}[l]{@{}llllllllll@{}}
\toprule
& \textbf{C1} & \textbf{C2} & \textbf{C3} & \textbf{C4} & \textbf{C5} &
\textbf{C6} & \textbf{C7} & \textbf{C8} & \textbf{C9}\tabularnewline
\midrule
\endhead
\textbf{OG} & 0 & 0 & 0 & 0 & 0 & 0 & 0 & 0 & 0\tabularnewline
\textbf{A} & 1 & 1 & 0 & 0 & 0 & 0 & 0 & 1 & 1\tabularnewline
\textbf{B} & 1 & 1 & 0 & 0 & 1 & 1 & 1 & 1 & 1\tabularnewline
\textbf{C} & 1 & 1 & 0 & 1 & 1 & 0 & 1 & 1 & 0\tabularnewline
\textbf{D} & 1 & 0 & 1 & 0 & 0 & 0 & 0 & 0 & 0\tabularnewline
\textbf{E} & 1 & 1 & 0 & 1 & 1 & 0 & 1 & 1 & 0\tabularnewline
\textbf{F} & 1 & 0 & 1 & 0 & 0 & 0 & 0 & 0 & 0\tabularnewline
\textbf{G} & 1 & 1 & 0 & 0 & 1 & 1 & 1 & 1 & 1\tabularnewline
\bottomrule
\end{longtable}

\ifprintanswers

\newpage

\subsection*{Solutions}

\textsc{Question 1}

\begin{forest} mytree
	[[,name=og
	[OG]
	[,name=split1
	[A]
	[,name=split2
	[B]
	[,name=split3
	[C]
	[,name=split4
	[D]
	[,name=split5
	[E]
	[,name=split6
	[F]
	[G]
	]
	]
	]
	]
	]
	]]]
%	\filldraw (og) circle [radius=3pt, fill=black, xshift=-5mm] node [below, xshift=-5mm, yshift=-2pt] {C7};
	\filldraw (split1) circle [radius=3pt, fill=black, xshift=-5mm] node [below, xshift=-5mm, yshift=-2pt] {C1};
	\filldraw (split2) circle [radius=3pt, fill=black, xshift=-5mm] node [below, xshift=-5mm, yshift=-2pt] {C2};
	\filldraw (split3) circle [radius=3pt, fill=black, xshift=-5mm] node [below, xshift=-5mm, yshift=-2pt] {C3};
	\filldraw (split4) circle [radius=3pt, fill=black, xshift=-5mm] node [below, xshift=-5mm, yshift=-2pt] {C4};
	\filldraw (split5) circle [radius=3pt, fill=black, xshift=-5mm] node [below, xshift=-5mm, yshift=-2pt] {C5};
	\filldraw (split6) circle [radius=3pt, fill=black, xshift=-5mm] node [below, xshift=-5mm, yshift=-2pt] {C6};
\end{forest}

\quad

\textsc{Question 2}

\begin{forest} mytree
[
 [,name=og
  [OG]
  [,name=split1
   [B]
    [,name=ACDE,
     [,name=CA
      [C]
      [A]
     ]
     [,name=DE
      [D]
      [E]
     ]
    ]
   ]
  ]
]
	\filldraw (split1) circle [radius=3pt, fill=black, xshift=-5mm] node [below, xshift=-5mm, yshift=-2pt, align=left] {Site 3\\G$\rightarrow$T};
	\filldraw (ACDE) circle [radius=3pt, fill=black, xshift=-5mm] node [below, xshift=-5mm, yshift=-2pt, align=left] {Site 2\\A$\rightarrow$G};
	\filldraw (CA) circle [radius=3pt, fill=black, xshift=-5mm] node [below, xshift=-5mm, yshift=-2pt, align=left] {Site 1\\T$\rightarrow$C};
	\filldraw (DE) circle [radius=3pt, fill=black, xshift=-5mm] node [below, xshift=-5mm, yshift=-2pt, align=left] {Site 4\\C$\rightarrow$T};
\end{forest}

\newpage

\textsc{Question 3}

\begin{forest} mytree
[[,name=og
 [OG]
 [,name=split1
  [,name=ACF
   [C]
    [,name=AF
    [A]
    [F]
    ]]
    [,name=BDEGH
     [,name=BD
     [B]
     [D]
     ]
     [,name=EGH
      [H]
      [,name=GE
      [G]
      [E]
      ]
     ]
    ]
 ]
]]
\filldraw (split1) circle [radius=3pt, fill=black, xshift=-5mm] node [below, xshift=-5mm, yshift=-2pt] {C6};
\filldraw (ACF) circle [radius=3pt, fill=black, xshift=-5mm] node [below, xshift=-5mm, yshift=-2pt] {C5};
\filldraw (AF) circle [radius=3pt, fill=black, xshift=-5mm] node [below, xshift=-5mm, yshift=-2pt] {C2};
\filldraw (BDEGH) circle [radius=3pt, fill=black, xshift=-5mm] node [below, xshift=-5mm, yshift=-2pt] {C1};
\filldraw (BD) circle [radius=3pt, fill=black, xshift=-5mm] node [below, xshift=-5mm, yshift=-2pt] {C4};
\filldraw (EGH) circle [radius=3pt, fill=black, xshift=-5mm] node [below, xshift=-5mm, yshift=-2pt] {C3};
\filldraw (GE) circle [radius=3pt, fill=black, xshift=-5mm] node [below, xshift=-5mm, yshift=-2pt] {C7};
\end{forest}

\textsc{Question 4}

\begin{forest} mytree
[[,name=og
 [OG]
 [,name=split1
  [,name=ABC
   [C]
    [,name=AB
     [A]
     [B]
    ]
  ]
  [,name=DEFGH
   [,name=FGH
    [F]
     [,name=GH
      [G]
      [H]
     ]
    ]
    [D]
    [E]
   ]
  ]
 ]
]]
\filldraw (split1) circle [radius=3pt, fill=black, xshift=-5mm] node [below, xshift=-5mm, yshift=-2pt] {C1};
\filldraw (ABC) circle [radius=3pt, fill=black, xshift=-5mm] node [below, xshift=-5mm, yshift=-2pt] {C5};
\filldraw (AB) circle [radius=3pt, fill=black, xshift=-5mm] node [below, xshift=-5mm, yshift=-2pt] {C6};
\filldraw (DEFGH) circle [radius=3pt, fill=black, xshift=-5mm] node [below, xshift=-5mm, yshift=-2pt] {C4};
\filldraw (FGH) circle [radius=3pt, fill=black, xshift=-5mm] node [below, xshift=-5mm, yshift=-2pt] {C2};
\filldraw (GH) circle [radius=3pt, fill=black, xshift=-5mm] node [below, xshift=-5mm, yshift=-2pt] {C3};
\end{forest}


\newpage

\textsc{Question 5}: Tree 1 (C9 last)

Tree requires 10 steps. Most parsimonious.

\begin{forest} mytree
[[,name=og
 [OG]
 [,name=split1
  [,name=DF
   [D]
   [F]
  ]
   [,name=ABCEG
    [A,name=A]
     [,name=BCEG
      [,name=CE
       [C]
       [E]
      ]
      [,name=BG
       [B]
       [G]
      ]
     ]
   ]
 ]
]]
\filldraw (split1) circle [radius=3pt, fill=black, xshift=-5mm] node [below, xshift=-5mm, yshift=-2pt] {C1};
\filldraw (DF) circle [radius=3pt, fill=black, xshift=-5mm] node [below, xshift=-5mm, yshift=-2pt] {C3};
\filldraw (ABCEG) circle [radius=3pt, fill=black, xshift=-5mm] node [below, xshift=-5mm, yshift=-2pt] {C2,C8};
\filldraw (A) circle [radius=3pt, fill=black, xshift=-15mm] node [below, xshift=-15mm, yshift=-2pt] {C9};
\filldraw (BCEG) circle [radius=3pt, fill=black, xshift=-5mm] node [below, xshift=-5mm, yshift=-2pt] {C5,C7};
\filldraw (CE) circle [radius=3pt, fill=black, xshift=-5mm] node [below, xshift=-5mm, yshift=-2pt] {C4};
\filldraw (BG) circle [radius=3pt, fill=black, xshift=-5mm] node [below, xshift=-5mm, yshift=-2pt] {C6,C9};
\end{forest}

\bigskip

\bigskip

\textsc{Question 5}: Tree 2 (C5 \& C7 last)

Tree requires 11 steps. Less parsimonious.

\begin{forest} mytree
[[,name=og
 [OG]
  [,name=split1
   [,name=DF
    [D]
	[F]
   ]
   [,name=ABCEG
	[,name=CE
	 [C]
	 [E]
	]
	[,name=ABG
	 [A]
	  [,name=BG
	   [B]
	   [G]
	  ]
	]
  ]
 ]
]]
\filldraw (split1) circle [radius=3pt, fill=black, xshift=-5mm] node [below, xshift=-5mm, yshift=-2pt] {C1};
\filldraw (DF) circle [radius=3pt, fill=black, xshift=-5mm] node [below, xshift=-5mm, yshift=-2pt] {C3};
\filldraw (ABCEG) circle [radius=3pt, fill=black, xshift=-5mm] node [below, xshift=-5mm, yshift=-2pt] {C2,C8};
\filldraw (CE) circle [radius=3pt, fill=black, xshift=-5mm] node [below, xshift=-5mm, yshift=-2pt,align=center] {C4,\\C5,C7};
\filldraw (ABG) circle [radius=3pt, fill=black, xshift=-5mm] node [below, xshift=-5mm, yshift=-2pt] {C9};
\filldraw (BG) circle [radius=3pt, fill=black, xshift=-5mm] node [below, xshift=-5mm, yshift=-2pt,align=center] {C6,\\C5,C7};
\end{forest}

\fi



\end{questions}

\end{document}
