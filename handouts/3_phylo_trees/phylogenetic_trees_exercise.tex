%!TEX TS-program = lualatex
%!TEX encoding = UTF-8 Unicode

\documentclass[11pt, addpoints]{exam}
\usepackage{graphicx}
	\graphicspath{{/Users/goby/Pictures/teach/300/exercises/}} % set of paths to search for images

\usepackage{geometry}
\geometry{letterpaper, bottom=1in}                   
%\geometry{landscape}                % Activate for for rotated page geometry
%\usepackage[parfill]{parskip}    % Activate to begin paragraphs with an empty line rather than an indent
\usepackage{amssymb, amsmath}
\usepackage{mathtools}
	\everymath{\displaystyle}

\usepackage{fontspec}
\setmainfont[Ligatures={TeX}, BoldFont={* Bold}, ItalicFont={* Italic}, BoldItalicFont={* BoldItalic}, Numbers={Proportional}]{Linux Libertine O}
\setsansfont[Scale=MatchLowercase,Ligatures=TeX]{Linux Biolinum O}
\usepackage{microtype}

\usepackage{unicode-math}
\setmathfont[Scale=MatchLowercase]{Asana Math}
%\setmathfont[Scale=MatchLowercase]{XITS Math}

% To define fonts for particular uses within a document. For example, 
% This sets the Libertine font to use tabular number format for tables.
\newfontfamily{\tablenumbers}[Numbers={Monospaced}]{Linux Libertine O}
\newfontfamily{\libertinedisplay}{Linux Libertine Display O}

\usepackage[sc]{titlesec}
 
\usepackage{pdflscape}
\usepackage{booktabs}
%\usepackage{tabularx}
\usepackage{longtable}
%\usepackage{siunitx}
\usepackage{array}
\newcolumntype{L}[1]{>{\raggedright\let\newline\\\arraybackslash\hspace{0pt}}p{#1}}
\newcolumntype{C}[1]{>{\centering\let\newline\\\arraybackslash\hspace{0pt}}p{#1}}
\newcolumntype{R}[1]{>{\raggedleft\let\newline\\\arraybackslash\hspace{0pt}}p{#1}}

\usepackage{enumitem}
%\usepackage{hyperref}
%\usepackage{placeins} %PRovides \FloatBarrier to flush all floats before a certain point.
%\usepackage{hanging}




\renewcommand{\solutiontitle}{\noindent}
\unframedsolutions
\SolutionEmphasis{\bfseries}

\pagestyle{headandfoot}
\firstpageheader{BI 300: Evolution}{}{\ifprintanswers\textbf{KEY}\else Name: \enspace \makebox[2.5in]{\hrulefill}\fi}
\runningheader{}{}{\footnotesize{pg. \thepage}}
\footer{}{}{}
\runningheadrule

%\printanswers

\begin{document}

\subsection*{Interpreting Phylogenetic Trees (\numpoints\ points)}

Phylogenetic trees are hypotheses about the relationships among taxa,
based on morpological or genetic evidence. Phylogenetic trees are also
used to develop and test hypotheses about the evolutionary history of
taxa. Interpreting phylogenetic trees is, therefore, an essential part
of evolutionary biology. Learning to interpret phylogenetic trees is the
purpose of this exercise, which uses phylogenetic trees that are based
on the thesis research of one of my graduate students (Jameson and
Taylor, in prep). The purpose of his research was to determine whether
hovering behavior evolved once or twice in this group of fishes called
gobies (genus \emph{Elacatinus}). In addition to behavior, we can also
use the phylogenies to study the evolution of the colorful lateral
stripe found on these gobies (see the photos on screen).

Phylogenetic Tree A (on page~\pageref{treeA} at the end of this handout for easy
removal) shows the species name, the color of the lateral stripe (blue,
white, or yellow), and the type of behavior displayed (sponge-dwelling,
cleaning, hovering) by each species. Study Tree A and answer the
following questions.

\begin{questions}

\question[1]
How many species are included on this tree?\ifprintanswers\quad\textbf{14}\fi

\vspace*{\stretch{1}}

\question[1]
How many times did the cleaning behavior evolve?\ifprintanswers\quad\textbf{Once}\fi

\vspace*{\stretch{1}}

\question[1]
How many times did the sponge-dwelling behavior evolve?\ifprintanswers\quad\textbf{Once}\fi

\vspace*{\stretch{1}}

\question[1]
How many times did the hovering behavior evolve?\ifprintanswers\quad\textbf{Twice}\fi

\vspace*{\stretch{1}}

\question[1]
Identify a monophyletic group in Tree A that has at least three
different species in the group. List the species here, and then put a
large, solid dot on the branch in Tree A that represents the common
ancestor for your monophyletic group.

\vspace*{\stretch{1}}

\newpage

\question[1]
Assuming that a colored lateral stripe arose in the common ancestor
to all of these species, what is the most likely ancestral lateral
stripe color? Can you state with 100\% certainty that this color is the
ancestral lateral stripe color? Why or why not? Use the principle of
maximum parsimony to explain why the color you chose is the most likely
ancestral state.

\vspace*{\stretch{2}}

\question[1]
What two colors are the derived states relative to the ancestral
color you listed above. For each color, write how many times that color
evolved (use the principle of parsimony).

\ifprintanswers\quad\textbf{Blue: 3 Times.}\quad\textbf{White: 5 times}\fi

\vspace*{\stretch{1}}

\question[1]
According to the phylogeny, are \emph{illecebrosus} and
\emph{prochilos} sister species? Explain why or why not.

\vspace*{\stretch{2}}

\question[1]
According to the phylogeny, are \emph{figaro} and \emph{randalli}
sister species? Are they sister taxa? Explain why or why not.

\vspace*{\stretch{2}}

\newpage

\question[1]
According to the hypothesis, what is the sister group to
\emph{xanthiprora}? If the sister group has only one species that is
sister to \emph{xanthiprora}, then list that one species. If the group
contains more than one species, then list all of the species that would
be included in this group.

\vspace*{\stretch{1.5}}

\uplevel{%
The tree you just examined above is based on a combined analysis of four
genetic markers: \emph{cytochrome oxidase I} and \emph{cytochrome b}
(from the mitochondrial genome) and \emph{recombination activation gene
1} and \emph{rhodopsin} (from the nuclear genome). The genetic markers
were also analyzed separately. Trees B and C show the results for two of
the genetic markers. Tree B shows the results of the \emph{cytochrome
oxidase I} (\emph{COI)} analysis and Tree C shows the results of the
\emph{cytochrome b} (\emph{mtcytb}) analysis (both on page~\pageref{two_trees}).%
}

\question[1]
Compare trees B and C. Do they show the same relationships among the
different species? If not, describe a few ways in which they differ.

\vspace*{\stretch{1}}

\question[1]
Provide reasoned speculation about why the separate analyses
resulted in two different phylogenetic trees if they used exactly the
same individuals.

\vspace*{\stretch{1}}

\question[1]
Speculate how you might resolve this problem.

\vspace*{\stretch{1}}

\end{questions}

\newpage

\noindent This page could have been wastefully left blank but instead I give you
 pictures. Here is a picture of \emph{Elacatinus genie} cleaning
around the eye of a Nassau grouper.

\begin{center}
	\includegraphics[width=0.85\textwidth]{phylo_tree_genie}
\end{center}

\vspace*{1\baselineskip}

\noindent Here is \emph{Elacatinus prochilos} (?) cleaning a Powderblue Tang.
This picture is from an aquarium. This subgenus of \emph{Elacatinus} is
endemic to the western Atlantic Ocean, primarily the Caribbean Sea. The
tang is from the Indian Ocean.

\begin{center}
	\includegraphics[width=0.85\textwidth]{phylo_tree_prochilos}
\end{center}

\newpage

\includegraphics[width=\textwidth]{phylo_tree_treeA}\label{treeA}

\newpage

\begin{landscape}

\includegraphics[height=\textheight]{phylo_tree_treeB}\label{two_trees}%
\hfill\includegraphics[height=\textheight]{phylo_tree_treeC}

\end{landscape}


\end{document}  