%!TEX TS-program = lualatex
%!TEX encoding = UTF-8 Unicode

\documentclass[11pt, addpoints]{exam}

%\printanswers

\usepackage{graphicx}
	\graphicspath{{/Users/goby/Pictures/teach/300/exercises/}
	{img/}} % set of paths to search for images

\usepackage{geometry}
\geometry{letterpaper, bottom=1in}                   
%\geometry{landscape}                % Activate for for rotated page geometry
%\usepackage[parfill]{parskip}    % Activate to begin paragraphs with an empty line rather than an indent
\usepackage{amssymb, amsmath}
\usepackage{mathtools}
	\everymath{\displaystyle}

\usepackage{fontspec}
\setmainfont[Ligatures={TeX}, BoldFont={* Bold}, ItalicFont={* Italic}, BoldItalicFont={* BoldItalic}, Numbers={Proportional}]{Linux Libertine O}
\setsansfont[Scale=MatchLowercase,Ligatures=TeX]{Linux Biolinum O}
%\setmonofont[Scale=MatchLowercase]{Inconsolata}
\usepackage{microtype}

\usepackage{unicode-math}
\setmathfont[Scale=MatchLowercase]{Asana Math}
%\setmathfont[Scale=MatchLowercase]{XITS Math}

% To define fonts for particular uses within a document. For example, 
% This sets the Libertine font to use tabular number format for tables.
\newfontfamily{\tablenumbers}[Numbers={Monospaced}]{Linux Libertine O}
\newfontfamily{\libertinedisplay}{Linux Libertine Display O}

\usepackage{booktabs}
%\usepackage{tabularx}
\usepackage{longtable}
%\usepackage{siunitx}
\usepackage{array}
\newcolumntype{L}[1]{>{\raggedright\let\newline\\\arraybackslash\hspace{0pt}}p{#1}}
\newcolumntype{C}[1]{>{\centering\let\newline\\\arraybackslash\hspace{0pt}}p{#1}}
\newcolumntype{R}[1]{>{\raggedleft\let\newline\\\arraybackslash\hspace{0pt}}p{#1}}

\usepackage{enumitem}
\usepackage{hyperref}
%\usepackage{placeins} %PRovides \FloatBarrier to flush all floats before a certain point.
\usepackage{hanging}

%\usepackage[sc{titling}
%\setlength{\droptitle}{-60pt}
%\posttitle{\par\end{center}}
%\predate{}\postdate{}

\usepackage[sc]{titlesec}

\renewcommand{\solutiontitle}{\noindent}
\unframedsolutions
\SolutionEmphasis{\bfseries}

\pagestyle{headandfoot}
\firstpageheader{BI 300: Evolution}{}{\ifprintanswers\textbf{KEY}\else Name: \enspace \makebox[2.5in]{\hrulefill}\fi}
\runningheader{}{}{\footnotesize{pg. \thepage}}
\footer{}{}{}
\runningheadrule

\begin{document}

\subsection*{Hardy-Weinberg equilibrium: non-random mating (\numpoints\ points)}

The last assumption of \textsc{hwe} for us to explore is how haplotype
frequencies change in non-randomly mating populations. In a randomly
mating population, any individual has the same chance of mating with any
other individual in the population. For example, if bluebirds truly
mated at random, a bluebird near Cape Girardeau should have the same
chance of mating with a bluebird near Sikeston, St. Louis or Chicago as
it does with one near Cape Girardeau. In reality, however, two bluebirds
near Cape Girardeau have a much greater chance of mating with each other
than with bluebirds from Sikeston, St. Louis or Chicago.

Non-random mating affects populations in a manner similar to genetic
drift. This can be demonstrated with a simple mathematical exercise.

\begin{questions}

\question[1]
The diagram below represents a single population of randomly
mating individuals, called an \textbf{outbred population}. If all other assumptions of \textsc{hwe} are met, then what
are the expected genotype frequencies in this population?

\begin{center}
	\includegraphics[width=0.5\textwidth]{nonrandom_mating_outbred}
\end{center}

\qquad\emph{A}\textsubscript{1}\emph{A}\textsubscript{1}:\ifprintanswers\quad\textbf{0.25}\fi\vspace{\baselineskip}

\qquad\emph{A}\textsubscript{1}\emph{A}\textsubscript{2}:\ifprintanswers\quad\textbf{0.5}\fi\vspace{\baselineskip}

\qquad\emph{A}\textsubscript{2}\emph{A}\textsubscript{2}:\ifprintanswers\quad\textbf{0.25}\fi\vspace{0.5\baselineskip}


\fullwidth{Those genotype frequencies are expected if individuals in the outbreeding population are mating randomly.}

\question[3]
Assume that the population then becomes subdivided into two
subpopulations by an impermeable barrier to dispersal, and that
\emph{all other assumptions of \textsc{hwe} are met}. The population is now an \textbf{inbred population} because random mating can no longer occur among all individuals in the entire population. The haplotype frequencies
for each population are shown below. \textit{The frequency of each allele averaged across the subpopulations is still 0.5.} Calculate the genotype frequencies
for each inbred subpopulation to fill in the table on the next page.

\begin{center}
	\includegraphics[width=0.5\textwidth]{nonrandom_mating_inbred}
\end{center}

{\large
\begin{longtable}[c]{@{}lccccc@{}}
\toprule
& \emph{A}\textsubscript{1} &%
\emph{A}\textsubscript{2} &%
\emph{A}\textsubscript{1}\emph{A}\textsubscript{1} &%
\emph{A}\textsubscript{1}\emph{A}\textsubscript{2} &%
\emph{A}\textsubscript{2}\emph{A}\textsubscript{2} \\[0.35cm]
Inbred subpopulation 1 &%
	 0.25 &%
	 0.75 &%
	 \ifprintanswers\textbf{0.0625}\else\rule{0.5in}{0.4pt}\fi &%
	 \ifprintanswers\textbf{0.375}\else\rule{0.5in}{0.4pt}\fi &%
	 \ifprintanswers\textbf{0.5625}\else\rule{0.5in}{0.4pt}\fi \\[0.35cm] 
Inbred subpopulation 2 &%
	0.75 &%
	0.25 &%
	\ifprintanswers\textbf{0.5625}\else\rule{0.5in}{0.4pt}\fi &%
	\ifprintanswers\textbf{0.375}\else\rule{0.5in}{0.4pt}\fi &%
	\ifprintanswers\textbf{0.0625}\else\rule{0.5in}{0.4pt}\fi \\[0.35cm]
Inbred Average* &%
	0.5 &%
	0.5 &%
	\ifprintanswers\textbf{0.3125}\else\rule{0.5in}{0.4pt}\fi &%
	\ifprintanswers\textbf{0.375}\else\rule{0.5in}{0.4pt}\fi &%
	\ifprintanswers\textbf{0.3125}\else\rule{0.5in}{0.4pt}\fi \tabularnewline
\bottomrule
\end{longtable}
}%end Large

\fullwidth{%
* Calculate the average frequency of each genotype for both
subpopulations by averaging the two values in each column.
}%end fullwidth

\question
Fill in the table below with the results from Question 1.

{\large
\begin{longtable}[c]{@{}lccccc@{}}
\toprule
& \emph{A}\textsubscript{1} &%
 \emph{A}\textsubscript{2} &%
 \emph{A}\textsubscript{1}\emph{A}\textsubscript{1} &%
 \emph{A}\textsubscript{1}\emph{A}\textsubscript{2} &%
 \emph{A}\textsubscript{2}\emph{A}\textsubscript{2} \\[0.35cm]
 Outbred Population &%
 	0.5 &%
	0.5 &%
	\ifprintanswers\textbf{0.25}\else\rule{0.5in}{0.4pt}\fi &%
	\ifprintanswers\textbf{0.5}\else\rule{0.5in}{0.4pt}\fi &%
	\ifprintanswers\textbf{0.25}\else\rule{0.5in}{0.4pt}\fi \tabularnewline
\bottomrule
\end{longtable}
}%end Large

\question[1]
Describe how the genotype frequencies you calculated for
the inbred population differ from the genotype frequencies you calculated for
the outbred population. Which has greater heterozygosity and which has less heterozygosity?

\begin{minipage}[t][1.5in]{\textwidth}%
\begin{solution}
The inbred population has lower heterozygosity.
\end{solution}
\end{minipage}

\question[1]
What happens to heterozygosity in an inbred population? What happens to heterozygosity in a small population? What do inbreed and genetic drift have in common with respect to heterozygosity?

\begin{minipage}[t][1.5in]{\textwidth}%
\begin{solution}
Inbred populations have lower heterozygosity. Small populations have lower heterozygosity. Both inbreeding and genetic drift cause the loss of heterozygosity.
\end{solution}
\end{minipage}

\fullwidth{%
Small populations and inbred populations both lose heterozygosity.
Reduced genetic variation reduces the ability of the population to adapt
to changing environments.
}%end Fullwidth

\newpage

%\fullwidth{%
%\subsection*{Consequences of Inbreeding}
%}%end fullwidth

\fullwidth{%
\textbf{Consequences of Inbreeding}\vspace{\baselineskip}

Consider a population with a very rare recessive allele,
\emph{A}\textsubscript{2}, at a frequency of 0.001.
}%end Fullwidth

\question[1]
Calculate the following frequencies. Calculate the genotype frequencies to six digits after the decimal.\vspace{\baselineskip}

\emph{A}\textsubscript{1}: \ifprintanswers\quad\ \textbf{0.999}\fi\vspace{\baselineskip}

\emph{A}\textsubscript{1}\emph{A}\textsubscript{1}:\ifprintanswers\quad\textbf{0.998001}\fi\vspace{\baselineskip}

\emph{A}\textsubscript{1}\emph{A}\textsubscript{2}:\ifprintanswers\quad\textbf{0.001998}\fi\vspace{\baselineskip}

\emph{A}\textsubscript{2}\emph{A}\textsubscript{2}:\ifprintanswers\quad\textbf{0.000001}\fi\vspace{\baselineskip}

\fullwidth{%
Although the frequency of heterozygous individuals in the population is
very small, heterozygotes appear 2000 times more frequently in the
population (in this example) than individuals that are homozygous for
the rare recessive allele. Copies of rare alleles are most often found
in heterozygous individuals in outbreeding populations. This is known as
\textbf{hidden} or \textbf{concealed genetic variation}. The
variation is hidden because the recessive phenotype can be observed only
in individuals homozygous for the rare allele, and these homozygotes are
extremely rare. All other individuals express the dominant
\emph{A}\textsubscript{1} phenotype.
}%end fullwidth

\question[1]
Based on what you know about inbred populations and
heterozygosity, describe how the frequency of the
\emph{A}\textsubscript{2}\emph{A}\textsubscript{2} genotype should
change over future generations. (Assume no fitness difference among
genotypes.) Explain why.

\begin{minipage}[t][1.5in]{\textwidth}%
\begin{solution}
Should increase because of decreasing heterozygosity / increasing homozygosity found in non-randomly mating populations.\end{solution}
\end{minipage}

\vfill

\fullwidth{%
This result has two consequences. The first consequence to consider is how average fitness in the population 
would be affected if the rare recessive allele reduces fitness without causing death.
}% end fullwidth

\newpage

\question[1]
Describe what should happen to the average fitness of this
inbred population (average number of offspring produced by all breeding
individuals) if the rare recessive allele is detrimental (but is not lethal)? Explain why.

\begin{minipage}[t][2in]{\textwidth}%
\begin{solution}
The average fitness will decline because more offspring will be produced that are homozygous for the detrimental allele.\end{solution}
\end{minipage}

\fullwidth{%
The \textbf{heterozygote deficit} found in inbred populations is called
\textbf{inbreeding depression} because the average fitness of the entire
population (total number of offspring produced averaged across \emph{N}\textsubscript{e}) is generally lower than
outbreeding populations. This phenomenon is easily seen in domestic
animals bred for show quality. For example, many purebred dogs have
congenital hip defects because these breeds are highly inbred compared
to dogs overall. Some human groups also show abnormally high incidences
of some genetic diseases and birth defects, such as hemophilia in the
European Royal families during the 19th century or
Ellis-van Creveld Syndrome (dwarfism and polydactyly) in Old Order Amish
(founder effect and inbreeding).
}%end Fullwidth

\fullwidth{%
The second consequence to consider is what would happen in the
population if the rare allele were advantageous.
}%end fullwidth

\question[1]
What would happen to the average fitness of an inbred
population if the rare allele was beneficial? Explain why. Be sure to
recall the concepts learned in the selection exercise.

\begin{minipage}[t][2in]{\textwidth}%
\begin{solution}
For the same reason, fitness should increase. More homozygotes will appear with the rare, beneficial allele. Selection will have a greater chance of acting on the phenotype.\end{solution}
\end{minipage}


\fullwidth{%
You should now understand why non-random mating would cause a population
to deviate from \textsc{hwe}. The final point of this exercise is that, even in
large populations, many alleles occur at frequencies \textless{} 0.01
(1\%), which means that populations harbor an enormous amount of
concealed variation that is only expressed in homozygotes. While many of
these alleles may be detrimental, beneficial mutations also have a
greater chance of phenotypic expression. This possibility is greatest in
small populations or in non-randomly mating populations of any size.
}%end fullwidth

\end{questions}

\end{document}  