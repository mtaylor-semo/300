%!TEX TS-program = lualatex
%!TEX encoding = UTF-8 Unicode

\documentclass[11pt, addpoints]{exam}
\usepackage{graphicx}
	\graphicspath{{/Users/goby/Pictures/teach/300/exercises/}
	{img/}} % set of paths to search for images

%\printanswers

\usepackage{geometry}
\geometry{letterpaper, bottom=1in}                   
%\geometry{landscape}                % Activate for for rotated page geometry
%\usepackage[parfill]{parskip}    % Activate to begin paragraphs with an empty line rather than an indent
\usepackage{amssymb, amsmath}
\usepackage{mathtools}
	\everymath{\displaystyle}

\usepackage{fontspec}
\setmainfont[Ligatures={TeX}, BoldFont={* Bold}, ItalicFont={* Italic}, BoldItalicFont={* BoldItalic}, Numbers={Proportional, OldStyle}]{Linux Libertine O}
\setsansfont[Scale=MatchLowercase,Ligatures=TeX]{Linux Biolinum O}
%\setmonofont[Scale=MatchLowercase]{Inconsolata}
\usepackage{microtype}

\usepackage{unicode-math}
\setmathfont[Scale=MatchLowercase]{Asana Math}
%\setmathfont[Scale=MatchLowercase]{XITS Math}

% To define fonts for particular uses within a document. For example, 
% This sets the Libertine font to use tabular number format for tables.
\newfontfamily{\tablenumbers}[Numbers={Monospaced}]{Linux Libertine O}
\newfontfamily{\libertinedisplay}{Linux Libertine Display O}

\usepackage{booktabs}
%\usepackage{tabularx}
\usepackage{longtable}
%\usepackage{siunitx}
\usepackage{array}
\newcolumntype{L}[1]{>{\raggedright\let\newline\\\arraybackslash\hspace{0pt}}p{#1}}
\newcolumntype{C}[1]{>{\centering\let\newline\\\arraybackslash\hspace{0pt}}p{#1}}
\newcolumntype{R}[1]{>{\raggedleft\let\newline\\\arraybackslash\hspace{0pt}}p{#1}}

\usepackage{enumitem}
\usepackage{hyperref}
%\usepackage{placeins} %PRovides \FloatBarrier to flush all floats before a certain point.
\usepackage{hanging}

%\usepackage{titling}
%\setlength{\droptitle}{-60pt}
%\posttitle{\par\end{center}}
%\predate{}\postdate{}

\usepackage[sc]{titlesec}

\renewcommand{\solutiontitle}{\noindent}
\unframedsolutions
\SolutionEmphasis{\bfseries}

\pagestyle{headandfoot}
\firstpageheader{{\scshape bi}~300: Evolution}{}{\ifprintanswers\textbf{KEY}\else Name: \enspace \makebox[2.5in]{\hrulefill}\fi}
\runningheader{}{}{\footnotesize{pg. \thepage}}
\footer{}{}{}

\begin{document}

\subsection*{Hardy-Weinberg Equilibrium: Gene Flow (\numpoints\ points)}

This exercise will show how \textbf{gene flow} (the migration of
breeding individuals among populations) affects haplotype frequency
in a population. We will continue to look at a single locus with two
haplotypes (\emph{A} and \emph{a}). The rate at which individuals move 
among populations can vary depending on the dispersal ability of the 
organisms and any geographic barriers such as unsuitable habitat 
between populations.

\begin{questions}

\question
Assume four (4) populations, each with
$N_e$ = 1000, a starting frequency for Haplotype
\emph{A} of 0.5, and no mutations. Individuals can move freely among
populations so the migration rate is a high value of 0.6. That is, 60\%
of the individuals in a population migrate to any one of the other three
populations each generation. Draw on the graph below how you would
predict the frequency of Haplotype \emph{A} to change in \emph{each} of
the four populations (use one line for each population).

\ifprintanswers
	{\bfseries %
	POP SIZE = 1000\\
	\# POPS = 4\\
	FREQ = 0.5\\
	GENERATIONS = 1000
	
	First, run 4 pops with NO MIGRATION. 
	
	THEN MIGRATION = 0.6 }\vspace*{14\baselineskip}
\else
		\begin{center}
			\includegraphics[height=20\baselineskip]{prediction_graph}
		\end{center}
\fi

\question[1]
Explain the reasoning you used to arrive at your
prediction. Compare your prediction to those made by others in your
group. Discuss any differences with your group.

\newpage

\question On the graph below, draw the results of the simulation.

\begin{center}
	\includegraphics[width=0.9\textwidth]{prediction_graph}
\end{center}

\question
Why do you think the four populations showed similar
patterns of haplotype frequency change? Discuss ideas with your group
and explain below.

\vspace*{\stretch{1}}


\fullwidth{%
Gene flow keeps the populations genetically similar to each other.
Because of migration, the four populations of $N_e$ = 1000 individuals 
behave as a single population of $N_e$ = 4000. Therefore, the 
haplotype frequencies in each population change together and genetic 
drift is less pronounced because $N_e$ is (ahem) effectively larger. Would
you expect these results to differ for smaller effective population sizes? Find out next.
}%end fullwidth
\newpage

\question
On the graph below, draw your prediction for the frequency
change of Haplotype \emph{A} using the same assumptions as Question 1
but for four (4) populations, each with $N_e$ = 100.

\ifprintanswers
	{\bfseries %
	POP SIZE = 100\\
	\# POPS = 4\\
	FREQ = 0.5\\
	GENERATIONS = 1000\\
	MIGRATION = 0.6 }\vspace*{16\baselineskip}
\else
	\begin{center}
		\includegraphics[height=20\baselineskip]{prediction_graph}
	\end{center}
\fi

\question[1] 
Explain the reasoning you used to arrive at your predictions. 
Compare your prediction to those made by others in your
group. Discuss any differences with your group.

%\vspace*{\stretch{1}}
\newpage

\question 
On the graph below, draw the results of the simulation.

\begin{center}
	\includegraphics[width=0.9\textwidth]{prediction_graph}
\end{center}

\question 
Did the results agree with your predictions? Compare the
results with your predictions and those of your group. Discuss ideas
with your group, and then explain why you think your results did or did
not agree with your prediction.

\vspace{\stretch{1}}


\fullwidth{%
These simulations assumed a relatively high migration rate (0.6). This
means that gene flow is high among populations so it should not be
surprising that genetic variation among the four populations remained
similar, even if $N_e$ is very small. Would the results differ if gene flow was low?
Find out with the next simulation.
}%end fullwidth
\newpage

\question
Assume four populations, each with $N_e$ = 1000. On the graph below, 
draw your predictions for the frequency change of Haplotype \emph{A} 
(starting at 0.5) when the migration rate is very low (0.01).

\ifprintanswers
	{\bfseries %
	POP SIZE = 1000\\
	\# POPS = 4\\
	FREQ = 0.5\\
	GENERATIONS = 1000\\
	MIGRATION = 0.01 }\vspace*{16\baselineskip}
\else
	\begin{center}
		\includegraphics[height=20\baselineskip]{prediction_graph}
	\end{center}
\fi

\question[1]
Explain the reasoning you used to arrive at your predictions. Compare your 
prediction to those made by others in your group. Discuss any differences 
with your group.

\newpage

\question
Draw the results of the simulation on this graph.

\begin{center}
	\includegraphics[width=0.9\textwidth]{prediction_graph}
\end{center}

\question[2]
Did the results agree with your predictions? With your
group, write a general principle to explain how migration among
populations affects genetic variation and potential genetic divergence
of those populations? Be sure your principle is a clear, direct
statement. 

\begin{solution}
Migration can increase genetic variation by introducing new haplotypes into a population.
Migration reduces or prevents the potential for genetic divergence among populations
because they all share the same gene pool.
\end{solution}

\newpage

\question
Let's look more closely at the effects of genetic drift.
Assume four populations, each with $N_e$ = 100. On
the graph below, draw your predictions for the frequency change of
Haplotype \emph{A} when the migration rate is very low (0.01). Be sure
to consider the general principle that you wrote above.

\ifprintanswers
	{\bfseries %
	POP SIZE = 100\\
	\# POPS = 4\\
	FREQ = 0.5\\
	GENERATIONS = 1000\\
	MIGRATION = 0.01 }\vspace*{16\baselineskip}
\else
	\begin{center}
		\includegraphics[height=20\baselineskip]{prediction_graph}
	\end{center}
\fi


\question[1]
Explain the reasoning you used to arrive at your
predictions. Compare your prediction to those made by others in your
group. Discuss any differences with your group.

\newpage

\question
Draw the results of the simulation on this graph.

\begin{center}
	\includegraphics[width=0.9\textwidth]{prediction_graph}
\end{center}

\question[2]
Did the results agree with your general principle? What
effect did genetic drift have on the population? Do you need to modify
your general principle? If so, write your modified principle below to
account for both gene flow and genetic drift.

\newpage

\fullwidth{%
How much gene flow among populations is necessary to keep
them genetically similar? The genetic similarity of populations can be
measured with the \textbf{fixation index} ($F_{ST}$),

\[F_{ST} = \frac{1}{4N_e m + 1}\]

where $m$ is the migration rate and $N_e m$ is the total number
of migrants per generation. The fixation index values vary between 0 and
1. If $F_{ST}$ = 0, then the populations are
genetically identical. If $F_{ST}$ = 1, then the
populations are fixed for different haplotypes at a locus. That is, the
populations do not have any haplotypes in common at a locus. As a rule,
the closer $F_{ST}$ is to 0, the more genetically
similar the populations. The closer $F_{ST}$ is to 1,
the more genetically different the populations.
}%end fullwidth

\question[2]
Calculate $F_{ST}$ for the following
populations. Assume that a single individual migrates between
populations each generation. With this assumption, you can calculate the
migration rate as $m = 1/N_e$  per generation.

\begin{parts}
	\part $N_e$ = 100: 
		\ifprintanswers
			\qquad $m = \frac{1}{100} = 0.01$ so $F_{ST} = \frac{1}{(4 \cdot 100 \cdot 0.01)+1} = 0.20.$\vspace{2\baselineskip}
		\else
			\vspace{3\baselineskip}
		\fi

	\part $N_e$ = 2000:
		\ifprintanswers
			\qquad $m = \frac{1}{2,000} = 0.0005$ so $F_{ST} = \frac{1}{(4 \cdot 2,000 \cdot 0.0005)+1} = 0.20.$\vspace{2\baselineskip}
		\else
			\vspace{3\baselineskip}
		\fi
	
	\part $N_e$ = 50,000:
			\ifprintanswers
			\qquad $m = \frac{1}{50,000} = 0.00002$ so $F_{ST} = \frac{1}{(4 \cdot 50,000 \cdot 0.00002)+1} = 0.20.$\vspace{2\baselineskip}
		\else
			\vspace{3\baselineskip}
		\fi

\end{parts}

\fullwidth{%
As you can see, very low migration rates are sufficient to keep
populations genetically similar. This observation explains the results
of the simulations performed earlier. We will revisit this concept when
we consider speciation.

\subsection*{Mutation}
}%end fullwidth

\question[1]
We did not consider mutation in this simulation. Based on
what you know about the effects of mutation, how do you think it would
affect any of the results above? Do you think any one of these
assumptions (population size, migration rate, mutation rate) will have a
greater effect than the other assumptions on genetic variation among
populations? Explain.

\newpage

\question
Based on your reasoning above, what do you think would
happen to Haplotype \emph{A} over 1000 generations with
$N_e$ = 100, a migration rate of 0.01, a mutation
rate from \emph{A} to \emph{a} of 0.03 and a mutation rate from \emph{a}
to \emph{A} of 0.01. Draw your prediction on the graph below.

\ifprintanswers
	{\bfseries %
	POP SIZE = 100\\
	\# POPS = 4\\
	FREQ = 0.5\\
	GENERATIONS = 1000\\
	MIGRATION = 0.01 \\
	MUTATION from A to a = 0.03\\
	MUTATION from a to A = 0.01}\vspace*{14\baselineskip}
\else
	\begin{center}
		\includegraphics[height=20\baselineskip]{prediction_graph}
	\end{center}
\fi

\question[1]
Were you correct? Of population size, migration rate and
mutation rate, which appears to have the greatest effect on genetic
variation in the populations? Explain.

\begin{solution}
Mutation will have greatest effect. population will drift around equilibrium frequency. Mutation most prominent because independent of population size, and migration will distribute mutations among populations, keeping them homogeneous, so drifting together.
\end{solution}
\end{questions}


\end{document}  