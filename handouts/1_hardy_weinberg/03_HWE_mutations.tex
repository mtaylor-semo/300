%!TEX TS-program = lualatex
%!TEX encoding = UTF-8 Unicode

\documentclass[11pt, addpoints]{exam}
\usepackage{graphicx}
	\graphicspath{{/Users/goby/Pictures/teach/300/}
	{img/}} % set of paths to search for images

\usepackage{geometry}
\geometry{letterpaper, bottom=1in}                   
%\geometry{landscape}                % Activate for for rotated page geometry
%\usepackage[parfill]{parskip}    % Activate to begin paragraphs with an empty line rather than an indent
\usepackage{amssymb, amsmath}
\usepackage{mathtools}
	\everymath{\displaystyle}

\usepackage{fontspec}
\setmainfont[Ligatures={TeX}, BoldFont={* Bold}, ItalicFont={* Italic}, BoldItalicFont={* BoldItalic}, Numbers={Proportional}]{Linux Libertine O}
\setsansfont[Scale=MatchLowercase,Ligatures=TeX]{Linux Biolinum O}
\setmonofont[Scale=MatchLowercase]{Linux Libertine Mono O}
\usepackage{microtype}

\usepackage{unicode-math}
\setmathfont[Scale=MatchLowercase]{Asana Math}
%\setmathfont[Scale=MatchLowercase]{XITS Math}

% To define fonts for particular uses within a document. For example, 
% This sets the Libertine font to use tabular number format for tables.
\newfontfamily{\tablenumbers}[Numbers={Monospaced}]{Linux Libertine O}
\newfontfamily{\libertinedisplay}{Linux Libertine Display O}

\usepackage{booktabs}
%\usepackage{tabularx}
\usepackage{longtable}
%\usepackage{siunitx}
\usepackage{array}
\newcolumntype{L}[1]{>{\raggedright\let\newline\\\arraybackslash\hspace{0pt}}p{#1}}
\newcolumntype{C}[1]{>{\centering\let\newline\\\arraybackslash\hspace{0pt}}p{#1}}
\newcolumntype{R}[1]{>{\raggedleft\let\newline\\\arraybackslash\hspace{0pt}}p{#1}}

\usepackage{enumitem}
\usepackage{hyperref}
%\usepackage{placeins} %PRovides \FloatBarrier to flush all floats before a certain point.
\usepackage{hanging}

\usepackage{titling}
\setlength{\droptitle}{-60pt}
\posttitle{\par\end{center}}
\predate{}\postdate{}

\renewcommand{\solutiontitle}{\noindent}
\unframedsolutions
\SolutionEmphasis{\bfseries}

\pagestyle{headandfoot}
\firstpageheader{BI 300: Evolution}{}{\ifprintanswers\textbf{KEY}\else Name: \enspace \makebox[2.5in]{\hrulefill}\fi}
\runningheader{}{}{\footnotesize{pg. \thepage}}
\footer{}{}{}
\runningheadrule

%\printanswers

\begin{document}

\subsection*{Hardy-Weinberg Equilibrium: Mutations (\numpoints\ points)}

You have seen how the number of breeding individuals
(\emph{N}\textsubscript{e}) in a population affects the number of new mutations that can occur in the population every generation. This exercise will examine how mutations
that occur in a population affect haplotype frequencies. We'll continue to look 
at a single locus with two haplotypes (\emph{A} and \emph{a}). 
The rate at which new mutations occur in a population is generally very
low. Even low mutation rates, however, can dramatically affect haplotype
frequencies in a population over many generations. We'll use both low
and high mutation rates to demonstrate this concept.

\begin{questions}

\question
Assume $N_e$ = 1000 individuals and a
starting frequency for Haplotype \emph{A} of 0.5, for 1000 generations. On the graph below,
draw your prediction for how the frequency of Haplotype \emph{A} might
change over 1000 generations when the mutation rate from \emph{A} to \emph{a} is very low
(0.01 mutations per generation) and there is no mutation from \emph{a}
to \emph{A}. 

\ifprintanswers
	{\bfseries Initial Freq of A to 0.5 \\	
	Pop size = 1000\\	
	Generations = 1000\\
	Mutation from A to a = 0.01\\
	Mutation from a to A = 0.0

	DO Five Simulations
	
	\vspace{\baselineskip}
	
	REPEAT But Change:\\	
	Mutation from A to a = 0.0\\
	Mutation from a to A = 0.01
	
	DO Five Simulations}\vspace{9\baselineskip}
\else
		\begin{center}
			\includegraphics[height=20\baselineskip]{prediction_graph}
		\end{center}
\fi

\question[2]
Explain the reasoning you used to arrive at your
prediction. Compare your prediction to those made by others in your
group. Discuss any differences with your group.%\vspace{1cm}

\newpage

\question On the graph below, draw the results from several runs of
the simulation software.

\begin{center}
	\includegraphics[width=0.9\textwidth]{prediction_graph}
\end{center}

\question
Did the results agree with your predictions? Compare the
results with your predictions and those of your group. Discuss ideas
with your group, and then explain why you think your results did or did
not agree with your prediction.%\vspace{3cm}

\begin{solution}
BE SURE TO HAVE STUDENTS EXPLAIN SHAPE: WHY INITIAL RAPID DROP, FOLLOWED BY TAPER

Becomes fixed for allele. Lots of copies of A, so even low mutation rate has lots to work with, but as more alleles become a / A, then fewer copies for mutations so rate of change slows down.
\end{solution}

\newpage

\question
On the graph below, draw with a solid line your prediction
given the following assumptions: $N_e$ = 1000, a
starting frequency of 0.5 for Haplotype \emph{A}, a mutation rate from
\emph{A} to \emph{a} of 0.6 and a mutation rate from \emph{a} to
\emph{A} of 0.3, for 1000 generations. Draw a dashed line to show your prediction for the
reverse scenario: the mutation rate from \emph{A} to \emph{a} is 0.3 and
the mutation rate from \emph{a} to \emph{A} is 0.6. Both mutations rates
are unnaturally high but will serve to illustrate a point.

\ifprintanswers
	{\bfseries 
	Pop size = 1000\\
	Freq = 0.5\\
	Generations = 1000\\
	Mutation from A to a = 0.6,\\
	a to A = 0.3\\
	Five simulations
	
	THEN
	
	Mutations from A to a = 0.3,\\
	a to A = 0.6\\
	Five simulations}\vspace{10\baselineskip}
\else
	\begin{center}
		\includegraphics[height=20\baselineskip]{prediction_graph}
	\end{center}
\fi

\question[2] 
Explain the reasoning you used to arrive at your
predictions. Compare your prediction to those made by others in your
group. Discuss any differences with your group.

%\vspace*{\stretch{1}}
\newpage

\question 
On the graph below, draw the results from several runs of
the simulation software.

\begin{center}
	\includegraphics[width=0.9\textwidth]{prediction_graph}
\end{center}

\question Did the results agree with your predictions? Compare the
results with your predictions and those of your group. Discuss ideas
with your group, and then explain why you think your results did or did
not agree with your prediction.

%\ifprintanswers
\begin{solution}[\stretch{1}]
Will hit an equilibrium point. There will be more copies of a but a lower mutation rate balances against fewer copies of A which has a higher mutation rate. After the run, the frequencies will settle at $0.33\overline{3}$ for A and $0.66\overline{6}$ for a. To show how equilibrium is reached, multiple each frequencies by it’s mutation rate. The numbers will be equal.

A: $0.333333 \times 0.6 = 0.2$ \qquad There ends up being an equal number

a: $0.666666 \times 0.3 = 0.2$ \qquad of alleles mutating between the two states.
\end{solution}
%\else
%	\vspace*{\stretch{1}}
%\fi

\question[2]
Would you expect the results to change if the mutation rate
is very low (say, 0.06 and 0.03)? Explain.
\begin{solution} Reaches same equilibrium point but more slowly.\end{solution}
\vspace*{\stretch{1}}

\newpage

\fullwidth{%
The two haplotypes will reach a point of equilibrium where the rate of
mutation from \emph{a} to \emph{A} is balanced by the rate of mutation
from \emph{A} to \emph{a}. These observations show that the magnitude
(0.6 vs 0.06) of the mutation rate does not affect the final point of
equilibrium achieved by the two haplotypes. Only the time necessary to
reach the point of equilibrium will change. Faster mutation rates will
reach equilibrium between the haplotypes more quickly than slower
mutation rates.

\subsection*{Population size}

Does population size matter? Would any of the results we observed with
the above simulations change if we changed population size? For example,
the first simulation started with $N_e$ = 1000, a mutation rate from \textit{A} to
\textit{a} of 0.01 (no mutations from \textit{a} to \textit{A}), and a starting frequency
for Haplotype A of 0.5. What would happen if $N_e$ = 100 or $N_e$ = 10,000, but the
mutation rate did not change? 
Find out with the next simulation.
}% end fullwidth

\question[2]
Explain below the graph how you would expect the results to
differ in a smaller population of $N_e$ = 100
individuals and a larger population of $N_e$ =
10,000 individuals. The mutation rate is still 0.01 for \textit{A} to
\textit{a}, with no mutations from \textit{a} to \textit{A}, and the starting frequency is still
0.5. If you would not expect the results to differ, then
explain why not. Draw your predictions on the graph. 
For both runs, use a mutation rate from \emph{A} to \emph{a} of 0.01, no mutations from
\emph{a} to \emph{A}, and a starting frequency for Haplotype \emph{A} of
0.5, for 1000 generations.

\ifprintanswers
	{\bfseries 
	Initial Freq of A to 0.5\\
	Pop size = 100\\
	Generations = 1000\\
	Mutation from A to a = 0.01\\
	Mutation from a to A = 0.0\\
	Run several, THEN\vspace{\baselineskip}
	
	Pop size = 10,000\\
	Run several}\vspace{11\baselineskip}
\else
	\begin{center}
		\includegraphics[height=20\baselineskip]{prediction_graph}
	\end{center}
\fi

\newpage

\question
On the graph below, draw the results of the simulation.

\begin{center}
	\includegraphics[width=0.9\textwidth]{prediction_graph}
\end{center}

\question
Did the results agree with your predictions? Compare the
results with your predictions and those of your group. Discuss ideas
with your group, and then explain why you think your results did or did
not agree with your prediction.

\begin{solution}
Mutation rate is independent of population size. Drift is more pronounced in small population, so more random variation in frequency but ultimately still lost.
\end{solution}

\newpage

\fullwidth{
You can think about the relationship between mutation rate and
population size another way (recalling back to the exercise on mutation
rates). The mutation rate is relatively constant, such as $2.2 \times
10^{-9}$ mutations per base pair per generation, and is
independent of population size. Each diploid individual in a population
of size $N_e$ has a chance of getting a new mutation
in one of its two haplotype copies. (Although mutations can occur in all 
individuals in a population, only heritable mutations are important so the 
total number of mutations is considered for effective population size and not census
population size.) Thus, if $2N_e$ represents the haplotype copy number in the population, and $\mu$
represents the mutation rate, $2N_e\mu$ represents the total number of new mutations that will occur, on
average, each generation in a population of breeding individuals. If
$2N_e$ is large, then a large number of total
mutations will occur in the population in each generation. If
$2N_e$ is small, then only a small number of total
mutations will occur in each generation in the population.
}%end fullwidth

\question[2]
Convince yourself by multiplying a mutation rate ($\mu$) of
0.02 by: 
\begin{parts}
	\part $N_e$ = 100,000:
		\ifprintanswers
			\qquad $2N_e \times \mu = 2 \times 100,000 \times 0.02 = 4,000$\vspace{3\baselineskip}
		\else
			\vspace{3\baselineskip}
		\fi

	\part $N_e$ = 1,000:
		\ifprintanswers
			\qquad $2N_e \times \mu = 2 \times 1,000 \times 0.02 = 40$\vspace{3\baselineskip}
		\else
			\vspace{3\baselineskip}
		\fi
\end{parts}

\fullwidth{%
Thus, although all populations may have the same mutation \emph{rate},
the absolute number of new mutations that occur in a population is
proportional to effective population size. This will become important
later when we consider natural selection. From the lecture on genetic drift, you should recall that the neutral
mutation rate is the basis of the \textbf{Neutral Theory} and the rate
at which \textbf{nonadaptive evolution} occurs in populations via
genetic drift.
}%end fullwidth

\end{questions}


\end{document}  