%!TEX TS-program = lualatex
%!TEX encoding = UTF-8 Unicode

\documentclass[11pt, addpoints]{exam}
\usepackage{graphicx}
	\graphicspath{{/Users/goby/Pictures/teach/300/}
	{img/}} % set of paths to search for images

\usepackage{geometry}
\geometry{letterpaper, bottom=1in}                   
%\geometry{landscape}                % Activate for for rotated page geometry
%\usepackage[parfill]{parskip}    % Activate to begin paragraphs with an empty line rather than an indent
\usepackage{amssymb, amsmath}
\usepackage{mathtools}
	\everymath{\displaystyle}

\usepackage{fontspec}
\setmainfont[Ligatures={TeX}, BoldFont={* Bold}, ItalicFont={* Italic}, BoldItalicFont={* BoldItalic}, Numbers={Proportional}]{Linux Libertine O}
\setsansfont[Scale=MatchLowercase,Ligatures=TeX]{Linux Biolinum O}
\setmonofont[Scale=MatchLowercase]{Inconsolata}
\usepackage{microtype}

\usepackage{unicode-math}
\setmathfont[Scale=MatchLowercase]{Asana Math}
%\setmathfont[Scale=MatchLowercase]{XITS Math}

% To define fonts for particular uses within a document. For example, 
% This sets the Libertine font to use tabular number format for tables.
\newfontfamily{\tablenumbers}[Numbers={Monospaced}]{Linux Libertine O}
\newfontfamily{\libertinedisplay}{Linux Libertine Display O}

\usepackage{booktabs}
%\usepackage{tabularx}
\usepackage{longtable}
%\usepackage{siunitx}
\usepackage{array}
\newcolumntype{L}[1]{>{\raggedright\let\newline\\\arraybackslash\hspace{0pt}}p{#1}}
\newcolumntype{C}[1]{>{\centering\let\newline\\\arraybackslash\hspace{0pt}}p{#1}}
\newcolumntype{R}[1]{>{\raggedleft\let\newline\\\arraybackslash\hspace{0pt}}p{#1}}

\usepackage{enumitem}
\usepackage{hyperref}
%\usepackage{placeins} %PRovides \FloatBarrier to flush all floats before a certain point.
\usepackage{hanging}

\usepackage{titling}
\setlength{\droptitle}{-60pt}
\posttitle{\par\end{center}}
\predate{}\postdate{}

\renewcommand{\solutiontitle}{\noindent}
\unframedsolutions
\SolutionEmphasis{\bfseries}

\pagestyle{headandfoot}
\firstpageheader{BI 300: Evolution}{}{\ifprintanswers\textbf{KEY}\else Name: \enspace \makebox[2.5in]{\hrulefill}\fi}
\runningheader{}{}{\footnotesize{pg. \thepage}}
\footer{}{}{}
\runningheadrule

%\printanswers

\begin{document}

\subsection*{Selection for Favorable Mutations (\numpoints\ points)}

In the exercise on mutation rates, we saw that even small mutation rates
can generate a large number of mutations in populations of reasonable
size. Considering only mutations that affect phenotype, most are likely
to be detrimental and removed by natural selection but some mutations
will provide an advantage and will be favored by natural selection.
Let's look at a specific example.

Michael Nachman and colleagues have studied populations of rock pocket
mice (\textit{Chaetodipus intermedius}) in the deserts of Mexico, Arizona
and New Mexico. The desert is a patchwork mosaic of light-colored
granitic rock and dark basaltic rock from ancient lava flows.
Interestingly, the population of rock pocket mice consists of
individuals with light-colored fur that live primarily on light-colored
rocks and dark-colored individuals that live primarily on dark-colored
rocks.

Nachman and colleagues discovered that the difference in fur color of
rock pocket mice is due to a mutation in a gene called the
melanocortin-1 receptor, or \textit{MC1R}. Several independent studies of
this gene have revealed that any one of at least 10 different mutations
will change the fur color of mice from light to dark. That is, there are
at least 10 different ways that a mutation can change the fur color.

\textbf{Show your work throughout. Include units of measure (mutations, year, mice, etc.)}
\begin{questions}

\question[1]
\label{mutes_per_mouse}For mathematical simplicity, assume the mutation rate is $2.0
\times 10^{-9}$ per bp per generation (which is slightly lower
than the $2.2 \times 10^{-9}$ per bp per generation mutation
rate used in an earlier exercise). There are 10 types of mutations per
allele copy that can cause dark fur color, and two copies of each gene
in each mouse. Multiply the figures together to calculate the number of
mutations that would occur per mouse. What result do you get? Show your work.

\ifprintanswers\vspace{\baselineskip}
	\textbf{0.000000002 mutations $\times$ 10 sites $\times$ 2 copies = 0.00000004 mutations per mouse.}
	\vspace*{\stretch{1}}
\else
	\vspace*{\stretch{1}}
\fi

\question[1]
\label{how_many_mice}Fractional mutations in a single mouse are impossible so
convert this to the chance of a singe mouse getting one of the favorable
mutations (equivalent to one mutation per how many mice?) by dividing 1
by the result from question (\ref{mutes_per_mouse}). Show your work.
%\vspace*{\stretch{1}}

\ifprintanswers\vspace{\baselineskip}
	\textbf{1/0.00000004 mice = 1 in 25,000,000 per mouse.}
	\vspace*{\stretch{1}}
\else
	\vspace*{\stretch{1}}
\fi

\fullwidth{%
These are long odds for a single mouse but as before, we have to account
for all of the offspring produced in a population each generation. We
previously assumed that small mammals had an effective population size
of about 100,000 individuals. Let's be more conservative this time, and
use the following assumptions: $N_e$ = 10,000
individuals, of which exactly half is female, and each female produces 5
offspring per year. (In real life, female \textit{Chaetodipus intermedius}
produce 2--3 clutches per year with 3--6 offspring in each clutch.)
}%end fullwidth
\newpage

\question[1]
\label{how_many_offspring}How many offspring are produced in the population each year? Show your work.

\ifprintanswers\vspace{\baselineskip}
	\textbf{5000 female mice $\times$ 5 offspring per female = 25,000 offspring.}
	\vspace*{\stretch{1}}
\else
	\vspace*{\stretch{1}}
\fi

\question[1]
\label{mutes_per_year}Using the figures you calculated for questions (\ref{how_many_mice}) and (\ref{how_many_offspring}),
a new mutation should occur once every how many years? Show your work.

\ifprintanswers\vspace{\baselineskip}
	\textbf{25,000 mice/year $\times$ 1 mutation/25,000,000 mice = 0.001 mutations per year.}
	
	\textbf{0.001 mutations per year $\times$ 1000 = 1 mutation every 1000 years}
	\vspace*{\stretch{1}}
\else
	\vspace*{\stretch{1}}
\fi

\fullwidth{%
Although this is a long time to us, this is a ``blink of an eye'' for
evolutionary time scales. Again, the numbers we used are conservative so
the actual values could be much higher.
}%end fullwidth

\question[1]
\label{ten_mutations}Calculate the amount of time for one of the 10 mutations to
appear in the population using the following assumptions. Show your work.

\begin{itemize}
\item
  breeding females: 50,000
\item
  offspring produced per female per year: 10
\end{itemize}

\ifprintanswers\vspace{\baselineskip}
	\textbf{500,000 offspring per year} 
	
	\textbf{500,000/25,000,000 = 50 years}
	\vspace*{\stretch{2}}
\else
	\vspace*{\stretch{2}}
\fi

\fullwidth{%
Although less conservative, the assumptions given for question (\ref{ten_mutations}) are
reasonable for many species, including \emph{C. intermedius}. The results of your calculations for questions (\ref{mutes_per_year}) and (\ref{ten_mutations}) show you that advantageous
mutations to specific genes can occur very often over evolutionary time
scales.
}%end fullwidth
\newpage

\fullwidth{%
Any one of these mutations to \emph{MC1R} conveys a selective
advantage to rock pocket mice that live on the dark basalt. You can use
the mathematics of population genetics to estimate the number of
generations that are needed for all of the individuals in a population
to have the new advantageous mutation. The equation is 

\[t = \frac{2}{s}\ln(2N_e) \]

where \emph{t} is the time in generations, \emph{s} is the selection
coefficient, and $N_e$ is the effective population size.

Nachman and colleagues used population genetic mathematics and empirical
haplotype frequencies to estimate that the effective population size of
\emph{C. intermedius} was between 10,000 and 100,000 individuals. They
also estimated that the selection coefficients (\emph{s}) ranged from
0.39--0.013 against light-colored mice on dark rocks, and from
0.020--0.0002 against dark-colored mice on light rocks.\footnotemark
}

\question[2]
Using the equation above, fill in the four blanks in the table below to show
the number of generations needed for dark fur color to spread through a
population following the new mutation. Round your results to whole
generations. You do not have to show your work for this question.


\begin{longtable}[c]{@{}lcc@{}}
\toprule
& $N_e$ = 10,000 & $N_e$ = 100,000\\
\midrule\\[1em]
$s = 0.01$ &%
	\ifprintanswers
		\textbf{1981}
	\else
		\rule{2.5cm}{0.4pt} 
	\fi &
	\ifprintanswers
		\textbf{2441}
	\else
		\rule{2.5cm}{0.4pt} 
	\fi \\[2.2em]

$s = 0.2$ &%
	\ifprintanswers
		\textbf{99}
	\else
		\rule{2.5cm}{0.4pt} 
	\fi &
	\ifprintanswers
		\textbf{122}
	\else
		\rule{2.5cm}{0.4pt} 
	\fi \\
\bottomrule
\end{longtable}

\fullwidth{%
You should now understand that favorable mutations could occur often and
spread quickly in populations of reasonable size. Even if some
advantageous mutations are lost due to genetic drift, further mutations
to the same gene are very likely to occur again in a fairly short period
of evolutionary time. In this example, most estimates of \emph{s} are
much greater than $1/4N_e$, so natural selection has
a much greater chance of overriding the effects of genetic drift (see
the previous handout on the strength of selection).
}%
\end{questions}


\vfill

\footnotetext{For more details of the study and calculations, see Hoekstra, H.E.,
K.E. Drumm, and M.W. Nachman, 2004. Ecological genetics of adaptive
color polymorphism in pocket mice: geographic variation in selected and
neutral genes. Evolution 58: 1329--1341, and related information in your
text.}

\end{document}  