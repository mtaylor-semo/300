%!TEX TS-program = lualatex
%!TEX encoding = UTF-8 Unicode

\documentclass[11pt, addpoints]{exam}
\usepackage{graphicx}
	\graphicspath{{/Users/goby/Pictures/teach/300/}
	{img/}} % set of paths to search for images

\usepackage{geometry}
\geometry{letterpaper, bottom=1in}                   
%\geometry{landscape}                % Activate for for rotated page geometry
%\usepackage[parfill]{parskip}    % Activate to begin paragraphs with an empty line rather than an indent
\usepackage{amssymb, amsmath}
\usepackage{mathtools}
	\everymath{\displaystyle}

\usepackage{fontspec}
\setmainfont[Ligatures={TeX}, BoldFont={* Bold}, ItalicFont={* Italic}, BoldItalicFont={* BoldItalic}, Numbers={Proportional}]{Linux Libertine O}
\setsansfont[Scale=MatchLowercase,Ligatures=TeX]{Linux Biolinum O}
%\setmonofont[Scale=MatchLowercase]{Inconsolata}
\usepackage{microtype}

\usepackage{unicode-math}
\setmathfont[Scale=MatchLowercase]{Asana Math}

\newfontfamily{\tablenumbers}[Numbers={Monospaced}]{Linux Libertine O}
\newfontfamily{\libertinedisplay}{Linux Libertine Display O}

\usepackage{booktabs}
%\usepackage{tabularx}
\usepackage{longtable}
%\usepackage{siunitx}
\usepackage{array}
\newcolumntype{L}[1]{>{\raggedright\let\newline\\\arraybackslash\hspace{0pt}}p{#1}}
\newcolumntype{C}[1]{>{\centering\let\newline\\\arraybackslash\hspace{0pt}}p{#1}}
\newcolumntype{R}[1]{>{\raggedleft\let\newline\\\arraybackslash\hspace{0pt}}p{#1}}

\usepackage{enumitem}

\usepackage{titling}
\setlength{\droptitle}{-60pt}
\posttitle{\par\end{center}}
\predate{}\postdate{}

\renewcommand{\solutiontitle}{\noindent}
\unframedsolutions
\SolutionEmphasis{\bfseries}

\pagestyle{headandfoot}
\firstpageheader{BI 300: Evolution}{}{\ifprintanswers\textbf{KEY}\else Name: \enspace \makebox[2.5in]{\hrulefill}\fi}
\runningheader{}{}{\footnotesize{pg. \thepage}}
\footer{}{}{}
\runningheadrule

%\printanswers

\begin{document}

\subsection*{Hardy-Weinberg Equilibrium: Natural Selection (\numpoints\ points)}

This exercise will look at how \textbf{natural selection}, both positive
and negative, can alter genetic variation in a population. We'll
continue to look at a single locus with two haplotypes (\emph{A} and
\emph{a}). Natural selection can act on both the genotype and the
phenotype. For this exercise, we will assume that natural selection is
acting only on the phenotype but that the phenotype is determined
entirely by the genotype. In other words, the environment does not
affect the phenotype. When this is so, natural selection is effectively
acting directly on the genotype through the phenotype. Later in class,
we'll consider the relative contribution of the environment on the
phenotype.

\begin{questions}

\question[1]
What do think positive natural selection, or simply
\textbf{positive selection}, means? Similarly, what do you think
\textbf{negative selection} means?
\vspace*{\stretch{1}}

\question[1]
Based on your reading and group discussion, define
\textbf{relative} \textbf{fitness}.
\vspace*{\stretch{1}}

\question[1]
Describe the relationship between natural selection and
relative fitness.
\vspace*{\stretch{1}}

\question[1]
Do you think you can predict the number of generations it would take for a beneficial mutation to become fixed in a population? Explain
\vspace*{\stretch{1}}

\fullwidth{%
\textbf{Relative fitness} is represented by \emph{\textbf{w}}, which can
vary from 0 (no fitness) to 1 (maximum fitness). In models like the HWE,
one genotype is assumed to have the highest fitness. The other genotypes
may have equal or lower fitness. For example, the \emph{AA} genotype may
have the highest fitness, so \emph{w} = 1.0. \emph{Aa} may have a lower
relative fitness (\emph{w} = 0.9) and \emph{aa} may have the lowest
relative fitness (\emph{w} = 0.8). In previous HWE exercises, fitness
was assumed to be equal (\emph{w} = 1) for each genotype to satisfy the
assumption of no natural selection. We will now see how fitness
differences among genotypes affects the genetic variation of
populations.
}%end fullwidth
\newpage

\question
Assume \emph{N}\textsubscript{e} = 10,000 and a starting
frequency for Haplotype \emph{A} of 0.3. The fitness of genotype
\emph{AA} is \emph{w} = 1, the fitness of genotype \emph{Aa} is \emph{w}
= 0.75 and the fitness of genotype \emph{aa} is \emph{w} = 0.5. Predict
on the graph below how the frequency of Haplotype \emph{A} will change
in the population in response to selection. All other assumptions of
HWE, except for infinite population size, are met. (Mental question:
will drift be significant?)

\ifprintanswers
	{\bfseries %
	POP SIZE = 10,000\\
	\# POPS = 10\\
	FREQ = 0.3\\
	GENERATIONS = 1000\\	
	FITNESS = 1, 0.75, and 0.5, top to bottom. }\vspace*{16\baselineskip}
\else
	\begin{center}
		\includegraphics[height=20\baselineskip]{prediction_graph}
	\end{center}
\fi

\question
Explain the reasoning you used to arrive at your prediction.
Compare your prediction to those made by others in your group. Discuss
any differences with your group.

\newpage

\question
On the graph below, draw the results of the simulation.

	\begin{center}
		\includegraphics[height=20\baselineskip]{prediction_graph}
	\end{center}

\question
Did the results agree with your predictions? Discuss ideas
with your group, and then explain why you think your results did or did
not agree.
\vspace*{\stretch{1}}

\question[1]
Based on these results, can you tell if Haplotype \emph{A}
is subject to positive selection or if Haplotype \emph{a} is subject to
negative selection? Think about this carefully, discuss with your group,
and then explain.

\ifprintanswers 
	\textbf{No.}
	\vspace*{\stretch{1}}
\else
	\vspace*{\stretch{1}}
\fi

\newpage

\fullwidth{%
Without addition information we cannot determine whether natural
selection is acting positively on \emph{A} or negatively on \emph{a}. We
do know, however, that the \emph{AA} genotype has the selective
advantage in the population because it has the highest fitness. The
relative fitness difference between the genotype with the highest
fitness (\emph{w} = 1.0) and genotypes with lower fitness is called the
\textbf{selection} \textbf{coefficient} (or the \textbf{strength of
selection}), represented by \emph{\textbf{s}}. In the first simulation,
the relative fitness of \emph{Aa} was 0.75, so the selection coefficient
relative to \emph{AA} was \emph{s} = 1 -- 0.75 = 0.25.
}%end fullwidth

\question[1]
What was the selection coefficient for \emph{aa}?
\ifprintanswers
	\textbf{0.5}
	\vspace*{\stretch{0.5}}
\else
	\vspace*{\stretch{0.5}}
\fi

\fullwidth{%
Selection coefficients this large are known in natural populations but
are probably uncommon for most loci. What if the selective advantage of
a genotype is very slight, such as 0.01 or 0.02?
}% end fullwidth

\question
Assume \emph{N}\textsubscript{e} = 10,000. The starting
frequency of Haplotype \emph{A} is 0.3. The relative fitness (\emph{w})
of the three genotypes are 1.0, 0.99 and 0.98 for \emph{AA}, \emph{Aa}
and \emph{aa}, respectively. Draw your prediction for the change in
frequency for Haplotype \emph{A} on the graph.

\ifprintanswers
	{\bfseries %
	POP SIZE = 10,000\\
	\# POPS = 10\\
	FREQ = 0.3\\
	GENERATIONS = 1000\\	
	FITNESS = 1, 0.99, and 0.98, top to bottom. }\vspace*{16\baselineskip}
\else
	\begin{center}
		\includegraphics[height=20\baselineskip]{prediction_graph}
	\end{center}
\fi

\question
Explain the reasoning you used to arrive at your
predictions. Compare your prediction to those made by others in your
group. Discuss any differences with your group.
\vspace*{\stretch{1}}
\newpage

\question
Draw the results of the simulation on this graph.
	\begin{center}
		\includegraphics[height=20\baselineskip]{prediction_graph}
	\end{center}

\question
Did the results agree with your predictions? Discuss ideas
with your group, and then explain why you think your results did or did
not agree.
\vspace*{\stretch{1}}

\question[1]
Let's now consider a small population with an effective
size of 100 diploid individuals. In the absence of natural selection,
calculate the probability that a new mutation will become fixed in the
population. Show your work.

\ifprintanswers
	\textbf{1/200 = 0.005} \vspace*{\stretch{1}}
\else
	\vspace*{\stretch{1}}
\fi
\newpage

\question
Assume an effective population size of 100, a starting
frequency of Haplotype \emph{A} of 0.005, and relative fitnesses of 1.0,
0.99 and 0.98 for \emph{AA}, \emph{Aa} and \emph{aa} respectively. On
the graph below, draw your prediction of the frequency change of
Haplotype \emph{A}. Draw 4 different lines (four populations) to
consider different possible outcomes.

\ifprintanswers
	{\bfseries %
	Pop Size = 100\\
	\# {\LARGE POPS = 100}\\
	FREQ = 0.005\\
	Generations = 1000\\	
	Fitness = 1, 0.99, and 0.98, top to bottom. }\vspace*{16\baselineskip}
	
	\textbf{Run 10 times with just drift, then 10 times with selection. Compare results.}
\else
	\begin{center}
		\includegraphics[height=20\baselineskip]{prediction_graph}
	\end{center}
\fi

\question
Describe the results and compare them to your four
predictions.

\newpage

\fullwidth{%
Given our starting assumptions, Haplotype \emph{A} had a 0.005
probability of going to fixation due to genetic drift alone (0.5\%
chance). But, the slight selective advantage of the \emph{AA} genotype
increased the probability of fixation to about 0.02. So, instead of
fixation in about 1 of every 200 populations (on average), the slight
selective advantage led to fixation in about 2 of every 100 populations
(2\%).
}%end fullwidth

\fullwidth{%
Will natural selection always override the effects of genetic drift? No.
Even in the above simulation, genetic drift was still a factor but
selection increased the chances of fixation for the genotype with the
selective advantage. If the effective population size is small enough,
or if the strength of selection is very weak, drift will have a greater
effect than selection. If \emph{s} \textless{}\textless{}
1/(4\emph{N}\textsubscript{e}) then drift will have a greater effect
than selection on genetic variation in the population.
}%end fullwidth

\fullwidth{%
For example, if \emph{N}\textsubscript{e} is 50, then the selection
coefficient must be greater than about 0.005 (= 1/200) or genetic drift
will be the principle cause of haplotype frequency change.
}%end fullwidth


\question[2]
What should be the strength of selection in a population of
\emph{N}\textsubscript{e} = 100 to overcome the effects of genetic
drift? If \emph{N}\textsubscript{e} = 10,000 or 50,000? Show your work.

\begin{parts}
	\part $N_e =$ 100: \ifprintanswers\textbf{0.0025}\fi\vspace*{\stretch{1}}
	
	\part $N_e =$ 10,000: \ifprintanswers\textbf{0.000025}\fi\vspace*{\stretch{1}}

	\part $N_e =$ 50,000: \ifprintanswers\textbf{0.000005}\fi\vspace*{\stretch{1}}
\end{parts}

\fullwidth{%
You must remember that genetic drift is present in populations of all
sizes. When considering both genetic drift and natural selection, drift
is more prevalent in small populations unless selection is strong. As
population size increases, even weak selection can override the effects
of genetic drift because 1/(4\emph{N}\textsubscript{e}) will be very
small.
}%end fullwidth

\fullwidth{%
This relationship between genetic drift and natural selection can have
important consequences in natural populations. For example, a beneficial
mutation in a population of an endangered species (very small
\emph{N}\textsubscript{e}) that increases survivorship in marginal
habitat is more likely to be lost due to genetic drift. In contrast,
even a slight selective advantage for resistance to antibiotics in a
population of disease-causing bacteria may spread rapidly throughout the
population in relatively few generations because the effective
population sizes of bacteria tend to be extremely large.
}

\fullwidth{%
\textbf{Heterozygotes: advantages and disadvantages}

In a few cases, individuals that are heterozygous at a locus have a
selective advantage over individuals that are homozygous at that locus.
That is, \emph{w} = 1.0 for \emph{Aa} and \emph{w} \textless{} 1 for
\emph{AA} and \emph{aa}. This is called \textbf{heterozygote advantage}.
Heterozygote advantage is also called overdominance or heterosis but I
prefer heterozygote advantage because it is self-descriptive.
}%end

\newpage

\question
Assume a population with \emph{N}\textsubscript{e} =
10,000 individuals. Heterozygotes have the selective advantage and the
coefficient of selection against the two homozygous genotypes is 0.05.
(What is \emph{w} for each homozygous genotype?). Draw your predictions
for the frequency change of Haplotype \emph{A}, given a low starting
frequency of 0.05.

\ifprintanswers
	{\bfseries %
	Pop Size = 10,000\\
	\#Pops = 10\\
	FREQ = 0.05\\
	Generations = 1000\\	
	Fitness = 0.95, 1, and 0.95, top to bottom. }\vspace*{16\baselineskip}
\else
	\begin{center}
		\includegraphics[height=20\baselineskip]{prediction_graph}
	\end{center}
\fi

\question
Explain the reasoning you used to arrive at your
prediction. Compare your prediction to those made by others in your
group. Discuss any differences with your group.

\newpage

\question
Draw the results of the simulation on this graph.
	\begin{center}
		\includegraphics[height=20\baselineskip]{prediction_graph}
	\end{center}

\question[2]
Did the results agree with your predictions? Based on these
results, write a general principle to explain the effect that a
heterozygote advantage will have on the frequencies of two haplotypes at
a locus. What do you predict if the selection coefficient differs for
the two homozygotes (say, 0.05 and 0.1).

\ifprintanswers
	{\bfseries %
	Pop Size = 10,000\\
	\#Pops = 10\\
	FREQ = 0.05\\
	Generations = 1000\\	
	Fitness = 0.95, 1, and 0.9, top to bottom. }
\fi

\newpage

\question
Draw the results of the simulations using s = 0.05 for
\emph{AA} and s = 0.1 for \emph{aa} on the graph below. Did they agree
with your general principle?
	\begin{center}
		\includegraphics[height=20\baselineskip]{prediction_graph}
	\end{center}

\question[1]
Describe the effect that the heterozygote advantage has on
genetic variation in a population.
\vspace*{\stretch{1}}

\fullwidth{%
Heterozygotes can also have a selective disadvantage relative to both
homozygotes, called \textbf{heterozygote disadvantage}, or
underdominance. (Guess which term I prefer?)
}%end fullwidth

\newpage

\question
Assume \emph{N}\textsubscript{e} = 10,000 individuals, and
the starting frequency of Haplotype \emph{A} is 0.5. Assume that both
homozygous genotypes have the same fitness (\emph{w} = 1) but the
fitness of the \emph{Aa} genotype is 0.99. What do you think will happen
to the frequency of Haplotype \emph{A} in the population? You should be
able to think of two alternatives.

\ifprintanswers
	{\bfseries %
	Pop Size = 10,000\\
	\#Pops = 10\\
	FREQ = 0.5\\
	Generations = 1000\\	
	Fitness = 1, 0.99, and 1, top to bottom. }\vspace*{16\baselineskip}
\else
	\begin{center}
		\includegraphics[height=20\baselineskip]{prediction_graph}
	\end{center}
\fi

\question
Explain the reasoning you used to arrive at your
predictions. Compare your prediction to those made by others in your
group. Discuss any differences with your group.

\newpage

\question
Draw the results of the simulation.
	\begin{center}
		\includegraphics[height=20\baselineskip]{prediction_graph}
	\end{center}

\question[1]
Describe what has happened to the haplotype frequencies in
the simulated populations. How does this differ, if at all, from your
predictions?

\newpage

\question[1]
What if we assume an equal, low mutation rate of 0.01 from
\emph{A} to \emph{a} and from \emph{a} to \emph{A}. Describe how you
think the mutation rate will affect the outcome of heterozygote
disadvantage, assuming the other conditions are the same. (Hint:
remember what happened in the mutation exercise.)
\vspace*{\stretch{1}}

\question
Describe what actually happened after the simulation is
run.
\vspace*{\stretch{1}}

\question[1]
Mutations that occur in coding DNA such as the exons of a
gene are often detrimental and therefore subject to negative selection.
Assume that haplotype \emph{a} is detrimental such that any individual
that is homozygous \emph{aa} dies before reproducing (\emph{w} = 0.0)
but the effect in heterozygotes is minimal (\emph{s} = 0.1). What should
happen to haplotype \emph{A} in the population? 
\vspace*{\stretch{1}}

\newpage

\question
Now assume that mutations from \emph{A} to \emph{a} occur
at a very low rate (0.01) in a population. On the graph below, draw your
prediction for the frequency of Haplotype \emph{A} in the population, if
Ne = 10,000, starting frequency of Haplotype A is 0.5, and the fitness
is as described in the previous question.

\ifprintanswers
	{\bfseries %
	Pop Size = 10,000\\
	\#Pops = 10\\
	FREQ = 0.5\\
	Generations = 1000\\	
	Fitness = 1, 0.9, and 0, top to bottom. 
	
	Mutation from A to a = 0.01\\
	Mutation from a to A = 0}\vspace*{13\baselineskip}
\else
	\begin{center}
		\includegraphics[height=20\baselineskip]{prediction_graph}
	\end{center}
\fi


\question[1]
Describe the results of the simulation. Did the results
agree with your prediction? Why do you think that Haplotype \emph{A} did
not become fixed in the population?
\vspace*{\stretch{1}}

\question[2]
Finally, discuss within your group how you think these
results might be affected by gene flow among populations. Write a brief
summary of your ideas here.

\ifprintanswers
	\textbf{No change. Pops would still evolve together.}
	\vspace*{\stretch{1}}
\else
	\vspace*{\stretch{1}}
\fi
\end{questions}

\end{document}  