%!TEX TS-program = lualatex
%!TEX encoding = UTF-8 Unicode

\documentclass[11pt, addpoints]{exam}
\usepackage{graphicx}
	\graphicspath{{/Users/goby/Pictures/teach/300/}
	{/Users/goby/Pictures/teach/300/exercises/}
	{img/}} % set of paths to search for images

\usepackage{geometry}
\geometry{letterpaper, bottom=1in}                   
%\geometry{landscape}                % Activate for for rotated page geometry
\usepackage[parfill]{parskip}    % Activate to begin paragraphs with an empty line rather than an indent
\usepackage{amssymb, amsmath}
\usepackage{mathtools}
	\everymath{\displaystyle}

\usepackage{fontspec}
\setmainfont[Ligatures={TeX}, BoldFont={* Bold}, ItalicFont={* Italic}, BoldItalicFont={* BoldItalic}, Numbers={Proportional, OldStyle}]{Linux Libertine O}
\setsansfont[Scale=MatchLowercase,Ligatures=TeX]{Linux Biolinum O}
%\setmonofont[Scale=MatchLowercase]{Inconsolata}
\usepackage{microtype}

\usepackage{unicode-math}
\setmathfont[Scale=MatchLowercase]{Asana Math}

\newfontfamily{\tablenumbers}[Numbers={Monospaced}]{Linux Libertine O}
\newfontfamily{\libertinedisplay}{Linux Libertine Display O}

\usepackage{booktabs}
%\usepackage{tabularx}
%\usepackage{longtable}
%\usepackage{siunitx}

\usepackage{caption}
	\captionsetup{labelsep=period} % Removes colon following figure / table number.

\usepackage{array}
\newcolumntype{L}[1]{>{\raggedright\let\newline\\\arraybackslash\hspace{0pt}}p{#1}}
\newcolumntype{C}[1]{>{\centering\let\newline\\\arraybackslash\hspace{0pt}}p{#1}}
\newcolumntype{R}[1]{>{\raggedleft\let\newline\\\arraybackslash\hspace{0pt}}p{#1}}

\usepackage{enumitem}
\setlist{resume, label=\textsc{\alph*}., leftmargin=*}
\setlist[1]{labelindent=\parindent}

\usepackage[sc]{titlesec}


\renewcommand{\solutiontitle}{\noindent}
\unframedsolutions
\SolutionEmphasis{\bfseries}

\renewcommand{\questionshook}{%
	\setlength{\leftmargin}{-\leftskip}%
}

\pagestyle{headandfoot}
\firstpageheader{\textsc{bi} 300: Evolution}{}{\ifprintanswers\textbf{KEY} \else Name \rule{2.5in}{0.4pt}\fi}
\runningheader{}{}{\footnotesize{pg. \thepage}}
\footer{}{}{}
\runningheadrule

%\printanswers

\begin{document}

\subsection*{Giraffes: one species or many? Part 2. (\numpoints\ points)}

In addition to the phylogenetic analysis of mitochondrial \textsc{dna}, Brown~et~al.~used 
other population genetic analyses to estimate the number of
giraffe species. Here are some of their results, which will help you
further refine your decision.

\begin{enumerate}
\item
  The authors used a measure similar to $F_\mathrm{ST}$,
  called $\Phi_{\mathrm{CT}}$, to estimate the amount of gene flow among
  the six regions (Fig.~2; previous handout). Their calculated $\Phi_{\mathrm{CT}}$ value,
  based on only the mitochondrial control region, was 0.754. We looked
  at $F_\mathrm{ST}$ when we studied migration as part of our
  Hardy-Weinberg exercises. $\Phi_{\mathrm{CT}}$ is interpreted the same
  as $F_\mathrm{ST}$. Values close to zero indicate high
  migration among populations. Values close to one indicate little to no migration.
\item
  The authors also used nuclear \textsc{dna}. Instead of nuclear DNA sequence,
  they used microsatellites. Microsatellites evolve rapidly (very high
  mutation rate) so are very useful in population genetic studies like the
  giraffes. The authors developed a “neighbor-joining network” (Fig.~\ref{fig:njnetwork}), 
  which is different from a phylogenetic tree but can be interpreted
  in a similar way.
\end{enumerate}


\begin{center}
	\includegraphics[width=0.55\textwidth]{fig9_microsatellites}
	\captionof{figure}{Neighbor-joining network, derived from 14 microsatellite loci
sampled from 381 giraffes. The numbers correspond to the patterns shown
in Fig.~2 (from Part 1). Look up locus on your vocabulary sheet if you don't remember
what is a locus.\label{fig:njnetwork}}
\end{center}


\begin{enumerate}
\item
  Crossbreeding between individuals of different patterns or from
  different regions is rare. Of the 381 giraffes sampled for
  microsatellites, only three individuals showed evidence of
  hybridization (0.8\%).
\item
  Some of the greatest observed genetic differences was between the geographically closest
  populations.
\end{enumerate}

\newpage

Answer the following questions, based on your previous ideas and
the new information from this exercise. Upload your typed answers to the dropbox called “Upload Giraffe Part 2
Files Here” by the due date.  Please name the file LastName\_FirstName\_Giraffe2.doc.

\subsection*{Questions}

\begin{questions}

\question[1]
Based on the new information, how many species of giraffes do you now
think? (You do not have to change your mind.)

\question[10]
As you did in Part 1, describe why or how the additional evidence caused you to
revise your estimate of the number of species. If you did not revise
your estimate, explain why you were not convinced by the evidence. Be 5
clear and precise, and use examples from the new evidence and the previous phylogenetic tree.
Treat this answer as you would an essay question on an exam. I am not
looking for a single correct answer because this question has many answers. I
want you to apply your knowledge and proper vocabulary gained from
class. 

\end{questions}



\end{document}  