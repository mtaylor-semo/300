%!TEX TS-program = lualatex
%!TEX encoding = UTF-8 Unicode

\documentclass[t]{beamer}

%%%% HANDOUTS For online Uncomment the following four lines for handout
%\documentclass[t,handout]{beamer}  %Use this for handouts.
%\usepackage{handoutWithNotes}
%\includeonlylecture{student}
%\pgfpagesuselayout{3 on 1 with notes}[letterpaper,border shrink=5mm]
%	\setbeamercolor{background canvas}{bg=black!5}


%%% Including only some slides for students.
%%% Uncomment the following line. For the slides,
%%% use the labels shown below the command.

%% For students, use \lecture{student}{student}
%% For mine, use \lecture{instructor}{instructor}

% FONTS
\usepackage{fontspec}
\def\mainfont{Linux Biolinum O}
\setmainfont[Ligatures={Common, TeX}, BoldFont={* Bold}, ItalicFont={* Italic}, Numbers={Proportional, OldStyle}]{\mainfont}
%\setmonofont[Scale=MatchLowercase]{Inconsolata} 
\setsansfont[Scale=MatchLowercase]{Linux Biolinum O} 
\usepackage{microtype}

\usepackage{graphicx}
	\graphicspath{%
	{/Users/goby/Pictures/teach/300/lectures/}%
	{/Users/goby/Pictures/teach/163/common/}} % set of paths to search for images

\usepackage{amsmath,amssymb}

\usepackage{booktabs}
\usepackage{multicol}
%	\setlength{\columnsep=1em}


%\usepackage{textcomp}
%\usepackage{setspace}
\usepackage{tikz}
	\tikzstyle{every picture}+=[remember picture,overlay]
\usetikzlibrary{positioning}
\usetikzlibrary{calc,shapes.callouts,shapes.arrows}
\newcommand{\bubblethis}[2]{
    \tikz[remember picture,baseline]{\node[anchor=base,inner sep=0,outer sep=0]%
    (#1) {\phantom{#1}};\node[overlay,cloud callout,callout relative pointer={(0.2cm,-0.7cm)},%
    aspect=2.5,fill=yellow!90] at ($(#1.north)+(-0.5cm,1.6cm)$) {#2};}%
}%

\mode<presentation>
{
  \usetheme{Lecture}
  \setbeamercovered{invisible}
  \setbeamertemplate{items}[square]
}

\usepackage{xcolor}
\definecolor{giraffe_blue}{HTML}{5991db}

\usepackage{etoolbox}
\makeatletter
\patchcmd{\beamer@calculateheadfoot}{\advance\footheight by 4pt}{\advance\footheight by 0pt}{}{}
\makeatother

\begin{document}

{
\setbeamercolor{background canvas}{bg=giraffe_blue}
\begin{frame}[t]{\textcolor{white}{\Huge Giraffes: one species or many?}}

\vfilll

\includegraphics[width=\linewidth]{giraffe_splash}	
\end{frame}
}

\begin{frame}[t,plain]
\begin{multicols}{2}

Giraffes have a patchy distribution across western, eastern-central and southern Africa.

\medskip

The distribution may be remnants of a once wider distribution.

\bigskip

\highlight{Giraffes tend to have a distinct coat pattern specific to each region.}

\columnbreak

\includegraphics[width=\linewidth]{giraffe_range}
\end{multicols}

\end{frame}


\begin{frame}[t, plain]{Phenotypic variation or geographic correlation?}
\centering
\includegraphics[width=0.98\linewidth]{giraffe_patterns}
\end{frame}

\begin{frame}[t]

\hfill\includegraphics[height=0.9\textheight]{patterns_geography}

\begin{tikzpicture}
\node (a) at (1, 4) [anchor=west] {\large 35 mt\textsc{dna} haplotypes};

\node (b) [below = of a.west, anchor = west]  {\large 6 regions};

\node (c) [below = of b.west, anchor = west]  {\large 5 coat patterns};

\end{tikzpicture}
	
\end{frame}

\begin{frame}[t]{Analyze the data and develop a hypothesis.}

\hangpara Obtain the \textsc{dna} sequences

\hangpara Align the sequences

\hangpara Perform the phylogenetic analysis

\hangpara Interpret the phylogenetic tree

\hangpara Estimate the number of species based on the tree.

\end{frame}

\begin{frame}[t]{The \textsc{dna} sequences are in \textsc{fasta} format.}

\vspace{-\baselineskip}

\hangpara \texttt{>pattern3\_h3\newline
ATGATCAACATCCGAAAGTCCCACCCACTAATAAAAATCGTAAATAACGC\newline
ACTAATCGATCTACCAGCCCCATCAAATATCTCATCATGATGAAACTTCG\newline
GCTCCCTACTAGGCATCTGCCTCATTTTACAAATTCTAAC}

\hangpara \texttt{>pattern2\_h29\newline
ATGATCAACATCCGAAAGTCCCACCCACTAATAAAAATCGTAAATAACGC\newline
ACTAATCGATCTACCAGCCCCATCAAATATCTCATCATGATGAAACTTCG\newline
GCTCCCTACTAGGCATCTGTCTCATCTTACAAATCCTAAC}

\hangpara \texttt{>pattern5\_h11\newline
ATGATCAACATCCGAAAGTCCCACCCACTAATAAAAATCGTAAATAACGC\newline
ACTAATCGATCTACCAGCCCCATCAAATATCTCATCATGATGAAACTTCG\newline
GCTCCCTACTAGGCATTTGTCTCATTTTACAAATTCTAAC}

\bigskip

etc\dots

\end{frame}

\begin{frame}[t]{The aligned sequences will be in Clustal format.}

\vspace{-\baselineskip}

\hangpara \texttt{pattern3\_h3\ \ \  ATGATCAACATCCGAAAGTCCCACCCACTAATAAAAATCGT\newline
pattern2\_h29\ \  ATGATCAACATCCGAAAGTCCCACCCACTAATAAAAATCGT\newline
pattern5\_h11\ \ ATGATCAACATCCGAAAGTCCCACCCACTAATAAAAATCGT
}

\bigskip

etc\dots

\end{frame}

{
\setbeamercolor{background canvas}{bg=giraffe_blue}
\begin{frame}[t]{\textcolor{white}{Upload your results for Part~1 to the drop box.}}

\vfilll

\includegraphics[width=\linewidth]{giraffe_splash}	
\end{frame}
}

{\setbeamercolor{background canvas}{bg=black}
\begin{frame}
\end{frame}
}

{
\setbeamercolor{background canvas}{bg=giraffe_blue}
\begin{frame}[t]{\textcolor{white}{\Huge Giraffes: one species or many?}}

\vfilll

\includegraphics[width=\linewidth]{giraffe_splash}	
\end{frame}
}

\begin{frame}[t]

\hspace{5em}\includegraphics[height=0.9\textheight]{example_result}

\begin{tikzpicture}
\onslide<2-3>{
\draw [ultra thick] (8,1) -- (8,4.4) node [midway, right] {Species 1};

\draw [ultra thick] (8,4.55) -- (8,6.1) node [midway, right] {Species 2};

\draw [ultra thick] (8,6.35) -- (8,9) node [midway, right] {Species 3};
}

\onslide<3>{
\draw [<-, ultra thick] (7.5, 6.24) -- (8.05, 6.24) node [right] {Species 4?};
}
\end{tikzpicture}

\end{frame}

{
\setbeamercolor{background canvas}{bg=giraffe_blue}
\begin{frame}[t]{\textcolor{white}{\Huge But wait\dots \hfill there's more!}}

\vfilll

\includegraphics[width=\linewidth]{giraffe_splash}	
\end{frame}
}

\begin{frame}[t]

\hfill\includegraphics[height=0.9\textheight]{patterns_geography}

\begin{tikzpicture}
\node (a) at (1, 4) [anchor=west] {\large \highlight{Microsatellite \textsc{dna}}};

\node (b) at (1, 4) [below = of a.west, anchor = west] {\large 35 mt\textsc{dna} haplotypes};

\node (c) [below = of b.west, anchor = west]  {\large 6 regions};

\node (d) [below = of c.west, anchor = west]  {\large 5 coat patterns};

\end{tikzpicture}
	
\end{frame}

\begin{frame}[t]{\onslide*<1>{Microsatellites are segments of repeated nucleotides.}\onslide*<2>{Their microsatellite results are intriguing.}}
\begin{multicols}{2}
\includegraphics[width=\linewidth]{microsat_slippage}

\columnbreak

\onslide*<2>{
\includegraphics[width=\linewidth]{microsat_result}
}

\end{multicols}

\visible<2>{
\begin{tikzpicture}
\filldraw [fill=white,opacity=0.75, draw opacity=0] (0,2) rectangle (6,7.5);
\end{tikzpicture}
}
\vfilll



\tiny Modified from Ellgren, H. Trends in Genetics 16: 551.
\end{frame}

{\setbeamercolor{background canvas}{bg=black}
\begin{frame}
\end{frame}
}

{
\setbeamercolor{background canvas}{bg=giraffe_blue}
\begin{frame}[t]{\textcolor{white}{\Huge Giraffes: one species or many?}}

\vfilll

\includegraphics[width=\linewidth]{giraffe_splash}	
\end{frame}
}

\begin{frame}[t]{What did all y'all hypothesize?}

\vspace{-0.5\baselineskip}
\centering
\includegraphics<1>[width=\linewidth]{num_giraffe_spp1}
\includegraphics<2>[width=\linewidth]{num_giraffe_spp2}

\end{frame}


\begin{frame}[t]
\begin{multicols}{2}

\hangpara Very strong agreement between coat pattern and microsat loci.

\hangpara Little to no evidence of gene flow.

\hangpara Strong support for 5 species, at least.

\hangpara Is there a 6th species? More than 6?

\columnbreak

\noindent\includegraphics[width=\linewidth]{microsat_result}

\end{multicols}

\vfilll

\tinyfill Brown et al.~2007.
\end{frame}

\begin{frame}[t]
\centering
\includegraphics[height=0.88\textheight]{structure1}

\vfilll

\tinyfill Brown et al.~2007.

\end{frame}

\begin{frame}[t]
\centering
\includegraphics[height=0.88\textheight]{structure2}

\vfill

\tinyfill Brown et al.~2007.

\end{frame}

{
\setbeamercolor{background canvas}{bg=giraffe_blue}
\begin{frame}[t]{\textcolor{white}{\Huge But wait\dots \hfill there's more!}}

\vfilll

\includegraphics[width=\linewidth]{giraffe_splash}	
\end{frame}
}

\begin{frame}[t]
\centering
\includegraphics[height=0.88\textheight]{fennessy_result}

\begin{tikzpicture}
\node at (-2,5.5) [align = left] {Fennessy et al.~2016\\ argue four species.};
\end{tikzpicture}

%\vfilll
\vfill

\tinyfill Fennessy et al.~2016

\end{frame}

\begin{frame}
\begin{tikzpicture}
\node [inner sep=0] at (7.85,-2.9) {\includegraphics[width=9cm]{fennessy_range}};

\node [inner sep=0] at (2.2,-6.3) {\includegraphics[width=5cm]{fennessy_species}};

\end{tikzpicture}

\vfill

\tinyfill Fennessy et al.~2016
\end{frame}

{
\setbeamercolor{background canvas}{bg=giraffe_blue}
\begin{frame}[t]{\textcolor{white}{\Huge Does it matter?}}

\vfilll

\includegraphics[width=\linewidth]{giraffe_splash}	
\end{frame}
}

\end{document}
