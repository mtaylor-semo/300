%!TEX TS-program = lualatex
%!TEX encoding = UTF-8 Unicode

%\documentclass[t]{beamer}

%%%% HANDOUTS For online Uncomment the following four lines for handout
\documentclass[t,handout]{beamer}  %Use this for handouts.
\usepackage{handoutWithNotes}
\includeonlylecture{student}
\pgfpagesuselayout{3 on 1 with notes}[letterpaper,border shrink=5mm]

%\usefonttheme{professionalfonts}


%%% Including only some slides for students.
%%% Uncomment the following line. For the slides,
%%% use the labels shown below the command.

%% For students, use \lecture{student}{student}
%% For mine, use \lecture{instructor}{instructor}

% FONTS
\usepackage{fontspec}
\def\mainfont{Linux Biolinum O}
\setmainfont[Ligatures={Common, TeX}, Contextuals={NoAlternate}, BoldFont={* Bold}, ItalicFont={* Italic}, Numbers={Proportional, OldStyle}]{\mainfont}
%\setmonofont[Scale=MatchLowercase]{Inconsolata} 
\setsansfont[Scale=MatchLowercase]{Linux Biolinum O} 
\setmonofont{Linux Libertine Mono O}

\usepackage{microtype}

\parindent=0pt

\usepackage{unicode-math}
\setmathfont[Scale=MatchLowercase]{Asana Math}

\usepackage{graphicx}
	\graphicspath{%
	{/Users/goby/Pictures/teach/300/lectures/}%
	{/Users/goby/Pictures/teach/163/lecture/}} % set of paths to search for images

\usepackage{amsmath,amssymb}

%\usepackage{units}

\usepackage{booktabs}
\usepackage{array}
\newcolumntype{L}[1]{>{\raggedright\let\newline\\\arraybackslash\hspace{0pt}}p{#1}}
\newcolumntype{C}[1]{>{\centering\let\newline\\\arraybackslash\hspace{0pt}}p{#1}}
\newcolumntype{R}[1]{>{\raggedleft\let\newline\\\arraybackslash\hspace{0pt}}p{#1}}


\usepackage{multicol}
%	\setlength{\columnsep=1em}
\usepackage{enumitem}
\usepackage{textcomp}
\usepackage{setspace}
\usepackage{tikz}
	\tikzstyle{every picture}+=[remember picture,overlay]
\usetikzlibrary{arrows}

\mode<presentation>
{
  \usetheme{Lecture}
  \setbeamercovered{invisible}
  \setbeamertemplate{items}[square]
}

\usepackage{calc}
\usepackage{hyperref}

\newcommand\HiddenWord[1]{%
	\alt<handout>{\rule{\widthof{#1}}{\fboxrule}}{#1}%
}


\usepackage{xifthen}
\newcommand{\futuyma}[1]{%
	\ifthenelse{\isempty{#1}}%
	{Futuyma \& Kirkpatrick 2017, 4th ed.}%
	{Fig.~#1~Futuyma \& Kirkpatrick 2017, 4th ed.}%
}

% This defines \amper for the fancy ampersand
% to be used in the header. See
% https://tex.stackexchange.com/a/58185/39194
\usepackage{xspace}
\newfontfamily\amperfont[Style=Alternate]{Linux Libertine O}    
\makeatletter
\DeclareRobustCommand{\amper}{{\amperfont\ifx\f@shape\scname\smaller[1.2]\fi\&}\xspace}
\makeatother

\newcommand{\backskip}{\vspace{-0.5\baselineskip}}

%% Remove indent from multicol.
\parindent=0pt

\begin{document}

\lecture{student}{student}

{
\usebackgroundtemplate{\includegraphics[width=\paperwidth]{evodevo_intro}}
\begin{frame}[b]

%\tinyfill  \textcolor{white}{\futuyma{}}
\end{frame}
}

%%
\begin{frame}{Here are embryos of human, alligator, and mouse. Which is which?}

\backskip

\includegraphics[width=\linewidth]{evodevo_embryos}

\pause

\mode<beamer>{%
\begin{tikzpicture}
\node at (2,0) {Alligator};
\node at (5.35,0) {Human};
\node at (9.4,0) {Mouse};
\end{tikzpicture}
}


\tinyfill \futuyma{15.1}

\end{frame}

%%

\begin{frame}{\highlight{Allometry} is the differential rate of growth of different parts or dimensions during development.}

\backskip

\begin{multicols}{2}

\includegraphics[width=0.9\linewidth]{evodevo_allometry_beetles}
	
\columnbreak

\includegraphics[width=0.9\linewidth]{evodevo_allometry_human}

\end{multicols}

\vfilll

\tiny \futuyma{15.2\textsc{b}} \hfill \textcopyright\,Pearson

\end{frame}

%%

\begin{frame}{\highlight{Heterochrony} is an evolutionary change in the timing or rate of developmental events.}

\backskip

\begin{multicols}{2}
\hangpara \highlight{Paedomorphosis} is the retention of juvenile characteristics in the adult form.

\medskip

\centering

\reflectbox{\includegraphics[width=0.9\linewidth]{evodevo_paedomorphosis}}

\columnbreak

\reflectbox{\includegraphics[width=0.8\linewidth]{evodevo_paedomorphosis_chimp}}

\end{multicols}
	
\vfilll

\tiny\futuyma{15.3\textsc{c}}\hfill Original source lost with the sands of time.
	
\end{frame}

%%

\begin{frame}{Organisms are modular.}
\vspace{-\baselineskip}

\begin{multicols}{2}

\centering

\includegraphics[width=0.85\linewidth]{evodevo_trilobite}

\vspace{0.5ex}

\noindent\includegraphics[width=0.85\linewidth]{evodevo_millipede}

\columnbreak

\noindent\includegraphics[width=0.9\linewidth]{evodevo_fish}

\vspace{0.5ex}

\includegraphics[width=0.9\linewidth]{evodevo_human_hand}

\end{multicols}
	
\end{frame}

%%

\begin{frame}{Modules have distinct genetic specifications, developmental patterns, locations and interactions.}
\vspace{-\baselineskip}

\begin{multicols}{2}

\centering

\includegraphics[width=0.85\linewidth]{evodevo_plotosids}

\vspace{0.5ex}

\noindent\includegraphics[width=0.85\linewidth]{evodevo_zebras}

\columnbreak

\includegraphics[width=0.85\linewidth]{evodevo_butterfly}

\vspace{0.5ex}

\includegraphics[width=0.85\linewidth]{evodevo_peacock_feathers}

\end{multicols}
	
\end{frame}

%%

\begin{frame}{Repeated modules are \highlight{serially homologous.}}

\vspace{-\baselineskip}

\begin{multicols}{2}

\centering

\includegraphics[width=0.85\linewidth]{evodevo_flower}

\vspace{0.5ex}

\noindent\includegraphics[width=0.85\linewidth]{evodevo_trillium}

\columnbreak

\includegraphics[width=0.9\linewidth]{evodevo_seed_pod}

\vspace{0.5ex}

\includegraphics[width=0.9\linewidth]{evodevo_compound_leaf}

\end{multicols}
	
\end{frame}

%%

\begin{frame}{\highlight{Serial homology:} repeating structures with similar developmental origins.}

\vspace{-\baselineskip}

\begin{multicols}{2}

\centering

\includegraphics[width=0.9\linewidth]{evodevo_velvet_worm}

\columnbreak

\vspace*{-2\baselineskip}

\includegraphics[width=0.85\linewidth]{evodevo_lobster_exploded}

\end{multicols}

\vfilll

\tiny\href{https://commons.wikimedia.org/wiki/File:Velvet_worm.jpg}{Velvet Worm, Geoff Gallice, Wikimedia, \ccby{2}}

	
\end{frame}

%%

\begin{frame}{Gene transcription is regulated by other genes.}

\backskip

%\vspace{-\baselineskip}

\centering

\includegraphics[height=0.85\textheight]{futuyma_fig15-7}

\tinyfill \futuyma{15.7}


\end{frame}

%%

\begin{frame}{Let's build a fly.}

\backskip

\begin{multicols}{2}

\includegraphics[width=\linewidth]{futuyma_fig15-8a}

\columnbreak

\includegraphics[width=\linewidth]{futuyma_fig15-8b}

\end{multicols}

\tinyfill \futuyma{15.8}

\end{frame}


%%

\begin{frame}{Initial polarity is determined by \highlight{maternal effect} genes.}

\backskip

%\vspace{-\baselineskip}

\centering

\includegraphics[height=0.85\textheight]{evodevo_fly_egg}

\tinyfill Carroll et al., 2001. \textit{From DNA to Diversity.}


\end{frame}

%%

\begin{frame}{\highlight{Regulatory genes} control expression of other genes.}

\vspace{-\baselineskip}

\begin{multicols}{2}

\includegraphics[width=\linewidth]{evodevo_maternal_effects1}

\columnbreak

\includegraphics[width=\linewidth]{evodevo_maternal_effects2}

\end{multicols}

\hangpara Concentration gradient determines expression.

\tinyfill Carroll et al., 2001. \textit{From DNA to Diversity.}


\end{frame}

%%

\begin{frame}

\begin{multicols}{2}

\centering

\includegraphics[width=0.95\linewidth]{futuyma_fig15-10b}

\includegraphics[width=0.95\linewidth]{futuyma_fig15-10c}

\includegraphics[width=0.95\linewidth]{futuyma_fig15-10d}

\columnbreak

\includegraphics[height=0.88\textheight]{evodevo_hierarchy}
\end{multicols}

\vfilll

\tiny \futuyma{15.10} \hfill Carroll et al., 2001. \textit{From DNA to Diversity.}

\begin{tikzpicture}
\draw [<-, ultra thick] (5.7,6.94) -- (6.4,6.94);

\draw [<-, ultra thick] (5.7,4.93) -- (8,4.93);

\draw [<-, ultra thick] (5.7,1.7) -- (8.1,1);
\end{tikzpicture}

\end{frame}

%%

\begin{frame}{\highlight{Hox genes} determine fate of cells in each segment.}

\backskip

\centering

\includegraphics[width=\linewidth]{futuyma_fig15-10d}

\tinyfill \futuyma{15.10}
\end{frame}

%%

\begin{frame}{Hox \emph{expression} order corresponds to anterior-posterior axis.}

\vspace*{-1.5\baselineskip}

\centering

\includegraphics[height=0.84\textheight]{futuyma_fig15-9}

\tinyfill \futuyma{15.9}
\end{frame}

%%



\begin{frame}{Hox expression in mouse is similar to Drosophila.}

\backskip

\centering

\includegraphics[height=0.84\textheight]{futuyma_fig15-11}

\tinyfill \futuyma{15.11}
\end{frame}

%%

\begin{frame}

\centering

\includegraphics[height=0.93\textheight]{evodevo_hox_phylogeny}

\tinyfill Carroll et al., 2001. \textit{From DNA to Diversity.}
\end{frame}

%%

\begin{frame}{The \highlight{genetic toolkit} is the set of regulatory genes that “builds” the organism.}

\backskip
 
\centering

\includegraphics[height=0.8\textheight]{evodevo_toolkit}

\tinyfill Carroll et al., 2001. \textit{From DNA to Diversity.}
\end{frame}

%%

\begin{frame}{Sonic hedgehog \highlight{(shh)} is a toolkit gene.}

\vspace*{-\baselineskip}

\begin{multicols}{2}

\centering

\includegraphics[width=0.9\linewidth]{evodevo_shh1}

\vspace{0.5ex}

\includegraphics[width=0.9\linewidth]{evodevo_shh2}

\columnbreak

\vspace*{-\baselineskip}

\includegraphics[width=0.5\linewidth]{evodevo_sonic_hedgehog}

\vspace{0.5ex}

\includegraphics[width=0.8\linewidth]{evodevo_shh3}


\end{multicols}

\end{frame}

%%

\begin{frame}{Bone morphogenetic protein \highlight{(bmp)} is a toolkit gene.}

\vspace*{-\baselineskip}

\begin{multicols}{2}

\centering

\includegraphics[width=0.7\linewidth]{evodevo_bat_wing}

\vspace{0.5ex}

\includegraphics[width=\linewidth]{evodevo_bmp}

\vspace{0.5ex}

\includegraphics[width=0.7\linewidth]{evodevo_duck_feet}

\columnbreak

\includegraphics[width=0.85\linewidth]{evodevo_bmp_effects}

\end{multicols}

\vfilll
\tiny Weatherbee 2006, PNAS 103:15103 
\hfill
Sears 2008. Cells Tissues Organs 187:6


\end{frame}

%%

\begin{frame}{Different eye types evolved multiple times.}

\backskip

\centering

\includegraphics[height=0.85\textheight]{evodevo_eye_types}

\end{frame}

%%

\begin{frame}{Eve development regulated by \highlight{\emph{pax6}} in all species studied.}

\backskip

\begin{multicols}{2}

\includegraphics[width=0.9\linewidth]{evodevo_eye_phylogeny}

\columnbreak

\includegraphics[width=0.78\linewidth]{evodevo_misplaced_eyes}\newline
\footnotesize Mouse \emph{pax6} expressed in \textit{Drosophila.}

\end{multicols}


\end{frame}

%%

\begin{frame}{Existing genes can be \highlight{recruited} or \highlight{co-opted} to serve new functions.}

\backskip

\includegraphics[width=\linewidth]{evodevo_crystallin_phylogeny}

\tinyfill Futuyma 2004.

\end{frame}

%%

\begin{frame}

\centering

\includegraphics[height=0.9\textheight]{evodevo_crystallin_table}

\tinyfill Futuyma 2004.

\end{frame}

%%

\begin{frame}{Eye spots in butterfly wings controlled by pigment genes and co-opted toolkit genes.}

%\backskip

\includegraphics[width=\linewidth]{evodevo_toolkit_cooption}

\tinyfill Futuyma 2004.
\end{frame}
%%

{
\usebackgroundtemplate{\includegraphics[width=\paperwidth]{evodevo_columbines}}
\begin{frame}[b]

%\tinyfill  \textcolor{white}{\futuyma{}}
\end{frame}
}
%%

\begin{frame}{Five classes of \textsc{mads} genes regulate development of four flower whorls.}

\vspace{-\baselineskip}

\centering
\includegraphics[width=0.8\linewidth]{evodevo_floral_whorls}

\includegraphics[width=0.36\linewidth]{evodevo_flower_cutaway}

\end{frame}
%%

\begin{frame}{The floral quartet model explains flower development.}

\backskip

\centering

\includegraphics[width=\linewidth]{evodevo_floral_quartet_model}

\tinyfill Theißen et al.~2016. Development 143:3259

\end{frame}
%%

\begin{frame}{M\textsc{ads} genes generate floral diversity in Ranunculaceae, including columbines.}

\backskip

\centering

\includegraphics[width=\linewidth]{evodevo_ranuncs}

\tinyfill Kramer et al.~2003. Intl J Plant Sci 164:1  

\end{frame}

%%
\begin{frame}{Spur size increases with pollinator tongue length in columbine flowers.}
\backskip

\centering

\includegraphics[width=\linewidth]{evodevo_columbine_phylogeny}

\tinyfill Whittall \& Hodges 2007. Nature 447:706  

\end{frame}
%%
\end{document}


