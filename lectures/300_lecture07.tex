%!TEX TS-program = lualatex
%!TEX encoding = UTF-8 Unicode

\documentclass[t]{beamer}

%%%% HANDOUTS For online Uncomment the following four lines for handout
%\documentclass[t,handout]{beamer}  %Use this for handouts.
%\usepackage{handoutWithNotes}
%\includeonlylecture{student}
%\pgfpagesuselayout{3 on 1 with notes}[letterpaper,border shrink=5mm]
%	\setbeamercolor{background canvas}{bg=black!5}
\usefonttheme{professionalfonts}


%%% Including only some slides for students.
%%% Uncomment the following line. For the slides,
%%% use the labels shown below the command.

%% For students, use \lecture{student}{student}
%% For mine, use \lecture{instructor}{instructor}

% FONTS
\usepackage{fontspec}
\def\mainfont{Linux Biolinum O}
\setmainfont[Ligatures={Common, TeX}, Contextuals={NoAlternate}, BoldFont={* Bold}, ItalicFont={* Italic}, Numbers={Proportional, OldStyle}]{\mainfont}
%\setmonofont[Scale=MatchLowercase]{Inconsolata} 
\setsansfont[Scale=MatchLowercase]{Linux Biolinum O} 
\usepackage{microtype}

\usepackage{unicode-math}
\setmathfont[Scale=MatchLowercase]{Asana Math}

\usepackage{graphicx}
	\graphicspath{%
	{/Users/goby/Pictures/teach/300/lectures/}%
	{/Users/goby/Pictures/teach/163/common/}} % set of paths to search for images

\usepackage{amsmath,amssymb}

%\usepackage{units}

\usepackage{booktabs}
\usepackage{multicol}
%	\setlength{\columnsep=1em}

\usepackage{textcomp}
\usepackage{setspace}
\usepackage{tikz}
	\tikzstyle{every picture}+=[remember picture,overlay]

\mode<presentation>
{
  \usetheme{Lecture}
  \setbeamercovered{invisible}
  \setbeamertemplate{items}[square]
}

\usepackage{calc}
\usepackage{hyperref}

\newcommand\HiddenWord[1]{%
	\alt<handout>{\rule{\widthof{#1}}{\fboxrule}}{#1}%
}

\newcommand{\futuyma}[1]{Fig.~#1~Futuyma \& Kirkpatrick 2017, 4th ed.}

% This defines \amper for the fancy ampersand
% to be used in the header. See
% https://tex.stackexchange.com/a/58185/39194
\usepackage{xspace}
\newfontfamily\amperfont[Style=Alternate]{Linux Libertine O}    
\makeatletter
\DeclareRobustCommand{\amper}{{\amperfont\ifx\f@shape\scname\smaller[1.2]\fi\&}\xspace}
\makeatother

\begin{document}

\lecture{student}{student}


{
\usebackgroundtemplate{\includegraphics[width=\paperwidth]{qt_evolution_intro}}
\begin{frame}[b]{\textcolor{white}{Phenotypic evolution \amper quantitative traits}}

\tinyfill{\textcolor{white}{\textit{Donax variabilis}, Debivort, \href{https://en.wikipedia.org/wiki/Phenotype\#/media/File:Coquina_variation3.jpg}{Wikimedia Commons}, \ccbysa{3}.}}
%	\tiny\hfill \textcolor{white}{\textit{Donax variabilis}, Debivort, \href{https://en.wikipedia.org/wiki/Phenotype#/media/File:Coquina_variation3.jpg}{Wikimedia Commons}}
\end{frame}}

\begin{frame}[t,plain]{True or false: eye color is determined by one gene?}

\centering

\includegraphics[height=0.85\textheight]{human_eye_colors}
\end{frame}

%%

\begin{frame}[t,plain]{True or false: height is determined by one gene?}
\vspace{-\baselineskip}
\includegraphics[width=\linewidth]{human_height}
\end{frame}

%%

\begin{frame}[t,plain]{Mendel worked with \highlight{discrete traits.}}
\vspace{-0.5\baselineskip}
\includegraphics[width=\linewidth]{qt_mendel_peas}
\end{frame}

%%

\begin{frame}[t,plain]{\highlight{Quantitative traits} are \highlight{polygenic.}}

\vspace{-0.5\baselineskip}

%\centering

\includegraphics[height=0.87\textheight]{futuyma_fig6-3}

\begin{tikzpicture}
\node at (9.5,7) [text width=4.1cm, align=left] {Quantitative traits show a continuous distribution.};
\end{tikzpicture}

\vfilll
\tinyfill{\futuyma{6.3}}
\end{frame}

%%

{
\usebackgroundtemplate{\includegraphics[width=\paperwidth]{qt_horned_lizard}}
\begin{frame}[b]

\vfilll

\tiny \textcolor{white}{\textit{Phrynosoma mcalli}, Jim Rorabaugh/\textsc{usfws}, \href{https://www.flickr.com/photos/54430347@N04/5162160238}{Flickr}, public domain}

\end{frame}}

%%

{
\usebackgroundtemplate{\includegraphics[width=\paperwidth]{qt_loggerhead_shrike}}
\begin{frame}[b]

\vfilll

\tinyfill{\textcolor{white}{Loggerhead Shrike, Marshall Hedin,  \href{https://www.flickr.com/photos/23660854@N07/23552219433}{Flickr}, \ccbysa{2}}}

\end{frame}}

%%

\begin{frame}[t]{\highlight{Fitness functions} describe selection on quantitative traits.}

\vspace{-0.5\baselineskip}

\centering
\includegraphics[height=0.85\textheight]{qt_fitness_function}

\vfilll

\tinyfill{Brodie et al.~2004. Science 304: 65.}
\end{frame}

%%

\begin{frame}[t]{\highlight{Fitness functions} change the mean or variance of a trait.}

\vspace{-0.5\baselineskip}

\centering
\includegraphics[height=0.9\textheight]{futuyma_fig6-6}

\end{frame}

{
\usebackgroundtemplate{\includegraphics[width=\paperwidth]{qt_directional_selection}}
\begin{frame}[t]

\vfilll

\tiny \textit{Male Long-tailed Widowbird}, Ashley Tubs,  \href{https://www.flickr.com/photos/47745688@N05/14808464640}{Flickr}, \ccby{2} 

\end{frame}}

{
\usebackgroundtemplate{\includegraphics[width=\paperwidth]{qt_stabilizing_selection}}
\begin{frame}[b]

\vfilll

\tiny \futuyma{6.8}

\end{frame}}

%%

{
\usebackgroundtemplate{\includegraphics[width=\paperwidth]{qt_disruptive_selection}}
\begin{frame}[t]

\vfilll

\tiny \futuyma{6.9}

\end{frame}}

%%

\begin{frame}[t]{Selection can favor combinations of traits.}

\vspace{-0.5\baselineskip}

\centering
\includegraphics[height=0.9\textheight]{futuyma_fig6-10}

\end{frame}

%%

\begin{frame}[t]{The \highlight{selection gradient} measures the strength of directional selection on a quantitative trait.}

\vspace{-0.5\baselineskip}

\centering
\includegraphics[height=0.84\textheight]{futuyma_fig6-11}\par

\pause

\begin{tikzpicture}
\node at (-2.1,7.4) [text width=3cm] {\small \highlight{$\beta$ is the slope of the fitness function.}};
\end{tikzpicture}


\end{frame}


%%
%% RECALL
\begin{frame}[t,plain]{Evolution \emph{by} selection requires a correlation between$\dots$ }

\includegraphics[width=\linewidth]{futuyma_fig5-4}

\hangpara (1) a phenotypic trait and fitness, and\\
 (2) phenotypes of parents and offspring.
 
 \vfilll
 
 \tinyfill{\futuyma{5.4}}

\end{frame}

%%

\begin{frame}[t,plain]{How much will a trait mean evolve in one generation?}

\begin{multicols}{2}
\noindent\includegraphics[width=\linewidth]{futuyma_fig6-13b}

\columnbreak

\noindent \includegraphics[width=\linewidth]{futuyma_fig6-13c}

\end{multicols}

\pause

\hangpara \highlight{Breeder's equation: $\Delta \overline{z} = \overline{z}' - \overline{z} = h^2S$ }
 
 \vfilll
 
 \tinyfill{\futuyma{6.13}}

\end{frame}

%%

\begin{frame}[t]{\highlight{Heritability $\left(h^2\right)$} is the proportion of phenotypic variation in a population due to genotypic variation.}

\includegraphics[height=0.8\textheight]{futuyma_fig6-14}


\end{frame}



\begin{frame}[t]{The \highlight{breeder's equation} predicts the phenotypic response to selection \emph{between} generations.}

%\vspace{-2\baselineskip}
\begin{multicols}{2}
\noindent\begin{align*}
\Delta\overline{z} &= h^2S\\
\\
S &= P\beta
\end{align*}

\columnbreak
\noindent\begin{align*}
\color{orange5}\Delta\overline{z} &\color{orange5}= G\beta \\
\\
G &= h^2P \\
\\
h^2 &= G/P
\end{align*}

\end{multicols}

where $P$ is phenotypic variance and $G$ is \highlight{additive genetic variance.}
\end{frame}

%%% EXAMPLE PROBLEM

\begin{frame}[t]{Example problem: What is new mean leg length?}
\vspace{-0.5\baselineskip}
\begin{multicols}{2}

\noindent Mean leg length $= 18.6\,\mathrm{mm}$\\[1ex]
$\beta$ = $-0.13/\mathrm{mm}$ \\[1ex]
$P = 1.4\,\mathrm{mm}^2$ \\[1ex]
$h^2 = 0.37$\\[1ex]

\vspace{\baselineskip}
Recall:

$\beta$ is the fitness function slope.\\[1ex]
$\Delta\overline{z} = G\beta$ is predicted response, and\\[1ex]
$G = h^2P$ is additive genetic variance.\\

\columnbreak

\centering
\noindent\includegraphics[width=0.95\linewidth]{qt_migratory_locust}

\end{multicols}

\vfilll

\tinyfill{Migratory Locust by ChriKo, \href{https://commons.wikimedia.org/wiki/File:Locusta_migratoria_migratorioides_male.jpg}{Wikimedia Commons}, \ccbysa{3}}
\end{frame}


%% ANSWER
\lecture{instructor}{instructor}
\begin{frame}[t]{Example problem: What is new mean leg length?}
\vspace{-0.5\baselineskip}
\begin{multicols}{2}

Mean leg length $= 18.6\,\mathrm{mm}$\\[1ex]
$\beta$ = $-0.13/\mathrm{mm}$ \\[1ex]
$P = 1.4\,\mathrm{mm}^2$ \\[1ex]
$h^2 = 0.37$\\[1ex]

\vspace{\baselineskip}
Recall:

$\beta$ is the fitness function slope.\\[1ex]
$\Delta\overline{z} = G\beta$ is predicted response, and\\[1ex]
$G = h^2P$ is additive genetic variance.\\

\columnbreak

\mode<beamer>{%
$G = 0.37 \times 1.4 = 0.518$\\[1ex]
\pause

$\Delta\overline{z} = 0.518 \times -0.13 = -0.067\,\mathrm{mm} $\\[1ex]

\pause
New mean length:\\
$18.6 - 0.067 = 18.533\,\mathrm{mm.}$
}

\end{multicols}
\end{frame}

\end{document}
