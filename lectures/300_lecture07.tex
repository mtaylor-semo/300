%!TEX TS-program = lualatex
%!TEX encoding = UTF-8 Unicode

\documentclass[t]{beamer}

%%%% HANDOUTS For online Uncomment the following four lines for handout
%\documentclass[t,handout]{beamer}  %Use this for handouts.
%\usepackage{handoutWithNotes}
%\includeonlylecture{student}
%\pgfpagesuselayout{3 on 1 with notes}[letterpaper,border shrink=5mm]
%	\setbeamercolor{background canvas}{bg=black!5}


%%% Including only some slides for students.
%%% Uncomment the following line. For the slides,
%%% use the labels shown below the command.

%% For students, use \lecture{student}{student}
%% For mine, use \lecture{instructor}{instructor}


%\usepackage{pgf,pgfpages}
%\pgfpagesuselayout{4 on 1}[letterpaper,border shrink=5mm]

% FONTS
\usepackage{fontspec}
\def\mainfont{Linux Biolinum O}
\setmainfont[Ligatures={Common, TeX}, Contextuals={NoAlternate}, BoldFont={* Bold}, ItalicFont={* Italic}, Numbers={Proportional, OldStyle}]{\mainfont}
%\setmonofont[Scale=MatchLowercase]{Inconsolata} 
\setsansfont[Scale=MatchLowercase]{Linux Biolinum O} 
\usepackage{microtype}

\usepackage{graphicx}
	\graphicspath{%
	{/Users/goby/Pictures/teach/300/lectures/}%
	{/Users/goby/Pictures/teach/163/common/}} % set of paths to search for images

\usepackage{amsmath,amssymb}

%\usepackage{units}

\usepackage{booktabs}
\usepackage{multicol}
%	\setlength{\columnsep=1em}

\usepackage{textcomp}
\usepackage{setspace}
\usepackage{tikz}
	\tikzstyle{every picture}+=[remember picture,overlay]

\mode<presentation>
{
  \usetheme{Lecture}
  \setbeamercovered{invisible}
  \setbeamertemplate{items}[square]
}

\usepackage{calc}
\usepackage{hyperref}

\newcommand\HiddenWord[1]{%
	\alt<handout>{\rule{\widthof{#1}}{\fboxrule}}{#1}%
}



\begin{document}

\lecture{student}{student}


{
\usebackgroundtemplate{\includegraphics[width=\paperwidth]{mutation_variation_intro}}
\begin{frame}[b]

	\tiny\hfill \textcopyright\,Futuyma \& Kirkpatrick 2017, 4th ed ed.
\end{frame}
}
%

\begin{frame}[t,plain]{Genetic variation is created by \highlight{segregation} and \highlight{recombination.}}

\begin{multicols}{2}

\hangpara Segregation: a gamete gets only one of two alleles at a locus.

\hangpara Recombination: creates new, random combinations of alleles across loci.

\columnbreak

\includegraphics[width=0.9\linewidth]{futuyma_fig4-7}

\end{multicols}

\vfilll

\tiny Shout out to Gregor Mendel!

\end{frame}


{
\usebackgroundtemplate{\includegraphics[width=\paperwidth]{/Users/goby/pictures/teach/163/lecture/mutation_types}}
\begin{frame}[b]{Mutations are the ultimate source of genetic variation.}
\end{frame}
}
%

\begin{frame}[t,plain]{Point mutations in an exon can be \highlight{synonymous} or \highlight{non-synonymous.}}

\begin{tabular}[t]{ll}
\includegraphics[width=0.48\linewidth]{/Users/goby/pictures/teach/300/review/genetic_code} & \includegraphics[width=0.48\linewidth]{mutation_point_nonsynonymous}
\end{tabular}

\bigskip

Non-synonymous mutations are subject to natural selection!


\vfilll

\hfill Figs.~4.2 \& 4.3 \textcopyright\,Futuyma \& Fitzpatrick 2017.
\end{frame}

%%

\begin{frame}[t,plain]{Indels in an exon can cause \highlight{frameshift} mutations.}

\centering

\begin{tabular}{@{}cccccc@{}}

\multicolumn{5}{l}{Original sequence} \\[1ex]
ATG & CCC & GAT & ATA & AAA & $\dots$ \\
Met & Pro & Asp & Ile & Lys & $\dots$ \\[0.5em]

\midrule \addlinespace[0.5em]

\multicolumn{5}{l}{Deletion (first C)} \\[1ex]
ATG & CCG & ATA & TAA & AAx & $\dots$ \\
Met & Pro & Asp & STOP & & \\ [0.5em]

\midrule \addlinespace[0.5em]

\multicolumn{5}{l}{Insertion (T before first C)} \\[1ex]
ATG & TCC & CGA & TAT & AAA & $\dots$ \\
Met & Ser & Arg & Tyr & Lys & $\dots$ \\ [1em]

\end{tabular}\par

\end{frame}


%%
\begin{frame}[t,plain]{A new duplicate gene has one of four fates.}
\includegraphics[width=\linewidth]{mutation_duplication_fates}

\vfilll

\tiny\hfill Rensing 2014, Curr Opin Plant Biol 17:43.

\end{frame}

\begin{frame}[t]{Low mutation rates vary among organismal groups.}
\centering
\includegraphics[height=0.85\textheight]{futuyma_fig4-15}
\end{frame}

\begin{frame}[t]{Large genomes still have lots of mutations every generation.}

\hangpara Human mutation rate ($\mu$) is $\approx 10^{-8}$ per base pair (bp).

\hangpara Human genome size is $\approx 3 \times 10^{9}.$

\hangpara $\left(10^{-8}\right)\left(3 \times 10^9\right) = 30$ new mutations in \emph{every} gamete.

\hangpara Not all mutations can be inherited.

%% Most animals have germ line cells that produce gametes. However,
%% plants and some animals do not have germ lines. (See page 94
%% of Futuyma and Kirkpatrick, 4th ed.

\end{frame}

\begin{frame}[t]{\highlight{Allele frequency} is a measure of how common an allele is for a locus.}

\vspace{-\baselineskip}
\centering

\includegraphics[height=0.85\textheight]{futuyma_fig4-5}

\begin{tikzpicture}
\node at (2,7.5) [font=\large, text width = 5.5cm, align=left] {$A_1$ frequency = 10/16 = 0.675.};

\node at (2, 6.5) [font=\large, text width = 5.5cm, align=left] {$A_2$ frequency = 6/16 = 0.325.};

\end{tikzpicture}

\end{frame}

%

\begin{frame}[t]{\highlight{Genotype frequency} is a measure of how common a genotype is for one or more loci.}

\vspace{-\baselineskip}
\centering

\includegraphics[height=0.85\textheight]{futuyma_fig4-5}

\begin{tikzpicture}
\node at (2,7.5) [font=\large, text width = 5.5cm, align=left] {$A_1A_1$ frequency = 4/8 = 0.5.};

\node at (2, 6.75) [font=\large, text width = 5.5cm, align=left] {$A_1A_2$ frequency = 2/8 = 0.25.};

\node at (2, 6) [font=\large, text width = 5.5cm, align=left] {$A_2A_2$ frequency = 2/8 = 0.25.};

\end{tikzpicture}

\end{frame}


\begin{frame}[t]{Consider a population where$\dots$}
\hangpara Half the individuals are $A_1A_1$

\hangpara Half the individuals are $A_2A_2$

\hangpara What is the frequency of each allele?

\hangpara What is the frequency of each genotype?

\end{frame}

%%

\begin{frame}[t]{If a very large population mates at random, what will be$\dots$}

\hangpara the allele frequencies in the next generation?

\hangpara the genotype frequencies in the next generation?

\hangpara Will subsequent generations maintain the same frequencies?

\vfilll

\hangpara \textcolor{gray}{Starting freqs in first generation were 0.5 each.}
\end{frame}

\begin{frame}[t]{A population will reach or maintain Hardy-Weinberg equilibrium if}

\hangpara the population is infinitely large,

\hangpara mating is random,

\hangpara mutations do not occur,

\hangpara gene flow does not occur, and

\hangpara natural selection does not occur.

\hangpara \highlight {A population will evolve if \emph{any} one of these assumptions is violated.}

\vfilll

\textcolor{gray}{Equilibrium will be reached in one generation.}

\end{frame}

%%

\begin{frame}[t]

\hangpara If $p$ and $q$ are the frequencies of two alleles at one locus, then

\hangpara $p^2$ and $q^2$ are the frequencies of each possible homozygote, and 

\hangpara $2pq$ is the frequency of heterozygotes.

\hangpara \highlight{$p + q = 1$} and \highlight{$p^2 + 2pq + q^2 = 1.$}

\hangpara \textcolor{gray}{Recall that $(p + q)^2 = 1^2$}

\end{frame}

%%

\begin{frame}[t]{Hardy-Weinberg equations can be extended to more than two alleles.}

\hangpara If $p + q + r = 1,$ then because

\hangpara $(p + q + r)^2 = 1^2,$ so

\hangpara $p^2 + 2pq + q^2 + 2pr + 2qr + r^2 = 1$


\end{frame}


\begin{frame}[t]{Assume a sample of 1000 individuals in \textsc{hwe}.}

\hangpara $q^2 = 200$ and $r^2 = 0.05.$

\hangpara Calculate the three allele frequencies and the other four genotype frequencies.
\end{frame}


\lecture{instructor}{instructor}

\begin{frame}[t]{Solution: If $q^2 = 0.2$ and $r^2 = 0.05$ then $\dots$}

\vspace{-0.5\baselineskip}

\hangpara $q = \sqrt{0.2} = 0.447$ and $r = \sqrt{0.05} = 0.224.$

\pause \hangpara $p = 1 - 0.447 - 0.224 = 0.329.$

\pause \hangpara $p^2 = 0.329^2 = 0.108.$

\pause \hangpara $2pq = (2)(0.329)(0.447) = 0.294.$

\pause \hangpara $2pr = (2)(0.329)(0.224) = 0.147.$

\pause \hangpara $2qr = (2)(0.447)(0.224) = 0.200.$

\pause \hangpara $0.108 + 0.200 + 0.050 + 0.294 + 0.147 + 0.200 = 1.$ (with rounding error)

\end{frame}




\end{document}
