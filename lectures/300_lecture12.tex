%!TEX TS-program = lualatex
%!TEX encoding = UTF-8 Unicode

\documentclass[t]{beamer}

%%%% HANDOUTS For online Uncomment the following four lines for handout
%\documentclass[t,handout]{beamer}  %Use this for handouts.
%\usepackage{handoutWithNotes}
%\includeonlylecture{student}
%\pgfpagesuselayout{3 on 1 with notes}[letterpaper,border shrink=5mm]

\usefonttheme{professionalfonts}


%%% Including only some slides for students.
%%% Uncomment the following line. For the slides,
%%% use the labels shown below the command.

%% For students, use \lecture{student}{student}
%% For mine, use \lecture{instructor}{instructor}

% FONTS
\usepackage{fontspec}
\def\mainfont{Linux Biolinum O}
\setmainfont[Ligatures={Common, TeX}, Contextuals={NoAlternate}, BoldFont={* Bold}, ItalicFont={* Italic}, Numbers={Proportional, OldStyle}]{\mainfont}
%\setmonofont[Scale=MatchLowercase]{Inconsolata} 
\setsansfont[Scale=MatchLowercase]{Linux Biolinum O} 
\setmonofont{Linux Libertine Mono O}

\usepackage{microtype}

\usepackage{unicode-math}
\setmathfont[Scale=MatchLowercase]{Asana Math}

\usepackage{graphicx}
	\graphicspath{%
	{/Users/goby/Pictures/teach/300/lectures/}%
	{/Users/goby/Pictures/teach/163/common/}} % set of paths to search for images

\usepackage{amsmath,amssymb}

%\usepackage{units}

\usepackage{booktabs}
\usepackage{array}
\newcolumntype{L}[1]{>{\raggedright\let\newline\\\arraybackslash\hspace{0pt}}p{#1}}
\newcolumntype{C}[1]{>{\centering\let\newline\\\arraybackslash\hspace{0pt}}p{#1}}
\newcolumntype{R}[1]{>{\raggedleft\let\newline\\\arraybackslash\hspace{0pt}}p{#1}}


\usepackage{multicol}
%	\setlength{\columnsep=1em}
\usepackage{enumitem}
\usepackage{textcomp}
\usepackage{setspace}
\usepackage{tikz}
	\tikzstyle{every picture}+=[remember picture,overlay]
\usetikzlibrary{arrows}

\mode<presentation>
{
  \usetheme{Lecture}
  \setbeamercovered{invisible}
  \setbeamertemplate{items}[square]
}

\usepackage{calc}
\usepackage{hyperref}

\newcommand\HiddenWord[1]{%
	\alt<handout>{\rule{\widthof{#1}}{\fboxrule}}{#1}%
}


\usepackage{xifthen}
\newcommand{\futuyma}[1]{%
	\ifthenelse{\isempty{#1}}%
	{Futuyma \& Kirkpatrick 2017, 4th ed.}%
	{Fig.~#1~Futuyma \& Kirkpatrick 2017, 4th ed.}%
}

% This defines \amper for the fancy ampersand
% to be used in the header. See
% https://tex.stackexchange.com/a/58185/39194
\usepackage{xspace}
\newfontfamily\amperfont[Style=Alternate]{Linux Libertine O}    
\makeatletter
\DeclareRobustCommand{\amper}{{\amperfont\ifx\f@shape\scname\smaller[1.2]\fi\&}\xspace}
\makeatother

\newcommand{\backskip}{\vspace{-0.5\baselineskip}}

\begin{document}

\lecture{student}{student}

{
\usebackgroundtemplate{\includegraphics[width=\paperwidth]{geneflow_intro}}
\begin{frame}[b]

\tiny \textcolor{white}{\futuyma{}}
\end{frame}
}


\begin{frame}{\highlight{Gene flow} is the movement of alleles among populations.}

\vspace{-\baselineskip}

\begin{multicols}{2}
\centering
\noindent\includegraphics[width=0.945\linewidth]{geneflow_pollen_release}

\vspace{1ex}

\noindent\includegraphics[width=0.945\linewidth]{geneflow_pollinator}

\columnbreak

\noindent\includegraphics[width=0.92\linewidth]{geneflow_dandelion}

\vspace{1ex}

\noindent\reflectbox{\includegraphics[width=0.92\linewidth]{geneflow_bird_berry}}

\end{multicols}

\vfilll

\tiny Buzzfeed \hfill \href{https://www.flickr.com/photos/22786627@N04/3584415133}{Erlend Schei, \ccby{2}}\newline
\href{https://nature.mdc.mo.gov/discover-nature/field-guide/megachilid-bees}{\textcopyright\,Tim Rather, \textsc{mdc}} \hfill \href{https://www.flickr.com/photos/yuri_timofeyev/3140287109}{Yuri Timofeyev, \ccbync{2}}
%\begin{tabular}{@{}cc@{}}
%\includegraphics[width=0.45\]

\end{frame}

%%


\begin{frame}[t]{Why could these populations drift independently?}
\centering

\includegraphics[width=0.97\linewidth]{futuyma_fig7-4}


\end{frame}

%%

\begin{frame}{This species has little gene flow between ridges.}
\backskip

\centering
\includegraphics[height=0.85\textheight]{geneflow_populations_isolated}


\vfilll

\tinyfill Millar et al.~2013. Heredity 111: 437.

\end{frame}

%%

\begin{frame}{\highlight{Clines} show gradual change of genetic variation.}
\backskip

\centering

\includegraphics[height=0.85\textheight]{futuyma_fig8-1}

\vfilll\tinyfill \futuyma{8.1}
\end{frame}

%%

\begin{frame}{Bergman's Rule is an example of a cline.}
\backskip

\centering

\includegraphics[height=0.85\textheight]{futuyma_fig8-2}

\tinyfill \futuyma{8.2}

\end{frame}

%%

\begin{frame}[c]{Migration rate can be used to estimate gene flow.}

\begin{tikzpicture}[->,>=stealth',auto,node distance=4cm,
  thick,main node/.style={circle,draw,align=center, text width=1.2cm}]
  
\node at (1.5,0) [main node] (1) {A {$p=0.7$}};
\node[main node] (2) [right of=1] {B $p = 0.4$};

\draw [->] (1)  to [out=30,in=150] node [midway, above] {0.2} (2) ;

\onslide<1>{\node [right of=2, text width=3cm] {How will $p$ change after migration?}; }

\onslide<2->{\node [right of=2, text width=3cm] {What is new $p$ for B after migration?};}

\end{tikzpicture}

\vspace{1.5\baselineskip}
\onslide<2->{%
{\Large
\begin{align*}
\Delta p &= m(p_m - p)\\
\Delta p &= 0.2(0.7 - 0.4) = 0.06
\end{align*}}
}
\end{frame}

%%
\begin{frame}{\highlight{Fixation index $\left(F_{ST}\right)$} compares heterozygosity within populations to heterozygosity of all populations.}

\backskip

\begin{multicols}{2}

\hangpara $F_{ST}$ measures \emph{difference} between probability of sampling the same allele \emph{within} a population and sampling the same allele from the species as a whole.


%\onslide<2>{%
%\vspace*{\baselineskip}
%$p_1 = 0.8$

%$p_2 = 0.2$

%$p = 0.5$ (overall)
%}

\columnbreak

\centering
\noindent\includegraphics[height=0.68\textheight]{futuyma_fig8-6}

\end{multicols}

%\onslide<2>{
%\begin{tikzpicture}
%\draw (6,2.8) rectangle (12.1,4.9);
%\end{tikzpicture}}


\tinyfill \futuyma{8.6}

\end{frame}

%%

%% This math from Gillespie tends to emphasize
%% homozygosity even though I emphasize heterozygosity.
%% Save for future perhaps but don't use as is.
%%
%\begin{frame}
%\begin{multicols}{2}
%\raggedbottom
%%\begin{align*}
%\hangpara $F_{ST} = \dfrac{G_S - G_T}{1-G_T}$
%%\end{align*}
%
%\hangpara where
%\begin{align*}
%G_T &= p^2 + q^2,\\
%G_S &= \sum c_i(p^2_i + q^2_i)\\
%c_i &=\,\mathrm{weight\ of\ population}\ i.
%\end{align*}
%
%\hangpara Sample size was equal so each population contributes (weights) equally to the calculations. So,
%
%\hangpara $c_1 = c_2 = 0.5.$
%
%\columnbreak
%\noindent\begin{align*}
%G_T &= 0.5^2 + 0.5^2 = 0.5\\
%& \\
%%& \text{Pop.~1, with $p_1 = 0.8$ and $q_1 = 0.2$,}\\
%G_{S1} &= 0.5(0.8^2 + 02^2)\\
%G_{S1} &= 0.5(0.68)\\
%G_{S1} &= 0.34\\[1em]
%G_{S2} &= 0.34\\[1em]
%G_S &= 0.34 + 0.34 = 0.68\\[1em]
%%
%F_{ST} &= \dfrac{0.68 - 0.5}{1 - 0.5} = 0.36
%\end{align*}
%
%\end{multicols}
%\end{frame}

\begin{frame}{Sidebar: use weighted means for unequal sample sizes.}

\vspace{-\baselineskip}
\begin{multicols}{2}
\raggedbottom

\onslide<1->{%
\qquad\begin{tabular}{@{}rcc@{}}
Overall & $\overline{Y}=$& 3.5 \tabularnewline
\midrule
 & 1 & 2 \tabularnewline 
 & 2 & 3 \tabularnewline 
 & 3 & 4 \tabularnewline 
 & 4 & 5 \tabularnewline 
 & 5 & 6 \tabularnewline 
\midrule
$\overline{Y}_i$ & 3 & 4 \tabularnewline
%$w_i$ & 0.5 & 0.5 \tabularnewline
%$w_i\overline{Y_i}$ & 3	& 1.4
\end{tabular}

\hangpara Average of $\{3,4\} = 3.5.$
}% End slide 1

\onslide<3->{%
\quad\begin{tabular}{@{}rcc@{}}
\phantom{Overall} & & \tabularnewline
$c_i$	& 0.5 & 0.5 \tabularnewline
$c_i\overline{Y}_i$ & 1.5 & 2 \tabularnewline
\end{tabular}

\hangpara Sum of $\{1.5,2\} = 3.5$
}

\onslide<1->{
\par\columnbreak
}

\onslide<2->{%
\quad\begin{tabular}{@{}rcc@{}}
Overall & $\overline{Y}=$& 3.1\tabularnewline
\midrule
 & 1 & 1 \tabularnewline 
 & 2 & 2 \tabularnewline 
 & 3 & 3 \tabularnewline 
 & 4 & 4 \tabularnewline 
 & 5 &  \tabularnewline 
 & 6 &  \tabularnewline 
\midrule
$\overline{Y_i}$ & 3.5 & 2.5 \tabularnewline
$c_i$ & 0.6 & 0.4 \tabularnewline
$c_i\overline{Y}_i$ & 2.1	& 1.0
\end{tabular}

\hangpara Average of $\{3.5,2.5\} = 3.0 \ne 3.1$

\hangpara Sum of $\{2.1,1.0\} = 3.1$
} % End slide 2

\end{multicols}

\end{frame}

%% Quick reminder
\lecture{instructor}{instrutor}

%%
\begin{frame}{\highlight{Fixation index $\left(F_{ST}\right)$} compares heterozygosity within populations to heterozygosity of all populations.}

\backskip

\begin{multicols}{2}


\hangpara $F_{ST}$ measures \emph{difference} between probability of sampling the same allele \emph{within} a population and sampling the same allele from the species as a whole.

\vspace*{\baselineskip}
$p_1 = 0.8$ \\ 
$p_2 = 0.2$ \\
$p = 0.5$ (overall)

\onslide<2>{%
\hangpara What is $2pq$ for population~1 and overall?
}

\columnbreak

\centering
\noindent\includegraphics[height=0.68\textheight]{futuyma_fig8-6}

\end{multicols}

\begin{tikzpicture}
\draw (6,2.8) rectangle (12.1,4.9);
\end{tikzpicture}

\tinyfill \futuyma{8.6}
\end{frame}

\lecture{student}{student}

%% Emphasize heterozygosity

\begin{frame}
\begin{multicols}{2}
\raggedbottom
%\begin{align*}
\hangpara $F_{ST} = \dfrac{2pq - \sum c_i2p_iq_i}{2pq}$
%\end{align*}

\hangpara where

\hangpara $2pq =$ overall heterozygosity across populations.

\hangpara $c_i2p_iq_i =$ heterozygosity of each population, weighted for sample size.

%\begin{align*}
%G_T &= p^2 + q^2,\\
%G_S &= \sum c_i(p^2_i + q^2_i)\\
%c_i &=\,\mathrm{weight\ of\ population}\ i.
%\end{align*}

\hangpara Sample size was equal so each population has equal weight,

\hangpara $c_1 = c_2 = 0.5.$

\columnbreak
\noindent\begin{align*}
2pq &= 0.5\ \text{overall}\\ 
& \\
%& \text{Pop.~1, with $p_1 = 0.8$ and $q_1 = 0.2$,}\\
c_12p_1q_1 &= 0.5\left(2\cdot0.8\cdot0.2\right)\\
c_12p_1q_1 &= 0.5(0.32)\\
c_12p_1q_1 &= 0.16\\[1em]
c_22p_2q_2 &= 0.16\\[1em]
\sum c_i2p_iq_i &= 0.16 + 0.16 = 0.32\\[1em]
%
F_{ST} &= \dfrac{0.5 - 0.32}{0.5} = 0.36
\end{align*}

\end{multicols}
\end{frame}

%%

\begin{frame}
\begin{multicols}{2}
\raggedbottom
\begin{tabular}{@{}lrrr@{}}
Pop	& $N$	& $p$ & $q$ \tabularnewline
A & 500	& 0.5 & 0.5 \tabularnewline
B & 100 & 0.65 & 0.35 \tabularnewline
C & 1000 & 0.35 & 0.65 \tabularnewline
\end{tabular}
\vskip-1.5em
\noindent\begin{flalign*}
\onslide<2->{%
\intertext{For weights, divide each $N$ by total.}
c_\mathrm{\textsc{a}} &= 0.3125, \\
c_\mathrm{\textsc{b}} &= 0.0625, \\
c_\mathrm{\textsc{c}} &= 0.625} \\[-1.5em]
\onslide<3->{%
\intertext{Calculate mean $p$ and $q$ overall.}
p &= 0.31(0.5) + 0.06(0.65) + 0.62(0.35) \\
  &= 0.415625\ \text{overall}\\[1ex]
q &= 1 - 0.415625 \\
  &= 0.584375\ \text{overall}}
\end{flalign*}

\columnbreak
\noindent\begin{align*}
\onslide<4->{%
2pq &= 2\cdot0.416\cdot0.584 \\
    &= 0.48576172\\[1em]}
\onslide<5->{%
c_\mathrm{\textsc{a}}2p_\mathrm{\textsc{a}}q_\mathrm{\textsc{a}} &= 0.3125\cdot2\cdot0.5\cdot0.5\\
&=0.15625\\[1ex]
c_\mathrm{\textsc{b}}2p_\mathrm{\textsc{b}}q_\mathrm{\textsc{b}} &= 0.0284375\\[1ex]
c_\mathrm{\textsc{c}}2p_\mathrm{\textsc{c}}q_\mathrm{\textsc{c}} &= 0.284375\\[1em]
\sum c_i2p_iq_i &= 0.4690625 \\[1em]}
\onslide<6->{%
F_{ST} &= \dfrac{0.48576172-4690625}{0.48576172}\\
       &= 0.0344}
\end{align*}

\end{multicols}
\end{frame}

%%

\begin{frame}{\highlight{Wahlund effect} is the reduction of heterozygosity due to population substructure (or inbreeding).}

\vspace{-\baselineskip}

\hangpara Suppose I sample 100 individuals from Bollinger and 100 individuals from Cape Girardeau populations in \textsc{hwe}.

\bigskip

\centering
\begin{tabular}{@{}rccc@{}}
\toprule
 & Bollinger & Cape & BollCape \tabularnewline
\midrule
$AA$ & 49 & 16 & 65 \tabularnewline
$Aa$ & 42 & 48 & 90 \tabularnewline
$aa$ & 9 & 36 & 45 \tabularnewline
\midrule 
\onslide<2->{%
$p$ & 0.7 & 0.4 & 0.55 \tabularnewline
$q$ & 0.3 & 0.6 & 0.45 \tabularnewline
$2pq$ & 0.42 & 0.48 & 0.495 \tabularnewline
\midrule }
\onslide<3->{%
Expected & 42 & 48 & 99 \tabularnewline
\bottomrule}
\end{tabular}

\end{frame}

\begin{frame}{\highlight{Isolation by distance} describes how $F_{ST}$ increases as geographic distance increases.}

\begin{tikzpicture}[->,>=stealth',auto,node distance=2cm,
  thick,main node/.style={circle,draw,align=center}]
%  
\node at (2,-1) [main node] (1) {A};
\node[main node] (2) [right of=1] {B};
\node[main node] (3) [right of=2] {C};
\node[main node] (4) [right of=3] {D};
\node[main node] (5) [right of=4] {E};
%%
\draw [<->] (1) to [out=30,in=150]  node [midway, above] {$m$} (2);
\draw [<->] (2) to [out=30,in=150]  node [midway, above] {$m$} (3) ;
\draw [<->] (3)  to [out=30,in=150]  node [midway, above] {$m$} (4) ;
\draw [<->] (4)  to [out=30,in=150]  node [midway, above] {$m$} (5) ;
%%
\draw [<->] (1)  to [out=330,in=210]  node [midway, below] {$m$} (5) ;
%

\node at (5,-4) [text width=8cm] {Migration rate $m$ decreases with distance.\\For example, $m_{\mathrm{AE}} \ll m_{\mathrm{AB}}$ or $m_{\mathrm{AC}}$};
\end{tikzpicture}

\end{frame}

%%

\begin{frame}{Human populations show evidence of isolation by distance.}
\centering
\includegraphics[width=\linewidth]{futuyma_fig8-7}

\tinyfill \futuyma{8.7}

\end{frame}

%%

\begin{frame}{$F_{ST}$ can be estimated from $N_e$ and migration rate $m$.}

\begin{equation*}
F_{ST} = \dfrac{1}{1+4N_em}
\end{equation*}

\pause
\hangpara List two ways that a \emph{new} allele can be introduced into a population.

\pause
\hangpara Therefore, $F_{ST}$ can also be estimated with mutation rate $\mu$.

\begin{equation*}
F_{ST} = \dfrac{1}{1+4N_e\mu}
\end{equation*}



\end{frame}

%%

\begin{frame}{Genomic $F_{ST}$ can identify chromosomal regions underyling population differences.}

\backskip
\centering
\includegraphics[height=0.80\textheight]{futuyma_fig8-8}

\tinyfill \futuyma{8.8}

\end{frame}

%%

\begin{frame}{Can gene flow offset the effects of selection?}

\hangpara If $m \gg s$ then \highlight{gene swamping} can occur. Local adaptation is overwhelmed by gene flow.

\hangpara If $m \ll s$ then local adaptation can occur.

\end{frame}

%%

\begin{frame}{Strong selection will overcome migration when $s \gg m$}

\backskip
\centering
\includegraphics[height=0.8\textheight]{futuyma_fig8-10}

\begin{tikzpicture}
\node at (4,7) [align=left, text width=2cm] {$s \gg 0.1 $\\$m \ll 0.01 $};
%\node at (4,6.5) [align=left, text width=1.5cm] {};
\end{tikzpicture}

\tinyfill \futuyma{8.10}
\end{frame}

%%

\begin{frame}[t]{Underdominance can form \highlight{tension zones.}}

\centering
\includegraphics[width=0.97\linewidth]{geneflow_bombina_tension_zone}

{\tinyfill \textcopyright\,Pearson Education, Inc.}

\begin{tikzpicture}
\node at (4,4.7) (1) {20\%};
\node [right of=1] (2) {2\%};
\node [above right of=1, yshift=-9pt,xshift=-8pt] {Mortality};
\end{tikzpicture}


\end{frame}


%%



\end{document}
