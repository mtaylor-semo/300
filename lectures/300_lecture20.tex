%!TEX TS-program = lualatex
%!TEX encoding = UTF-8 Unicode

%\documentclass[t]{beamer}

%%%% HANDOUTS For online Uncomment the following four lines for handout
\documentclass[t,handout]{beamer}  %Use this for handouts.
\usepackage{handoutWithNotes}
\includeonlylecture{student}
\pgfpagesuselayout{3 on 1 with notes}[letterpaper,border shrink=5mm]

\usefonttheme{professionalfonts}


%%% Including only some slides for students.
%%% Uncomment the following line. For the slides,
%%% use the labels shown below the command.

%% For students, use \lecture{student}{student}
%% For mine, use \lecture{instructor}{instructor}

% FONTS
\usepackage{fontspec}
\def\mainfont{Linux Biolinum O}
\setmainfont[Ligatures={Common, TeX}, Contextuals={NoAlternate}, BoldFont={* Bold}, ItalicFont={* Italic}, Numbers={Proportional, OldStyle}]{\mainfont}
%\setmonofont[Scale=MatchLowercase]{Inconsolata} 
\setsansfont[Scale=MatchLowercase]{Linux Biolinum O} 
\setmonofont{Linux Libertine Mono O}

\usepackage{microtype}

\usepackage{unicode-math}
\setmathfont[Scale=MatchLowercase]{Asana Math}

\usepackage{graphicx}
	\graphicspath{%
	{/Users/goby/Pictures/teach/300/lectures/}%
	{/Users/goby/Pictures/teach/163/lecture/}} % set of paths to search for images

\usepackage{amsmath,amssymb}

%\usepackage{units}
%\usepackage{mhchem}
\usepackage{booktabs}
\usepackage{array}
\newcolumntype{L}[1]{>{\raggedright\let\newline\\\arraybackslash\hspace{0pt}}p{#1}}
\newcolumntype{C}[1]{>{\centering\let\newline\\\arraybackslash\hspace{0pt}}p{#1}}
\newcolumntype{R}[1]{>{\raggedleft\let\newline\\\arraybackslash\hspace{0pt}}p{#1}}


\usepackage{multicol}
%	\setlength{\columnsep=1em}
\usepackage{enumitem}
\usepackage{xcolor}
\usepackage{textcomp}
\usepackage{setspace}
\usepackage{tikz}
	\tikzstyle{every picture}+=[remember picture,overlay]
\usetikzlibrary{arrows}

\mode<presentation>
{
  \usetheme{Lecture}
  \setbeamercovered{invisible}
  \setbeamertemplate{items}[square]
}

\usepackage{calc}
\usepackage{hyperref}

\newcommand\HiddenWord[1]{%
	\alt<handout>{\rule{\widthof{#1}}{\fboxrule}}{#1}%
}


\usepackage{xifthen}
\newcommand{\futuyma}[1]{%
	\ifthenelse{\isempty{#1}}%
	{Futuyma \& Kirkpatrick 2017, 4th ed.}%
	{Fig.~#1~Futuyma \& Kirkpatrick 2017, 4th ed.}%
}

% This defines \amper for the fancy ampersand
% to be used in the header. See
% https://tex.stackexchange.com/a/58185/39194
\usepackage{xspace}
\newfontfamily\amperfont[Style=Alternate]{Linux Libertine O}    
\makeatletter
\DeclareRobustCommand{\amper}{{\amperfont\ifx\f@shape\scname\smaller[1.2]\fi\&}\xspace}
\makeatother

\newcommand{\backskip}{\vspace{-0.5\baselineskip}}

%% Remove indent from multicol.
\parindent=0pt

\begin{document}

\lecture{student}{student}

{
\usebackgroundtemplate{\includegraphics[width=\paperwidth]{homo_intro}}
\begin{frame}[b]

\tinyfill  \textcolor{white}{\futuyma{}}
\end{frame}
}


%%

\begin{frame}{Humans belong to the family \highlight{Hominidae.}}

\includegraphics[width=\linewidth]{futuyma_fig21-2}

\tinyfill \futuyma{21.2}

\end{frame}
%%

\begin{frame}{\highlight{Hominins} are \highlight{bipedal.} What are the adaptive advantages?}

\backskip

\centering

\includegraphics[width=\linewidth]{futuyma_fig21-4}

\tinyfill \futuyma{21.4}
\end{frame}
%%

\begin{frame}{How many hominin species to you see?}

\backskip

\includegraphics[width=\linewidth]{homo_skulls}

	\pause
	\backskip
	\alt<handout>{}{%
	\begin{multicols}{3}
	\begin{enumerate}[leftmargin=*,label=\textsc{\Alph*}.]
	{\tiny
	\item Chimpanzee, modern\\
	\item \textit{Australopithecus africanus}, 2.6 \textsc{mya}\\
	\item \textit{A. africanus}, 2.5 \textsc{mya}\\
	\item \textit{Homo habilis}, 1.9 \textsc{mya}\\
	\item \textit{H. habilis}, 1.8 \textsc{mya}\\
	%
	\item \textit{H. rudolfensis}, 1.8 \textsc{mya}\\
	\item \textit{H. erectus}, 1.75 \textsc{mya}\\
	\item \textit{H. ergaster}, 1.75 \textsc{mya}\\
	\item \textit{H. heidelbergensis}, 300–125 \textsc{kya}\\
	\item \textit{H. neanderthalensis}, 70 \textsc{kya}\\
	%
	\item \textit{H. neanderthalensis}, 60 \textsc{kya}\\
	\item \textit{H. neanderthalensis}, 45 \textsc{kya}\\
	\item \textit{H. sapiens}, Cro-Magnon, 30 \textsc{kya}\\
	\item \textit{H. sapiens}, modern\\
	}
	\end{enumerate}
	\end{multicols}
}

\end{frame}
%%

\begin{frame}%{How many species of \textit{Homo} have been described?}

\includegraphics[width=0.8\linewidth]{homo_species_list2}

%\begin{multicols}{2}
%\textit{habilis} (2.3–1.4 \textsc{mya})*\\
%\textit{rudolfensis} (1.9)*\\
%\textit{ergaster} (1.9–1.4)\\
%\textit{georgicus} (1.8)\\
%\textit{erectus} (1.5–0.2)\\
%\textit{antecessor} (1.2–0.8)\\
%\textit{cepranensis} (0.9–0.8)\\
%
%\columnbreak
%
%\textit{heidelbergensis} (0.6-0.3 \textsc{mya})\\
%\textit{rhodesiensis} (0.3–0.12)\\
%\textit{neanderthalensis} (0.4–0.04)\\
%\textit{sapiens} (0.8–?)\\
%\textit{floresiensis} (0.10–0.012)\\
%Denisovans (0.5–0.03)\\
%\end{multicols}


\tinyfill Martin et al.\,2024.  Evolutionary Anthropology 33e22018

%\tiny * Australopithecines?

\end{frame}

%%

%\begin{frame}{Many \textit{Homo} species are no longer recognized.}
%
%\includegraphics[width=\linewidth]{homo_jaw}
%
%\begin{tikzpicture}
%\node at (3.5,7) [text width=4.5cm]{\textit{capensis, dubius, leakeyi, kanamensis, mauritanicus, modjokertensis, pekinensis}};
%\end{tikzpicture}
%
%\end{frame}

%%

\begin{frame}

\centering
%\includegraphics[height=0.94\textheight]{futuyma_fig21-6}

%\tinyfill \futuyma{21.6}

\includegraphics[height=0.94\textheight]{homo_phylogeny}

\tinyfill Martin et al.\,2024.  Evolutionary Anthropology 33e22018

\end{frame}

%%


\begin{frame}{Two waves of \textit{Homo} left Africa 1.9 and 0.6 \textsc{mya}.}

%\backskip

\includegraphics[width=\linewidth]{futuyma_fig21-9}

\tinyfill \futuyma{21.9}

\end{frame}

%%

\begin{frame}{\textit{Homo sapiens} left Africa 60 \textsc{kya}.}

%\backskip

\includegraphics[width=\linewidth]{futuyma_fig21-12}

\tinyfill \futuyma{21.12}

\end{frame}

%%

\begin{frame}{\textit{Homo sapiens} interbred with other species of \textit{Homo.}}

\backskip

\centering

\includegraphics[height=0.83\textheight]{futuyma_fig21-14}

\tinyfill \futuyma{21.14}

\end{frame}

%%

\begin{frame}[t]{Bipedalism evolved before large brain size}
	\includegraphics[width=\textwidth]{hominin_brain_size}

\end{frame}

%%

\begin{frame}{As brain size increased, skull structure changed.}

\centering

\includegraphics[width=\linewidth]{futuyma_fig21-8}

\smallskip

\includegraphics[width=0.8\linewidth]{futuyma_fig21-10}

\tinyfill \futuyma{21.8, 21.10}
\end{frame}

%%

\begin{frame}{Larger brain size might be adaptive for ecological or social reasons.}

\centering

\includegraphics[height=0.75\textheight]{futuyma_fig21-15}

\tinyfill \futuyma{21.15}
\end{frame}

%%

\begin{frame}{Cognitive ability for rudimentary language is old\ldots}

\backskip

\centering

\includegraphics[height=0.83\textheight]{futuyma_fig21-16}

\tinyfill \futuyma{21.16}

\end{frame}

%%

\begin{frame}{\ldots but complex communication is young.}

\backskip

\centering

\includegraphics[height=0.83\textheight]{futuyma_fig21-23}

\tinyfill \futuyma{21.23}

\end{frame}

%%

\begin{frame}{Large brains require high metabolic rates.}

\backskip

\centering

\includegraphics[height=0.83\textheight]{futuyma_fig21-17}

\tinyfill \futuyma{21.17}

\end{frame}

%%

\begin{frame}{Agriculture had enormous evolutionary consequences.}

\backskip

\centering

\includegraphics[width=\linewidth]{futuyma_fig21-18}

\tinyfill \futuyma{21.18}

\end{frame}

%%

\begin{frame}{Twelve genes have signature of adaptive evolution.}

\backskip

\centering

\includegraphics[height=0.83\textheight]{futuyma_fig21-19}

\tinyfill \futuyma{21.19}

\end{frame}

%%

\begin{frame}{Selection still acts today.}

\backskip

\centering


\begin{multicols}{2}

Cholesterol

\includegraphics[width=\linewidth]{futuyma_fig21-21a} 

\columnbreak

Height

\includegraphics[width=\linewidth]{futuyma_fig21-21b}

\end{multicols}


\tinyfill \futuyma{21.21}

\end{frame}

%%

\begin{frame}{Mutational load increases with distance from Africa.}

\backskip

\centering


\begin{multicols}{2}

\includegraphics[width=\linewidth]{futuyma_fig21-20a}

\columnbreak

\includegraphics[width=\linewidth]{futuyma_fig21-20b}

\end{multicols}


\tinyfill \futuyma{21.21}

\end{frame}

%%

\begin{frame}{Global $F_{ST} \approx 0.12.$ Apparent differences are often adaptive.}

\centering
\includegraphics[width=0.95\linewidth]{homo_global_pops}

\begin{tikzpicture}
\node at (-3,1.5) [text width=4cm]{Nigeria, SE~Asia, Europe};
\end{tikzpicture}


\end{frame}


%%
\end{document}


