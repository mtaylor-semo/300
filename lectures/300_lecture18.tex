%!TEX TS-program = lualatex
%!TEX encoding = UTF-8 Unicode

%\documentclass[t]{beamer}

%%%% HANDOUTS For online Uncomment the following four lines for handout
\documentclass[t,handout]{beamer}  %Use this for handouts.
\usepackage{handoutWithNotes}
\includeonlylecture{student}
\pgfpagesuselayout{3 on 1 with notes}[letterpaper,border shrink=5mm]

%\usefonttheme{professionalfonts}


%%% Including only some slides for students.
%%% Uncomment the following line. For the slides,
%%% use the labels shown below the command.

%% For students, use \lecture{student}{student}
%% For mine, use \lecture{instructor}{instructor}

% FONTS
\usepackage{fontspec}
\def\mainfont{Linux Biolinum O}
\setmainfont[Ligatures={Common, TeX}, Contextuals={NoAlternate}, BoldFont={* Bold}, ItalicFont={* Italic}, Numbers={Proportional, OldStyle}]{\mainfont}
%\setmonofont[Scale=MatchLowercase]{Inconsolata} 
\setsansfont[Scale=MatchLowercase]{Linux Biolinum O} 
\setmonofont{Linux Libertine Mono O}

\usepackage{microtype}

\parindent=0pt

\usepackage{unicode-math}
\setmathfont[Scale=MatchLowercase]{Asana Math}

\usepackage{graphicx}
	\graphicspath{%
	{/Users/goby/Pictures/teach/300/lectures/}%
	{/Users/goby/Pictures/teach/438/lectures/}} % set of paths to search for images

\usepackage{amsmath,amssymb}

%\usepackage{units}

\usepackage{booktabs}
\usepackage{array}
\newcolumntype{L}[1]{>{\raggedright\let\newline\\\arraybackslash\hspace{0pt}}p{#1}}
\newcolumntype{C}[1]{>{\centering\let\newline\\\arraybackslash\hspace{0pt}}p{#1}}
\newcolumntype{R}[1]{>{\raggedleft\let\newline\\\arraybackslash\hspace{0pt}}p{#1}}


\usepackage{multicol}
%	\setlength{\columnsep=1em}
\usepackage{enumitem}
\usepackage{textcomp}
\usepackage{setspace}
\usepackage{tikz}
	\tikzstyle{every picture}+=[remember picture,overlay]
\usetikzlibrary{arrows}

\mode<presentation>
{
  \usetheme{Lecture}
  \setbeamercovered{invisible}
  \setbeamertemplate{items}[square]
}

\usepackage{calc}
\usepackage{hyperref}

\newcommand\HiddenWord[1]{%
	\alt<handout>{\rule{\widthof{#1}}{\fboxrule}}{#1}%
}


\usepackage{xifthen}
\newcommand{\futuyma}[1]{%
	\ifthenelse{\isempty{#1}}%
	{Futuyma \& Kirkpatrick 2017, 4th ed.}%
	{Fig.~#1~Futuyma \& Kirkpatrick 2017, 4th ed.}%
}

% This defines \amper for the fancy ampersand
% to be used in the header. See
% https://tex.stackexchange.com/a/58185/39194
\usepackage{xspace}
\newfontfamily\amperfont[Style=Alternate]{Linux Libertine O}    
\makeatletter
\DeclareRobustCommand{\amper}{{\amperfont\ifx\f@shape\scname\smaller[1.2]\fi\&}\xspace}
\makeatother

\newcommand{\backskip}{\vspace{-0.5\baselineskip}}

%% Remove indent from multicol.
\parindent=0pt

\begin{document}

\lecture{student}{student}

{
\usebackgroundtemplate{\includegraphics[width=\paperwidth]{biogeo_intro}}
\begin{frame}[b]

%\tinyfill  \textcolor{white}{\futuyma{}}
\end{frame}
}

%%
{
\usebackgroundtemplate{\includegraphics[width=\paperwidth]{regions_terrestrial}}
\begin{frame}{Eight \highlight{biogeographic realms} are recognized.}

\tinyfill Modified from \futuyma{18.2}
\end{frame}
}


%%

\begin{frame}{Higher level taxa are usually \highlight{endemic} to a realm.}

\backskip

\includegraphics[width=\linewidth]{futuyma_fig18-3}
	

\tinyfill \futuyma{18.3}

\end{frame}

%%

\begin{frame}{Realms are divided into \highlight{provinces}.}

\backskip

\centering

\includegraphics[height=0.83\textheight]{regions_north_american_provinces}
	

\tinyfill \textcopyright\,Lomolino et al.~2005. \emph{Biogeography}

\end{frame}


%%

\begin{frame}{Species ranges are endemic at smaller scales.}

\backskip

\includegraphics[height=0.83\textheight]{range_marmota}
	

\tinyfill \textcopyright\,Lomolino et al.~2005. \emph{Biogeography}


\end{frame}

%
{
\usebackgroundtemplate{\includegraphics[width=\paperwidth]{range_allopatric}}
\begin{frame}[t]{Closely related species have \highlight{allopatric} distributions.}

\tinyfill \textcopyright\,Lomolino et al.~2005. \emph{Biogeography}

\end{frame}
}
%
%

\begin{frame}{Ranges are hierarchical at higher taxonomic levels.}

\backskip

\centering 

\includegraphics[height=0.83\textheight]{regions_endemism_hierarchy}
	

\tinyfill \textcopyright\,Lomolino et al.~2005. \emph{Biogeography}


\end{frame}


%
%%

\begin{frame}{Some taxa expand their range by \highlight{dispersal.}}

\backskip

\centering

\includegraphics[height=0.83\textheight]{dispersal_jump_diffusion}
	

\tinyfill \textcopyright\,Lomolino et al.~2005. \emph{Biogeography}

\end{frame}

%%

\begin{frame}{Humans have caused some range expansion.}

\backskip

\centering

\includegraphics[height=0.83\textheight]{futuyma_fig18-5}
	

\tinyfill \futuyma{18.5}

\end{frame}

%%

\begin{frame}{Some taxa have a \highlight{disjunct} distribution.}

\backskip

\centering

\includegraphics[height=0.83\textheight]{futuyma_fig18-4}
	

\tinyfill \futuyma{18.4}

\end{frame}
%%

\begin{frame}{Disjunct distributions can be caused by \highlight{vicariance.}}

\backskip

\centering

\includegraphics[height=0.83\textheight]{vicariance_reptile_mammals}
	

\tinyfill \textcopyright\,Lomolino et al.~2005. \emph{Biogeography}

\end{frame}

%%

\begin{frame}{Many modern taxa have vicariant distributions.}

\backskip

\centering

\includegraphics[width=\linewidth]{vicariance_ratites}
	

\tinyfill \textcopyright\,Lomolino et al.~2005. \emph{Biogeography}

\end{frame}


%%

\begin{frame}{Vicariance is reflected in phylogenetic trees.}

\backskip

\centering

\includegraphics[width=\linewidth]{futuyma_fig18-9}
	
\tinyfill \futuyma{18.9}

\end{frame}

%%

\begin{frame}{Dispersal can also be reflected in phylogenetic trees.}

\backskip

\centering

\includegraphics[height=0.83\textheight]{futuyma_fig18-12}
	
\tinyfill \futuyma{18.12}

\end{frame}

%%

\begin{frame}{\highlight{Phylogeography} studies the geographic distribution of gene lineages.}

\backskip

\centering

\includegraphics[height=0.79\textheight]{futuyma_fig18-13}
	
\tinyfill \futuyma{18.13}

\end{frame}

%%

\begin{frame}{Gene lineages are not always resolved by phylogenies.}

\backskip

\centering

\includegraphics[height=0.83\textheight]{biogeo_gene_phylogeny}
	
\tinyfill Guo et al.~2019. Korean J Parasitol 57:153

\end{frame}

%%

\begin{frame}{Gene lineages can be represented as a network.}

\backskip

\centering

\includegraphics[height=0.83\textheight]{biogeo_gene_network}
	
\tinyfill Guo et al.~2019. Korean J Parasitol 57:153

\end{frame}



%%

\begin{frame}{Phylogeography reflects evolutionary history, such as refugia and range expansion.}

\backskip
	\begin{columns}[T]
		\begin{column}{0.6\textwidth}
			\includegraphics[width=0.9\textwidth]{pleistocene_macgillivray_range} %\hspace*{1cm}
		\end{column}
		\begin{column}{0.4\textwidth}
			\includegraphics[width=0.9\textwidth]{pleistocene_macgillivray_picture}
		\end{column}
	\end{columns}
	\begin{tikzpicture}[overlay, line width=2pt]
		\draw [<-] (1.7,1.5) -- (0.3,1.5) ;
	\end{tikzpicture}
	\vspace{\baselineskip}
	
	Is the breeding resident population of MacGillivray's Warbler in Mexico a refugial population?
	
\end{frame}
%
\lecture{instructor}{instructor}
{\setbeamercolor{background canvas}{bg=black}
\begin{frame}[plain]
%	Placeholder to work out parsimony network
\end{frame}}
%
\lecture{student}{student}
\begin{frame}{\highlight{Parsimony network} supports refuge and range expansion for MacGillivray's Warbler.}
	\vspace{-\baselineskip}
	\begin{center}
		\includegraphics[width=0.9\textwidth]{pleistocene_macgillivray_network}
	\end{center}
\end{frame}

%%

\begin{frame}{Similar patterns of divergence suggest shared biogeographic histories.}

%\backskip

\centering

\includegraphics[width=\linewidth]{futuyma_fig18-14}
	
\tinyfill \futuyma{18.13}

\end{frame}

%%

\begin{frame}{Ecologically similar species are allopatric. Dissimilar species can be sympatric.}

\backskip

\begin{multicols}{2}
\includegraphics[width=\linewidth]{biogeo_morpho_divergence}

\columnbreak

\includegraphics[width=\linewidth]{futuyma_fig18-16}
\end{multicols}

\vfilll
	
\tiny Pigot \& Tobias 2013. Ecology Letters 16:330 \hfill \futuyma{18.16}

\end{frame}

%

\begin{frame}{Taxa show a \highlight{latitudinal diversity gradient.}}

%\backskip

\centering

\includegraphics[width=\linewidth]{futuyma_fig18-18}
	
	
\begin{tikzpicture}

\node at (-1,6) [text width=3.5cm] {Diversity is greatest in the tropics.};
\end{tikzpicture}
	
\tinyfill \futuyma{18.18}

\end{frame}

%
\begin{frame}[t,plain]{\highlight{Net primary productivity} is highest in the tropics.}

\backskip
\centering
		\includegraphics[width=\textwidth]{diversity_gradient_npp}
\end{frame}
%

{
\usebackgroundtemplate{\includegraphics[width=\paperwidth]{diversity_gradient_niche_conservatism_phylogeny}}
\begin{frame}[t]{Oldest lineages are tropical. Younger lineages are temperate.}


	\vfilll

	\tiny \hfill Fig.~14.55\,\textcopyright Sinauer Associates, Inc.
\end{frame}
}

%%

\begin{frame}{The oldest tree frog lineages are tropical.}

%\backskip

\centering

\includegraphics[width=\linewidth]{futuyma_fig18-20}
	
\tinyfill \futuyma{18.20}

\end{frame}

%
{
\usebackgroundtemplate{\includegraphics[width=\paperwidth]{diversity_gradient_out_of_tropics}}
\begin{frame}[t]{Tropics serve as cradle, museum, and immigration pump.}


	\vfilll

	\tiny \hfill Fig.~14.56\,\textcopyright Sinauer Associates, Inc.
\end{frame}
}


%%

\begin{frame}[t]{More marine bivalves tended to evolve first in tropics.}
	\begin{center}
		\includegraphics[width=\textwidth]{diversity_gradient_oot_first_occurrence}
	\end{center}
	
	\vfilll
	
	\hfill \tiny Jablonski et al. 2013. Proc. Natl. Acad. Sci. 110: 10487.
\end{frame}
%
\begin{frame}[t]{Marine bivalves appear later in temperate zones.}
	\vspace{-\baselineskip}
	\begin{center}
		\includegraphics[height=0.75\textheight]{diversity_gradient_oot_first_occurrence2}
	\end{center}
	
	\vfilll
	
	\hfill \tiny Jablonski et al. 2013. Proc. Natl. Acad. Sci. 110: 10487.
\end{frame}
%

\begin{frame}{Diversity on “islands” varies by \highlight{area and isolation.}}
	\centering
		\includegraphics[height=0.8\textheight]{island_equilibrium}
\end{frame}


%%
\end{document}


