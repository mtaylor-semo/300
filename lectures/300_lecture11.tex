%!TEX TS-program = lualatex
%!TEX encoding = UTF-8 Unicode

\documentclass[t]{beamer}

%%%% HANDOUTS For online Uncomment the following four lines for handout
%\documentclass[t,handout]{beamer}  %Use this for handouts.
%\usepackage{handoutWithNotes}
%\includeonlylecture{student}
%\pgfpagesuselayout{3 on 1 with notes}[letterpaper,border shrink=5mm]

\usefonttheme{professionalfonts}


%%% Including only some slides for students.
%%% Uncomment the following line. For the slides,
%%% use the labels shown below the command.

%% For students, use \lecture{student}{student}
%% For mine, use \lecture{instructor}{instructor}

% FONTS
\usepackage{fontspec}
\def\mainfont{Linux Biolinum O}
\setmainfont[Ligatures={Common, TeX}, Contextuals={NoAlternate}, BoldFont={* Bold}, ItalicFont={* Italic}, Numbers={Proportional, OldStyle}]{\mainfont}
%\setmonofont[Scale=MatchLowercase]{Inconsolata} 
\setsansfont[Scale=MatchLowercase]{Linux Biolinum O} 
\setmonofont{Linux Libertine Mono O}

\usepackage{microtype}

\usepackage{unicode-math}
\setmathfont[Scale=MatchLowercase]{Asana Math}

\usepackage{graphicx}
	\graphicspath{%
	{/Users/goby/Pictures/teach/300/lectures/}%
	{/Users/goby/Pictures/teach/163/common/}} % set of paths to search for images

\usepackage{amsmath,amssymb}

%\usepackage{units}

\usepackage{booktabs}
\usepackage{array}
\newcolumntype{L}[1]{>{\raggedright\let\newline\\\arraybackslash\hspace{0pt}}p{#1}}
\newcolumntype{C}[1]{>{\centering\let\newline\\\arraybackslash\hspace{0pt}}p{#1}}
\newcolumntype{R}[1]{>{\raggedleft\let\newline\\\arraybackslash\hspace{0pt}}p{#1}}


\usepackage{multicol}
%	\setlength{\columnsep=1em}
\usepackage{enumitem}
\usepackage{textcomp}
\usepackage{setspace}
\usepackage{tikz}
	\tikzstyle{every picture}+=[remember picture,overlay]
\usetikzlibrary{arrows}

\mode<presentation>
{
  \usetheme{Lecture}
  \setbeamercovered{invisible}
  \setbeamertemplate{items}[square]
}

\usepackage{calc}
\usepackage{hyperref}

\newcommand\HiddenWord[1]{%
	\alt<handout>{\rule{\widthof{#1}}{\fboxrule}}{#1}%
}


\usepackage{xifthen}
\newcommand{\futuyma}[1]{%
	\ifthenelse{\isempty{#1}}%
	{Futuyma \& Kirkpatrick 2017, 4th ed.}%
	{Fig.~#1~Futuyma \& Kirkpatrick 2017, 4th ed.}%
}

% This defines \amper for the fancy ampersand
% to be used in the header. See
% https://tex.stackexchange.com/a/58185/39194
\usepackage{xspace}
\newfontfamily\amperfont[Style=Alternate]{Linux Libertine O}    
\makeatletter
\DeclareRobustCommand{\amper}{{\amperfont\ifx\f@shape\scname\smaller[1.2]\fi\&}\xspace}
\makeatother

\newcommand{\backskip}{\vspace{-0.5\baselineskip}}

\begin{document}

\lecture{student}{student}

{
\usebackgroundtemplate{\includegraphics[width=\paperwidth]{species_intro}}
\begin{frame}[b]

\tiny \hfill \textcolor{white}{\futuyma{}}
\end{frame}
}


\begin{frame}{What is a species?}


\includegraphics[width=\linewidth]{species_what_is_species}

\tinyfill \futuyma{9.2}

\end{frame}

%%


\begin{frame}{Many species concepts have been defined.}

\vspace{-2\baselineskip}

\hangpara\begin{multicols}{2}
Morphological\\
Phylogenetic (I)\\
Biological\\
Ecological\\
Phylogenetic (II)\\
Evolutionary\\
General Lineage\\
Typological\\
Cohesion\\
%\columnbreak
Genealogical\\
Genotypic\\
Recognition\\
Phenetic\\
Cladistic\\
Diagnostic\\
Polytypic\\
Population\\[1ex]

artificial concept\\
\end{multicols}
\end{frame}


%%

\begin{frame}{Two widely followed species concepts are$\dots$}


\noindent\begin{multicols}{2}

\reflectbox{\includegraphics[width=\linewidth]{species_evelynae1}}
\vspace{1ex}
\includegraphics[width=\linewidth]{species_evelynae2}

\columnbreak

\hangpara \highlight{morphological} species, and

\hangpara \highlight{biological} species.
\end{multicols}

\tinyfill \textcopyright\,Paul Humann

\end{frame}

%%

\begin{frame}{Morphological species can be diagnosed by morphology.}

\backskip

\includegraphics[width=\linewidth]{futuyma_fig9-6}

\tinyfill \futuyma{9.6}

\end{frame}

\begin{frame}{The \textsc{bsc} defines species as$\dots$}

\vspace{-\baselineskip}

\hangpara groups of actually or potentially interbreeding populations, which are \emph{reproductively isolated} from other such groups.

\hangpara What are the advantages of the \textsc{bsc?}

\hangpara What are the disadvantages of the \textsc{bsc?}

\hangpara Does the \textsc{bsc} actually tell us what \emph{is} a species?

\end{frame}

%%

\begin{frame}[t]{\highlight{Allopatric} populations are difficult to test for reproductively isolation.}

\backskip
\centering

\includegraphics[height=0.8\textheight]{futuyma_fig9-3}

\tinyfill \futuyma{9.3}

\end{frame}

%%
\begin{frame}

\backskip

\begin{multicols}{2}

\centering
\noindent \includegraphics[width=\linewidth]{species_fox_sparrow_forms}

\columnbreak

\centering
\noindent\includegraphics[width=\linewidth]{species_fox_sparrow_map}

\end{multicols}

\tinyfill\textcopyright\,iBird Pro 12.5, Mitchell Waite Group

\end{frame}

%%


\begin{frame}{\highlight{Hybrid zones} can form between \highlight{parapatric} species.}

\backskip

\centering

\includegraphics[height=0.85\textheight]{futuyma_fig9-4}

\tinyfill \futuyma{9.4}
\end{frame}

%%
\begin{frame}{\highlight{Introgression} in hybrid zones introduces alleles into a species from another.}

\backskip

\centering
\includegraphics[width=0.95\linewidth]{futuyma_fig9-5}

\tinyfill \futuyma{9.5}

\end{frame}


%%

\begin{frame}{\highlight{Speciation} is the evolution of two species from one ancestor.}

\vspace{-\baselineskip}

\centering

\includegraphics[height=0.85\textheight]{species_speciation}

\end{frame}

%%

\begin{frame}{Speciation occurs when populations evolve \highlight{reproductive isolating barriers.}}

\backskip
\centering

\includegraphics[width=0.9\linewidth]{futuyma_tab9-1}

\tinyfill Table 9.1 Futuyma \& Kirkpatrick 2017, 4th ed.

\end{frame}

%%

\begin{frame}{\highlight{Temporal isolation:} species reproduce at different times. }

\includegraphics[width=\linewidth]{species_temporal_cicada}

\vfilll

\tiny \highlight{prezygotic} \hfill \futuyma{9.9\textsc{b}} \quad \href{https://commons.wikimedia.org/wiki/File:Periodical_Cicada_Broods_of_the_United_States.png}{U\textsc{sda}, Wikimedia}

\end{frame}

%%

\begin{frame}{\highlight{Mechanical isolation:} reproductive structures are incompatible.}

\backskip

\centering

\includegraphics[height=0.8\textheight]{species_mechanical}

\vfilll

\tiny \highlight{Prezygotic} \hfill \futuyma{9.10}
\end{frame}

%%

\begin{frame}{\highlight{Pollinator isolation}: Species use different pollinators.}

\backskip

\centering

\includegraphics[height=0.85\textheight]{species_pollinator}

\vfilll

\tiny \highlight{Prezygotic} \hfill \futuyma{9.7}

\end{frame}

%%

\begin{frame}{Species can be isolated by multiple barriers.}

\backskip

\centering

\includegraphics[width=\linewidth]{species_multiple_barriers}

\vfilll

\tiny \highlight{Prezygotic} \hfill \futuyma{9.8}

\end{frame}

%%

\begin{frame}{Sperm competition occurs post-mating.}

\backskip

\begin{multicols}{2}

{\centering
\noindent\includegraphics[width=0.95 \linewidth]{species_sperm_speed}
}

%\vspace{2em}

\hangpara Sperm swim faster in promiscuous species (above).

\columnbreak

\onslide<2>{
\noindent\includegraphics[width=\linewidth]{species_sperm_competition}
}
\end{multicols}

\vfilll

\tiny \highlight{Prezygotic}\quad Nascimento et al.~2008. J.R.~Soc.~Interface 5:297. \onslide<2>{\hfill Copper \& Phadnis 2017. Genome Bio.~Evol.~9:1938.}

\end{frame}

%%

\begin{frame}{\highlight{Ecological inviability:} hybrids have reduced survival.}
\includegraphics[width=\linewidth]{futuyma_fig9-11}

\vfilll

\tiny\highlight{postzygotic} \hfill \futuyma{9.11}

\end{frame}

%%

\begin{frame}{\highlight{Hybrid sterility:} hybrids are incapable of reproducing.}

\vspace{-\baselineskip}

\begin{center}
\includegraphics[height=0.85\textheight]{species_hybrid_sterility}
\end{center}

\begin{tikzpicture}
\node at (2,4) {2n = 64};
\node at (10,4) {2n = 62};
\node at (9.7,1.5) {2n = 63};

\end{tikzpicture}

\vfilll

\tiny \highlight{postzygotic}
\end{frame}


\begin{frame}{\highlight{Reinforcement} reduces hybridization.}

\backskip

\begin{multicols}{2}
\includegraphics[width=\linewidth]{futuyma_fig9-17a}


%\medskip

%\includegraphics[width=\linewidth]{futuyma_fig9-17b}

\columnbreak

\includegraphics[width=\linewidth]{futuyma_fig9-17c}


%\medskip

%\includegraphics[width=\linewidth]{species_reinforcement_criteria}

\end{multicols}

\pause 

\hangpara \highlight{Character displacement} is the divergence of a phenotype in \highlight{sympatry.}

\tinyfill \futuyma{9.17}

\end{frame}

%%
\begin{frame}
\centering

\includegraphics[width=\linewidth]{species_reinforcement_criteria}

\tinyfill Hopkins 2013. New Phytologist 197: 1095.

\end{frame}

%%

\begin{frame}{\highlight{Allopatric speciation} is the evolution of reproductive barriers between geographically separated populations.}

\hangpara Geographic isolation is \emph{not} a biological reproductive barrier.

\hangpara Gene flow is disrupted between allopatric populations.

\end{frame}

%%

\begin{frame}{\highlight{Vicariance} divides a large population into smaller, isolated populations.}

\vspace{-\baselineskip}

\centering

\includegraphics[width=0.85\linewidth]{/Users/goby/pictures/teach/163/lecture/allopatric_shrimp2}

\end{frame}

%%

\begin{frame}{\highlight{Founder events} by dispersal can isolate populations.}

\backskip

\centering

\includegraphics[height=0.87\textheight]{futuyma_fig9-21}

\tinyfill \futuyma{9.21}
\end{frame}

%%

\begin{frame}{\highlight{Sympatric speciation} occurs \emph{without} geographic isolation.}

\backskip

\centering

\includegraphics[height=0.8\textheight]{futuyma_fig9-27}

\tinyfill \futuyma{9.27}

\end{frame}

%%

\begin{frame}{\textit{Rhagoletis pomonella} is the poster child for sympatric speciation.}

\backskip

\includegraphics[width=\linewidth]{futuyma_fig9-26}

\tinyfill \futuyma{9.26}

\end{frame}

%%

\begin{frame}{\highlight{Parapatric speciation} can occur when gene flow is reduced between adjacent populations.}

\backskip

\includegraphics[width=\linewidth]{futuyma_fig9-28}

\tinyfill \futuyma{9.28}

\end{frame}
\end{document}
