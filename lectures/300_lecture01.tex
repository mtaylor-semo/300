%!TEX TS-program = lualatex
%!TEX encoding = UTF-8 Unicode

\documentclass[t]{beamer}

%%%% HANDOUTS For online Uncomment the following four lines for handout
%\documentclass[t,handout]{beamer}  %Use this for handouts.
%\usepackage{handoutWithNotes}
%\includeonlylecture{student}
%\pgfpagesuselayout{3 on 1 with notes}[letterpaper,border shrink=5mm]
%	\setbeamercolor{background canvas}{bg=black!5}


%%% Including only some slides for students.
%%% Uncomment the following line. For the slides,
%%% use the labels shown below the command.

%% For students, use \lecture{student}{student}
%% For mine, use \lecture{instructor}{instructor}


%\usepackage{pgf,pgfpages}
%\pgfpagesuselayout{4 on 1}[letterpaper,border shrink=5mm]

% FONTS
\usepackage{fontspec}
\def\mainfont{Linux Biolinum O}
\setmainfont[Ligatures={Common, TeX}, Contextuals={NoAlternate}, BoldFont={* Bold}, ItalicFont={* Italic}, Numbers={Proportional, OldStyle}]{\mainfont}
%\setmonofont[Scale=MatchLowercase]{Inconsolata} 
\setsansfont[Scale=MatchLowercase]{Linux Biolinum O} 
\usepackage{microtype}

\usepackage{graphicx}
	\graphicspath{%
	{/Users/goby/Pictures/teach/300/lectures/}%
	{/Users/goby/Pictures/teach/163/common/}} % set of paths to search for images

\usepackage{amsmath,amssymb}

%\usepackage{units}

\usepackage{booktabs}
\usepackage{multicol}
%	\setlength{\columnsep=1em}

\usepackage{textcomp}
\usepackage{setspace}
\usepackage{tikz}
	\tikzstyle{every picture}+=[remember picture,overlay]

\mode<presentation>
{
  \usetheme{Lecture}
  \setbeamercovered{invisible}
  \setbeamertemplate{items}[square]
}

\usepackage{calc}
\usepackage{hyperref}

\newcommand\HiddenWord[1]{%
	\alt<handout>{\rule{\widthof{#1}}{\fboxrule}}{#1}%
}

\usepackage{csquotes}

\begin{document}
	
\lecture{student}{student}

{
	\usebackgroundtemplate{\includegraphics[width=\paperwidth]{evolution_intro}}
	\begin{frame}[t]{\textcolor{white}{\textbf{BI 300: Evolution}}}
\end{frame}
}

{
\usebackgroundtemplate{\includegraphics[width=\paperwidth]{mike_snake}}
\begin{frame}[t,plain]
\large
\vspace{5ex}
\hangpara\hspace{17em} Mike Taylor

\hangpara\hspace{17em} \textsc{rh} 217

\hangpara\hspace{17em} mtaylor@semo.edu

%\hangpara\hspace{17em} Zoom office hours

%	\hangpara \hspace{17em} \includegraphics[width=0.4cm]{twitter_icon} @MikeTaylor\textsc{semo}\\
%\hspace{17em} \#\textsc{bi}348\textsc{taylor}
\end{frame}
}



%\textsc{\lecture{instructor}{instructor}
%\begin{frame}{\highlight{Trigger alert.}}
%
%\hangpara Some of you may find the following images disturbing.
%
%\end{frame}
%
%
%{
%\usebackgroundtemplate{\includegraphics[width=\paperwidth]{post_surgery_hematoma} }
%\begin{frame}[t]
%\end{frame}
%}
%
%{
%\usebackgroundtemplate{\includegraphics[width=\paperwidth]{pre_surgery_hematoma} }
%\begin{frame}[t]
%\end{frame}
%}
%}


\begin{frame}[t]{You \highlight{earn} your grade with}
	\begin{center}\large\begin{tabular}{@{}ll@{}}
	Four 75-point exams & 60\% \\[1ex]
	Home/in-class assignments & 40\% \\
	\end{tabular}
	\end{center}
\end{frame}


% Lecture goals
\lecture{student}{student}

\begin{frame}[t]{Your \highlight{objectives} for this course are to}
	
	\hangpara explain the process of evolution, 
	
	\hangpara explain random and non-random evolutionary processes,
	
	\hangpara explain the evidence for evolution, and
	
	\hangpara apply your knowledge to solve evolutionary problems.
	
\end{frame}




{
\usebackgroundtemplate{\includegraphics[width=\paperwidth]{charles_darwin} }
	\begin{frame}[t]
\end{frame}
}

{
\usebackgroundtemplate{\includegraphics[width=\paperwidth]{beagle_voyage} }
\begin{frame}[t]{}
\end{frame}
}

{
\begin{frame}[t]{Darwin became famous as a naturalist during and after the voyage.}

\begin{multicols}{2}

\hangpara Published more than 20 books and hundreds of articles.

\hangpara Developed his thoughts on the “species problem.”

\hangpara Published \textit{On the Origin of Species} in 1859.

\columnbreak

\centering
\includegraphics[width=0.33\textwidth]{beagle_cover}\\
Voyage of the \textit{Beagle}

\end{multicols}

\end{frame}
}

{
\begin{frame}[t]{Natural selection is similar to artificial selection.}
	
	\begin{multicols}{2}
		
		\hangpara Artificial selection for favored traits in domestic breeds.
		
		\hangpara Domestic breeds stem from wild-type species.
				
		\columnbreak
		
		\centering
		\includegraphics[width=0.33\textwidth]{artificial_pigeons1}\\
		\includegraphics[width=0.33\textwidth]{artificial_pigeons2}
		
	\end{multicols}
	
\end{frame}
}

{
\begin{frame}[t]{Population growth is limited by resources.}
	
	\begin{multicols}{2}
		
		\hangpara \textit{Principle of Population} by Robert Malthus.
		
		\hangpara If human population growth can be limited by resources, then why not natural populations?
		
		\columnbreak
		
		\centering
		\includegraphics[width=0.4\textwidth]{malthus_graph}
%		\includegraphics[width=0.25\textwidth]{malthus_cover}
		
	\end{multicols}
	
\end{frame}
}

{
\usebackgroundtemplate{\includegraphics[width=\paperwidth]{darwin_descent_modification} }
\begin{frame}[t]{}
\end{frame}
}


{
\usebackgroundtemplate{\includegraphics[width=\paperwidth]{descent_with_modification} }
\begin{frame}[t]{}
\end{frame}
}

\begin{frame}
%\begin{tabular}{cc}
\centering
\includegraphics[width=\textheight]{feline_phylogeny} %&
%\includegraphics[width=0.3\textwidth]{domestic_cat_phylogeny} \\
%\end{tabular}
\vfilll
\end{frame}

{
\usebackgroundtemplate{\includegraphics[width=\paperwidth]{adaptation_by_natural_selection} }
	\begin{frame}[t]{}
\end{frame}
}

\begin{frame}[t]{What was missing from Darwin's concepts?}
\vspace{-\baselineskip}
\begin{multicols}{2}
	\highlight{Species descend from common ancestor.}
\bigskip

\highlight{Adaptations evolve by natural selection.}

\vspace{7\baselineskip}

Charles Darwin, \textit{On the Origin of Species}, 1st edition, 1859

	\columnbreak
	\begin{displayquote}There is grandeur in this view of life, with its several powers, having been originally breathed into a few forms or into one; and that, whilst this planet has gone cycling on according to the fixed law of gravity, from so simple a beginning endless forms most beautiful and most wonderful have been, and are being, evolved.\end{displayquote}
%— Charles Darwin, \textit{On the Origin of Species}, 1st edition, 1859
	
	
\end{multicols}
\end{frame}

\begin{frame}{The \highlight{evolutionary synthesis} built upon Darwin's foundation.}

\hangpara Natural selection and genetic processes that operate within species account for the \highlight{\textit{origin of new species.}}

\end{frame}

\begin{frame}[t]{The \highlight{evolutionary synthesis} determined that}
\hangpara the unit of evolution is the population,

\hangpara genetic variation is due to random mutation and genetic recombination,

\hangpara genetic variation creates phenotypic variation, and

\hangpara natural selection acting on phenotypic variation is the primary cause of evolutionary change.

\end{frame}

\begin{frame}[t]{The \highlight{evolutionary synthesis} determined that}
\hangpara species are pools of shared alleles, 

\hangpara speciation occurs when populations become reproductively isolated, and

\hangpara macroevolution is the same process as microevolution.


\end{frame}


\end{document}
