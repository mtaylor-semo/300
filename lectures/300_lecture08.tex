%!TEX TS-program = lualatex
%!TEX encoding = UTF-8 Unicode

\documentclass[t]{beamer}

%%%% HANDOUTS For online Uncomment the following four lines for handout
%\documentclass[t,handout]{beamer}  %Use this for handouts.
%\includeonlylecture{student}
%\usepackage{handoutWithNotes}
%\pgfpagesuselayout{3 on 1 with notes}[letterpaper,border shrink=5mm]
%	\setbeamercolor{background canvas}{bg=black!5}
\usefonttheme{professionalfonts}


%%% Including only some slides for students.
%%% Uncomment the following line. For the slides,
%%% use the labels shown below the command.

%% For students, use \lecture{student}{student}
%% For mine, use \lecture{instructor}{instructor}

% FONTS
\usepackage{fontspec}
\def\mainfont{Linux Biolinum O}
\setmainfont[Ligatures={Common, TeX}, Contextuals={NoAlternate}, BoldFont={* Bold}, ItalicFont={* Italic}, Numbers={Proportional, OldStyle}]{\mainfont}
%\setmonofont[Scale=MatchLowercase]{Inconsolata} 
\setsansfont[Scale=MatchLowercase]{Linux Biolinum O} 
\usepackage{microtype}

\usepackage{unicode-math}
\setmathfont[Scale=MatchLowercase]{Asana Math}

\usepackage{graphicx}
	\graphicspath{%
	{/Users/goby/Pictures/teach/300/lectures/}%
	{/Users/goby/Pictures/teach/163/common/}} % set of paths to search for images

\usepackage{amsmath,amssymb}

%\usepackage{units}

\usepackage{booktabs}
\usepackage{multicol}
%	\setlength{\columnsep=1em}

\usepackage{textcomp}
\usepackage{setspace}
\usepackage{tikz}
	\tikzstyle{every picture}+=[remember picture,overlay]

\mode<presentation>
{
  \usetheme{Lecture}
  \setbeamercovered{invisible}
  \setbeamertemplate{items}[square]
}

\usepackage{calc}
\usepackage{hyperref}

\newcommand\HiddenWord[1]{%
	\alt<handout>{\rule{\widthof{#1}}{\fboxrule}}{#1}%
}

\newcommand{\futuyma}[1]{Fig.~#1~Futuyma \& Kirkpatrick 2017, 4th ed.}

\begin{document}

\lecture{student}{student}


{
\usebackgroundtemplate{\includegraphics[width=\paperwidth]{gentheory_intro}}
\begin{frame}[b]

	\tiny\hfill \textcolor{white}{Photo \textcopyright\,Futuyma \& Kirkpatrick 2017, 4th ed.}
\end{frame}
}
%

\begin{frame}[t,plain]{“Natural selection is not evolution.” --- R.A.~Fisher}

\includegraphics[width=\linewidth]{futuyma_fig5-1}


\end{frame}
%%

\begin{frame}[t,plain]{Evolution \emph{by} selection requires a correlation between$\dots$ }

\includegraphics[width=\linewidth]{futuyma_fig5-4}

\hangpara (1) a phenotypic trait and fitness, and\\
 (2) phenotypes of parents and offspring.
 
 \vfilll
 
 \tinyfill{Fig.~5.4 Futuyma \& Kirkpatrick 2017, 4th ed.}

\end{frame}


%
\begin{frame}[t]{\highlight{Absolute fitness $\left(W\right)$}  is a measure of two fitness components.}
\begin{multicols}{2}
\includegraphics[width=\linewidth]{futuyma_fig5-5}

\columnbreak

(1) probability the  individual survives to maturity

\quad {\Large $\times$}\\

(2) expected number of offspring produced.
\end{multicols}

\vfilll

\tinyfill{Fig.~5.5 Futuyma \& Kirkpatrick 2017, 4th ed.}

\end{frame}

%% Relative fitness
\begin{frame}[t]{\highlight{Relative fitness $\left(w\right)$} is absolute fitness divided by a reference fitness.}

\hangpara If $A_1A_1 \left(W_{11}\right)$ is the reference genotype, then

\begin{align*}
w_{11} &= W_{11}/W_{11} = 1\\
w_{12} &= W_{12}/W_{11}\\
w_{22} &= W_{22}/W_{11}
\end{align*}

\end{frame}

%% Selection coefficient

\begin{frame}[t]{\highlight{Selection coefficient $\left(s\right)$} is a measure of the strength of selection.}

\vspace{-\baselineskip}

\includegraphics[width=\linewidth]{gentheory_selection_coefficient_zimmer}

\begin{tikzpicture}
\node at (2.2,2) [text width = 3.5cm, align=left] {$s =$ fitness difference relative to reference genotype.};
\end{tikzpicture}

\vfilll

\tinyfill{Zimmer and Emlen, 2nd ed.}
\end{frame}


\begin{frame}[t]{\highlight{Positive selection} favors alleles that increase fitness.}
\includegraphics[width=\linewidth]{futuyma_fig5-6}

\vfilll

\tinyfill{\futuyma{5.6}}
\end{frame}

%%

\begin{frame}[t]{$s$ determines the rate of adaptation: \highlight{$\Delta p = sp(1-p)$}}

\includegraphics[width=\linewidth]{futuyma_fig5-7}

\begin{tikzpicture}
\node at (10.85,7.5) {\highlight{$A_2$ is fixed.}};
\end{tikzpicture}

\vfilll

\tinyfill{\futuyma{5.7}}
\end{frame}

%%
\begin{frame}[t]{We've been discussing beneficial mutations but$\dots$}

\hangpara {\Large \highlight{$\dots$does this apply to detrimental mutations?}}

\pause

\mode<beamer>{\Large \hangpara Yes! Purifying selection $\left(-s\right)$ removes detrimental mutations.}
\end{frame}

%%
\begin{frame}[t]{Rate of adaptation depends on dominance of allele.}

\includegraphics[width=\linewidth]{futuyma_fig5-10}

\begin{tikzpicture}
\node at (3.9,8.45) {\highlight{$w_{12} = w_{22}$}};
\node at (8.3,4.6) {\highlight{$w_{12} = w_{11}$}};
\node at (9,8) [text width=5.5cm] {If recessive $p = 0.001$ then most copies of $A_2$ are in heterozygotes.};
\end{tikzpicture}

\vfilll

\tinyfill{\futuyma{5.10}}
\end{frame}


\begin{frame}[t]{Positive selection can lead to \highlight{hitchhiking} among linked genes.}

\includegraphics[width=\linewidth]{gentheory_selective_sweep_zimmer}


\vfilll

\tinyfill{Zimmer and Emlen 2nd ed.}
\end{frame}


%%

{
\usebackgroundtemplate{\includegraphics[width=\paperwidth]{gentheory_lactose_persistence_zimmer}}
\begin{frame}[t]{Lactose persistance in humans is the result of a \highlight{selective sweep.}}

\vfilll

\tinyfill{Zimmer and Emlen 2nd ed.}
\end{frame}
}
%%

\begin{frame}[t]{\highlight{Balancing selection} maintains genetic polymorphism.}

\includegraphics[width=\linewidth]{futuyma_fig5-19b}


\vfilll

\tinyfill{\futuyma{5.19}}
\end{frame}

%%

\begin{frame}[t]{\highlight{Overdominance} occurs when heterozygotes have highest fitness.}

\includegraphics[width=\linewidth]{futuyma_fig5-18}

\vfilll

\tinyfill{\futuyma{5.18}}
\end{frame}

%%

\begin{frame}[t]{Relative fitness of homozygotes determines \highlight{polymorphic equilibrium} frequency.}


\begin{center}
\begin{tabular}{lccc}
Genotype & $AA$ & $AS$ & $SS$ \tabularnewline
Fitness $\left(w\right)$ & 0.88 & 1 & 0.14 \tabularnewline
\end{tabular}
\end{center}

\hangpara Let $\hat{p} =$ equilibrium frequency of $S$ allele.

\vspace{\baselineskip}
\begin{multicols}{2}
\noindent\begin{equation*}
\hat{p} = \dfrac{1-w_{AA}}{2-w_{AA} - w_{SS}}
\end{equation*}

\columnbreak
{\setlength{\jot}{1em}
\noindent\begin{align*}
\hat{p} &= \dfrac{1-0.88}{2-0.88 - 0.14} \\
        &= \mathbf{0.122}
\end{align*}
}

\end{multicols}

\end{frame}

%%
{
\usebackgroundtemplate{\includegraphics[width=\paperwidth]{gentheory_frequency_dependent_cichlids}}
\begin{frame}[t]{\highlight{Negative frequency dependence:} common alleles have lower fitness.}

	\vfilll

	\tiny Ichijo, \href{https://doi.org/10.3389/fnins.2016.00595}{Front. Neurosci. 4 Jan 2017} \hfill Hori 1993. Science 260:216.
\end{frame}
}

%%

\begin{frame}[t]{\highlight{Underdominance} eliminates genetic polymorphism.}

\includegraphics[width=\linewidth]{futuyma_fig5-19c}

\vfilll

\tinyfill{\futuyma{5.19}}
\end{frame}

%%

\begin{frame}[t]{\highlight{Positive frequency dependence:} common alleles have higher fitness.}

\vspace{-0.5\baselineskip}
\centering

\includegraphics[height=0.8\textheight]{futuyma_fig5-24}

\vfilll

\tinyfill{\futuyma{5.24}}
\end{frame}

%%

\begin{frame}[t]{If purifying selection removes detrimental alleles, then why do so many genetic disorders remain?}

Common disorders:

\begin{multicols}{3}
{\small 
Angelman syndrome\\
Canavan disease\\
Charcot–Marie–Tooth\\
Color blindness\\
Cri du chat syndrome\\
Cystic fibrosis\\
DiGeorge syndrome\\
Down syndrome\\
Duchenne muscular dyst.\\
Hypercholesterolemia\\
Haemochromatosis\\
Hemophilia\\
Klinefelter syndrome\\
Neurofibromatosis\\
Phenylketonuria\\
Polycystic kidney disease\\
Prader–Willi syndrome\\
Sickle cell disease\\
Spinal muscular atrophy\\
Tay–Sachs disease\\
Turner syndrome\\
}
\end{multicols}

\hrule

\hangpara The Genetics and Rare Disease Center (NIH) lists nearly 3000 known genetic disorders.

\vfilll

\tiny \href{https://en.wikipedia.org/wiki/List_of_genetic_disorders}{Wikipedia List of Genetic Disorders} \hfill \href{https://rarediseases.info.nih.gov/diseases/diseases-by-category/5/congenital-and-genetic-diseases}{\textsc{Genetics and Rare Disease Center}}

\end{frame}

%%

\begin{frame}[c]{\highlight{Mutation-selection balance} is an equilibrium between new mutations and purifying selection.}

\vspace{-1.5\baselineskip}
\begin{tikzpicture}
\filldraw[left color=white, right color=black!70!white] 
(0.5,-0.5)-- (4.5,-0.5)--(4.5,-1)--(5.5,0)--(4.5,1)--(4.5,0.5)--(0.5,0.5)--cycle;
\filldraw[right color=white, left color=black!70!white] 
(11,-0.5)--(7,-0.5)--(7,-1)--(6,0)--(7,1)--(7,0.5)--(11,0.5)--cycle;
\draw [dashed, thick] (5.75,-1.5) -- (5.75,1.5);
\node at (5.75,2.2) {$\hat{p}\approx \dfrac{\mu}{s}$};
\node at (2.5,1) {New mutations $\left(\mu\right)$};
\node at (9.1,1) {Purifying selection $\left(-s\right)$};
%
\node at (6.4,-2.5) {If $\mu = 10^{-6}$ and $s = 0.01$, then $\hat{p} \approx 0.0001.$};
\end{tikzpicture}
\end{frame}

{\setbeamercolor{math text}{fg=gray}
\begin{frame}[t]{\highlight{Mutation load} is the reduction of mean fitness due to deleterious mutations in a population.}

\hangpara Mutation load in humans might reduce $\overline{w}$ by as much as 89\%. (There are a lot of caveats!)

\hangpara \textcolor{gray}{If $U = 2.2$ new mutations per genome per generation\footnotemark, then}
\begin{align*}
L &= 1 - e^{-U}\\
  &= 1-e^{-2.2}\\
  &= 1-0.11 = 0.89.
\end{align*}
 
 \vfilll
 
 \tinyfill{\textsuperscript{1}\,Keightly 2012.~Genetics 190:295}
\end{frame}
}
\end{document}
