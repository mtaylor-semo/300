%!TEX TS-program = lualatex
%!TEX encoding = UTF-8 Unicode

\documentclass[t]{beamer}

%%%% HANDOUTS For online Uncomment the following four lines for handout
%\documentclass[t,handout]{beamer}  %Use this for handouts.
%\includeonlylecture{student}
%\usepackage{handoutWithNotes}
%\pgfpagesuselayout{3 on 1 with notes}[letterpaper,border shrink=5mm]
%	\setbeamercolor{background canvas}{bg=black!5}


%%% Including only some slides for students.
%%% Uncomment the following line. For the slides,
%%% use the labels shown below the command.

%% For students, use \lecture{student}{student}
%% For mine, use \lecture{instructor}{instructor}


%\usepackage{pgf,pgfpages}
%\pgfpagesuselayout{4 on 1}[letterpaper,border shrink=5mm]

% FONTS
\usepackage{fontspec}
\def\mainfont{Linux Biolinum O}
\setmainfont[Ligatures={Common, TeX}, Contextuals={NoAlternate}, BoldFont={* Bold}, ItalicFont={* Italic}, Numbers={Proportional}]{\mainfont}
%\setmonofont[Scale=MatchLowercase]{Inconsolata} 
\setsansfont[Scale=MatchLowercase]{Linux Biolinum O} 
\usepackage{microtype}

\usepackage{graphicx}
	\graphicspath{%
	{/Users/goby/Pictures/teach/300/lectures/}%
	{/Users/goby/Pictures/teach/163/common/}} % set of paths to search for images

\usepackage{amsmath,amssymb}

%\usepackage{units}

\usepackage{booktabs}
\usepackage{multicol}
%	\setlength{\columnsep=1em}

\usepackage{textcomp}
\usepackage{setspace}
\usepackage{tikz}
	\tikzstyle{every picture}+=[remember picture,overlay]

\mode<presentation>
{
  \usetheme{Lecture}
  \setbeamercovered{invisible}
  \setbeamertemplate{items}[square]
}

\usepackage{calc}
\usepackage{hyperref}

\newcommand\HiddenWord[1]{%
	\alt<handout>{\rule{\widthof{#1}}{\fboxrule}}{#1}%
}



\begin{document}

% Lecture goals
\lecture{student}{student}

{
\usebackgroundtemplate{\includegraphics[width=\paperwidth]{selection_flying_frog}}
	\begin{frame}[t]{\textcolor{white}{Natural selection and adaptations}}
	
	\vfilll
	
	\tiny\hfill \textcolor{white}{\textcopyright\,sFutuyma 2017. \textit{Evolution}, 4th ed.}
\end{frame}
}


{
\usebackgroundtemplate{\includegraphics[width=\paperwidth]{adaptation_barrel_cactus}}
	\begin{frame}[b,plain]{\textcolor{white}{How would you define} \highlight{adaptation?}}
	
	\vfilll
	
	\tiny \hfill \rotatebox{90}{\textcolor{white}{André Karwath, \href{https://commons.wikimedia.org/w/index.php?curid=369777}{Wikimedia Commons}. \ccbysa{2.5}}}
	
% Foxes	\tiny\hfill Jonatan Pie, Wikimedia Commons. \cc~v1.0
\end{frame}
}


\begin{frame}[t, plain]{\highlight{Adaptations} are traits that increase reproductive success.}

\vspace{-\baselineskip}

\hangpara Adaptations have been \highlight{modified} from the ancestral form.\bigskip

\includegraphics[width=0.95\textwidth]{adaptation_chough}\\
Are the red eyes of this bird adaptive?

\vfilll

\hfill \tiny White-winged Chough, \textcopyright\,Futuyma 2017. \textit{Evolution}, 4th ed.

\end{frame}


\begin{frame}[t,plain]{Natural selection results in \highlight{fitness} differences among phenotypes.}

\vspace{-\baselineskip}

\begin{multicols}{2}

\hangpara Natural selection occurs when any consistent differences among different groups of \emph{heritable} phenotypes results in \highlight{differential reproductive success.}

\hangpara Fitness has two components. What are they?

\columnbreak

\includegraphics[width=0.9\linewidth]{chap3_fig7a}

\end{multicols}
%\hangpara Natural selection causes variation in \highlight{reproductive success} among individuals that vary in \highlight{heritable traits.}
\end{frame}

\begin{frame}[t,plain]{Fitness is the \textit{average} contribution of allele copies to the next generation.}

\vspace{-\baselineskip}

\begin{multicols}{2}
\hangpara Annual plants grow from seeds each year.

\hangpara Assume that 1 in 500 seeds survive to reproduce. 

\hangpara Those that survive produce an average of 2000 seeds. 

\hangpara What is the average fitness of the plant?

\hangpara \textcolor{gray}{survival frequency $\times$ average number of offspring produced.}
\columnbreak

\begin{center}
\includegraphics[width=0.9\linewidth]{fitness_garden_pea}
\end{center}


\end{multicols}

\vfilll

\tiny \hfill Common garden pea. Photo by Rasbak, \href{https://commons.wikimedia.org/w/index.php?curid=194762}{Wikimedia Commons}. \ccbysa{3}

\end{frame}

\begin{frame}[t,plain]{Selection can occur at the level of }

\vspace{-\baselineskip}

\begin{multicols}{2}
\hangpara genes,

\hangpara \highlight{individuals,}

\hangpara populations, and

\hangpara species.

\columnbreak

\begin{center}
\includegraphics[width=0.8\linewidth]{selection_selfish_gene_cover}
\end{center}

\end{multicols}

\vfilll

\tiny \hfill Source, Fair use, \href{https://en.wikipedia.org/w/index.php?curid=20132319}{Wikipedia} 
\end{frame}

\begin{frame}[t,plain]{\highlight{Genic selection} increases copies of certain genes.}

\begin{multicols}{2}
\hangpara Gene copies increase regardless of effect on individual fitness.

\hangpara Other forms of genic selection include segregation distortion and meiotic drive.

\columnbreak

\includegraphics[width=0.95\linewidth]{selection_genic_transposons}

\end{multicols}

\vfilll

\tiny \hfill Transposons. Modified from Kazasian, H.H. Science 303:1626.

\end{frame}


\begin{frame}[t]{\highlight{Kin selection} increases fitness of relatives.}

\vspace{-\baselineskip}

\begin{multicols}{2}

\hangpara Relatives carry allele copies \highlight{identical by descent.}

\hangpara Individuals sacrifice their own fitness to increase fitness of relatives.

\hangpara Increases copies of their shared alleles.

\columnbreak

\includegraphics[width=0.9\linewidth]{selection_kin_flsj}

\end{multicols}

\vfilll

\tiny \hfill \textcopyright\,Reanna Thomas, \href{https://macaulaylibrary.org/asset/199768971}{Macaulay Library ML199768971}

\end{frame}


\begin{frame}[t]{\highlight{Species selection} increases diversity of characteristics among species.}

\vspace{-\baselineskip}

\begin{multicols}{2}

The proportion of \textit{species} with a particular trait increases over time compared to species without that trait.


\columnbreak

\includegraphics[width=0.9\linewidth]{asexual_rotifer}\\
\includegraphics[width=0.9\linewidth]{asexual_dandelion}
\end{multicols}

\end{frame}


\begin{frame}[t]{Selection \emph{for} traits. \hfill Are all traits adaptive?}

\vspace{-\baselineskip}

\begin{multicols}{2}
\includegraphics[width=0.95\linewidth]{chap3_fig14}

\columnbreak

\includegraphics[width=\linewidth]{chap3_fig15b}
\end{multicols}


\end{frame}

\begin{frame}[t]{Adaptative significance can be tested by experiment and comparative methods.}

\vspace{-\baselineskip}

\begin{multicols}{2}
\includegraphics[width=\linewidth]{/Users/goby/pictures/teach/163/case_studies/pollen_grains_remaining}

\columnbreak

\includegraphics[width=\linewidth]{selection_female_bird_song}

\end{multicols}

\end{frame}

\end{document}
