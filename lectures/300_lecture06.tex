%!TEX TS-program = lualatex
%!TEX encoding = UTF-8 Unicode

\documentclass[t]{beamer}

%%%% HANDOUTS For online Uncomment the following four lines for handout
%\documentclass[t,handout]{beamer}  %Use this for handouts.
%\includeonlylecture{student}
%\usepackage{handoutWithNotes}
%\pgfpagesuselayout{3 on 1 with notes}[letterpaper,border shrink=5mm]
%	\setbeamercolor{background canvas}{bg=black!5}


%%% Including only some slides for students.
%%% Uncomment the following line. For the slides,
%%% use the labels shown below the command.

%% For students, use \lecture{student}{student}
%% For mine, use \lecture{instructor}{instructor}


%\usepackage{pgf,pgfpages}
%\pgfpagesuselayout{4 on 1}[letterpaper,border shrink=5mm]

% FONTS
\usepackage{fontspec}
\def\mainfont{Linux Biolinum O}
\setmainfont[Ligatures={Common, TeX}, Contextuals={NoAlternate}, BoldFont={* Bold}, ItalicFont={* Italic}, Numbers={Proportional}]{\mainfont}
%\setmonofont[Scale=MatchLowercase]{Inconsolata} 
\setsansfont[Scale=MatchLowercase]{Linux Biolinum O} 
\usepackage{microtype}

\usepackage{graphicx}
	\graphicspath{%
	{/Users/goby/Pictures/teach/300/lectures/}%
	{/Users/goby/Pictures/teach/163/common/}} % set of paths to search for images

\usepackage{amsmath,amssymb}

%\usepackage{units}

\usepackage{booktabs}
\usepackage{multicol}
%	\setlength{\columnsep=1em}

\usepackage{textcomp}
\usepackage{setspace}
\usepackage{tikz}
	\tikzstyle{every picture}+=[remember picture,overlay]

\mode<presentation>
{
  \usetheme{Lecture}
  \setbeamercovered{invisible}
  \setbeamertemplate{items}[square]
}

\usepackage{calc}
\usepackage{hyperref}

\newcommand\HiddenWord[1]{%
	\alt<handout>{\rule{\widthof{#1}}{\fboxrule}}{#1}%
}



\begin{document}

% Lecture goals
\lecture{student}{student}

\begin{frame}[t]{Estimating phylogenetic trees.}
	
\includegraphics[width=\linewidth]{origin_phylo_tree}	
\end{frame}

\begin{frame}[t]{But first, some essential vocabulary\dots}

Relate these to “descent with modification.”

\hangpara \highlight{Plesiomorphic} (plesiomorphy): an ancestral character state.

\hangpara \highlight{Apomorphic} (apomorphy): a derived character, changed from the ancestral state.

\hangpara \highlight{Synapomorphic} (synapomorphy): a derived character shared by two or more taxa. Provides evidence of common ancestry.

\hangpara \highlight{Autapomorphic} (autapomorphy): a derived character unique to one species or group. Cannot provide evidence of ancestry because not shared.

\end{frame}

\begin{frame}[t]{But first, some essential vocabulary\dots}

\hangpara \highlight{Outgroup:} a taxon outside the group of interest that helps determine the direction of ancestral change; i.e., distinguish between plesiomorphic and apomorphic characters.

\hangpara \highlight{Mosaic evolution:} characters evolve independently. Species have combinations of plesiomorphic and apomorphic characters.

\end{frame}


\begin{frame}[t,plain]{Clades are united by \highlight{synapomorphies.}}
	
	\centering
	
	\includegraphics[height=0.85\textheight]{futuyma_fig16-2a_synapomorphy}
	
\vfilll

\tiny \hfill Fig.~16-2, \textcopyright\,Futuyma and Kirkpatrick.
\end{frame}
%
\begin{frame}[t,plain]{Homoplasies are the result of \highlight{convergent evolution.}}

	\centering

	\includegraphics[height=0.85\textheight]{futuyma_fig16-2a_homoplasy}

	\vfilll

	\tiny \hfill Fig.~16-2, \textcopyright\,Futuyma and Kirkpatrick.
\end{frame}
%
%
\begin{frame}[t,plain]{Trees built using homoplasies are incorrect.}

\includegraphics[width=\linewidth]{futuyma_fig16-2}

\hangpara The left tree requires \emph{fewer} evolutionary changes.

\vfilll

\tiny \hfill Fig.~16-2, \textcopyright\,Futuyma and Kirkpatrick.
\end{frame}
%
%
\begin{frame}[t,plain]{Trees can be estimated with different techniques.}

\hangpara \highlight{Maximum Parsimony:} the best estimate of the phylogeny requires the fewest evolutionary changes.

\hangpara \highlight{Maximum Likelihood:} uses evolutionary models of genetic evolution.

\hangpara \highlight{Bayesian:} also uses evolutionary models of genetic evolution.

\end{frame}
%
\begin{frame}[t,plain]{Confidence: how good is the hypothesis?}

\hangpara \highlight{Bootstrap:} \textcolor{gray}{(non-parametric)} 

\quad Evaluates confidence of each node.
Values between 50-100.\\  

\smallskip

\quad Values above 70 are good.

\hangpara \highlight{Bayesian:} \textcolor{gray}{(parametric)} 

\quad Simultaneously infers phylogeny \& evaluates confidence of each node.
\\

\quad Values above 80\% are good (above 95\% for greatest confidence).

\end{frame}
%
\begin{frame}[t,plain]

	\includegraphics[height=0.94\textheight]{taylor_hellberg_phylogeny}

	\vfilll
	
	\tiny \hfill Taylor and Hellberg 2005.
	
	\begin{tikzpicture}
	\node at (10.5cm,8cm) [text width=6cm]{\large Bootstrap (upper or left)\\Bayesian (lower or right)};
	\end{tikzpicture}
\end{frame}

{
\usebackgroundtemplate{\includegraphics[width=\paperwidth]{zimmer_fig14-13}}
\begin{frame}[b]

	\begin{tikzpicture}
\node at (10cm,2cm) [align=left, text width=6cm]{0 = plesiomorphic character.\\1 = apomorphic character.};
\end{tikzpicture}


	
%	\vfilll

	\tiny\hfill \textcopyright\,Zimmer and Emlen 2016, 2nd ed.
\end{frame}
}
%
{
	\usebackgroundtemplate{\includegraphics[width=\paperwidth]{zimmer_fig14-15}}
	\begin{frame}[b]
	
	%	\vfilll
	
	\tiny \textcopyright\,Zimmer and Emlen 2016, 2nd ed.
\end{frame}
}



\end{document}
