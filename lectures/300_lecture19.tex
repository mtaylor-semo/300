%!TEX TS-program = lualatex
%!TEX encoding = UTF-8 Unicode

\documentclass[t]{beamer}

%%%% HANDOUTS For online Uncomment the following four lines for handout
%\documentclass[t,handout]{beamer}  %Use this for handouts.
%\usepackage{handoutWithNotes}
%\includeonlylecture{student}
%\pgfpagesuselayout{3 on 1 with notes}[letterpaper,border shrink=5mm]

\usefonttheme{professionalfonts}


%%% Including only some slides for students.
%%% Uncomment the following line. For the slides,
%%% use the labels shown below the command.

%% For students, use \lecture{student}{student}
%% For mine, use \lecture{instructor}{instructor}

% FONTS
\usepackage{fontspec}
\def\mainfont{Linux Biolinum O}
\setmainfont[Ligatures={Common, TeX}, Contextuals={NoAlternate}, BoldFont={* Bold}, ItalicFont={* Italic}, Numbers={Proportional, OldStyle}]{\mainfont}
%\setmonofont[Scale=MatchLowercase]{Inconsolata} 
\setsansfont[Scale=MatchLowercase]{Linux Biolinum O} 
\setmonofont{Linux Libertine Mono O}

\usepackage{microtype}

\usepackage{unicode-math}
\setmathfont[Scale=MatchLowercase]{Asana Math}

\usepackage{graphicx}
	\graphicspath{%
	{/Users/goby/Pictures/teach/300/lectures/}%
	{/Users/goby/Pictures/teach/438/lectures/}} % set of paths to search for images

\usepackage{amsmath,amssymb}

%\usepackage{units}
\usepackage{mhchem}
\usepackage{booktabs}
\usepackage{array}
\newcolumntype{L}[1]{>{\raggedright\let\newline\\\arraybackslash\hspace{0pt}}p{#1}}
\newcolumntype{C}[1]{>{\centering\let\newline\\\arraybackslash\hspace{0pt}}p{#1}}
\newcolumntype{R}[1]{>{\raggedleft\let\newline\\\arraybackslash\hspace{0pt}}p{#1}}


\usepackage{multicol}
%	\setlength{\columnsep=1em}
\usepackage{enumitem}
\usepackage{textcomp}
\usepackage{setspace}
\usepackage{tikz}
	\tikzstyle{every picture}+=[remember picture,overlay]
\usetikzlibrary{arrows}

\mode<presentation>
{
  \usetheme{Lecture}
  \setbeamercovered{invisible}
  \setbeamertemplate{items}[square]
}

\usepackage{calc}
\usepackage{hyperref}

\newcommand\HiddenWord[1]{%
	\alt<handout>{\rule{\widthof{#1}}{\fboxrule}}{#1}%
}


\usepackage{xifthen}
\newcommand{\futuyma}[1]{%
	\ifthenelse{\isempty{#1}}%
	{Futuyma \& Kirkpatrick 2017, 4th ed.}%
	{Fig.~#1~Futuyma \& Kirkpatrick 2017, 4th ed.}%
}

% This defines \amper for the fancy ampersand
% to be used in the header. See
% https://tex.stackexchange.com/a/58185/39194
\usepackage{xspace}
\newfontfamily\amperfont[Style=Alternate]{Linux Libertine O}    
\makeatletter
\DeclareRobustCommand{\amper}{{\amperfont\ifx\f@shape\scname\smaller[1.2]\fi\&}\xspace}
\makeatother

\newcommand{\backskip}{\vspace{-0.5\baselineskip}}

%% Remove indent from multicol.
\parindent=0pt

\begin{document}

\lecture{student}{student}

{
\usebackgroundtemplate{\includegraphics[width=\paperwidth]{diversity_intro}}
\begin{frame}[b]

%\tinyfill  \textcolor{white}{\futuyma{}}
\end{frame}
}

%%

\begin{frame}{Fossil formation requires specific conditions.}

\backskip

\includegraphics[height=0.83\textheight]{diversity_fossil_formation}
	

\tinyfill \textcopyright Zimmer and Emlen, 2015. \textit{Evolution}

\end{frame}

%%


\begin{frame}{The fossil record will always be incomplete.}

\backskip

\begin{multicols}{2}

\hangpara Some periods are poorly represented.

\hangpara Many lineages represented only by early and late stages.

\hangpara Some taxa known from few specimens.

\columnbreak

\centering
\noindent\includegraphics[width=\linewidth]{diversity_brachiopod}\\
Brachiopod\,—\,early Permian

\end{multicols}

\end{frame}

%%

\begin{frame}{The fossil record has three types of bias.}

\backskip

\begin{multicols}{2}

\hangpara \highlight{Geographic:} most fossils come from lowland or marine habitats.


\hangpara \highlight{Taxonomic:} many species lack hard parts.


\hangpara \highlight{Temporal:} Older geologic formations tend to be obscured.

\columnbreak

\centering
\noindent\includegraphics[width=0.7\linewidth]{diversity_fossil_fish}

\smallskip

\noindent\includegraphics[width=0.7\linewidth]{diversity_fossil_insect}\\
fishes and insect\,—\,Eocene

\end{multicols}

\end{frame}


%%

\begin{frame}{\highlight{Diversification rate $\left(D\right)$} in the fossil record can be calculated by:}

\vspace*{-2\baselineskip}

{\Large
\begin{equation*}
D = S - E
\end{equation*}
}% LARGE

%\bigskip

\hangpara $S =$ speciation (origination of new taxa),
and\\
$E =$ extinction of existing taxa.

\medskip


\hangpara If $D > 0$, then diversity increases. 

\hangpara If $D < 0$, then diversity
decreases.

\end{frame}

%%

\begin{frame}{The rate of change for taxa \emph{per unit time} is}

\vspace*{-\baselineskip}

{\Large
\begin{equation*}
\dfrac{\Delta N}{\Delta T} = \left(S - E\right)N = DN
\end{equation*}
}% LARGE

%\bigskip

\hangpara $N =$ number of current taxa.


\end{frame}

%%

\begin{frame}{Marine diversity has increased over the last 600 \textsc{mya}$\dots$}

\backskip

\centering

\includegraphics[height=0.83\textheight]{diversity_marine}
	
\tinyfill See \futuyma{19.4}

\end{frame}

%%

\begin{frame}{$\dots$as has terrestrial diversity.}

\backskip

\begin{multicols}{2}

\centering

\includegraphics[width=0.9\linewidth]{diversity_insects}

\smallskip

\includegraphics[width=0.9\linewidth]{diversity_tetrapods}

\columnbreak

\includegraphics[width=0.9\linewidth]{diversity_plants}
\end{multicols}

\tinyfill \textcopyright Sinauer Assoc.~2005.
\end{frame}

%%

\begin{frame}{Have origination and extinction rates changed over time?}

\backskip

\includegraphics[width=\linewidth]{diversity_orig_ext}
	
\tinyfill \textcopyright Zimmer and Emlem, 2015. \textit{Evolution}

\end{frame}


%%


\begin{frame}{Origination rates $\left(S\right)$ have tended to decrease.}

\backskip

\centering

\includegraphics[width=\linewidth]{diversity_originations}
	
\tinyfill \textcopyright Sinauer Assoc.~2005

\end{frame}

%%

\begin{frame}{Extinction rates $\left(E\right)$ have tended to decrease.}

\backskip

\centering

\includegraphics[height=0.83\textheight]{diversity_extinctions}

\tinyfill \textcopyright Zimmer and Emlem 2015. \textit{Evolution}

\end{frame}

%

\begin{frame}{\highlight{Turnover rate} is the link between S and E within taxa.}

\backskip

\begin{multicols}{2}
{\Large
\begin{equation*}
D = S - E
\end{equation*}
}% LARGE

\columnbreak

\hangpara \textcolor{gray}{$D > 0$: increasing diversity}

\hangpara \textcolor{gray}{$D < 0$: decreasing diversity}

\end{multicols}

\hangpara When compared among taxa:
\\
\hspace*{1em} Low origination = low extinction
\\
\hspace*{1em} High origination = high extinction

\end{frame}

%
\begin{frame}{How can turnover rate be explained?}

\backskip

\begin{multicols}{2}

\hangpara \highlight{Ecological specialization}

\hangpara \highlight{Population dynamics}

\hangpara \highlight{Geographic range}

\columnbreak

\centering
\noindent\includegraphics[width=\linewidth]{diversity_ediacaran_example}

\end{multicols}
\end{frame}
%

\begin{frame}{How can turnover rate be explained?}

\backskip

\begin{multicols}{2}

\noindent\includegraphics[width=\linewidth]{diversity_ediacaran_example}

\columnbreak

Did taxa with high turnover rates have specialized ecologies, low pop sizes, or small ranges?


\bigskip

Does loss of taxa with high turnover rates explain decreased rate of extinction and origination?

\end{multicols}
\end{frame}

%%

\begin{frame}{Mass extinctions cause diversity to decrease rapidly.}

\backskip

\centering

\includegraphics[height=0.83\textheight]{diversity_marine}
	
\tinyfill See \futuyma{19.4}

\end{frame}

%%


\begin{frame}{Five mass extinctions have occurred since the Cambrian.}

\backskip

\centering

\includegraphics[height=0.83\textheight]{diversity_mass_extinctions}
	
\tinyfill \textcopyright Pearson Prentice Hall 2004.

\end{frame}

%%

\begin{frame}{At least 75\% of species went extinct at each event.}



\centering

\begin{tabular}{@{}lcc@{}}
\toprule
Event & Genera & Species \tabularnewline
\midrule
End Ordovician & 57\% & 86\% \tabularnewline
Late Devonian & 35\%  & 75\% \tabularnewline
End Permian & 56\%  &  96\% \tabularnewline
End Triassic & 47\% & 80\% \tabularnewline
End Cretaceous \textsc{(k/t)} & 40\% & 76\% \tabularnewline
\bottomrule
\end{tabular}
\end{frame}

%%

\begin{frame}{Rapid climate change is the ultimate cause of mass extinctions.}

\backskip
\begin{multicols}{2}
\hangpara Asteroid impact

\hangpara Volcanism

\hangpara glaciation

\hangpara Methane gas

\hangpara Changing ocean currents\\
\hspace*{1em} Anoxia

\columnbreak

\noindent Not all of these apply to all mass extinction events

\end{multicols}

\end{frame}

%%


\begin{frame}{Did mass extinctions coincide with impacts or volcanism?}

\backskip

\centering

\includegraphics[width=\linewidth]{diversity_volcanism_asteroids}


\end{frame}

%%


\begin{frame}{A 65 \textsc{my} old global iridium layer supports an asteroid impact.}

\backskip

\centering

\includegraphics[height=0.78\textheight]{diversity_iridium_layer}
	
\tinyfill \textcopyright Zimmer and Emlen 2015. \textit{Evolution}

\end{frame}

%%


\begin{frame}{Sediments, $\delta$\textsuperscript{13}C, and foraminiferan changes support impact.}

%\backskip

\centering

%\includegraphics[height=0.8\textheight]{diversity_impact_evidence2}

\includegraphics[width=\linewidth]{diversity_impact_evidence}
	
\tinyfill Keller 2007. Geol Soc Am Spec Pap 437:147

\end{frame}

%%



\begin{frame}{180~km Chicxulub\textsuperscript{1} crater caused by 10~km asteroid.}

\backskip

\begin{multicols}{2}

\includegraphics[width=\linewidth]{diversity_impact_site}

\columnbreak

\includegraphics[width=\linewidth]{diversity_impact_sonar}
\end{multicols}

\centering

\vfilll
	
\tiny \textcolor{gray}{1. Cheek-she-loob (Wikipedia)}


\end{frame}

%%

\begin{frame}{The Chicxulub impact caused negative global effects.}

\backskip

\begin{multicols}{2}

\hangpara Atmospheric debris blocked sunlight for at least 6 months.
\\
\hspace*{1em} No photosynthesis.
\\
\hspace*{1em} Cooling, followed by heating.

\hangpara Global wildfires.

\hangpara Acid rain?

\columnbreak

\noindent \includegraphics[width=0.7\linewidth]{diversity_shocked_quartz}\\
\noindent Shocked quartz in \textsc{k/t} boundary. Shocked quartz known only from impact sites (and nuclear test sites).


\end{multicols}

\vfilll
	
\tiny \textcolor{gray}{1. Cheek-she-loob (Wikipedia)}

\end{frame}


%%

\begin{frame}{The Siberian traps produced lava ca.~251–250 \textsc{mya.}}

\backskip

\includegraphics[width=\linewidth]{diversity_siberian_traps}

\end{frame}


%%

\begin{frame}{Volcanism caused global warming.}

\backskip

\begin{multicols}{2}

\hangpara Temperature spike could have caused release of \ce{CH4}, \ce{CO2}, and water vapor.


\hangpara Warming altered ocean circulation, causing an anoxic ocean.


\columnbreak

\centering

\noindent\includegraphics[width=0.8\linewidth]{diversity_geological_temps}

\end{multicols}

\vfilll
	
\end{frame}

%%


\begin{frame}{\ce{CO2} increase coincides with End Permian.}

\backskip

\centering

\includegraphics[height=0.83\textheight]{diversity_permian_co2}
	
\tinyfill \textcopyright Zimmer and Emlen 2015. \textit{Evolution}

\end{frame}

%%

%%%


\begin{frame}{Why has diversity increased for some taxa but not others?  Why has diversity increased since the Permian?}

\backskip

\begin{multicols}{2}

\includegraphics[width=\linewidth]{diversity_marine}


\columnbreak

%\centering
%\includegraphics[width=\linewidth]{diversity_ammonite}\\
%ammonite


\hangpara \highlight{Ecological Release}

\hangpara \highlight{Ecological Divergence}

\hangpara \highlight{Coevolution}


\end{multicols}

\end{frame}

%%

\begin{frame}{\highlight{Ecological release} allows taxa to exploit new ecological opportunities.}

%\backskip

\centering

\includegraphics[width=\linewidth]{diversity_ecological_release}
	
\tinyfill \textcopyright Sinauer Assoc.~2005.

\end{frame}

%

\begin{frame}{\highlight{Ecological divergence} can occur when taxa adapt to new niches.}

%\backskip

\centering

\includegraphics[width=\linewidth]{futuyma_fig19-13}
	
\tinyfill \futuyma{19.13}

\end{frame}

%%

\begin{frame}{Taxa coevolve as part of a community.}

\hangpara Species exist in a community.

\hangpara Species are resources for other species.

\hangpara As one group diversifies, so does the other group.

\hangpara Species interactions drives evolution of diversity.

\end{frame}

%


%%
\end{document}


