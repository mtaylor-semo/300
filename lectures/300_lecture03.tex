%!TEX TS-program = lualatex
%!TEX encoding = UTF-8 Unicode

\documentclass[t]{beamer}

%%%% HANDOUTS For online Uncomment the following four lines for handout
%\documentclass[t,handout]{beamer}  %Use this for handouts.
%\includeonlylecture{student}
%\usepackage{handoutWithNotes}
%\pgfpagesuselayout{3 on 1 with notes}[letterpaper,border shrink=5mm]
%	\setbeamercolor{background canvas}{bg=black!5}


%%% Including only some slides for students.
%%% Uncomment the following line. For the slides,
%%% use the labels shown below the command.

%% For students, use \lecture{student}{student}
%% For mine, use \lecture{instructor}{instructor}


%\usepackage{pgf,pgfpages}
%\pgfpagesuselayout{4 on 1}[letterpaper,border shrink=5mm]

% FONTS
\usepackage{fontspec}
\def\mainfont{Linux Biolinum O}
\setmainfont[Ligatures={Common, TeX}, Contextuals={NoAlternate}, BoldFont={* Bold}, ItalicFont={* Italic}, Numbers={Proportional}]{\mainfont}
%\setmonofont[Scale=MatchLowercase]{Inconsolata} 
\setsansfont[Scale=MatchLowercase]{Linux Biolinum O} 
\usepackage{microtype}

\usepackage{graphicx}
	\graphicspath{%
	{/Users/goby/Pictures/teach/300/lectures/}%
	{/Users/goby/Pictures/teach/163/common/}} % set of paths to search for images

\usepackage{amsmath,amssymb}

%\usepackage{units}

\usepackage{booktabs}
\usepackage{multicol}
%	\setlength{\columnsep=1em}

\usepackage{textcomp}
\usepackage{setspace}
\usepackage{tikz}
	\tikzstyle{every picture}+=[remember picture,overlay]

\mode<presentation>
{
  \usetheme{Lecture}
  \setbeamercovered{invisible}
  \setbeamertemplate{items}[square]
}

\usepackage{calc}
\usepackage{hyperref}

\newcommand\HiddenWord[1]{%
	\alt<handout>{\rule{\widthof{#1}}{\fboxrule}}{#1}%
}



\begin{document}

% Lecture goals
\lecture{student}{student}

{
\usebackgroundtemplate{\includegraphics[width=\paperwidth]{hwe_gorilla}}
\begin{frame}[t]

\tinyfill\textcolor{white}{\href{https://www.flickr.com/photos/34209020@N02/9612772316}{Elizabeth Haslam, Flickr,}~\ccbync{2.0}} 

\end{frame}
}


\lecture{student}{student}

\begin{frame}[t]{Questions! Answers?}

\vspace{-\baselineskip}

\hangpara What is a genotype?  \mode<beamer>{\pause \newline \hspace*{1em} \textbf{The combination of alleles at one or more genes. More broadly, the sequence of \textsc{dna} in an individual's chromosomes.}}

\pause

\hangpara  What is a phenotype? \mode<beamer>{\pause \newline \hspace*{1em} \textbf{The physical expression of a trait.}}

\pause

\hangpara Do proteins have phenotypes? \mode<beamer>{\pause \quad \textbf{Absolutely!}}

\hangpara (T/F) The phenotype is determined only by the genotype of a gene. \mode<beamer>{\pause \quad \textbf{False. The environment can influence the phenotype of some traits.}}

\pause

\hangpara Extra credit: What is a locus? \mode<beamer>{\pause \quad \textbf{The location of a \textsc{dna} sequence on a chromosome.}}


\end{frame}


\begin{frame}
\frametitle<1>{What is allele frequency?}
\frametitle<2>{\highlight{Allele frequency} is a measure of how common an allele is for a locus in a population.}

\onslide<2>{%
\vspace{-0.5\baselineskip}
\centering

\includegraphics[height=0.82\textheight]{futuyma_fig4-5}

\begin{tikzpicture}
\node at (2,7.5) [font=\large, text width = 5.5cm, align=left] {$A_1$ frequency = 10/16 = 0.675.};

\node at (2, 6.5) [font=\large, text width = 5.5cm, align=left] {$A_2$ frequency = 6/16 = 0.325.};

\end{tikzpicture}

}% end onslide<2>
\end{frame}


%

\begin{frame}[t]
\frametitle<1>{What is genotype frequency?}
\frametitle<2>{\highlight{Genotype frequency} is a measure of how common a genotype is for one or more loci in a population.}

\onslide<2>{%
\vspace{-0.5\baselineskip}
\centering

\includegraphics[height=0.82\textheight]{futuyma_fig4-5}

\begin{tikzpicture}
\node at (2,7.5) [font=\large, text width = 5.5cm, align=left] {$A_1A_1$ frequency = 4/8 = 0.5.};

\node at (2, 6.75) [font=\large, text width = 5.5cm, align=left] {$A_1A_2$ frequency = 2/8 = 0.25.};

\node at (2, 6) [font=\large, text width = 5.5cm, align=left] {$A_2A_2$ frequency = 2/8 = 0.25.};

\end{tikzpicture}
} %end onslide<2>
\end{frame}


\begin{frame}{Assume a population of diploid individuals with only two alleles for a gene locus.}

\hangpara What is the Hardy-Weinberg equation for allele frequency?

\pause

\hangpara \highlight{$p + q = 1$} 

\pause

\hangpara What is the Hardy-Weinberg equation for genotype frequency?

\pause

\hangpara \highlight{$p^2 + 2pq + q^2 = 1.$}

\vspace{2\baselineskip}

\hangpara \textcolor{gray}{Recall that $(p + q)^2 = 1^2$}


\end{frame}

%

{
\usebackgroundtemplate{\includegraphics[width=\paperwidth]{practice_progress}}
\begin{frame}[t]

\tinyfill \textcolor{white}{\href{https://www.flickr.com/photos/98195299@N00/296747958}{Steven S., Flickr}, \ccby{2.0}}
\end{frame}
}

%

%\begin{frame}[t]{Consider a population where$\dots$}
%\hangpara Half the individuals are $A_1A_1$
%
%\hangpara Half the individuals are $A_2A_2$
%
%\hangpara What is the frequency of each allele?
%
%\hangpara What is the frequency of each genotype?
%
%\end{frame}
%
%%%
%
%%
%
%\begin{frame}[t]{If a very large population mates at random, what will be$\dots$}
%
%\hangpara the allele frequencies in the next generation?
%
%\hangpara the genotype frequencies in the next generation?
%
%\hangpara Will subsequent generations maintain the same frequencies?
%
%\vfilll
%
%\hangpara \textcolor{gray}{Starting freqs in first generation were 0.5 each.}
%\end{frame}



%\begin{frame}[t]
%
%\hangpara If $p$ and $q$ are the frequencies of two alleles at one locus, then
%
%\hangpara $p^2$ and $q^2$ are the frequencies of each possible homozygote, and 
%
%\hangpara $2pq$ is the frequency of heterozygotes.
%
%\hangpara \highlight{$p + q = 1$} and \highlight{$p^2 + 2pq + q^2 = 1.$}
%
%\hangpara \textcolor{gray}{Recall that $(p + q)^2 = 1^2$}
%
%\end{frame}

%%

\begin{frame}[t]{Hardy-Weinberg equations can be extended to more than two alleles.}

For a diploid population with three alleles in equilibrium, then

\hangpara $p + q + r = 1,$ and

\hangpara $(p + q + r)^2 = 1^2,$ so

\hangpara $p^2 + 2pq + q^2 + 2pr + 2qr + r^2 = 1$


\end{frame}


\begin{frame}[t]{Assume a sample of 1000 individuals in \textsc{hwe}.}

\hangpara $q^2 = 0.2$ and $r^2 = 0.05.$

\hangpara Calculate the three allele frequencies and the other four genotype frequencies.
\end{frame}


\lecture{instructor}{instructor}

\begin{frame}[t]{Solution: If $q^2 = 0.2$ and $r^2 = 0.05$ then $\dots$}

\vspace{-0.5\baselineskip}

\hangpara $q = \sqrt{0.2} = 0.447$ and $r = \sqrt{0.05} = 0.224.$

\pause \hangpara $p = 1 - 0.447 - 0.224 = 0.329.$

\pause \hangpara $p^2 = 0.329^2 = 0.108.$

\pause \hangpara $2pq = (2)(0.329)(0.447) = 0.294.$

\pause \hangpara $2pr = (2)(0.329)(0.224) = 0.147.$

\pause \hangpara $2qr = (2)(0.447)(0.224) = 0.200.$

\pause \hangpara $0.108 + 0.200 + 0.050 + 0.294 + 0.147 + 0.200 = 1.$ (with rounding error)

\end{frame}

%

\begin{frame}[t]
\frametitle<1>{What are the five assumptions of Hardy-Weinberg equilibrium?}
\frametitle<2>{A population will reach or maintain Hardy-Weinberg equilibrium if}

\onslide<2>{%
\hangpara the population is infinitely large,

\hangpara mating is random,

\hangpara mutations do not occur,

\hangpara gene flow does not occur, and

\hangpara natural selection does not occur.

\hangpara \highlight {A population will evolve if \emph{any} one of these assumptions is violated.}

\vfilll

\textcolor{gray}{Equilibrium will be reached in one generation.}
}%end onslide<2>
\end{frame}

%%
\end{document}
