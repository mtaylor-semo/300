%!TEX TS-program = lualatex
%!TEX encoding = UTF-8 Unicode

\documentclass[t]{beamer}

%%%% HANDOUTS For online Uncomment the following four lines for handout
%\documentclass[t,handout]{beamer}  %Use this for handouts.
%\usepackage{handoutWithNotes}
%\includeonlylecture{student}
%\pgfpagesuselayout{3 on 1 with notes}[letterpaper,border shrink=5mm]
%	\setbeamercolor{background canvas}{bg=black!5}


%%% Including only some slides for students.
%%% Uncomment the following line. For the slides,
%%% use the labels shown below the command.

%% For students, use \lecture{student}{student}
%% For mine, use \lecture{instructor}{instructor}


%\usepackage{pgf,pgfpages}
%\pgfpagesuselayout{4 on 1}[letterpaper,border shrink=5mm]

% FONTS
\usepackage{fontspec}
\def\mainfont{Linux Biolinum O}
\setmainfont[Ligatures={Common, TeX}, Contextuals={NoAlternate}, BoldFont={* Bold}, ItalicFont={* Italic}, Numbers={Proportional}]{\mainfont}
%\setmonofont[Scale=MatchLowercase]{Inconsolata} 
\setsansfont[Scale=MatchLowercase]{Linux Biolinum O} 
\usepackage{microtype}

\usepackage{graphicx}
	\graphicspath{%
	{/Users/goby/Pictures/teach/300/lectures/}%
	{/Users/goby/Pictures/teach/163/common/}} % set of paths to search for images

\usepackage{amsmath,amssymb}

%\usepackage{units}

\usepackage{booktabs}
\usepackage{multicol}
%	\setlength{\columnsep=1em}

\usepackage{textcomp}
\usepackage{setspace}
\usepackage{tikz}
	\tikzstyle{every picture}+=[remember picture,overlay]

\mode<presentation>
{
  \usetheme{Lecture}
  \setbeamercovered{invisible}
  \setbeamertemplate{items}[square]
}

\usepackage{calc}
\usepackage{hyperref}

\newcommand\HiddenWord[1]{%
	\alt<handout>{\rule{\widthof{#1}}{\fboxrule}}{#1}%
}



\begin{document}

% Lecture goals
\lecture{student}{student}

{
\usebackgroundtemplate{\includegraphics[width=\paperwidth]{darwin_phylo_tree}}
\begin{frame}[t]
\end{frame}
}


\begin{frame}[t,plain]{\highlight{Phylogenetic systematics} estimates evolutionary relationships.}
	
	\vspace{-0.75\baselineskip}
	
	\includegraphics[width=\linewidth]{phylogeny_genealogy_comparison}
	
	\vfilll
	\tiny \hfill \textcopyright\,Zimmer and Emlen
	
% Foxes	\tiny\hfill Jonatan Pie, Wikimedia Commons. \cc~v1.0
\end{frame}

\begin{frame}[t,plain]{You must learn the basic parts of a phylogenetic tree.}
	
	\centering
%	\vspace{-0.5\baselineskip}
	
	\includegraphics[width=\linewidth]{futuyma_fig2-6a}
	
	\vfilll
	
	\tiny \hfill Fig. 2.6~\textcopyright\,Futuyma and Kirkpatrick.
	
\end{frame}

\begin{frame}[t,plain]{\highlight{Clades} include an ancestor and \emph{all} of its descendants.}
	
	\centering
	\vspace{-0.5\baselineskip}
	
	\includegraphics[width=\linewidth]{phylogeny_clades}
	
	\hangpara Trees can be rearranged but still show the same relationships.
	
	\vfilll
	\tiny \hfill \textcopyright\,Zimmer and Emlen.
\end{frame}

\begin{frame}[t,plain]{\highlight{Clades} can include one or more branches.}
	
	\centering
	\vspace{-0.5\baselineskip}
	
	\includegraphics[width=\linewidth]{clades_collapse}
	
	\vfilll
	\tiny \hfill \textcopyright\,Zimmer and Emlen.
\end{frame}

\begin{frame}[t,plain]{Clades are \highlight{monophyletic} groups.}
	
	\centering
	\vspace{-0.5\baselineskip}
	
	\includegraphics[width=\linewidth]{monophyly_paraphyly_polyphyly}
	
	\vfilll
	\tiny \hfill \textcopyright\,Zimmer and Emlen.
\end{frame}
%
\begin{frame}[t,plain]{Linnean classification is based on nested clades.}
	
	\centering
	\vspace{-0.5\baselineskip}
	
	\includegraphics[height=0.85\textheight]{phylogeny_classification}
	
	\vfilll
	\tiny \hfill \textcopyright\,Zimmer and Emlen.
\end{frame}
%
\begin{frame}[t,plain]
	
	\centering
%	\vspace{-0.5\baselineskip}
	
	\includegraphics[width=\linewidth]{phylogeny_example}
	
	\vfilll
	\tiny \hfill \textcopyright\,Zimmer and Emlen.
	
	\pause
	
	\begin{tikzpicture}
	\draw [draw=white, fill=white] (-0.5,1) rectangle ++(6.7,8.5);
	\node at (3,8) [draw=white, align=left, font=\large, text width=4cm] {Which groups are \emph{not} monophyletic?};
	\pause
	\node at (3,6) [draw=white, align=left, font=\large, text width=4cm] {Are they paraphyletic\\ or polyphyletic?};
%	\node (rect) at (-0.5,1) [draw, minimum width=6.7cm, minimum height=2cm] {TEST};
	\end{tikzpicture}
`\end{frame}
%
\begin{frame}[t,plain]{\highlight{Gene trees} show the evolutionary history of \textsc{dna} sequences.}

	\centering
	\vspace{-0.75\baselineskip}

	\includegraphics[height=0.85\textheight]{futuyma_fig2-13}

\end{frame}
%
\begin{frame}[t,plain]{\highlight{Paralogous} genes arise via \highlight{gene duplication}.}

	\centering
	\vspace{-0.25\baselineskip}

	\includegraphics[height=0.85\textheight]{futuyma_fig2-14}

\end{frame}

%
\begin{frame}[t,plain]{\highlight{Orthologous} genes arise from speciation.}

	\centering
	\vspace{-0.25\baselineskip}

	\includegraphics[height=0.85\textheight]{futuyma_fig2-14}

\end{frame}
%
\begin{frame}[t,plain]{Gene duplication can give rise to gene families.}

	\centering
	\vspace{-0.25\baselineskip}

	\includegraphics[height=0.85\textheight]{futuyma_fig2-15}

\end{frame}
%
%
\begin{frame}[t,plain]{Orthologs have a shared evolutionary history.\\ Paralogs do not.}

	\centering
	\vspace{-0.75\baselineskip}

	\includegraphics[height=0.82\textheight]{orthologs_paralogs_hemoglobin}

\end{frame}
%
\begin{frame}[t,plain]{Gene trees can have different topologies than their species trees.}

	\centering
%	\vspace{-0.75\baselineskip}

	\includegraphics[width=\linewidth]{futuyma_fig16-6}

	\vfilll
	
	\tiny \hfill Fig.~16.6 \textcopyright\,Futuyma and Fitzpatrick
\end{frame}
%
%
\begin{frame}[t,plain]{Rapid diversification to fill empty niches is an \highlight{adaptive radiation.}}

	\centering
	\vspace{-0.75\baselineskip}

	\includegraphics[height=0.85\textheight]{futuyma_fig2-24}

\end{frame}
%
\begin{frame}[t,plain]

	\centering
%	\vspace{-0.75\baselineskip}

	\includegraphics[height=0.9\textheight]{taylor_hellberg_phylogeny}
	
	\vfilll
	
	\tiny \hfill Taylor and Hellberg~2005.

\end{frame}
%
\begin{frame}[t,plain]{\highlight{Incomplete lineage sorting} is the retention of genetic polymorphisms after speciation.}

	\hangpara Descendant species retain ancestral genetic states.
	
	\hangpara I\textsc{ls} can cause incorrect inference of the evolutionary history \emph{of the species.}

\end{frame}
%

\begin{frame}[t,plain]{Summary}
\hangpara Phylogenetic trees are hypotheses of evolutionary history.

\hangpara Monophyletic groups include the ancestor and all descendants.

\hangpara Paralogous genes arise from gene duplication.

\hangpara Orthologous genes arise from speciation.

\hangpara Adaptive radiations result from rapid diversification to fill empty ecological niches.

\hangpara Gene trees can differ from their species trees due to incomplete lineage sorting.

\end{frame}


\end{document}
