%!TEX TS-program = lualatex
%!TEX encoding = UTF-8 Unicode

\documentclass[t]{beamer}

%%%% HANDOUTS For online Uncomment the following four lines for handout
%\documentclass[t,handout]{beamer}  %Use this for handouts.
%\usepackage{handoutWithNotes}
%\includeonlylecture{student}
%\pgfpagesuselayout{3 on 1 with notes}[letterpaper,border shrink=5mm]

%\usefonttheme{professionalfonts}


%%% Including only some slides for students.
%%% Uncomment the following line. For the slides,
%%% use the labels shown below the command.

%% For students, use \lecture{student}{student}
%% For mine, use \lecture{instructor}{instructor}

% FONTS
\usepackage{fontspec}
\def\mainfont{Linux Biolinum O}
\setmainfont[Ligatures={Common, TeX}, Contextuals={NoAlternate}, BoldFont={* Bold}, ItalicFont={* Italic}, Numbers={Proportional, OldStyle}]{\mainfont}
%\setmonofont[Scale=MatchLowercase]{Inconsolata} 
\setsansfont[Scale=MatchLowercase]{Linux Biolinum O} 
\setmonofont{Linux Libertine Mono O}

\usepackage{microtype}

\parindent=0pt

\usepackage{unicode-math}
\setmathfont[Scale=MatchLowercase]{Asana Math}

\usepackage{graphicx}
	\graphicspath{%
	{/Users/goby/Pictures/teach/300/lectures/}%
	{/Users/goby/Pictures/teach/163/lecture/}} % set of paths to search for images

\usepackage{amsmath,amssymb}

%\usepackage{units}

\usepackage{booktabs}
\usepackage{array}
\newcolumntype{L}[1]{>{\raggedright\let\newline\\\arraybackslash\hspace{0pt}}p{#1}}
\newcolumntype{C}[1]{>{\centering\let\newline\\\arraybackslash\hspace{0pt}}p{#1}}
\newcolumntype{R}[1]{>{\raggedleft\let\newline\\\arraybackslash\hspace{0pt}}p{#1}}


\usepackage{multicol}
%	\setlength{\columnsep=1em}
\usepackage{enumitem}
\usepackage{textcomp}
\usepackage{setspace}
\usepackage{tikz}
	\tikzstyle{every picture}+=[remember picture,overlay]
	\usetikzlibrary{arrows, arrows.meta}
	\usetikzlibrary{positioning, backgrounds}

\usepackage{forest}
\forestset{
    every leaf node/.style={
        if n children=0{#1}{}
    },
    every tree node/.style={
        if n children=0{}{#1}
    },
    mytree/.style={
        for tree={
            edge path={
            \noexpand\path [draw, very thick, \forestoption{edge}] (!u.parent anchor) |- (.child anchor)\forestoption{edge label};
            },
            every tree node={draw=none,inner sep=0, outer sep=0, minimum size=0},
            every leaf node/.style={align=left},
            grow=east,
            parent anchor=east, 
            child anchor=west,
            anchor=west,
            l sep=5mm,
            s sep=3mm,
            draw=none,
    			if n children=0{tier=word}{}
        }
    }
}

\mode<presentation>
{
  \usetheme{Lecture}
  \setbeamercovered{invisible}
  \setbeamertemplate{items}[square]
}

\usepackage{calc}
\usepackage{hyperref}

\newcommand\HiddenWord[1]{%
	\alt<handout>{\rule{\widthof{#1}}{\fboxrule}}{#1}%
}


\usepackage{xifthen}
\newcommand{\futuyma}[1]{%
	\ifthenelse{\isempty{#1}}%
	{Futuyma \& Kirkpatrick 2017, 4th ed.}%
	{Fig.~#1~Futuyma \& Kirkpatrick 2017, 4th ed.}%
}

% This defines \amper for the fancy ampersand
% to be used in the header. See
% https://tex.stackexchange.com/a/58185/39194
\usepackage{xspace}
\newfontfamily\amperfont[Style=Alternate]{Linux Libertine O}    
\makeatletter
\DeclareRobustCommand{\amper}{{\amperfont\ifx\f@shape\scname\smaller[1.2]\fi\&}\xspace}
\makeatother

\newcommand{\backskip}{\vspace{-0.5\baselineskip}}

%% Remove indent from multicol.
\parindent=0pt

\begin{document}

\lecture{student}{student}

{
\usebackgroundtemplate{\includegraphics[width=\paperwidth]{macro_intro}}
\begin{frame}[b]

\vfilll

\tiny  \textcolor{white}{\futuyma{}}
\end{frame}
}

%%
{
\usebackgroundtemplate{\includegraphics[width=\paperwidth]{macroevolution_dinosaurs} }
\begin{frame}{\highlight{Macroevolution} is broad evolutionary changes over time between different lineages.}
\end{frame}
}

\begin{frame}{\highlight{Macroevolution} happens because of microevolution.}

	\centering
	\begin{tikzpicture}
		[remember picture, overlay,
		myLine/.style={color=black,very thick},
		myArrow/.style={color=orange6, thick, ->}]

			\draw [myLine] (-3,-2) -- (0,-2);
			\draw [myLine] (0,-2) -- (0,-0.7); % vertical lines
			\draw [myLine] (0,-2) -- (0,-3.3);
			\draw [myLine] (0,-0.7) -- (3,-0.7); % descendants
			\draw [myLine] (0,-3.3) -- (3,-3.3); 

		% Draw the tree:
		
			\draw [myLine] (-3,-2) -- (0,-2); % ancestor

			\draw [myLine] (0,-2) -- (0,-0.7); % vertical lines
			\draw [myLine] (0,-2) -- (0,-3.3);
		
			\draw [myLine] (0,-0.7) -- (3,-0.7); % descendants
			\draw [myLine] (0,-3.3) -- (3,-3.3); 

		% Ancestor microevolution
			\node [color=blue6] at (-1.5,-1.8) {Microevolution}; % ancestor
	
		% Speciation
		%	\node [color=orange6] (speciation) at (-1.5,-3.1) {Speciation};
		%	\node [circle, draw=orange6, thick, minimum size=3mm] (speccirc) at (0,-2) {};
		%	\draw [color=orange6, thick] (speciation) edge (speccirc);
		
		
		% Descendant microevolution
			\node [color=blue6] at (1.5,-0.5) {Microevolution}; % descendants
			\node [color=blue6] at (1.5,-3.5) {Microevolution};

		% Macroevolution
			\node [color=orange6] (macro) at (1.7,-2) {Macroevolution};
			\draw [myArrow] (macro.north) to (1.7,-0.8);
			\draw [myArrow] (macro.south) to (1.7,-3.2);

	\end{tikzpicture}

\end{frame}

%%

\begin{frame}

\centering

\includegraphics[height=0.93\textheight]{evodevo_hox_phylogeny}

\tinyfill Carroll et al., 2001. \textit{From DNA to Diversity.}
\end{frame}

%%


\begin{frame}{Body plan complexity increased via \emph{Hox} expansion and co-option.}

\backskip

\centering

\includegraphics[width=0.88\linewidth]{macro_hox_expansion}

\tinyfill Peterson \& Davidson 2000. PNAS 97:4430.

\end{frame}

%%

\begin{frame}{Jaws evolved before transition to land.}
\backskip

\centering

\includegraphics[width=0.83\linewidth]{macro_jaws_first}

\end{frame}
%%

\begin{frame}{Jaws evolved by specialization of toolkit genes.}

%\backskip

\centering

\includegraphics[width=\linewidth]{macro_jaws_toolkit}

\tinyfill Kuratani 2004. J Anat 205:335. Mehta \& Wainwright 2007. Nature 449:79.

\end{frame}
%%

\begin{frame}{\highlight{CaM1} and \highlight{BMP4} influence jaw morphology of vertebrates.}

\backskip

\centering

\includegraphics[width=0.95\linewidth]{macro_jaws_cam1}

\tinyfill Parson \& Albertson 2009. Annu Rev Genet 43:369


\end{frame}
%%

\begin{frame}{Bill shape influenced by \highlight{CaM1} and \highlight{BMP4.}}

\backskip

\centering

\includegraphics[height=0.82\textheight]{macro_bill_shape}

\tinyfill Abzhanov et al.~2004 Science 305:1462. Abzhanov et al.~2006 Nature 442:563.

\end{frame}

%%

\begin{frame}{\highlight{CaM1} correlated with jaw width in fishes.}

\backskip

\centering

\includegraphics[height=0.82\textheight]{macro_jaw_width}

\tinyfill Parson \& Albertson 2009. Annu Rev Genet 43:369


\end{frame}


%%

\begin{frame}{Tetrapod limbs evolved via co-option of Hox genes.}

\backskip

\centering

\includegraphics[height=0.82\textheight]{macro_hox_limbs}

\tinyfill \futuyma{15.12}

\end{frame}

%%

{
\usebackgroundtemplate{\includegraphics[width=\paperwidth]{macro_hox_limbs_colinear}}
\begin{frame}[b]

\tinyfill Modified from Shubin et al.~1997. Nature 388:639

\end{frame}
}

%%

\begin{frame}{Homologous toolkit genes control limb patterning for arthropods and vertebrates.}

\backskip

\centering

\includegraphics[width=\linewidth]{macro_animal_appendages}

\tinyfill Shubin et al.~1997. Nature 388:639

\end{frame}

%%

\begin{frame}

\centering

\includegraphics[height=0.95\textheight]{macro_therapod_phylogeny}

\end{frame}

%%

\begin{frame}{\highlight{Heterotopy} is an evolutionary change in the spatial position of a character.}

{\centering

\includegraphics[width=\linewidth]{futuyma_fig20-7}
}

\hangpara The gene networks for condensations 2–4 in therapods shift to form digits 2–4 in birds.
\tinyfill \futuyma{20.7}

\end{frame}

%%


\begin{frame}{\highlight{Shh} and \highlight{Bmp2} determine growth of alligator and bird scales, and feathers.}

\backskip

\includegraphics[width=\linewidth]{macro_scales_feathers}


\tinyfill Harris et al.~2002. J Exp Zool 294:160

\end{frame}

%%

\begin{frame}{Would the same toolkit genes determine growth of mammalian hair?}

\backskip

\includegraphics[width=\linewidth]{macro_toolkit_feathers}

\vfilll

\tiny Chang et al.~2009. Int J Dev Biol 53:813
\hfill Widelitz et al.~2003. J Exp Zool 298B:109



\end{frame}



%%

\end{document}


