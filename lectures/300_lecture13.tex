%!TEX TS-program = lualatex
%!TEX encoding = UTF-8 Unicode

\documentclass[t]{beamer}

%%%% HANDOUTS For online Uncomment the following four lines for handout
%\documentclass[t,handout]{beamer}  %Use this for handouts.
%\usepackage{handoutWithNotes}
%\includeonlylecture{student}
%\pgfpagesuselayout{3 on 1 with notes}[letterpaper,border shrink=5mm]

%\usefonttheme{professionalfonts}


%%% Including only some slides for students.
%%% Uncomment the following line. For the slides,
%%% use the labels shown below the command.

%% For students, use \lecture{student}{student}
%% For mine, use \lecture{instructor}{instructor}

% FONTS
\usepackage{fontspec}
\def\mainfont{Linux Biolinum O}
\setmainfont[Ligatures={Common, TeX}, Contextuals={NoAlternate}, BoldFont={* Bold}, ItalicFont={* Italic}, Numbers={Proportional, OldStyle}]{\mainfont}
%\setmonofont[Scale=MatchLowercase]{Inconsolata} 
\setsansfont[Scale=MatchLowercase]{Linux Biolinum O} 
\setmonofont{Linux Libertine Mono O}

\usepackage{microtype}

\parindent=0pt

\usepackage{unicode-math}
\setmathfont[Scale=MatchLowercase]{Asana Math}

\usepackage{graphicx}
	\graphicspath{%
	{/Users/goby/Pictures/teach/300/lectures/}%
	{/Users/goby/Pictures/teach/163/lecture/}} % set of paths to search for images

\usepackage{amsmath,amssymb}

%\usepackage{units}

\usepackage{booktabs}
\usepackage{array}
\newcolumntype{L}[1]{>{\raggedright\let\newline\\\arraybackslash\hspace{0pt}}p{#1}}
\newcolumntype{C}[1]{>{\centering\let\newline\\\arraybackslash\hspace{0pt}}p{#1}}
\newcolumntype{R}[1]{>{\raggedleft\let\newline\\\arraybackslash\hspace{0pt}}p{#1}}


\usepackage{multicol}
%	\setlength{\columnsep=1em}
\usepackage{enumitem}
\usepackage{textcomp}
\usepackage{setspace}
\usepackage{tikz}
	\tikzstyle{every picture}+=[remember picture,overlay]
\usetikzlibrary{arrows}

\mode<presentation>
{
  \usetheme{Lecture}
  \setbeamercovered{invisible}
  \setbeamertemplate{items}[square]
}

\usepackage{calc}
\usepackage{hyperref}

\newcommand\HiddenWord[1]{%
	\alt<handout>{\rule{\widthof{#1}}{\fboxrule}}{#1}%
}


\usepackage{xifthen}
\newcommand{\futuyma}[1]{%
	\ifthenelse{\isempty{#1}}%
	{Futuyma \& Kirkpatrick 2017, 4th ed.}%
	{Fig.~#1~Futuyma \& Kirkpatrick 2017, 4th ed.}%
}

% This defines \amper for the fancy ampersand
% to be used in the header. See
% https://tex.stackexchange.com/a/58185/39194
\usepackage{xspace}
\newfontfamily\amperfont[Style=Alternate]{Linux Libertine O}    
\makeatletter
\DeclareRobustCommand{\amper}{{\amperfont\ifx\f@shape\scname\smaller[1.2]\fi\&}\xspace}
\makeatother

\newcommand{\backskip}{\vspace{-0.5\baselineskip}}

%% Remove indent from multicol.
\parindent=0pt

\begin{document}

\lecture{student}{student}

{
\usebackgroundtemplate{\includegraphics[width=\paperwidth]{sexsel_intro}}
\begin{frame}[b]

\tinyfill  \textcolor{white}{\futuyma{}}
\end{frame}
}

%%


{
	\usebackgroundtemplate{\includegraphics[width=\paperwidth]{intrasexual_selection_giraffes} }
	\begin{frame}{\highlight{Intrasexual} selection is competition between males.}
	
	\vfilll
	
	\tiny \textcolor{white}{Luca Galuzzi, \href{http://www.galuzi.it}{www.galuzi.it}, \ccbysa{2.5}
		\hfill \href{http://www.youtube.com/watch?v=C7HCIGFdBt8}{Link to Video}}
	
\end{frame}
}

%%

{
\usebackgroundtemplate{\includegraphics[width=\paperwidth]{superb_bird_of_paradise} }
\begin{frame}{\textcolor{orange5}{Intersexual} \textcolor{white!95!black}{selection is females choosing mates.}}

\vfilll

\tiny \textcolor{white}{\href{https://youtu.be/IPfW7iolmgc}{Link to Video} 
	\hfill \href{https://youtu.be/1XkPeN3AWIE}{Bonus Video}}

\end{frame}
}


%%

\begin{frame}{\highlight{Sexual dimorphism} is the difference of  traits between sexes of the same species.}

\backskip

\begin{multicols}{2}

\hangpara \highlight{Primary sexual traits} have a direct role in  reproduction.

\hangpara \highlight{Secondary sexual traits} do not have a \emph{direct} role in reproduction.

\vspace{\baselineskip}

\centering

\includegraphics[width=0.9\linewidth]{sex_dimorphism_lions}

\columnbreak

\includegraphics[width=\linewidth]{sex_dimorphism_fishes}


\end{multicols}


\vfilll

\tiny \href{https://www.flickr.com/photos/11847703@N05/8234519412}{George Lamson, Flickr, \ccbyncsa{2}} \hfill
\textcopyright\,M.G.~Simpson, Plant Systematics 2010.
\end{frame}

%%

\begin{frame}{Why are males sexually selected?}

\backskip

\begin{multicols}{2}
	\hangpara \highlight{Bateman's principle:} male \textit{Drosophila} sired more offspring if they mated with more females.

	\hangpara \highlight{Operational sex ratio:} females can be limiting resource for males.

	\hangpara \highlight{Anisogamy}
	
\columnbreak

\includegraphics[width=0.9\linewidth]{sex_bateman}

\end{multicols}

\vfilll

\tinyfill Bateman 1948. Heredity 2:349. 

\end{frame}

%%

\begin{frame}{Intrasexual selection can be direct.}

\backskip

\begin{multicols}{2}
\hangpara Visual ornaments can enhance apparent body size.

\vspace{6\baselineskip}

\hangpara Same-sex contests based on size, weapons, or other traits.

\columnbreak

\centering

\includegraphics[width=0.7\linewidth]{sex_rwbl}

\smallskip

\includegraphics[width=0.8\linewidth]{sex_elk_combat}

\end{multicols}
	

	
\end{frame}

%%

\begin{frame}{Intrasexual selection can be indirect}
\backskip

\begin{multicols}{2}
\hangpara Territories that limit movement.

\hangpara External mechanisms that prevent copulation.

\hangpara Internal mechanisms post-copulation.

\columnbreak

\centering

\includegraphics[width=0.65\linewidth]{sex_mating_damselflies}

\smallskip

\includegraphics[width=0.65\linewidth]{sex_damselfly_penis}

\end{multicols}
	
\end{frame}

%%

\begin{frame}{Female choice for male traits can be counteracted by other selective pressures.}

\backskip

\includegraphics[width=\linewidth]{sex_trinidad_guppies}

\end{frame}

%%

\begin{frame}{Predators may cue in on same cues that females prefer.}

\includegraphics[width=\linewidth]{sex_tungara_frogs}

\tinyfill \href{https://www.smithsonianmag.com/science-nature/frogs-mating-call-also-attracts-predators-180949463/}{Audio of Tungara frogs calls.}
\end{frame}

%%

\begin{frame}{Some females may be able to retain choice at the cellular level.}

\backskip

\begin{multicols}{2}

\hangpara Multiple sperm enter egg of \textit{Beroë} (a comb jelly).

\hangpara The egg pronuclei (red) visits all sperm pronuclei (black), then fuses with one.

\columnbreak

\vspace*{-2.5\baselineskip}\includegraphics[width=0.85\linewidth]{sex_beroe}

\end{multicols}

\vfilll

\tiny Birkhead T.~2000. Promiscuity. Harvard Press.

\end{frame}

%%

\begin{frame}{Why does sexual selection cause secondary sexual traits?}

\hangpara \highlight{Direct benefits:} males display obvious benefits to female or her offspring.

\hangpara \highlight{Indirect benefits:} Runaway sexual selection or good genes.

\hangpara \highlight{Sensory bias:} females have an existing bias for trait, males later evolve trait.

\hangpara \highlight{Sensory drive:} females prefer traits that standout from the background.

\hangpara \highlight{Antagonistic coevolution:} conflicting self-interest leads to battle of the sexes.

\end{frame}

%%

\begin{frame}{\highlight{Direct benefits:} males display obvious benefits to female or her offspring.}

\backskip
\begin{multicols}{2}

\centering

\includegraphics[width=\linewidth]{sex_bower}

{\small Bower of Satin Bowerbird}

\columnbreak

\includegraphics[width=0.93\linewidth]{sex_hofi}

\smallskip

\includegraphics[width=0.93\linewidth]{sex_hofi_feeding}

{\small House Finch}

\end{multicols}

\end{frame}


%%

\begin{frame}{\highlight{Runaway sexual selection} occurs if female preference is linked to male trait.}

\vspace*{-\baselineskip}

\includegraphics[width=\linewidth]{sex_runaway_selection}

\end{frame}

%%

\begin{frame}{\highlight{Good genes:} Males sport traits that are energetically or ecologically costly.}

\vspace{-\baselineskip}

\begin{multicols}{2}

\includegraphics[width=\linewidth]{sex_rnph}

\smallskip

\includegraphics[width=\linewidth]{sex_rnph_spurs}

\columnbreak

\begin{tabular}{@{}R{2cm}cc@{}}
\toprule
& \multicolumn{2}{c}{Genetic quality} \tabularnewline
\cmidrule{2-3}
Male traits & Good & Bad \tabularnewline
\midrule
Short tail (cheap) & Alive & Alive \tabularnewline
Long tail (costly) & \highlight{Alive} & Dead \tabularnewline
\bottomrule
\end{tabular}

\end{multicols}

\end{frame}

%%

\begin{frame}{\highlight{Sensory bias} is when females have a pre-existing bias for a particular trait.}

\backskip

\includegraphics[width=\linewidth]{sex_sensory_bias}

\end{frame}

%%

\begin{frame}{\highlight{Sensory drive} is a male response to female perception of signal relative to environment.}

\hangpara Females prefer conspicuous signals that standout from the environment.

\hangpara Perceptual sensitivity varies among environments.
e.g., ability to perceive color varies with light regime

\hangpara Male signals that match female perception are favored.

\end{frame}

%%

\begin{frame}
\begin{multicols}{2}

\includegraphics[width=\linewidth]{sex_phylloscopus}

\columnbreak

\includegraphics[width=\linewidth]{sex_phylloscopus_graphs}

\end{multicols}

\begin{tikzpicture}
\draw [->, ultra thick] (5.8,7.74) to (7.7,7.74);
\draw [->, ultra thick] (5.8,2.35) to (9.65,2.62);
\end{tikzpicture}


\vfilll

\tiny Martin Vavrik, Wikimedia (upper). David Cook, Flickr (lower)
\end{frame}


%%


\begin{frame}{\highlight{Antagonistic coevolution} results from sexual conflict due to anisogamy.}

\hangpara Females produce few, large eggs that are energetic costly.

\hangpara Male produce many small sperm that are energetically cheap.

\end{frame}

%%

{
\usebackgroundtemplate{\includegraphics[width=\paperwidth]{sex_social_monogamy}}
\begin{frame}{\textcolor{white}{Are monogamous birds really monogamous?}}

\vfilll

\tiny \textcolor{white}{Sandysphotos2009, Wikimedia}
\end{frame}
}

%%

\begin{frame}{Flatworm penis fencing: a courtship duel.}

\backskip

\centering

\includegraphics[width=0.95\linewidth]{sex_penis_fencing}

\vfilll

\tiny \href{http://www.youtube.com/watch?v=5fx-YgcP8Gg}{Link to video} \hfill Nico Michiels, University Tuebingen


\end{frame}

%%

{
\usebackgroundtemplate{\includegraphics[width=\paperwidth]{sex_aggressive_mallards}}
\begin{frame}

\vfilll

\tiny \href{https://www.youtube.com/watch?v=dQs1cw25dqw}{Link to video} \hfill AnemoneProjectors, Wikimedia
\end{frame}
}

%%

\begin{frame}

\begin{multicols}{2}

\centering

\includegraphics[height=0.85\textheight]{sex_duck_penis}

\columnbreak

\includegraphics[height=0.85\textheight]{sex_duck_genitalia}

\end{multicols}

\vfilll

\tiny McCracken et al. 2001. Nature 413: 128
\hfill 
Brennan et al. 2007. PLoS ONE 2(5):e418. 

\end{frame}

%%

{
\usebackgroundtemplate{\includegraphics[width=\paperwidth]{sex_suicide}}
\begin{frame}

\begin{tikzpicture}
\node at (2.5,-2) [text width=2cm] {\LARGE Sexual \highlight{suicide.}};
\end{tikzpicture}


\tinyfill Australian redback spider by Tony Hudson, Wikimedia Commons.
\end{frame}
}

%%

{
\usebackgroundtemplate{\includegraphics[width=\paperwidth]{sex_cannibalism}}
\begin{frame}


\begin{tikzpicture}
\node at (4.5,-0.7) [text width=3.2cm] {\LARGE \textcolor{white}{Sexual}\\ \highlight{cannibalism.}};
\end{tikzpicture}


\vfilll

\tiny \textcolor{white}{Oliver Koemmerling, Wikimedia Commons.\hfill \href{http://www.youtube.com/watch?v=KYp_Xi4AtAQ}{Link to video} \hfill \phantom{.}} 

\end{frame}
}

%%

{
\usebackgroundtemplate{\includegraphics[width=\paperwidth]{sex_humans}}
\begin{frame}{\textcolor{white}{Does sexual selection occur in humans?}}

%\vfilll

\tinyfill AnemoneProjectors, Wikimedia
\end{frame}
}


%%

\end{document}


