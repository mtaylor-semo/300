%!TEX TS-program = lualatex
%!TEX encoding = UTF-8 Unicode

\documentclass[t]{beamer}

%%%% HANDOUTS For online Uncomment the following four lines for handout
%\documentclass[t,handout]{beamer}  %Use this for handouts.
%\usepackage{handoutWithNotes}
%\includeonlylecture{student}
%\pgfpagesuselayout{3 on 1 with notes}[letterpaper,border shrink=5mm]

%\usefonttheme{professionalfonts}


%%% Including only some slides for students.
%%% Uncomment the following line. For the slides,
%%% use the labels shown below the command.

%% For students, use \lecture{student}{student}
%% For mine, use \lecture{instructor}{instructor}

% FONTS
\usepackage{fontspec}
\def\mainfont{Linux Biolinum O}
\setmainfont[Ligatures={Common, TeX}, Contextuals={NoAlternate}, BoldFont={* Bold}, ItalicFont={* Italic}, Numbers={Proportional, OldStyle}]{\mainfont}
%\setmonofont[Scale=MatchLowercase]{Inconsolata} 
\setsansfont[Scale=MatchLowercase]{Linux Biolinum O} 
\setmonofont{Linux Libertine Mono O}

\usepackage{microtype}

\parindent=0pt

\usepackage{unicode-math}
\setmathfont[Scale=MatchLowercase]{Asana Math}

\usepackage{graphicx}
	\graphicspath{%
	{/Users/goby/Pictures/teach/300/lectures/}%
	{/Users/goby/Pictures/teach/163/common/}} % set of paths to search for images

\usepackage{amsmath,amssymb}

%\usepackage{units}

\usepackage{booktabs}
\usepackage{array}
\newcolumntype{L}[1]{>{\raggedright\let\newline\\\arraybackslash\hspace{0pt}}p{#1}}
\newcolumntype{C}[1]{>{\centering\let\newline\\\arraybackslash\hspace{0pt}}p{#1}}
\newcolumntype{R}[1]{>{\raggedleft\let\newline\\\arraybackslash\hspace{0pt}}p{#1}}


\usepackage{multicol}
%	\setlength{\columnsep=1em}
\usepackage{enumitem}
\usepackage{textcomp}
\usepackage{setspace}
\usepackage{tikz}
	\tikzstyle{every picture}+=[remember picture,overlay]
\usetikzlibrary{arrows}

\mode<presentation>
{
  \usetheme{Lecture}
  \setbeamercovered{invisible}
  \setbeamertemplate{items}[square]
}

\usepackage{calc}
\usepackage{hyperref}

\newcommand\HiddenWord[1]{%
	\alt<handout>{\rule{\widthof{#1}}{\fboxrule}}{#1}%
}


\usepackage{xifthen}
\newcommand{\futuyma}[1]{%
	\ifthenelse{\isempty{#1}}%
	{Futuyma \& Kirkpatrick 2017, 4th ed.}%
	{Fig.~#1~Futuyma \& Kirkpatrick 2017, 4th ed.}%
}

% This defines \amper for the fancy ampersand
% to be used in the header. See
% https://tex.stackexchange.com/a/58185/39194
\usepackage{xspace}
\newfontfamily\amperfont[Style=Alternate]{Linux Libertine O}    
\makeatletter
\DeclareRobustCommand{\amper}{{\amperfont\ifx\f@shape\scname\smaller[1.2]\fi\&}\xspace}
\makeatother

\newcommand{\backskip}{\vspace{-0.5\baselineskip}}

\begin{document}

\lecture{student}{student}

{
\usebackgroundtemplate{\includegraphics[width=\paperwidth]{sexsel_intro}}
\begin{frame}[b]

\tinyfill  \textcolor{white}{\futuyma{}}
\end{frame}
}

\begin{frame}{\highlight{Sexual dimorphism} is the difference of  traits between sexes of the same species.}

\backskip

\begin{multicols}{2}

\hangpara \highlight{Primary sexual traits} have a direct role in  reproduction.

\hangpara \highlight{Secondary sexual traits} do not have a \emph{direct} role in reproduction.

\vspace{\baselineskip}

\centering

\includegraphics[width=0.9\linewidth]{sex_dimorphism_lions}

\columnbreak

\includegraphics[width=\linewidth]{sex_dimorphism_fishes}


\end{multicols}


\vfilll

\tiny \href{https://www.flickr.com/photos/11847703@N05/8234519412}{George Lamson, Flickr, \ccbyncsa{2}} \hfill
\textcopyright\,M.G.~Simpson, Plant Systematics 2010.
\end{frame}


\begin{frame}{\highlight{Asexual reproduction} occurs without meiosis and syngamy.}

\includegraphics[width=\linewidth]{futuyma_fig10-20}

\tinyfill \futuyma{10.20}

\end{frame}

%%


\begin{frame}{What are advantages and disadvantages of asexual reproduction?}

\begin{multicols}{2}

\includegraphics[width=\linewidth]{sex_binary_fission}

\columnbreak

\includegraphics[width=\linewidth]{sex_budding_yeast}
\end{multicols}

\vfilll

\tiny \href{https://www.sciencephoto.com/media/11784/view/dividing-bacterium-sem}{\textcopyright\,David Sharf, Science Photo Library} \hfill \href{https://commons.wikimedia.org/wiki/File:Saccharomyces_cerevisiae_SEM.jpg}{Mogana Das Murtey \& Patchamuthu Ramasamy, Wikimedia Commons, \ccbysa{3}}


\end{frame}


%%

\begin{frame}{Is it good to maintain the same genome?}

\begin{multicols}{2}
\includegraphics[width=\linewidth]{futuyma_fig10-20a}

\columnbreak

\includegraphics[width=\linewidth]{sex_mullers_ratchet}

\end{multicols}

\pause

\hangpara \highlight{Muller's Ratchet} is the accumulation of mildly deleterious mutations in asexual populations.

\vfilll


\tiny \futuyma{10.20} \hfill Brockhaus Konversations-Lexikon (1894, 14th edition, adapted by Georg Wiora.

\end{frame}

%%

\begin{frame}{Sexual reproduction includes both meiotic recombination and syngamy.}

%\hangpara Sex breaks up allele combinations. Is this good or bad?

%\pause

\hangpara What are the advantages and disadvantages of sexual reproduction?

\end{frame}

%%

\begin{frame}{All else equal, does asexual or sexual reproduction have greater relative fitness?}

\onslide<1-3>{
\includegraphics[width=\linewidth]{sex_two_fold_cost}
}

\begin{tikzpicture}
\onslide<1>{
\fill [white] (0,0) rectangle (12,3.5);
}


\onslide<2>{\fill [white] (0,0) rectangle (12,2.5);}

\end{tikzpicture}

\onslide<3>{%
\vspace{-\baselineskip}
\hangpara \highlight{Two-fold cost of sex:} males reduce reproductive potential by a factor of two every generation.
}


\tinyfill \textcopyright\,Sinauer Associates, Inc.~2005

\end{frame}

%%

\begin{frame}{If sex has a fitness cost, then why did sex evolve?}
\vspace{-\baselineskip}

\begin{multicols}{2}

\centering

\reflectbox{\includegraphics[width=0.9\linewidth]{sex_penguins}}

\smallskip

\includegraphics[width=0.9\linewidth]{sex_bees}

\columnbreak

\reflectbox{\includegraphics[width=0.9\linewidth]{sex_hoverflies}}

\smallskip

\includegraphics[width=0.8\linewidth]{sex_fleas}

\end{multicols}
\end{frame}

%%

\begin{frame}{Sexual reproduction breaks up allele combinations.}

\hangpara Is this good or bad?

\hangpara What if \emph{AB} is the ideal allele combination in an individual that is \emph{AB} on one chromosome and \emph{ab} on the other?

\hangpara What if an \emph{individual} had a theoretically “perfect” genome? 


\end{frame}


\begin{frame}{Sexual reproduction has advantages in changing environments.}

\hangpara \highlight{Red queen hypothesis}

\hangpara \highlight{Selective interference} is reduced.\newline
	\hspace*{1em} \highlight{Clonal interference}\newline
	\hspace*{1em} \highlight{Ruby-in-the-rubbish}
	
\hangpara Enhanced rate of adaptation

\end{frame}

%%

\begin{frame}{\highlight{Red queen hypothesis} describes the \emph{evolutionary arms race} between hosts and their pathogens.}

\backskip

\centering


\includegraphics[height=0.79\textheight]{futuyma_fig10-22}

\tinyfill \futuyma{10.22}


\end{frame}

%%

\begin{frame}{\highlight{Clonal interference} slows adaptation in asexual species.}
\backskip

\centering


\includegraphics[height=0.82\textheight]{futuyma_fig10-23}

\tinyfill \futuyma{10.23}
\end{frame}

%%

\begin{frame}{\highlight{Ruby-in-the-rubbish effect} links detrimental and beneficial mutations in asexual species.}
\backskip

\centering

\includegraphics[width=\linewidth]{futuyma_fig10-24}

\hangpara Analogous to selective sweep but genome wide.

\tinyfill \futuyma{10.24}
\end{frame}

%%

\begin{frame}{Muller's ratchet is a form of selective interference in areas of low recombination.}

\backskip

\begin{multicols}{2}
\includegraphics[width=\linewidth]{futuyma_fig10-27}

\columnbreak

\includegraphics[width=\linewidth]{futuyma_fig10-26}
\end{multicols}

\vfilll

\tiny \futuyma{10.27} \hfill \futuyma{10.26}

\end{frame}

%%

\begin{frame}{Recombination speeds the rate of adaptation.}

\backskip

\centering

\includegraphics[height=0.8\textheight]{futuyma_fig10-25}

\tinyfill \futuyma{10.25}

\end{frame}

%%


\begin{frame}{Many alternate mating strategies have evolved.}

\vspace{-\baselineskip}

\begin{multicols}{2}

\hangpara \highlight{Sneaking male} strategy increases fitness of non-territorial males.

\vspace{6\baselineskip}

\hangpara \highlight{Sequential hermaphroditism} increases fitness of different age or size classes.

\columnbreak

\centering

\noindent\includegraphics[width=0.94\linewidth]{sex_sneaking_males}

\smallskip

\reflectbox{\includegraphics[width=0.94\linewidth]{sex_blue-head_wrasse}}
\end{multicols}

\end{frame}

%%

\begin{frame}{Sequential hermaphroditism can be \highlight{protogynous} or \highlight{protandrous.}}


\includegraphics[width=\linewidth]{sex_sequential_hermaphroditism}

\vfilll

\tinyfill \textcopyright\,2005 Sinauer Associates, Inc.

\end{frame}

%%


\end{document}


