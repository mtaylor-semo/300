%!TEX TS-program = lualatex
%!TEX encoding = UTF-8 Unicode

\documentclass[t]{beamer}

%%%% HANDOUTS For online Uncomment the following four lines for handout
%\documentclass[t,handout]{beamer}  %Use this for handouts.
%\includeonlylecture{student}
%\usepackage{handoutWithNotes}
%\pgfpagesuselayout{3 on 1 with notes}[letterpaper,border shrink=5mm]
%	\setbeamercolor{background canvas}{bg=black!5}
\usefonttheme{professionalfonts}


%%% Including only some slides for students.
%%% Uncomment the following line. For the slides,
%%% use the labels shown below the command.

%% For students, use \lecture{student}{student}
%% For mine, use \lecture{instructor}{instructor}

% FONTS
\usepackage{fontspec}
\def\mainfont{Linux Biolinum O}
\setmainfont[Ligatures={Common, TeX}, Contextuals={NoAlternate}, BoldFont={* Bold}, ItalicFont={* Italic}, Numbers={Proportional, OldStyle}]{\mainfont}
%\setmonofont[Scale=MatchLowercase]{Inconsolata} 
\setsansfont[Scale=MatchLowercase]{Linux Biolinum O} 
\setmonofont{Linux Libertine Mono O}

\usepackage{microtype}

\usepackage{unicode-math}
\setmathfont[Scale=MatchLowercase]{Asana Math}

\usepackage{graphicx}
	\graphicspath{%
	{/Users/goby/Pictures/teach/300/lectures/}%
	{/Users/goby/Pictures/teach/163/common/}} % set of paths to search for images

\usepackage{amsmath,amssymb}

%\usepackage{units}

\usepackage{booktabs}
\usepackage{array}
\newcolumntype{L}[1]{>{\raggedright\let\newline\\\arraybackslash\hspace{0pt}}p{#1}}
\newcolumntype{C}[1]{>{\centering\let\newline\\\arraybackslash\hspace{0pt}}p{#1}}
\newcolumntype{R}[1]{>{\raggedleft\let\newline\\\arraybackslash\hspace{0pt}}p{#1}}


\usepackage{multicol}
%	\setlength{\columnsep=1em}
\usepackage{enumitem}
\usepackage{textcomp}
\usepackage{setspace}
\usepackage{tikz}
	\tikzstyle{every picture}+=[remember picture,overlay]

\mode<presentation>
{
  \usetheme{Lecture}
  \setbeamercovered{invisible}
  \setbeamertemplate{items}[square]
}

\usepackage{calc}
\usepackage{hyperref}

\newcommand\HiddenWord[1]{%
	\alt<handout>{\rule{\widthof{#1}}{\fboxrule}}{#1}%
}


\usepackage{xifthen}
\newcommand{\futuyma}[1]{%
	\ifthenelse{\isempty{#1}}%
	{Futuyma \& Kirkpatrick 2017, 4th ed.}%
	{Fig.~#1~Futuyma \& Kirkpatrick 2017, 4th ed.}%
}

% This defines \amper for the fancy ampersand
% to be used in the header. See
% https://tex.stackexchange.com/a/58185/39194
\usepackage{xspace}
\newfontfamily\amperfont[Style=Alternate]{Linux Libertine O}    
\makeatletter
\DeclareRobustCommand{\amper}{{\amperfont\ifx\f@shape\scname\smaller[1.2]\fi\&}\xspace}
\makeatother

\begin{document}

\lecture{student}{student}


{
\usebackgroundtemplate{\includegraphics[width=\paperwidth]{drift_intro}}
\begin{frame}[b]

\tiny \textcolor{white}{\futuyma{}}
\end{frame}
}

%%

{
\usebackgroundtemplate{\includegraphics[width=\paperwidth]{drift_cockroach}}
\begin{frame}[t]{\highlight{Genetic drift} is the \emph{random} change of allele frequencies.}

\vfilll

\tiny Photo by Furryscaly, \href{https://www.flickr.com/photos/98528214@N00/428682623}{Flickr}, \ccbysa{2}
\end{frame}
}

%%
{
\usebackgroundtemplate{\includegraphics[width=\paperwidth]{drift_meiosis_example}}
\begin{frame}[t]{A simple example of genetic drift.}

\end{frame}
}

%%

\begin{frame}[t]{Genetic drift$\dots$}

\vspace{-\baselineskip}

\begin{multicols}{2}
\noindent\includegraphics[width=\linewidth]{futuyma_fig7-2}

\columnbreak

\begin{itemize}[label=\textcolor{white}{\textbullet}, leftmargin=6pt]
\item is unbiased,
\item decreases \highlight{heterozygosity,}
\item is stronger in smaller populations,
\item fixes alleles without selection, and
\item increases differences among populations.
\end{itemize}

\end{multicols}

\end{frame}


%%

\begin{frame}
\centering

\includegraphics[height=0.97\textheight]{futuyma_fig7-3}
\end{frame}

%% HETEROZYGOSITY

\begin{frame}[t]{\highlight{Heterozygosity} is a measure of genetic variation in a population.}

%\vspace{-\baselineskip}

One measure is \highlight{expected heterozygosity.}

\begin{equation*}
H = 1-\sum_{i}^{k}p_i^2
\end{equation*}

\hangpara If $p_1 = 0.7$ and $p_2 = 0.3$, then 

\vspace{-0.5\baselineskip}

\begin{equation*}
H = 1 - 0.7^2 + 0.3^2 = 1-0.49-0.09 = 0.42
\end{equation*}

\hangpara If $p_1 = 0.35, p_2 = 0.41,$ and $p_3 = 0.24,$ then

\vspace{-0.5\baselineskip}

\begin{equation*}
H = 1- 0.35^2 + 0.41^2 + 0.24^2 = 0.652
\end{equation*}


\end{frame}

%%

\begin{frame}[t, fragile]{Another measure of heterozygosity is \highlight{nucleotide diversity.}}

\vspace{-0.5\baselineskip}



\begin{equation*}
\pi = \dfrac{1}{[n(n-1)]/2}\sum_{i<j}\pi_{ij}
\end{equation*}


\hangpara These two sequences are 39 nucleotides long with 11 differences.
{\footnotesize
\begin{verbatim}
   ATCTTCAGGTCTTGGACATTAAGACAACATGCATAGCAT
   ATGACAGGGTCATGGACAATAAGTCAACATCCACAGAAT
     *****    *      *    *      *  *  *
\end{verbatim}
}

\begin{multicols}{2}
\hangpara If one copy of each, then

\begin{equation*}
\pi = 0.282
\end{equation*}

\columnbreak

\hangpara If 3 copies and 1 copy, then

\begin{equation*}
\pi = 0.141
\end{equation*}

\end{multicols}


\end{frame}


%%
\lecture{instructor}{instructor}
{
\usebackgroundtemplate{\includegraphics[width=\paperwidth]{drift_pelagibacter}}
\begin{frame}[t]{\textcolor{white}{\textit{Pelagibacter ubique} may be the most abundant species on Earth $(10^{28}\ \mathrm{cells}).$}}

\vfilll

\tinyfill{\textcolor{white}{\textsc{noaa},\href{https://commons.wikimedia.org/wiki/File:Pelagibacter.jpg}{Wikimedia} public domain.}}

\end{frame}
}

%%

\lecture{student}{student}

\begin{frame}[t]{Not all individuals contribute alleles to the next generation.}

\hangpara  \highlight{Effective population size $\left(N_e\right)$} is the size of an ideal population that loses heterozygosity at the same rate as the \highlight{census population size.}

\end{frame}

%%

\begin{frame}[t]{$N_e$ example 1: monogamy}

\vspace{-\baselineskip}

\raggedcolumns
\begin{multicols}{2}

\noindent\includegraphics[width=\linewidth]{drift_canada_goose}

\columnbreak
Assume 500 Canada Goose

\begin{itemize}[label=\textcolor{white}{\textbullet}, leftmargin=6pt]
\item 200 males
\item 300 females
\item 1:1 mating ratio
\end{itemize}

\bigskip

\textcolor{orange5}{$N_e = 400$}

\end{multicols}

\end{frame}

%%


\begin{frame}[t]{$N_e$ example 2: harem polygyny}

\vspace{-\baselineskip}

\raggedcolumns
\begin{multicols}{2}

\noindent\includegraphics[width=\linewidth]{drift_elephant_seals}

\columnbreak
Assume 500 Elephant Seal

\begin{itemize}[label=\textcolor{white}{\textbullet}, leftmargin=6pt]
\item 500 males
\item 500 females
\item 1:50 male:female mating ratio
\end{itemize}

Only 10 males are able to breed.

\bigskip

\textcolor{orange5}{$N_e \approx 39$}


\vspace{3\baselineskip}

\textcolor{gray}{$N_e = 4(N_mN_f)/(N_m + N_f)$}
\end{multicols}


\end{frame}

%% BOTTLENECK

\begin{frame}[t]{Population \highlight{bottlenecks} reduce heterozygosity.}

\vspace{-\baselineskip}

\begin{multicols}{2}

\includegraphics[width=\linewidth]{drift_elephant_seal_pop_size}

\columnbreak
{\footnotesize
\begin{tabular}{@{}L{2.9cm}rr@{}}

\toprule
Species	& Loci	& Alleles \tabularnewline
\midrule
Nor.~elephant seal	 &  21	 &  3.19 \tabularnewline
Hawaiian monk seal	 &  8	 &  3.5 \tabularnewline
Med. monk seal	 &  15	 &  2.32 \\[0.75ex]
Weighted mean	& & 3.09 \tabularnewline
& & \tabularnewline
Sou.~elephant seal	 &  17	 &  3.65 \tabularnewline
Grey seal	 &  9	 &  9.8 \tabularnewline
Harbor seal	 &  15	 &  5.13 \\[0.75ex]
Weighted mean	& & 5.41 \tabularnewline
\bottomrule

\end{tabular}
}%@ End small font
\end{multicols}

\vfilll

\tinyfill{Abadio-Cardozo et al.~2017. J.~Hered.~108:618.}
\end{frame}
%%


%% FOUNDER

\begin{frame}[t]{\highlight{Founder effects} reduce heterozygosity.}

\vspace{-\baselineskip}

\begin{multicols}{2}

\includegraphics[width=\linewidth]{drift_zebra_finch_map}


\columnbreak

\includegraphics[width=\linewidth]{drift_zebra_finch_nucleotide_diversity}

\bigskip

\centering

\includegraphics[height=3.5cm]{drift_zebra_finch}

\end{multicols}




\vfilll

\tiny Photo by JJ Harrison, \href{https://commons.wikimedia.org/wiki/File:Taeniopygia_guttata_-_Bushell's_Lagoon.jpg}{Wikimedia}, \ccbysa{4} \hfill Balakrishnan and Edwards.~2009. Genetics~181:645.

\end{frame}
%%


\end{document}
