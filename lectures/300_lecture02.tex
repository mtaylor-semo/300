%!TEX TS-program = lualatex
%!TEX encoding = UTF-8 Unicode

%\documentclass[t]{beamer}

%%%% HANDOUTS For online Uncomment the following four lines for handout
\documentclass[t,handout]{beamer}  %Use this for handouts.
\includeonlylecture{student}
\usepackage{handoutWithNotes}
\pgfpagesuselayout{3 on 1 with notes}[letterpaper,border shrink=5mm]
%	\setbeamercolor{background canvas}{bg=black!5}


%%% Including only some slides for students.
%%% Uncomment the following line. For the slides,
%%% use the labels shown below the command.

%% For students, use \lecture{student}{student}
%% For mine, use \lecture{instructor}{instructor}


%\usepackage{pgf,pgfpages}
%\pgfpagesuselayout{4 on 1}[letterpaper,border shrink=5mm]

% FONTS
\usepackage{fontspec}
\def\mainfont{Linux Biolinum O}
\setmainfont[Ligatures={Common, TeX}, Contextuals={NoAlternate}, BoldFont={* Bold}, ItalicFont={* Italic}, Numbers={Proportional, OldStyle}]{\mainfont}
%\setmonofont[Scale=MatchLowercase]{Inconsolata} 
\setsansfont[Scale=MatchLowercase]{Linux Biolinum O} 
\usepackage{microtype}

\usepackage{graphicx}
	\graphicspath{%
	{/Users/goby/Pictures/teach/300/lectures/}%
	{/Users/goby/Pictures/teach/163/common/}} % set of paths to search for images

\usepackage{amsmath,amssymb}

%\usepackage{units}

\usepackage{booktabs}
\usepackage{multicol}
%	\setlength{\columnsep=1em}

\usepackage{textcomp}
\usepackage{setspace}
\usepackage{tikz}
	\tikzstyle{every picture}+=[remember picture,overlay]

\usepackage[version=3]{mhchem}

\usepackage{csquotes}

\mode<presentation>
{
  \usetheme{Lecture}
  \setbeamercovered{invisible}
  \setbeamertemplate{items}[square]
}

\usepackage{calc}
\usepackage{hyperref}

\newcommand\HiddenWord[1]{%
	\alt<handout>{\rule{\widthof{#1}}{\fboxrule}}{#1}%
}



\begin{document}

\lecture{student}{student}

\begin{frame}[t]{What was missing from Darwin's concepts?}
\vspace{-\baselineskip}
\begin{multicols*}{2}
	\highlight{Species descend from common ancestor.}
\bigskip

\highlight{Adaptations evolve by natural selection.}

\vfilll

\columnbreak

\noindent\textit{There is grandeur in this view of life, with its several powers, having been originally breathed into a few forms or into one; and that, whilst this planet has gone cycling on according to the fixed law of gravity, from so simple a beginning endless forms most beautiful and most wonderful have been, and are being, evolved.}

\vspace{0.5em}

— Charles Darwin, \textit{On the Origin of Species}, 1st edition, 1859

\end{multicols*}
\end{frame}

\begin{frame}{The \highlight{evolutionary synthesis} built upon Darwin's foundation.}

\hangpara Natural selection and genetic processes that operate within species account for the \highlight{\textit{origin of new species.}}

\end{frame}

\begin{frame}[t]{The \highlight{evolutionary synthesis} determined that}
\hangpara the unit of evolution is the population,

\hangpara genetic variation is due to random mutation and genetic recombination,

\hangpara genetic variation creates phenotypic variation, and

\hangpara natural selection acting on phenotypic variation is the primary cause of evolutionary change.

\end{frame}

\begin{frame}[t]{The \highlight{evolutionary synthesis} determined that}
\hangpara species are pools of shared alleles, 

\hangpara speciation occurs when populations become reproductively isolated, and

\hangpara macroevolution is the same process as microevolution.


\end{frame}

\lecture{instructor}{instructor}

{
\usebackgroundtemplate{\includegraphics[width=\paperwidth]{thinking_gorilla}}
\begin{frame}[b]

	\tiny\hfill\textcolor{white}{\href{https://www.flickr.com/photos/34209020@N02/9612772316}{Elizabeth Haslam, Flickr,}~\ccbync{2.0}} 
\end{frame}
}

\lecture{student}{student}

\begin{frame}[t]{Answers? Questions!}

\vspace{-\baselineskip}

\hangpara What are the four nucleotides that make up DNA? \mode<beamer>{\pause \quad\textbf{A C G T}}

\pause

\hangpara What is the fifth nucleotide found only in RNA? Which nucleotide from DNA does it replace? \mode<beamer>{\pause \quad \textbf{U replaces T}}

\pause

\hangpara Which nucleotides are purines? Which are pyrimidines? \mode<beamer>{\pause \newline \hspace*{1em} \textbf{Purines: A G; Pyrimidines: C T}}

\pause

\hangpara In DNA, which pairs of nucleotides bond together to form the double helix? \mode<beamer>{\pause \hspace*{1em}\textbf{\ce{A\bond{=}T \quad C\bond{3}G}}}

\end{frame}

\begin{frame}[t]{More questions? More answers!}

\vspace{-\baselineskip}

\hangpara What is transcription?  What is translation?
	\mode<beamer>{\pause \newline \hspace*{1em} \textbf{Transcript: DNA to mRNA; Translate: mRNA to protein}}

\pause

\hangpara How many different amino acids are used to build proteins in living organisms? \mode<beamer>{\pause \textbf{20.} How many can you name?}

\pause

\hangpara What is the universal genetic code? 
\mode<beamer>{\pause \hspace*{1em} \textbf{Matches 3-nucleotide codons to specific amino acids, start and stop codons.}}

%\pause

%\hangpara What is an allele?

\end{frame}


\begin{frame}{Speak truth to power (at the molecular level).}

\vspace{-\baselineskip}

\hangpara All DNA is in genes. \mode<beamer>{\pause \quad \textbf{False. Most \textsc{dna} is non-coding.}}

\pause

\hangpara Genes code for traits (e.g., a blue-eye gene, a brown-eye gene) \mode<beamer>{\pause \quad \textbf{False.}}

\pause

\hangpara All genes code for proteins. \mode<beamer>{\pause \quad \textbf{False. Some encode \textsc{rna.}}}

\pause

\hangpara Only genes evolve. \mode<beamer>{\pause \quad \textbf{False. All \textsc{dna} evolves by mutation.}}

\pause

\hangpara Only genes are subject to natural selection. \mode<beamer>{\pause \quad \textbf{True. Probably.}}

\end{frame}

\begin{frame}[t]{Next time: Hardy-Weinberg review.}


\end{frame}

\end{document}
